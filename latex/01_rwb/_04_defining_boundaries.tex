% !TeX root = ../dissertation.tex

\section{Defining the Boundaries of \RWB}

Early adopters of the \vermes's taxonomy experimented with applying the term \rwb to a wide range of \secondtemple Jewish literature. The discussion about which texts should fall under the rubric of \rwb has continued into the present and remains a point of scholarly contention. Insofar as ``rewritten'' texts can be measured by how closely they resemble their \emph{Vorlagen}, defining the boundaries of \rwb focuses on which texts are \emph{too far} from their \emph{Vorlagen} to meaningfully be considered ``rewritten,'' forming what I will refer to as the ``upper bound'' and texts which are \emph{too close} to their \emph{Vorlagen} to be considered distinct literary works, forming the ``lower bound.'' At the upper bound, for example, the \emph{Book of the Watchers} and the \emph{Book of Giants} clearly are rooted in the biblical tradition, yet most scholars do not consider them sufficiently dependent on the text of Genesis to be considered ``rewritten.'' Although they take Genesis 6:1--4 as a point of departure, their narratives do not explicitly return to the biblical text in a concrete way. Conversely, at the lower bound, the Samaritan Pentateuch and the \q{4}{ReworkedPent}{}, although they certainly modify their \emph{Vorlagen} (and in that sense are ``rewritten'') are more often considered examples of alternate textual \emph{editions} rather than rewritten works. Likewise, the Targums and \lxx, as translations, are frequently excluded from discussions of \rwb at the lower bound because they were, presumably, meant to be perceived as the same literary work as their \emph{Vorlagen}. 

%% TODO: Add footnote about the perception of \lxx and Targums as "Bible" using ancient sources.

\subsection{The Upper Bound}

\vermes's use of the term \rwb grew out of the concrete examples of texts that exhibited the sorts of exegetical practices relevant to later aggadic traditions. As others adopted the term, however, the question of how to abstract the concept to something meaningful that could be applied to other texts was explored by a number of scholars.%
%
\footnote{See especially \cite{nickelsburg_stone1984}, \cite{harrington_kraft-nickelsburg1986}, and more recent contributions such as \cite{crawford2008}, \cite{falk2007}.}
%
These early applications of the term \rwb, like \vermes's use, did not tend to carry a technical nuance and instead focused on the ways that numerous texts reappropriated biblical stories, figures and themes in their own works. 

In his 1984 article ``The Bible Rewritten and Expanded,'' George Nickelsburg discusses a number of texts which are ``very closely related to the biblical texts, expanding and paraphrasing them and implicitly commenting on them.''\autocite[89]{nickelsburg_stone1984} We should note that, although the article does deal with \rwb, it includes a discussion of texts which even Nickelsburg does not consider ``rewritten'' (as the title indicates) discussing texts which introduce wholly new material into the traditions of the Bible. \autocite[89--90]{nickelsburg_stone1984}

Nickelsburg does, however, provide a list of texts which he loosely describes as examples of biblical rewriting: \firstenoch, \emph{Book of Giants}, \jub, \ga, \ant, the Books of Adam and Eve (\emph{Apocalypse of Moses}, \emph{Life of Adam and Eve}), and some Hellenistic Jewish Poets including Philo's \emph{On Jerusalem}, Theodotus's \emph{On the Jews}, and the \emph{Exagoge} by one ``Ezekiel the Poet of Tragedies.'' Compared to \vermes's list, Nickelsburg's represents a maximalist understanding of the \rwb phenomenon. The inclusion, especially, of \firstenoch illustrates his tendency to include works that build off of the biblical text (in this case, Genesis 6:1--4), but do not track with the biblical narrative for long stretches. 

One of the more interesting contributions that Nickelsburg makes to the conversation is his idea that biblical rewriting followed a trajectory from rewriting smaller units of the Bible---involving short stories that deal with particular events from the biblical text---to longer, more systematic, treatments which span multiple biblical books. His treatment of \firstenoch (which is, at least in part, the earliest text that he deals with) is illustrative of this approach. Rather than dealing with \firstenoch as a whole, Nickelsburg addresses the various rewritings of the flood narrative throughout \firstenoch as well as in the Book of Giants (which is not formally a part of \firstenoch, but has a clear connection to the work). Setting aside for the moment that \firstenoch is a composite work, we can appreciate that the flood story from Gen 6--9 is retold and to varying degrees reinterpreted throughout \firstenoch.\footnote{By my count, there   are six retellings of the flood in \firstenoch: 6--11; 54:7; 64--69;   83--84; 86--89; and 106--107.} 

Although Nickelsburg generally accepts that the rewritten texts ``comment'' on the Bible, he notes that the posture toward the biblical text is also not uniform even among the agreed upon \rwb texts. For example, while the author of \jub's concerns are largely halakhic and the book makes explicit reference to the biblical text, the authority assumed by the author of \jub does not (at least rhetorically) originate in the exposition of the Torah, but in the ``immutable heavenly tablets.''\autocite[100--101]{nickelsburg_stone1984} Nickelsburg thus states: 

\begin{quote}
    This process of transmitting and revising the biblical text reflects a remarkable view of Scripture and tradition. The \psgraphic ascription of the book to an angel of the presence and the attribution of laws to the heavenly tablets invest the author's interpretation of Scripture with absolute divine authority."\autocite[101]{nickelsburg_stone1984}
\end{quote} 

In contrast, \ga seems to have very little interest in halakhic matters and instead seems to just elaborate on the story by giving detailed geographic information and providing the reader with more dramatic characters \autocite[106]{nickelsburg_stone1984}. Finally, he observes that \lab likewise differs with \jub in its omission of halakhic matters and its ``highly selective reproduction of the text.''\autocite[110]{nickelsburg_stone1984} This selectivity also differs from the \ga, which otherwise is ``characterized by the addition of lengthy non-biblical incidence.''\autocite[110]{nickelsburg_stone1984} 

Ultimately, Nickelsburg differs from \vermes mainly in the way he views the Bible during the late \secondtemple period. Although Nickelsburg observes that the preoccupation with certain texts suggests that they were held in high regard, he does not have the same interest in tying the exegetical practices of, for example, \jub, with earlier inner-biblical or later haggadic traditions. Because Nickelsburg treats \rwb as a process, he is able to highlight the fact that, for example, \firstenoch does indeed ``rewrite'' certain pericopae from Genesis despite the fact that the whole book (which, we should note, is a composite text to begin with) does not maintain a ``centripetal'' relationship with the biblical narrative. 

Daniel Harrington's 1986 contribution entitled ``Palestinian Adaptations of Biblical Narratives and Prophecies I: The Bible Rewritten (Narratives),'' adopts the term \rwb to talk about texts produced around the turn of the era by Palestinian Judaism that ``take as their literary framework the flow of the biblical text itself and apparently have as their major purpose the clarification and actualization of the biblical story.''\autocite[239]{harrington_kraft-nickelsburg1986} In this regard, he follows \vermes closely in how he imagines \rwb to function. Yet, compared to \vermes, he operates with a slightly expanded list of rewritten texts. In addition to \jub, \ga, Ps. Philo's \lab and Josephus's \ant, he also includes the \emph{Assumption of Moses} and the \templescroll. Furthermore, he makes a point to suggest that a number of other texts may be able to be included in the list, including \emph{Paralipomena of Jeremiah}, \emph{Life of Adam and Eve/Apocalypse of Moses}, and \emph{Ascension of Isaiah}. Harrington's major contribution is his explicit rejection of \rwb as a category or literary genre (more on this, below) in favor of a process-oriented approach. Because of this fact, Harrington takes a broad view of rewriting and allows, to some degree, that this process be understood similar to a reception history (although, this is my term, and not his). Harrington's inclusion of the \templescroll marked a significant deviation from \vermes's use of the term by including non-narrative material under the rubric of \rwb. While several of Harrington's other suggested text are not considered \rwb by many scholars, the inclusion of other non-narrative texts, in particular the \templescroll, has gained wide acceptance.%
%
\footnote{At the time that \citetitle{vermes1961} was published, the \templescroll had not yet been published, so it is hardly fair to expect \vermes to include it in his discussion. That said, my understanding of \vermes's conception of \rwb---even taking into account the existence of works such as the \templescroll---would preclude the inclusion of the \templescroll from the category of \rwb. This is a point at which even those who broadly agree with \vermes, such as Moshe Bernstein, take issue with \vermes's definition. See \cite[183--184]{bernstein_textus2005}.}
%
Building on the notion that \rwb could also include non-narrative material, George Brooke, in a more recent treatment of the topic, defines \rwb as ``any representation of an authoritative scriptural text that implicitly incorporates interpretive elements, large or small, in the retelling itself.''\autocite[777]{brooke_schiffman-vanderkam2000} Adopting a ``loose'' definition of the term, Brooke includes in his discussion biblical texts that rewrite other biblical texts such as Deuteronomy and Chronicles in addition to examples of texts which ``rewrite'' portions of each of the major divisions of the Hebrew Bible, most of which were found at \qumran.%
\footnote{Brooke categorizes the texts as follows: Reworked Pentateuchs, Rewritten Pentateuchal narratives, Rewritten Pentateuchal laws, Rewritten Former Prophets, Rewritten Latter Prophets, and Rewritten Writings. \cite[778--780]{brooke_schiffman-vanderkam2000}. See also \cite{brooke_herbert-tov2002} and the important work of \cite{falk2007}.}

The purposes of rewriting, according to Brooke, are manifold, but in each case the (re)writer augmented or repurposed an authoritative base text for some new context. He writes: 

\begin{quote}
    The rewriting seems to have a variety of purposes, among which are the following: to improve an unintelligible base text, making it more comprehensible (11Q19); to improve a text by removing inconsistencies---often through internal harmonization (\q{4}{paleoExod}{m}); to justify some particular content by providing explanations for certain features in the base text (1QapGen); to make an authoritative text serve a particular function, perhaps in a liturgical setting (4Q41); to encourage the practice of particular legal rulings (\jub); and to make an old text have contemporary appeal (\templescroll).\autocite[778]{brooke_schiffman-vanderkam2000}
\end{quote} 

While I am sympathetic to the more maximalist approaches of Nickelsburg, Harrington, and Brooke, none of these treatments offer any concrete criteria for delineating between \rwb and texts that merely allude to biblical stories. Philip Alexander has suggested that certain works which are primarily ``expansive'' (the Book of Giants, the Book of Noah) should not be considered \rwb because their relationship to the biblical text is ``centrifugal''---that is, they take the biblical text as a point of departure while formally \rwb texts show a ``centripetal'' relationship to the biblical text---that is, they expand beyond the biblical text, but remain tightly coupled to the text \emph{as it exists in the Bible.} Alexander writes: 

\begin{quote}
    Rewritten Bible texts are centripetal: they come back to the Bible again and again. The rewritten Bible texts make use of the legendary material, but by placing that material within an extended biblical narrative (in association with passages of more or less literal retelling of the Bible), they clamp the legends firmly to the biblical framework, and reintegrate them into the biblical history. \autocite[117]{alexander_carson-williamson1988}
\end{quote} 

This ``centripetal'' relationship to the biblical text, I believe, should form the upper bound of what is called \rwb. Therefore, for the purposes of this study, works such as \firstenoch, will not be treated because they do not exhibit this close centripetal relationship. On the other hand, I adopt a more expansive understanding of \rwb than that of \vermes and include works within the Hebrew Bible itself (Deuteronomy and Chronicles), as well as non-narrative works such as the \templescroll which, I believe, do exhibit a centripetal relationship to the biblical text. 

%% TODO: Add something about Falk?

\subsection{The Lower Bound: Between Bible and \RWB}

Another recent avenue of investigation has been to explore the boundaries between the biblical text, editions, translations, and rewritten biblical texts. \vermes, of course, utilized the targums liberally in \emph{Scripture and Tradition}, but his goal was to blur the line between post-biblical texts. Most scholars treating RwB, however, are not inclined to include the targums among RwB. But the targums---and for that matter the \lxx and Samaritan Pentateuch---do uniquely represent interpretive traditions. Furthermore, the instability of the biblical text during the late \secondtemple period, as exhibited by the varied editions of Jeremiah found at Qumran and other liminal texts, such as 4QReworkedPent, has problematized the question of what may have constituted ``Bible'' (or, more properly, ``scripture'') at the time. 

Unsurprisingly, Emanuel Tov has been at the forefront of this investigation. In his 1998 article, ``Rewritten Bible Compositions and Biblical Manuscripts, with Special Attention to the Samaritan Pentateuch,'' Tov's purpose is to specify the ``fine line between biblical manuscripts and rewritten Bible texts.''\autocite[334]{tov_dsd1998} By this, Tov means that he is concerned with what I have termed the ``lower bound'' of the definition of RwB, specifically, the distinction between a text \emph{edition} and a distinct composition, which Tov considers ``rewritten.'' The primary difference between these two categories of texts, according to Tov, is not how dramatically the daughter text diverges from its parent, but the \emph{purpose} of the daughter text \autocite[334]{tov_dsd1998}. According to Tov, this purpose is mirrored in the putatively authoritative status of the ``biblical'' text \visavis the rewritten text which, he says, is not authoritative (although, he seems to suggest that this is up for debate\autocite[337]{tov_dsd1998}). For example, he notes that the extant texts of Jeremiah, while widely divergent in length and order, still represent ``biblical Jeremiah'' which carries some authoritative weight. Tov is, however, careful to point out that the nature of this authority is not clear and ``the boundary between the biblical and non-biblical texts was probably not as fixed as we would have liked for the purpose of our scholarly analysis.''\autocite[335]{tov_dsd1998} 

Carrying a similar trajectory, Michael Segal's 2005 article ``Between Bible and Rewritten Bible,'' in the tradition of Alexander, attempts to enumerate a series of criteria by which scholars can define the lower bound and distinguish between ``editions'' of biblical texts and ``rewritten'' texts. 

Segal's understanding of the role of RwB is rooted in the conviction that a rewritten text is a ``new'' work that derives its own authority by means of its association with a biblical text. The new composition carries with it the purpose and any theological or ideological \emph{Tendenzen} of the new author, builds off of the authoritative status of the underlying text.\autocite[11]{segal_henze2005} Segal writes: 

\begin{quote}
    Even though these rewritten compositions sometimes contain material contradictory to their biblical sources, their inclusion within the existing framework of the biblical text bestows upon them legitimacy in the eyes of the intended audience \ldots{} the inclusion of this material within the framework of the biblical passages under interpretation transforms the ideas of the later writer into authoritative and accepted beliefs.\autocite[11]{segal_henze2005}
\end{quote} 

 And further: 

\begin{quote}
    The nature of the relationship between rewritten biblical compositions and their sources constitutes a paradox. On the one hand, the rewritten composition relies upon biblical texts for authority and legitimacy. The author claims that any new information included in the later work already appears in earlier sources. But simultaneously, the insertion of new ideas into the biblical text, ideas that may even contradict the beliefs and concepts of the original biblical authors, undermines the very authority that the rewriter hopes to utilize"\autocite[11-12]{segal_henze2005}
\end{quote} 

While I find Segal's characterization of RwB texts problematic,%
%
%[[TODO: WHY?]]
%
his main contribution to the discussion are his criteria for distinguishing between ``biblical'' and RwB texts. He distinguishes between ``external'' and ``internal'' characteristics. 

\subsubsection{External Characteristics}

Segal also identifies two external characteristics of RwB texts: ``language'' and ``relationship between the source and its revision.'' 

 \begin{enumerate}
    \item   Language: While he offers little rationale for this criterion, Segal categorically dismisses the possibility that any RwB text could have been written in a language other than its \emph{Vorlage}.

    \item The Textual Relationship between the Source and Its Revision: The underlying text must be ``visible'' in the RwB text. He uses the book of Chronicles as the parade example of this relationship and notes the caveats necessary in dealing with \emph{Vorlagen} from this period (i.e., it is difficult to say what is ``rewritten'' versus what is just another variant in the \emph{Vorlage}).
\end{enumerate} 

Segal notes that both of these criteria, in fact, apply to textual editions, as well as to RwB texts \autocite[20]{segal_henze2005}. In other words, these are not ``distinguishing'' criteria, so much as the baseline for consideration. One may demur, however, that if a single criterion, such as language, categorically excludes several texts which meet all the other criteria (below; by this definition he excludes \ga, Josephus's \ant, and Ps. Philo's \emph{LAB}), perhaps the problem is with the criterion.

\subsubsection{Literary Criteria for \RWB}

It is the ``Literary Criteria'' which Segal, ultimately, believes provide the \emph{definition} of RwB texts.\autocite[20]{segal_henze2005} Segal provides six internal criteria: 

\begin{enumerate}

    \item Scope of the Composition: ``Editions'' of texts cover the same material as their source. In other words, one expects an edition of Genesis to cover the same material as the book of Genesis; pluses and minuses do not stray into other works. On the other hand, rewritten texts ``do not generally correspond to the scope of their sources.''\autocite[20]{segal_henze2005} For example, he observes that \jub covers Genesis and part of Exodus, and Chronicles covers parts of Samuel and Kings. Oddly, he also notes that Ps. Philo---which is not written in Hebrew---runs from Genesis into 1 Samuel. He writes: ``In all these examples the change in the scope of the composition created a new literary unit.'' \autocite[20--21]{segal_henze2005}
  
    \item New Narrative Frame: Several of the RwB texts include a framing narrative. His examples include the \templescroll and \jub, both of which re-frame the ``biblical'' material. In the case of both works, the Torah is assumed and the new work presumed to be a reflection of a second, direct revelation of the law to Moses, albeit by different means (and fragmentary, in the case of the \templescroll). In \jub, the angel of the Presence revealed this ``second Torah'' during Moses's second ascent (Exod 24). On the other hand, the Temple scroll seems to begin in Exod 34 \autocite[22]{segal_henze2005}.
   
    \item Voice: While biblical narratives are generally written in a ``detached'' third person style, Segal observes that both \jub and the \templescroll ``change the voice of the narrator throughout.'' \autocite[22]{segal_henze2005} As far as I can tell, what Segal means is that in these RwB texts, certain events which are narrated in the third person in the biblical text are re-framed as, for instance, direct discourse in the first person by an angel, or even by God.\footnote{This may seem like a   minor quibble, but the ``narrator'' has a distinct and technical   meaning in narrative criticism which should be maintained. I would   note, however, that this sort of reframing is not unique to RwB,   since, e.g., Deuteronomy does something similar (perhaps Segal   considered Deuteronomy to be RwB?).}
 
    \item Expansion versus Abridgment: By-and-large, text editions are \emph{additive}. That is to say, when there is a discrepancy between the amount of content (as opposed to the order), typically the shorter text is considered older. Segal is here concerned with editorial changes, and not with scribal errors, which, of course could go in both directions (through parablepsis et al.). This property, he contends, is rooted in the conviction of the scribes that in order to reproduce a text, one must reproduce the \emph{entire} text.\autocite[24]{segal_henze2005} \rwb texts, however, felt free to add \emph{or remove} material because their authors understood themselves to be composing entirely new works.\autocite[24]{segal_henze2005} 

    \item Tendentious Editorial Layer: ``Editions'' do not change fundamental ideology of the work. For example, differing editions of Jeremiah may differ but those differences do not change the fundamental ideology of the work. Likewise, expansion and addition to the work (e.g.~additions to Daniel) are in line with the theological \emph{Tendenz} of the shorter book. On the other hand, RwB texts freely alter the ideologies of the text, for example, \jub.\autocite[25]{segal_henze2005}

    \item Explicit References to the Source Composition: ``Editions'' cannot (in a meta-discursive sense) reference its base text while RwB texts can (e.g., \jub reference to the Torah).
\end{enumerate} 

Although I disagree fundamentally with Segal's conclusions, he makes a number of useful observations about several literary features which are common to \rwb texts. If one ignores his linguistic criterion, his literary criteria form a solid formal baseline for discussing \rwb texts. Thus, while Segal would categorically dismiss \ga, \ant, \lab, et al. for not being composed in Hebrew, in fact, his literary criteria fit these works very well.

More recently, Tov has returned to the topic of text editions and their relationship to the phenomenon of RwB.\autocite{tov_krarrer-kraus2008} Tov addresses three ``strange'' texts from the \lxx which, for one reason or another, differ significantly from the preserved MT (3 Kingdoms, Esther, and Daniel). Evoking a number of Segal's criteria\autocite{segal_henze2005} for inclusion in the category of \rwb (which he acknowledges to be well accepted, if not terribly well defined), Tov suggests that these \lxx texts likewise may exhibit 1) a new narrative frame, 2) expansion and abridgment, and 3) a tendentious editorial layer and therefore may be candidates for RwB. 

It is important to think about what Tov and Segal are trying to accomplish in these articles: They are trying to connect scribal practices which allowed for exegetical additions and emendations to ``authoritative texts''---dramatic examples of which are provided by \sampent and \lxx---to the practices which produced the \emph{new compositions} which scholars refer to as RwB texts.%
%
\footnote{Although, one wonders why the Targums are not included here. Perhaps it is because Tov is arguing for Hebrew \emph{Vorlagen} of these texts, while the Targums represent a translation.}

What Tov's articles in particular demonstrate, however, is that the issue of authorial \emph{intent} and \emph{purpose} may be at the heart of the distinction between text edition and RwB. Of course, such intents are not things that can be objectively proven, but I can not help but feel that such considerations ought to factor into reconstructions of ancient practices, even if we must settle for speculation. Thus it may be that certain especially troublesome texts, such as 4QReworkedPent, cannot be meaningfully categorized as ``edition'' or ``rewritten'' based on formal characteristics at all. Instead, it may be that such distinctions, ultimately, must be addressed by how we reconstruct a text's function within its social context with due consideration to the fact that such contexts are not monoliths and differ across time and space. For example, I imagine that the author of \ga did not consider himself to be creating a new edition of Genesis, nor do I imagine that the author's contemporaries understood it to be such. The same goes for \jub and Chronicles (for Samuel--Kings). On the other hand, the editors and translators of the \sampent, Targums, and \lxx presumably \emph{were} producing texts whose social functions aligned with (albeit, not perfectly) the biblical text.%
%
\footnote{I am not claiming that the social function of the \lxx and (esp.) the Targums would have been identical or indistinguishable from their \emph{Vorlagen}, however, the fact that both the \lxx and Targums were at times used liturgically---in any capacity---speaks to a unique social function by comparison to \rwb texts. That Chronicles is used liturgically is only a function of the fact that it became a part of the Hebrew Bible on its own merits. Canonically speaking Chronicles' does not derive its authority based on its relationship to Samuel--Kings.}
%
Therefore, what matters for our purposes is not, necessarily, what formal characteristics distinguish \rwb texts, but rather what social \emph{function} such texts may have held in antiquity \visavis the biblical text.