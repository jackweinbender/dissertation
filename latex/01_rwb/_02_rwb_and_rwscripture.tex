% !TeX root = ../dissertation.tex

\section{\RWB and Rewritten Scripture}

Since \vermes coined the term \rwb a number of scholars have suggested that the term be modified to more accurately reflect the (now, well established) fact that there was no ``Bible'' in the late \secondtemple period and that many of the works that would eventually make up the Hebrew Bible did not have stable textual witnesses that could be meaningfully ``rewritten.'' Because of these difficulties, scholars have, in recent years, suggested alternate designations for the phenomenon under investigation, the most widely used of which is ``rewritten \emph{scripture}.'' \vermes's original term \rwb was a product of its time. It took for granted the existence of a canonical ``Bible'' that more-or-less resembled the Bible used by the rabbis in the early centuries CE and term rewritten scripture was intended to correct what scholars perceived as an anachronistic reference to this canon of scripture during the late \secondtemple period.\autocites[58--59]{campbell_zsengeller2014}[See also][]{ulrich_mcdonald-sanders2002}[and][]{ulrich_zsengeller2014} 

Apart from the anachronistic reference to a ``Bible,'' one of the primary objections to the use of the term \rwb is the implicit assertion that \rwb texts necessarily fall outside the Bible.\autocite[61]{campbell_zsengeller2014} The notion that a rewritten biblical text by definition, could not be considered ``Bible'' itself runs contrary to, on the one hand, texts such as Chronicles and Deuteronomy which---for all intents and purposes---``rewrite'' their biblical \emph{Vorlagen} but are themselves a part of ``the Bible'' and on the other hand texts such as \jub and the \templescroll which likely were considered ``scripture'' among certain groups in antiquity and, in the case of \jub, remains a part of the biblical canon of the Ethiopian Orthodox Church in the present. [[Nuance this see: R.W. Cowley  "the ethiopina canon of today" and Baynes "Enoch and Jubilees..."]] 

Yet, I am not at all convinced that substituting the term ``scripture'' for ``Bible'' meaningfully affects the way that scholars have continued to discuss the topic at hand. While I agree that ``the Bible'' as we know it from the early centuries CE did not exist during the late \secondtemple period, I likewise find the strict reading of ``Bible'' to mean ``the Hebrew Bible (as we know it)'' unnecessarily rigid. To say that \jub was a part of the \qumran Community's ``Bible'' does not carry a vastly different nuance, it seems to me, than to say that the \qumran Community considered \jub to be ``scripture.'' Insofar as a particular group---given a set of texts---can determine which it considers to be ``scripture'' it has, at least in common parlance, a ``Bible.'' That said, I can appreciate the desire to fine-tune our terminology to better reflect the scholarly discourse. 

It could, however, be argued that the term ``scripture'' is no more ancient a term than ``Bible.'' Scholars such as James VanderKam have done important work in trying to discern which texts may have been considered ``authoritative scripture'' at \qumran,\autocite{vanderkam_dsd1998} but the fact remains that such endeavors start with the assumption that the ancients utilized a notion at all similar to what we consider ``textual authority.'' While there is good reason to believe that some texts were more important than others during the \secondtemple period (e.g., the Pentateuch, Isa et al.), the degree to which they considered them ``scripture'' is not at all clear, much less as a binary category. Thus, replacing the term \rwb with Rewritten Scripture, it seems to me, may very well shift the semantic burden from a well defined modern category of text to an ill-defined ancient category. 

For the sake of simplicity, I will follow \vermes in this study and simply use the term ``\RwB.'' In doing so, I realize that I am deviating from what has become the common scholarly terminology. Yet, I find some comfort in \vermes's own take on the matter, who writes, ``Frankly, replacing `Bible' by `Scripture' strikes me as a mere quibble\ldots{}I suggest therefore that we stick with the `Rewritten \emph{Bible}' and let the music of the argument begin.''\autocite[original emphasis]{vermes_zsengeller2014} 