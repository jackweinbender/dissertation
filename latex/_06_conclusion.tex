% !TEX root = dissertation.tex

\chapter*{Conclusions}
\addcontentsline{toc}{chapter}{Conclusions}

% Restate the main claims of your argument and the interventions that you make

The primary purpose of this dissertation is to introduce memory studies as a theoretical framework for reading \rwb texts. Since \Vermes originally coined the term \rwb, the conversation about how so-called \rwb texts has largely focused on the relationship of \rwb texts to their putative biblical \vorlagen. \vermes understood \rwb as a form of early biblical exegesis that used rewritten narratives to ``anticipate questions, and to solve problems in advance.''%
    \autocite[95]{vermes1961}
His characterization of these texts sought to locate \rwb on a continuum plotted on a trajectory stretching from the so-called inner-biblical exegesis found in the Hebrew Bible to the later rabbinic forms of exegesis such as midrash.%
    \footnote{One of \vermes' primary goals in \citetitle{vermes1961} was to disabuse the idea that the exegetical practices of the rabbis were novel inventions of the rabbinic period. Instead, \vermes argued that many aggadic interpretations could be traced back into the \secondtemple period.}
Although the conversation has moved beyond \vermes' characterization to include other texts such as the \templescroll and (arguably) \q{4}{ReworkedPentateuch}{},%
    \footnote{These texts, of course, were not available for \vermes to consider when \citetitle{vermes1961} was written. All the same, \vermes was adamant about characterizing \rwb as a narrative form of exegesis.}    
the conversation about the function of \rwb has continued to focus largely on the ways that the \rwb texts reflect how \secondtemple Jews understood their sacred texts, in other words, how \rwb texts functioned \emph{exegetically}. I take Crawford's characterization of \rwb as representative of the mainstream approach. She writes:
\begin{quote}
    These Rewritten Scriptures constitute as category or group of texts which are characterized by a close adherence to a recognizable and already authoritative base text (narrative or legal) and a recognizable degree of scribal intervention into that base text for the purpose of exegesis.\autocite[12--13]{crawford2008}
\end{quote}
\noindent
Crawford is careful to recognize that the function of these works could have differed among different communities and acknowledges that the texts themselves exhibit diversity in how they characterize themselves as authoritative or revelatory as well as how closely they adhere to their putative base text.%
    \autocite[13]{crawford2008}
Yet, Crawford's characterization still implicitly places the onus of interpretation on the individual scribe and very little concern is given to the social and cultural forces may have informed those scribal practice. Thus, it remains the case that, according to most scholars, the central concern of these texts was \emph{exegetical} and this line of reasoning has largely gone unchallenged. 

The central claim of this dissertation, however, is built on the idea that the characterization of \rwb texts as primarily focused on \emph{exegesis} does not adequately address \rwb texts as independent works of literature and cultural products on their own terms. Although \rwb texts are certainly products of scribal culture, such activities can not be divorced from the broader social and cultural location of those scribes. Even the most novel scribal interventions into a text must be understood within the context of its social function.

In this dissertation, therefore, I have suggested that memory studies offers a compelling theoretical model for describing \rwb texts as cultural products that participated in discrete social discourses about Israel's remembered past. In other words, I argue that the purpose of \rwb texts was \emph{not} primarily about explicating the biblical text. Instead, \rwb participated in broader cultural discourses about Israel's remembered past. The biblical texts (or, more precisely, the texts that later became the Bible), of course, played a central role in this process, but there is a fundamental difference between saying that \rwb texts illustrate how \secondtemple Judaism interpreted the Bible and saying that \rwb \emph{is fundamentally} a form of biblical interpretation used in the \secondtemple period. Approaching \rwb through the lens of social memory studies attempts to take a step back from this latter position and address how these texts \emph{functioned} as the means by which \secondtemple Judaism participated in discourses about its remembered past.


% Significance for RwB
    % THis is my main conclusion, it says that Memory Studies can be a useful way to talk about RwB and offers a more compelling an holistic approach to the category by focusing on the processes of transmission and the ways that societies adapt and construct conceptions of the past

% Significance for Memory studies (esp. within HB)
    % I think using, in particular plainly "fictional" works is a somewhat unique offering. Taking seriously the effect that fiction has on the conceptions of the past

% Areas of further study