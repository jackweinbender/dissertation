% !TEX root = dissertation.tex

\chapter*{Conclusions}
\addcontentsline{toc}{chapter}{Conclusions}

% Restate the main claims of your argument and the interventions that you make

The primary purpose of this dissertation is to introduce memory studies as a theoretical framework for reading \rwb texts. Since \Vermes originally coined the term \rwb, the conversation about \rwb has largely focused on the relationship of \rwb texts to their putative biblical \vorlagen. \vermes understood \rwb as a form of early biblical exegesis that used rewritten narratives to ``anticipate questions, and to solve problems in advance.''%
    \autocite[95]{vermes1961}
His characterization of these texts sought to locate \rwb on a continuum plotted on a trajectory stretching from the so-called inner-biblical exegesis found in the Hebrew Bible to the later rabbinic forms of exegesis such as midrash.%
    \footnote{One of \vermes's primary goals in \citetitle{vermes1961} was to disabuse the idea that the exegetical practices of the rabbis were novel inventions of the rabbinic period. Instead, \vermes argued that many aggadic interpretations could be traced back into the \secondtemple period.}
Although the conversation has moved beyond \vermes's characterization to include other texts such as the \templescroll and (arguably) 4QReworkedPentateuch,%
    \footnote{These texts were not available for \vermes to consider when \citetitle{vermes1961} was written. All the same, \vermes was adamant about characterizing \rwb as a narrative form of exegesis.}    
the conversation about the function of \rwb has continued to focus largely on the ways that the \rwb texts reflect how \secondtemple Jews understood their sacred texts, in other words, how \rwb texts functioned \emph{exegetically}. I take Crawford's characterization of  \rwb as representative of the mainstream approach. She writes:
\begin{quote}
    These Rewritten Scriptures constitute as category or group of texts which are characterized by a close adherence to a recognizable and already authoritative base text (narrative or legal) and a recognizable degree of scribal intervention into that base text for the purpose of exegesis.\autocite[12--13]{crawford2008}
\end{quote}
\noindent
Crawford is careful to recognize that the function of these works could have differed among different communities and acknowledges that the texts themselves exhibit diversity in how they characterize themselves as authoritative or revelatory as well as how closely they adhere to their putative base text.%
    \autocite[13]{crawford2008}
Yet, Crawford's characterization still implicitly places the onus of interpretation on the individual scribe and very little concern is given to the social and cultural forces may have informed those scribal practices. Thus, it remains the case that, according to most scholars, the central concern of these texts was \emph{exegetical} and this line of reasoning has largely gone unchallenged. 

The central claim of this dissertation, however, is built on the idea that the characterization of \rwb texts as primarily focused on \emph{exegesis} does not adequately address \rwb texts as independent works of literature and cultural products on their own terms. Although \rwb texts are certainly products of scribal culture, such activities can not be divorced from the broader social and cultural location of those scribes. Even the most novel scribal interventions into a text must be understood within the context of its social function.

In this dissertation, therefore, I have suggested that memory studies offers a compelling theoretical model for describing \rwb texts as cultural products that participated in discrete social discourses about Israel's remembered past. In other words, I argue that the purpose of \rwb texts was \emph{not} primarily about explicating the biblical text. Instead, \rwb participated in broader cultural discourses about Israel's remembered past. The biblical texts (or, more precisely, the texts that later became the Bible) naturally played a central role in this process, but there is a fundamental difference between saying that \rwb texts illustrate how \secondtemple Judaism interpreted the Bible and saying that \rwb \emph{is fundamentally} a form of biblical interpretation used in the \secondtemple period. Approaching \rwb through the lens of social memory studies attempts to take a step back from this latter position and address how these texts \emph{functioned} as the means by which \secondtemple Judaism participated in discourses about its remembered past.

Although memory studies have recently become an area of interest to biblical studies more generally, I chose to begin my series of case studies with the book of \chronicles because it is uniquely situated as an exemplar of \rwb that has already been treated from the perspective of memory studies. This fact is largely a function of the work's formal station within the field of Hebrew Bible studies---a field that has seen an increased interest in memory studies in recent years. In other words, \chronicles has been treated from a memory perspective as \emph{biblical} literature, but not as an example of \rwb. In \hyperref[chap:chronicles]{chapter three} I noted that among scholars who approach the book of \chronicles as biblical literature, the question of its status as an example of \rwb is rarely treated as more than a passing curiosity, despite the fact that it is commonly counted among the \rwb texts by scholars of \rwb. Thus, in \chronicles, we have an important intersection of scholarly discourses. Separately---and by different groups of scholars---\chronicles has 1) been fruitfully analyzed from a memory perspective and 2) been treated as an exemplar of \rwb. The juxtaposition of these two lines of scholarly inquiry is, I think, a good indication that applying memory studies to the other \rwb texts is a worthwhile endeavor. I use the book of \chronicles, therefore, as a point of departure for applying memory theory to \ga and \jub. 

My analysis of the book of \chronicles, while conventional, made the case that \chronicles should not be viewed primarily as focused on explaining the books of Samuel--Kings. I argued that the book of \chronicles fundamentally reimagines aspects of Israel's remembered past in an effort to provide a narrative frame for the events of the \secondtemple period. In particular, I discussed the role that ``sites'' of memory play in evolving discourses about the remembered past and how social memory can be modeled as a complex network of symbolic meaning.  Drawing heavily on the work of Ben Zvi and Pierre Nora, I argued that the book of \chronicles engages with the major sites of memory from Israel's remembered past and draws together its major metanarratives as a way to make sense of the events and circumstances of the \secondtemple period.

My treatment of \ga built on these fundamental ideas and moves the discussion more formally into the realm of literary analysis. In this chapter, I focused on the three-step cycle of memory reception and transmission and how \ga displays these steps: 1) the reception of cultural memory, 2) the reshaping of memory by contemporary social frameworks, and 3) the active construction, codification, and reintegration of memory for future transmission. I argued that the characterization of \ga as a rewriting of Genesis---while appropriate for \cols{19}{22}, may not be appropriate for \cols{1}{18}, since their \vorlagen seem to be extra-biblical. I discussed how genres are examples of social frameworks and the adaptation of \ga narratives into these frameworks can be understood as a process of memory. Finally, I discussed \ga's \psgraphical qualities as a particular form of memory discourse that ``speaks into'' the received tradition in a way that was different from other third-person accounts.

My final case study treated the book of \jub. In this chapter I argued that the book of \jub portrays itself as an authoritative revelation which invited the reader to incorporate new knowledge into their conception of the past in an effort to affect the behavior of its readers and to reinforce the practices of its remembering community. This chapter makes the crucial transition from treating memory and memory construction as abstract ideas to a discussion of the effects of memory on \emph{practice}. I argued that \jub engages with ``prescriptive discourses'' which had concrete effects on the practices of its remembering community. To make this case, I engaged with Hindy Najman's concept of ``Mosaic Discourse'' and suggested that this idea was tantamount to treating Moses (or Mosaic teaching) as a ``site'' of memory. I also argued that the 364-day calendar system at work in \jub served as a means of reinforcing certain practices of \secondtemple Judaism, in particular Sabbath observance. The super-annual system of weeks (seven-year cycles) and jubilees (49-year cycles) similarly were used to reimagine Israel's remembered past by inviting the reader to infer that some coming jubilee would bring with it release for those in debt-bondage, from the bondage of foreign occupation, and a renewed Israelite state during the \secondtemple period.

% Significance for RwB
What I have shown through these case studies is that characterizing \rwb texts as primarily focused on the \emph{exegesis} of their \vorlagen is not an adequate model for talking about \rwb as a cultural phenomenon. To say that the purpose of \chronicles was to explain Samuel--Kings, or the purpose of \ga was to explain Genesis, or \jub to explain Gen 1--Exod 1--12 sells short these texts as products of culture and as participants in a much broader set of contemporary social discourses. Memory studies, while not a panacea for every problem in describing \rwb, is uniquely suited to address these texts from a variety of perspectives and to integrate different approaches to textual analysis. For example, my treatment of David, Goliath, and Elhanan in 1 \& 2 Samuel brings together issues of textual transmission, text criticism, and redaction criticism under the umbrella of memory. In particular, the \chronicler's decision to credit Elhanan with killing Goliath's \emph{brother} has traditionally been explained text-critically. But treating this change only through the lens of textual criticism ignores the other forces, which I discussed, that led the \chronicler to his ultimate conclusion.%
    \footnote{See, for example the discussion in \cite[368--369]{japhet1993}.}

Each of my chapters, I think, have contributed something to memory studies more broadly. In \hyperref[chap:chronicles]{chapter three}, my observation that the processes of magnetism function like scale-free networks is, to my knowledge, a completely original suggestion within social and cultural memory studies---certainly within the field of biblical studies. In \hyperref[chap:ga]{chapter four} I think my literary approach to \ga was fairly innovative. In particular, my treatment of literary genres as examples of how social frameworks shape memory and the function of \psy for memory construction were both novel suggestions that I hope will prompt future study. Finally, in \hyperref[chap:jubilees]{chapter five} I think my emphasis on the practical effects of social memory is an extremely important discursive move to make for discussing an otherwise very abstract set of ideas. 

This dissertation also contributes to the growing body of literature within \secondtemple studies that is trending away from a ``canonically centered'' discourse about Jewish literature of the \secondtemple period. Although problematizing the concept of ``canon'' during the \secondtemple period is hardly new, the past several years there has been a tenancy toward further disjunction between established notions of textual authority and the literary production during this period.%
    \footnote{I have in mind, for example, the recent work of Eva Mroczek and the more established work of Hindy Najman, among others. See \cite{mroczek2016}; \cite{najman_jsj2012}; \cite{najman2003}.}
The implications of this study, I think, further call into question whether traditional terminology of ``textual authority'' and even ``scripture'' can be taken for granted as accurately descriptive of how people thought about these texts in antiquity. Take, for example, Bernstein's suggestion that the Abram Memoir be characterized as  ``rewritten Bible'' but the Lamech and Noah Memoirs of \ga be characterized as ``parabiblical'' because the latter do not rewrite the \emph{biblical} book of Genesis. Although it is certainly possible for a text to be multigeneric, as I suggested earlier, it seems to me that this should call into question whether the Bible/Scripture/Authoritative distinction was meaningful in antiquity. The question of whether the \ga's ``rewriting'' of portions of \firstenoch implies that \firstenoch was somehow ``authoritative'' begs the question of whether such categories were meaningful. Although I am not willing to completely throw out these distinctions, treating the production of \rwb texts as engaging with ``memory'' is a useful way to acknowledge this difficulty without projecting anachronistic categories onto these texts.

% % Significance for Memory studies (esp. within Hebrew Bible Studies)
% \rwb texts also bring unique issues to bear on the topic of memory studies. The application of memory theory within Hebrew Bible studies has tended to focus on diachronic transmission of memory and the ability of traditions to persists through time. 

Although this dissertation is not an exhaustive treatment of the topic of \rwb through the lens of memory studies, I think I have accomplished my goal of taking a ``sounding'' into the feasibility of using memory as a theoretical framework for contextualizing \rwb within a broader framework of cultural practice. I think, ultimately, this dissertation shows the usefulness of a memory approach to reading \rwb texts, in particular as a means to explore the social and cultural significance of these texts that extend beyond traditional ``exegetical'' models. The language of memory  brings with it a taxonomy for discussing these literary products in terms that are also meaningful outside \secondtemple studies. In this respect, I hope that this dissertation can serve as a point of departure for my own future research as well as a convenient starting point for others. 
