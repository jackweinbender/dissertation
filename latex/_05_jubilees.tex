% !TEX root = dissertation.tex

\nocite{dillamnn_jbw_kleine}
\nocite{ewald_zkm1844}

\chapter{(Re)Structuring the Past: Remembering Genesis in the Book of Jubilees}
\label{chap:jubilees}

Like \ga, the book of \jub engages in a form of rewriting which participates in the construction of memory through \psgraphical discourse. \jub builds clearly from the biblical material (Gen--Exod 12) with the kinds of adaptations, harmonizations, and emendations we expect of \rwb and bears clear influences from other \secondtemple traditions such as the Astronomical Book of Enoch (\firstenoch 72--82). In this respect, \jub bears many of the same kinds of qualities that I considered in the previous chapter such as the sources of tradition, generic features, and narrative framing.%
    \footnote{One might argue, for example, that the genre of \jub is that  of Apocalypse. But Cf. \cite{hanneken2012}. Regarding the formal characteristics of Apocalypse, see John Collins's work on the topic, esp. 
        \cite{collins_mason-etal2012} and 
        \cite{collins_semeia1979}.}
Although not framed as a first-person account, \jub also portrays itself as the product of first-hand experience. The author presents his work as the result of God's repeated command to Moses to ``write down all that you hear'' (1:5, 7, 26) and to record the content of the \heavenlytablets dictated to him by the chief angelic being (2:1). Thus the narrator takes on the persona of Moses and proffers his work as a faithful record of Moses's experience atop Mt. Sinai and is therefore also counted among the \psa. The rewritten account of ``biblical history'' from Gen 1--Exod 12 is, like \ga, \emph{drawn from} biblical memory and \emph{speaks back into} biblical memory through the process of rewriting. 

Taking the book of \jub as my point of departure, in this chapter I will attempt to differentiate the \emph{manner} that \rwb texts may have engaged with cultural memory. In the previous chapter, I argued that the \psgraphical quality of \ga engaged with the cultural memory \emph{differently} than other non-\psgraphical texts. However, I left open the question of how, specifically, readers were intended to understand the ``authority'' or ``authenticity'' of the account. Were these \psgraphical texts believed to be ``genuine'' first-hand accounts from Lamech, Noah, and Moses? Or were readers otherwise queued into the genre, understanding them to be ``historical fiction,'' as it were? Or perhaps both, or neither? In this chapter I will focus on the ways that \jub portrayed itself as an authoritative revelation which invited the reader to incorporate new knowledge into their conception of the past in an effort to affect the behavior of its readers and to reinforce the practices of its remembering community. I will argue that the book of \jub engages with cultural memory in a distinct fashion from other texts such as \ga in part through rhetorical means (so-called ``authority conferring strategies''). This distinct form of engagement is significant because it illustrates the way that memory not only affects the intellectual conceptions of the past, but also carries with it \emph{concrete practical effects which can be concretely observed}. To accomplish this, I will draw on Hindy Najman's work on Mosaic Discourse and will discuss the ways that \jub portrays itself as authoritative literature, how this portrayal may have been understood in antiquity and how it could, in some sense, both authorize and rewrite the Torah with halakhic implications. Then, to illustrate the point, I will discuss the calendrical and chronological system of the book of \jub as an example of how memory construction extends beyond abstract or intellectual processes and into the realm of concrete social practice.


%%%%%%%%%%%%%%%%%%%%%%%%%%%%%
% Discovery and Publication %
%%%%%%%%%%%%%%%%%%%%%%%%%%%%%
\section{Jubilees: Discovery and Publication}
The work now referred to as the book of \jub was believed to have been lost forever by European scholars prior to the mid nineteenth Century. The work was ``rediscovered,'' however, in 1844 when Heinrich Ewald published a description of an Ethiopian (\geez) manuscript under the title ``the Book of the Division'' \eth{መጽሐፈ~፡ ኩፋሌ}{masḥafa kufāle}.%
        \footnote{All translations are my own. \geez citations are from \vanderkam's critical edition, \cite*{vanderkam1989}.}
Because the name followed the common convention using a work's first few (key) words as its title (in this case, \eth{ዝንቱ~፡ ነገረ~፡ ኩፋሌ}{zəntu nabara kufāle}), Ewald suggested that this manuscript may have been a copy of the work known from antiquity as both \greek{τά Ἰωβηλαϊα}, ``the Jubilee,'' and \greek{Λεπτὴ Γίνεσις}, the ``Little Genesis.''\autocite[176--179]{ewald_zkm1844} Although the work had been in continuous use within Ethiopian Christianity since antiquity, prior to Ewald's publication, European scholarship only knew of the work through secondary references in a few classical sources.%
        \footnote{\vanderkam offers a concise summary of the various late-antique citations and allusions in his commentary, most notably in the works of Epiphanius (\emph{Panarion}, \emph{Measures and Weights}) and Syncellus (\emph{Chronography}).
                \cite[1:10--14]{vanderkam2018}. See also 
                \cite{reed_kister-etal2015} and 
                \cite{kreps_ch2018}.
        It is also probable that more recently discovered texts, such as the Damascus Document (CD), refer to the book of \jub as 
        ``the Book of the Divisions of the Times into their Jubilees and Weeks'' Heb. \hebrew{ספר מחלקות העתים ליובליהם ובשבועותיהם}. It seems almost unimaginable that CD was not referring to \jub, though, some have questioned the notion. See \cite[242--248]{dimant_vanderkam-etal2006}.}
The work was published (supplemented with a second manuscript) by August Dillmann in 1859\autocite{dillmann1859} and by R.~H. Charles in 1895, who included two additional manuscripts in his edition (totaling four).\autocite{charles1895} More recently, \vanderkam's 1989 edition utilized twenty-seven copies of the text\autocite[1:xiv--xvi]{vanderkam1989} and since its publication over twenty more copies have been cataloged and imaged.%
        \footnote{%
                \cite{erho_bsoas2013}.
                \vanderkam helpfully lists the twenty-seven manuscripts he used for his critical edition in the introduction of his commentary where he also notes the additional manuscripts photographed since its publication. See 
                \cite[1:14--16]{vanderkam2018}.}

With the exception of the ``rediscovery'' of the text for European scholarship, the most significant find for the study of \jub was the discovery of several Hebrew fragments among the \dss.%
    \footnote{The Hebrew fragments contain approximately 15\% of the total work, measured against the Ethiopic text. See \cite[124]{stokl_henoch2006}.}
These fragments attest to the work's antiquity and confirmed that the original language of \jub was Hebrew and not Aramaic, as Dillmann originally supposed.%
        \footnote{\Cite[90]{dillamnn_jbw1850}. Though, as \vanderkam notes, he seems to have changed his mind later and supposed a Hebrew original. \cite[324]{dillmann_spaw1883}; \cite[1:1 n. 1]{vanderkam2018}.

        The Ethiopic text is a granddaughter translation of the Hebrew through Greek, though no Greek manuscripts of the text have been found. See especially \vanderkam's treatment of the textual history of \jub in \cite*[1--18]{vanderkam1977}. This fact was convincingly demonstrated by Dillmann who observed several Greek forms preserved as transliterations in the Ethiopic text, specifically: \greek{δρῦς}, \greek{βάλανος}, \greek{λίψ}, \greek{σχῖνος}, and \greek{φάραγξ}. See, \cite[88]{dillamnn_jbw1850}. Charles later added \greek{ἡλιου} to the list. \cite[xxx]{charles1902}.

        By the end of the nineteenth century, however, partial copies of \jub had also been uncovered in Latin, which similarly appear to have come through the Greek.
        See
                \cite[15--54]{ceriani1861} and
                \cite{ronsch1874}.
        See also the work of Todd Hanneken and the Jubilees Palimpsest Project (\href{http://jubilees.stmarytx.edu}{jubilees.stmarytx.edu}).

        Finally, although no direct manuscript evidence has been found, \jub scholars posit that a Syriac translation of the Hebrew was made in antiquity. This suggestion is tenuous, but is based on a number of Syriac citations of \jub which do not show any linguistic influence (loan words, etc.) from Greek.
        See especially
                \cite[231--232]{tisserant_rb1921} and 
                \cite[xxix]{charles1902} but also 
                \cite[2:ix--x]{ceriani1861} and 
                \cite[x]{charles1895}.}
Despite all of these finds, however, the Ethiopic text remains the only tradition to preserve \jub in its entirety. Thus, in my treatment of \jub, I will be relying primarily on the Ethiopic text, supplemented by the Hebrew and other versions when available.

% \subsection{Content and Character}
The book of \jub offers a rewriting of the book of Genesis and the first part of Exodus (Gen 1--Exod 12).\autocite[1:17]{vanderkam2018} The bulk of the book (2:1--50:13) is dedicated to recounting these ``biblical'' events in the form of a revelation given to Moses by \yahweh with special concern for halakhic matters and the division of time according to ``weeks'' of years (7-year units) and ``jubilees'' (49-year units). The particulars of the revelation are mediated by the ``\ap'' (1:27; Eth. \ethiopic{መልአከ~፡ ገጽ} [\translit{malʔaka gaṣṣ}]) who dictates the content of the ``heavenly tablets'' (4:5; Eth. \ethiopic{ጽላተ~፡ ሰማይ} [\translit{ṣəllāta samāy}]) to Moses to record what they revealed about the structure and terminus of the cosmos.\autocite{martinez_najman-tigchelaar2012} The treatment of Moses as a scribe places him within a chain of tradition---along with Enoch and Noah---that emphasizes writing and written works as essential sources of tradition and revelation.%
        \footnote{See especially
                \cite[381--388]{najman_jsj1999}.} 

The main body of \jub is framed by a brief prologue and an even briefer epilogue. The prologue offers a short description of the work as an account concerned with the division of time into units of years, weeks, and jubilees given to Moses when he ascended Mt. Sinai to receive the ``stone tablets'':

% !TEX root = dissertation.tex
\begin{ethiopictext}
        \versenum{Prologue}
        ዝንቱ ፡ ነገረ ፡ ኩፋሌ ፡
        መዋዕላተ ፡ ሕግ ፡ ወለስምዕ ፡
        ለግብረ ፡ ዓመታት ፡ ለተሳብዖቶሙ ፡ 
        ለኢዮቤልውሳቲሆሙ ፡ ውስተ ፡ ኲሉ ፡ ዓመታተ ፡ ዓለም ፡
        በከመ ፡ ተናገሮ ፡ ለሙሴ ፡ በደብረ ፡ ሲና ፡
        አመ ፡ ዐርገ ፡ ይንሣእ ፡ ጽላተ ፡ እብን ፡ ሕግ ፡ ወትእዛዝ ፡ 
        በቃለ ፡ አግዚአብሔር ፡ በከመ ፡ ይቤሎ ፡ ይዕርግ ውስተ ፡ ርእሰ ፡ ደብር ።
\end{ethiopictext}

\begin{transliteration}
        \versenum{Prologue}
        zəntu nagara kufālē
        % kufālē                division
        mawāʕəlāta [la-]ḥegg wa-la-səmʕ
        % mawāʕel           'period, era, time' √mʕl 'to pass the day' Les. 603
        % səmʿ               testimony
        la-gəbra ʕāmatāt la-tasābəʕotomu
        % tasābeʿot             tGL perf from √sbʕ; not in the dictionary, but √sbʿ is seven, so… weeks
        la-ʔiyyobēləwəsātihomu wəsta \kw{ə}llu ʕāmatāta ʕālam
        % ˀiyyobēlwelātihomu    ʾiyyobēl is Jubilee, the rest (-welāt) some extended plural?
        ba-kama tanāgaro la-Musē ba-dabra Sinā
        % ba-kama               Just as
        % tanāgaro              Glt perf 3ms + 3ms
        ʔama ʕarga yenšāʔ ṣəllāta ʔəbn---ḥəgg wa-təʔzāz---%
        % ʕarga                 √ʕrg G pf 3ms 'go up'
        % yenšāʔ                √nšʔ G subj 3ms 'raise, accept, receive*' 
        % ṣellē                 pl. ṣellāt    'tablet'
        ba-qāla ʔagziʔabḥēr ba-kama yəbēlo yəʕrəg wəsta rəʔsa dabr.
        % yebēlo                G perf + 3ms
        % yeˤreg                G subj 
\end{transliteration}

\begin{translation}
        \versenum{Prologue}
        These are the words%
        \footnote{Lit. ``This is the word.'' I've chosen to follow VanderKam and others by rendering this construction in the plural based on the probable underlying Hebrew \he{אלה הדברים}. See \cite[125]{vanderkam2018}}
        of the division 
        of the days for the law and for the testimony
        for the event[s] of the years; for their weeks,
        for their Jubilees in all the years of the world
        just as he spoke (them) to Moses on Mount Sinai 
        when he went up to receive the tablets of stone---the law and the commandment---%
        at the command of God, as he had said to him 
        that he should ascend to the top of the mountain.
\end{translation}%
\noindent
The work closes with a terse statement declaring ``Here the account of the division of time is ended'' (\jub 50:13; Eth. 
    \eth{ተፈጸመ~፡ በዝየ~፡ ነገር~፡ ዘኩፋሌ~፡ መዋዕል~።}
        {tafaṣṣama ba-zəyya nagar za-kufāle mawāʕəl}).
        % tafaṣṣama     tD fṣm 'to complete'
        % ba-zeyya      here
        % nagar         account, speech, etc.
        % kufāle        division(s)
        % mawāʕel       √mʕl 'to pass the day' here: period, era, time Les. 603
\noindent
It is important to note that this prologue as well as the first chapter of the book of \jub are preserved among the Qumran fragments (specifically \q{4}{216}{}), which represent some of the oldest extant \jub fragments. Thus, this early narrative frame was almost certainly a part of the work in its earliest form and cannot be attributed to a later editor; it is an integral part of the literary shape of the book of \jub. Although superscriptions were often added much later, in this case, we have no reason to doubt that the prologue/superscription and framing narrative of the work were not a part of the most ancient versions.%
        \footnote{See 
                \cite[1:125]{vanderkam2018};
                \cite[25]{vanderkam_metso-etal2010}.}

Thus the work as a whole is presented as a revelation given to Moses by \yahweh, framed by a brief prologue and epilogue which situates the story during Moses's first 40-days atop Mt. Sinai when Moses receives the Tablets of Stone (Exod 24:12).\autocite[1:129]{vanderkam2018}


\section{Jubilees Halakot: The Practical Effects of Memory Construction}

The primary difference between the ways that \ga and the book of \jub engage with cultural memory is that the book of \jub engages more directly in what I will refer to as ``prescriptive discourses.'' This is not to say, of course, that the author of \ga did not or could not have had halakhic intentions behind his writing, but only that such intentions are not, generally, able to be identified. \GA functions ``prescriptively'' only insofar as its characters \emph{demonstrate} good practice. Within this framework, \ga participates in prescriptive discourses in a comparatively \emph{indirect} manner. \GA portrays its main characters as good, Torah-following Jews but we cannot say for certain whether these portrayals are intended to be novel contributions to the image of the characters, or whether they merely \emph{reflect} a received tradition about how the patriarchs \emph{would have} conducted themselves.%
    \footnote{For example, Noah, in \col{10}, 13--18 makes sacrifices in accordance with the legal tradition from Lev 4--6. Thus, Noah---the righteous patriarch---followed Torah even before it was given to Moses. This particular incident, as it happens, also is seen in \jub. See \cite[112]{crawford2008}. See also \cite[419]{reeves_revq1986}.}
In other words, the question of whether the author of \ga was proffering a novel portrait of the patriarchs or was more passively projecting his understanding of the patriarchs goes to the question of authorial intent, which, lacking more concrete evidence one way or the other, is a dead end. What \emph{is} certain, however, is that the \emph{participation} of the author within a particular site of memory affected the broader cultural memory, either by injecting new ideas, interpretations, or literary color into the tradition, or by reinforcing a set of inherited traditions.

To be sure, the book of \jub engages in similar kinds of positive portrayals of the patriarchs. And in fact, one of the tacit assertions of the book of \jub seems to have been that---at least certain aspects of---the Torah had not only been revealed to the pre-Mosaic patriarchs, but was practiced by them. Nickelsburg writes:
    \begin{quote}
        Noah offered a proper sacrifice (7:3--5), and Levi discharged the office of priest (32:4--9). Major holidays were observed by the patriarchs: First fruits by Noah, Abraham, Isaac, Jacob, and Ishmael (6:18, 15:1--2; 22:1--5); Tabernacles by Abraham (16:20--31); and the Day of Atonement by Jacob (34:12--20). Special prescriptions are given for Passover, the Jubilee Year, and the Sabbath (chs. 49--50).%
        \footnote{\Cite[69]{nickelsburg2005};
            See also \cite[70]{crawford2008}.}
    \end{quote}
\noindent
As with \ga, these otherwise anachronistic displays of religious piety could be viewed as engaging with biblical memory in multiple ways---as either an innovation or a preservation of cultural memory. Although the intention of the author certainly \emph{may} have been to advocate for a particular halakhic point of view, such intentions are simply not knowable with any degree of certainty. We can, however, assert that the portrayal of the patriarchs as being observant of particular halakhic practices \emph{reinforced} such practices as normative and thus may have had a prescriptive \emph{function}. In other words, although it is fraught to speculate about what an author \emph{intended}, we \emph{can} reasonably speculate that the repetition and the positive portrayal of these practices would have helped to buttress the cultural support of particular practices for contemporary readers.
    \footnote{The degree to which such practices would serve as an example to be followed, of course, assumes that the text was a trusted source of tradition. Within social contexts where a particular text was \emph{not} a trusted source of tradition, of course, the text would not function in this way.}

What is unique about the book of \jub, by comparison to both Chronicles and \ga, is that \jub goes beyond \emph{demonstrating} good practice and engages in more direct forms of prescriptive discourses. It stops short, however, of taking the form of traditional legal material. Although there are examples of \rwb texts which may be categorized formally as legal writing (e.g., the \templescroll\autocite{fraade_goldstein-etal2017}), the book of \jub does not, formally, include such material. 

Instead, the book of \jub includes explicit halakhic instruction embedded within the narrative. These explicit instructions take the literary form of, for example, testaments or as references to precepts recorded on the ``\heavenlytablets.'' In both cases, although the instructions are not meta-discursively directed toward the reader, this is precisely the effect.%
    \footnote{One might question how this is different than the legal material within the Hebrew Bible, which is also embedded within various narratives. I would argue, however, that even embedded within the narrative, the \emph{form} of these sections is legal. Importantly, these sections are \emph{portrayed within the narrative} as legal material. Thus, although embedded within a narrative, it remains (generically speaking) ``legal'' in form.}

%% Testament %%
One of the best examples of the testamental form of halakhic instruction falls within \jub 20--22. In this section, a series of three testamental speeches are given by Abraham: the first to Ishmael (and his children), Isaac (and his children), and Keturah's children (20:1--20:13), the second to Isaac (21:1--22:8), and the third to Jacob (22:16--24). Common among the speeches is an admonishment to follow God's commands (20:2, 21:5; 21:21--24) and to abstain from idolatry (20:7; 21:5; 22:17--18) and sexual impurities (including especially intermarriage with Canaanites; 20:3--6; 22:20). The testament directed toward Isaac (21:1--22:8), which is the longest of the three, includes a lengthy digression on several specific sacrificial practices including the proper exsanguination and butchering of the sacrificial animal (21:7--8), what flour and oil to use (21:7), which woods are acceptable for the fire (21:12--14), and ritual washing (21:16).%
    \footnote{One cannot help but notice the conspicuousness of the fact that it is \emph{Isaac} to whom Abraham gives these specific instructions. VanderKam notes that the Aramaic Levi Document, which shares some considerable overlap with this particular testament (including a concern for which woods are to be used for sacrifice), describes Isaac teaching his grandson Levi proper sacrificial procedure. Thus, it could be that Isaac's connection to the sacrificial cult is more a way to get the instructions to Levi than an allusion to the Aqedah. On the relationship between the Aramaic Levi Document and \jub and the kinds of wood allowed for sacrifice, see \cite[625, 636--639]{vanderkam2018}.}
Notably, although many of the specifics align with biblical instructions on sacrifice, the sum-total of these instructions cannot be accounted for through purely exegetical moves. The instruction about which woods are acceptable, in particular, does not have any biblical precedent. Yet, traces of the practice can be seen within the Hebrew Bible.%
    \footnote{\Cite[]{vanderkam2018}.}
Nehemiah 10:33--35 reads:

\begin{hebrewtext}
    \versenum{Neh 10:33}
    וְהֶעֱמַדְנוּ עָלֵינוּ מִצְוֹת לָתֵת עָלֵינוּ שְׁלִשִׁית הַשֶּׁקֶל בַּשָּׁנָה לַעֲבֹדַת בֵּית אֱלֹהֵינוּ׃ 
    \versenum{34}
    לְלֶחֶם הַמַּעֲרֶכֶת וּמִנְחַת הַתָּמִיד וּלְעוֹלַת הַתָּמִיד הַשַּׁבָּתוֹת הֶחֳדָשִׁים לַמּוֹעֲדִים וְלַקֳּדָשִׁים וְלַחַטָּאוֹת לְכַפֵּר עַל־יִשְׂרָאֵל וְכֹל מְלֶאכֶת בֵּית־אֱלֹהֵינוּ׃
    \versenum{35}
    וְהַגּוֹרָלוֹת הִפַּלְנוּ עַל־קֻרְבַּן הָעֵצִים הַכֹּהֲנִים הַלְוִיִּם וְהָעָם לְהָבִיא לְבֵית אֱלֹהֵינוּ לְבֵית־אֲבֹתֵינוּ לְעִתִּים מְזֻמָּנִים שָׁנָה בְשָׁנָה לְבַעֵר עַל־מִזְבַּח יְהוָה אֱלֹהֵינוּ כַּכָּתוּב בַּתּוֹרָה׃
\end{hebrewtext}
\begin{translation}
    \versenum{Neh 10:33}
    And we have obligated ourselves with a command to give a third of a shekel yearly for the service of the \temple of our God:
    \versenum{34}
    for the showbread, the regular offering and for the regular burnt offering, the sabbaths, the new moons, for the appointed festivals and for the sacred things and for the sin offerings and for the atonement of Israel and all the work of the \temple of our God.
    \versenum{35}
    Now, we have cast lots---the priests, the Levites, and the people---concerning the gift of wood, to bring (it) to the \temple of our God by the ancestral houses for the appointed times, year by year to burn upon the altar of \yahweh our God \emph{as it is written in the law}. (emphasis added)
\end{translation}
\noindent
Thus the book of Nehemiah at least hints that there was some ritual associated with the wood for the altar during the (nascent) \secondtemple period. The fact that such a ritual is described as ``written in the law'' is a conspicuous difficulty given the absence of any such instruction within the Torah.%
    \footnote{Blenkinsopp states that the offering of wood was ``implicit'' in the command for the priests to keep the fire burning at all times (Lev. 6:2, 5--6). He further notes that the practice is described in Josephus' \emph{Jewish War} 2.425. See \cite[317]{blenkinsopp1988}. VanderKam notes further that a similar practice is described in the Mishnah (m. Taʿan 4:5, m. Tamid 2:3). See, \cite[636]{vanderkam2018}.}
Whatever the case, what is important for our purposes is that plainly halakhic material (for which we have evidence of practice elsewhere, but which does not show up in the Hebrew Bible) is clearly presented as \emph{instructive} for the reader, despite being embedded within a narrative.

%% Heavenly Tablets %%
Another way that the author(s) of \jub expresses these more direct forms of prescriptive discourses is through references to ``the \heavenlytablets'' (\HT; Eth. \eth{ጽላተ~፡ ሰማይ}{ṣəllāta samāy}). Florentino García Martinez has noted several functional valences to \jub's literary use of these tablets, including  the ``divine, pre-existing archetype of the Torah,'' a register of good and evil deeds, the ``book of destiny,'' a calendar for the proper observation of sabbaths and feasts, and a source for ``new halakot.''\autocite{martinez_najman-tigchelaar2012} It is this last function of the \HT which concerns us.

The author of \jub uses the \HT as a literary device to assert the absoluteness of a given halakhic practice. For instance, \jub 30:7 prohibits a man from giving his daughter or sister to a foreigner in marriage upon penalty of death by stoning. Moreover, the woman given is also to be killed by burning. Martinez notes that the prohibition of foreign marriage is an extremely important theme in \jub, however, there is no such prescription in the Torah (though not entirely foreign to the Hebrew Bible). Martinez writes:
\begin{quote}
    The biblical basis of the penalty imposed, that is, stoning, is deduced from the equivalence between delivering one's offspring to Moloch and delivering them to foreigners. The punishment of burning a woman can only be explained by a comparison with the Israelite woman who marries a foreigner to the daughter of a priest who prostitutes herself. In both cases the interpretation of the biblical text supposes, as does the halakah itself, that it can be accepted only by virtue of the authority that is conferred upon it by its inscription upon the ``heavenly tablets.''\autocite[67]{martinez_najman-tigchelaar2012}
\end{quote}
\noindent
In other words, the halakah that the author proffers is not presented as an interpretation of the Torah, but as an absolute precept known from an immutable and divine source. Martinez notes that the formula of referencing halakot as deriving from the \HT (with minor variations) can be found in at least six instances where the halakot cannot be directly linked to the Torah.%
    \footnote{The references are \jub 3:31; 4:32; 15:25; 28:6; 30:9; and 32:10--15. See \cite[64--68]{martinez_najman-tigchelaar2012}. Placing ``Torah backed'' halakot in a different category than ``new'' halakot, I think is a methodological problem with Martinez's study. Martinez goes so far as to say that the \HT ``do not represent a single notion,'' but I think this begs the question. The \HT \emph{are} presented as a single idea (with multiple functions). Nickelsburg, on the other hand, includes 6:17--22 and 33:10--20 in his discussion of the phrase and I think it helps to unify what the \HT were imagined to be.}

Thus, where \ga certainly offered \emph{examples} of good practice, \jub presents itself as an official, correct, understanding of the cosmic order which is divinely ordained, at times speaking literally in the imperative mood. \jub deals with legal and halakhic matters directly---it gives instructions about how and when to celebrate the sabbath, festivals, how to observe the yearly calendar, who to marry, and directly critiques the sinful behavior of Israel. While \ga may have \emph{implicitly} endorsed particular ideologies and halakhic practices by linking them with the foundational figures of Genesis (Lamech, Noah, and Abram), the book of \jub at times engages in direct imperative and presents itself as an authoritative text whose content comes directly from God, incised in the \heavenlytablets, mediated by God's chief angelic being (the \ap), and ultimately recorded by Israel's most authoritative legal figure, Moses. It stands to reason, therefore, that the purpose of \jub was not simply religious entertainment or vaguely edifying storytelling but was intended to be read, understood, and (importantly) to affect the behavior of its readers by reinforcing the practices of the remembering community.

At times, this places \jub at odds with a plain reading of the Torah and in such cases, the author of \jub insists to the reader that its characterizations are absolute and divinely sourced. But, of course, such a characterization is simply a rhetorical means of bolstering the author's particular claims about proper Jewish praxis. Such claims are framed not as innovations \emph{over and against} the Torah, but as the \emph{originally intended} or \emph{omitted} precepts from God. %
% TODO: Are the Torah and Jubilees assumed to be for the same audience?
Thus, the author still holds up the Torah as \emph{the} divine standard and frames obedience to the Torah as a central identifier of pious living. For example, God tells Moses in \jub 1:9--10 that the people will stray from the covenant in part by ``forgetting'' God's commandments and neglecting proper cultic activities. Furthermore, the persecution of those \emph{who study the law} is included in a catena of evil deeds that Israel will perpetrate. In other words, although the book of \jub seems to be claiming a particular kind of divine knowledge that in some sense ``trumps'' the Torah as-written, it does so while simultaneously holding up Torah observance as a fundamental measure of fidelity to God. \jub 1:9--14 illustrates this continuity in nearly Deuteronomistic language:

% !TEX root = dissertation.tex
\begin{ethiopictext}
    \versenum{Jubilees 1:9}
    እስመ~፡ ይረስዑ~፡ ኵሎ~፡ ትእዛዝየ~፡ 
    ኵሎ~፡ ዘአነ~፡ እኤዝዞሙ~፡ ወየሐውሩ~፡ ድኅረ~፡ አሕዛብ~፡ ወድኅረ~፡
    ርኵሶሙ~፡ ወድኅረ~፡ ኀሳሮሙ~፡ ወይትቀነዩ~፡ ለአማልክቲሆሙ~፡ 
    ወይከውንዎሙ~፡ ማዕቀፈ~፡ ወለምንዳቤ~፡ ወለፃዕር~፡ ወለመሥገርት~፡
    \versenum{10}
    ወይትሐጐሉ~፡ ብዙኃን~፡ ወይትአኀዙ~፡ ወይወድቁ~፡ ውስተ~፡
    እደ~፡ ፀር~፡ እስመ~፡ ኀደጉ~፡ ሥርዓትየ~፡ ወትእዛዝየ~፡ ወበዓላተ~፡
    ኪዳንየ~፡ ወሰንበታትየ~፡ ወቅድሳትየ~፡ ዘቀደስኩ~፡ ሊተ~፡ በማእከሎሙ~።
    ወደብተራየ~፡ ወመቅደስየ~፡ ዘቀደስኩ~፡ ሊተ~፡ በማእከለ~፡
    ምድር~፡ ከመ~፡ እሢም~፡ ስምየ~፡ ሳዕሌሁ~፡ ወይኅድር~።
    \versenum{11}
    ወገብሩ~፡
    ሎሙ~፡ ፍሥሐታተ~፡ ወኦመ~፡ ወግልፎ : ወሰገዱ~፡ ዘዘ~፡ ዚአሆሙ~፡ 
    ለስሒት~፡ ወይዘብሑ~፡ ውሉዶሙ~፡ ለአጋንንት~፡ ወለኵሉ~፡ ግብረ~፡
    ስሕተተ~፡ ልቦሙ~።
    \versenum{12}
    ወእፌኑ~፡ ኀቤሆሙ~፡ ሰማዕተ~፡ ከመ~፡
    አስምዕ~፡ ሎሙ~፡ ወኢይሰምዑ~፡ ወሰምዕተኒ~፡ ይቀትሉ~፡ ወለእለሂ~፡
    የኀሥሡ~፡ ሕገ~፡ ይሰድድዎሙ~፡ ወኵሎ~፡ ያፀርዑ~፡ ወይዌጥኑ~፡ ለገቢረ~፡
    እኩይ~፡ በቅድመ~፡ አዕይንትየ~፡
    \versenum{13}
    ወአኀብእ~፡ ገጽየ~፡
    እምኔሆሙ~፡ ወእሜጥዎሙ~፡ ውስተ~፡ እደ~፡ አሕዛብ~፡ ለፂዋዌ~፡
    ወለሕብል~፡ ወለተበልዖ~። ወአሴስሎሙ~፡ እማእከለ~፡ ምድር~፡
    ወእዘርዎሙ~፡ ማእከለ~፡ አሕዛብ~፡
    \versenum{14}
    ወይረስዑ~፡ ኵሎ~፡ ሕግየ~፡
    ወኵሎ~፡ ትእዛዝየ~፡ ወኵሎ~፡ ፍትሕየ~፡ ወይስሕቱ~፡ ሠርቀ~፡ ወሰንበተ~፡
    ወበዓለ~፡ ወኢዮቤለ~፡ ወሥርዓተ~።
\end{ethiopictext}

\begin{transliteration}
    \versenum{Jubilees 1:9}
    ʔəsma yərassəʕu \kw{ə}llo təʔzāzəya
    % yərassəʕu         g impf 3mp √rsʕ 'to forget' Les. 473
    % təʔzāzya          təʔzāz + 1cs 'law'
    \kw{ə}llo za-ʔana ʔəʔēzzəzomu wa-yaḥawwəru dəḫra ʔaḥzāb wa-dəḫra
    % ʔəʔēzzezomu       d impf 1cs +3mp √ ʔzz 'to command' Les. 53
    % yaḥawwəru         g impf 3mp √ḥwr 'to go' Les. 249
    % dəḫra             'back, past, after, then' Les. 129
    % ʔaḥzāb            pl. of ḥəzb 'nation, people, tribe' Les. 253
    rə\kw{ə}somu wa-dəḫra ḫasāromu wa-yətqan\-nayu la-ʔamāləktihomu
    % rəkwsomu          n. √rkws 'to be unclean, impure' Les. 470
    % ḫasārom           n. √ḫsr 'dishonor' Les. 265
    % yətqannayu        tG impf. 3mp 'serve' √qny Les. 437
    % ʕamālektihomu     pl. of ʔamlāk + 3mp '(false) gods, idols' Les. 344
    wa-yəkawwənəwwomu māʕəqafa wa-la-məndābē wa-la-ḍāʕr [ṣāʕr] wa-la-maśgart
    % yəkawwənəwwomu    g impf. 3mp (ənu- > ənəww- Lambd. 64) √kwn 'become'
    % māʕəqafa          √ʕqf 'to trip up' Les 67
    % məndābē           'tribulation, affliction' Les. 348
    % ḍāʕər             √ṣʕr 'anguish, trouble' Les. 544
    % maśgart           √śgr II 'snare, trap' Les 527
    \versenum{10}
    wa-yətḥag\gw{a}lu [yəthag\gw{a}lu] bəzuḫān wa-yətʔaḫḫazu wa-yəwaddəqu wəsta
    % yətḥagolu         tG impf √hgwl 'destroy'
    % bəzaḫān           'many' √bzḫ 'be numerous' Les 117
    % yətʔaḫḫazu        tG impf. 3mp √ʔḫz 'take, catch' Les. 14
    % yəwaddəqu         G impf 3mp √wdq 'collapse, go to ruin' Les.
    ʔəda ḍarr ʔəsma ḫadagu śərʕātəya wa-təʔzāzəya wa-baʕālāta
    % ʔəda              'hand' pl. ʔədaw Les. 7
    % ḍar               'enemy' pl. ʔaḍrār Les. 152
    % ḫadagu             g pf. 3mp √ḫdg 'to abandon' Les. 258
    % śərʕāteya          pl. 'ordinances' √śrʕ Les. 533
    % baʕālāta          pl. 'festivals' Les. 83
    kidānǝya wa-sanbatātǝya wa-qəddǝsātǝya za-qaddasku lita ba-māʔkalomu
    % kidāneya           'covenant' √kyd Les 301
    % sanbatāteya        'sabbaths'
    % qəddesāteya        'holy things' Les 422
    % qaddasku           d pf. 1cs √qds 'to make holy' Les. 422
    % ba-māʔkal          'amongst'
    wa-dabtarāya wa-maqdasǝya za-qaddasku lita ba-māʔkala
    % dabtarāya         'tabernacle'
    % maqdasya          'temple'
    % qadasku           g pf. 1cs √qds 'to make holy' Les. 422
    mədr kama ʔəśim səməya lāʕlēhu wa-yəḫdər
    % ʔəśim         g. subj. 1cs √śym 'put, place' Les. 539
    % səmya         'name' +1cs Les. 504
    % lāʕlēhu       prep. 'upon'
    % yəḫdər        g subj 3ms √ḫdr 'to dwell'  Les. 258
    \versenum{11}
    wa-gabru
    % gabru             g pf. 3mp √gbr 'to make'
    lomu fəśḥatāta wa-ʔoma [ʕoma] wa-gəlfo wa-sagadu zazza ziʔahomu
    % fəśḥatāta         pl. 'high place' √fśḥ II Les. 168
    % ʔoma              ʕoma 'grove' Les. 62
    % gəlfo             adj. 'carved' √glf 'to carve' Les. 190
    % sagadu            g pf 3mp √sgd 'bow down, prostrate'
    % zaza              each
    % ziʔahomu          each of them (probably zazazaiʔahomu)
    la-səḥit wa-yəzabbəḥu wəludomu la-ʔəgānənt wa-la-\kw{ə}llu gəbra
    % səḥit             'sin, error' √sḥt 'to stray, err' Les. 494
    % yəzabbəḥu         g impf 3mp √zbḥ 'to sacrifice' Les. 631
    % ʔəgānənt          'demons' sn. gānən Les 198
    % gəbra             'thing' √gbr 'to do, make' Les 178
    səḥtata ləbbomu
    % saḥtata           'sinful' √sḥt 'to stray, err' Les. 494
    % ləbbomu           'hear, mind'
    \versenum{12}
    wa-ʔəfēnnu ḫabēhomu samāʕta kama
    % ʔəfēnnu           d. impf 1cs √fnw 'to send' Les. 163
    % ḫabē-             'toward, near, to'          
    % samāʕta           'witnessses' samāʕi √smʕ 'hear'
    ʔasməʕ [ʔāsməʕ] lomu wa-ʔiyyəsamməʕu  wa-samāʕta-ni yəqattəlu wa-la-ʔəlla-hi
    % ʔasməʕ            CG Subj 1cs √smʕ 'to bear witness'
    % ʔiyyəsammeʕu      g impf 3mp √smʕ 'to listen' Les. 501
    % samāʕta           'witnessses' samāʕi √smʕ 'hear' 
    yaḫaśśəśu ḥəgga yəsaddədəwwomu wa-\kw{ə}llo yāḍarrəʕu wa-yəwēṭṭənu la-gabira
    % yaḫaśśəśu         g impf √ḫśś 'to seek' Les. 266
    % yəsaddədəwwomu    g impf 3mp √sdd 'persecute, drive away'
    % yāḍarrəʕu         CG impf 3mp √ḍrʕ 'to annul, leave aside'
    % yəwēṭṭənu         D impf 3mp √wṭn Les. 623
    % gabira            G inf √gbr 'to do'
    ʔəkuya ba-qədma ʔaʕəyyəntəya
    % ʔəkuya            'evil'
    % ba-qədmu          'before'
    % ʔaʕəyyəntəya      'eyes' sn. ʕayn +1cs Les. 79
    \versenum{13}
    wa-ʔaḫabbəʔ gaṣṣəya
    % ʔaḫabbəʔ          g impf 1cs √ḫbʔ 'hide, conceal' Les. 253
    % gaṣṣəya           'face' Les. 205
    ʔəmmənēhomu wa-ʔəmēṭṭəwomu wəsta ʔəda ʔaḥzāb la-ḍiwāwē
    % ʔəmēṭəwwomu       d impf 1cs √mṭw 'hand over, deliver' Les. 374
    % ʔaḥzāb            pl. of ḥəzb 'nation, people, tribe' Les. 253
    % ḍiwāwē            'captivity' √ḍww 'take prisoner' Les. 153         
    wa-la-ḥəbl wa-la-tabalʕā wa-ʔasēssəlomu [ʔāsēssəlomu] ʔəm-māʔkala mədr
    % ḥəbl              'spoils' Les. 223
    % tabalʕā           tG pf 3fs? 'devouring' √blʕ 'to eat' Les. 94
    % ʔasēssəlomu       CD impf 1cs +3mp 'remove' Les. 516
    wa-ʔəzarrəwomu māʔəkla ʔaḥzāb
    % ʔəzarrəwomu       G impf 1cs +3mp √zrw 'scatter, disperse' Les. 644
    % ʔaḥzāb            pl. of ḥəzb 'nation, people, tribe' Les. 253
    \versenum{14}
    wa-yərassəʕu \kw{ə}llo ḥəggəya
    % yərassəʕu         g impf 3mp √rsʕ 'to forget' Les. 473
    wa-\kw{ə}llo təʔzāzəya wa-\kw{ə}llo fətḥəya wa-yəsəḥḥətu śarqa wa-sanbata
    % təʔzāzya          təʔzāz + 1cs 'law'
    % fətḥəya           fətḥ 'precept' √ftḥ Les. 170
    % yəsəḥḥətu         G impf 3mp √sḥt 'to err, make a mistake' Les 494
    % śarqa             'rising' here 'beginning of the month' √śrq Les. 534
    % sanbata           'sabbath' Les. 505
    wa-baʕāla wa-ʔiyyobēla wa-śərʕāta
    % śərʕāta           'ordinance' √śrʕ Les. 533
\end{transliteration}

\begin{translation}
    \versenum{Jubilees 1:9}
    Therefore they will forget my law---%
    all that I am commanding them---and they will go after the nations and after
    their impurity and after their dishonor. And they will serve their (false) gods
    and they will become a hindrance and an affliction and a snare for them.
    \versenum{10}
    Many will be destroyed and they will be captured and will fall into
    the hand of the enemy because they abandoned my ordinances and my laws and festivals of 
    my covenant and my sabbaths and my sacred items which I sanctified for myself amongst them.
    Also my tabernacle and my \temple which I sanctified for myself in midst of
    the land that I might establish my name upon it and it might dwell (there).
    \versenum{11}
    And they made
    for themselves high places and a grove, and a carved image and they bowed down, each one of them,
    to error. And they will sacrifice their children to demons and to every thing
    of their sinful mind.
    \versenum{12}
    And I will send witnesses to them that
    I may testify to them, but they will not listen and (instead) they will kill the witnesses and even
    those (who) seek after the law, they will persecute. And they will leave aside everything and they will begin to do
    evil before my eyes.
    \versenum{13}
    And I will hide my face
    from them and I will deliver them into the hand of the nations for captivity,
    for spoils, and for their devouring. I will remove them from the midst of the land
    and I will scatter them among the nations.
    \versenum{14}
    They will forget my whole law,
    and all my commandments, and all my precepts. And they will err regarding the new moon and the sabbath
    and the festival and the jubilee and the ordinances.
\end{translation}
\noindent
Thus, \jub at once affirms the centrality of the Torah, while, in some sense, circumventing it by providing its own idiosyncratic rewriting of Gen 1--Exod 12. The juxtaposition of deference toward Torah while simultaneously circumventing its claim to primacy yields a sort of ``\psgraphical paradox'' (my term). It is not immediately clear how a \psgraphical author, knowingly writing under a false name, can simultaneously endorse one text, while offering novel embellishments and interpretations which seem to alter the plain meaning of the former. At least to the modern reader, this practice appears foreign and disingenuous by the \psgraphical author. The question should be raised, therefore, whether \jub \emph{was in fact} intending to supersede or circumvent the authority of the Torah (as some scholars suggest) or whether some other relationship existed between the texts.%
    \footnote{Wacholder, for example, understands \jub and the \templescroll to be a single unit and a work which was meant to supersede the Pentateuch. See \cite{wacholder_kampen-etal1997}. His theory has not been widely accepted.}

Although there is some question whether the book of \jub attained the status of ``scripture'' in antiquity, it is generally agreed that \psgraphical texts such as \jub were not intended as replacements for the more well-known scriptures (especially the Torah).%
    \footnote{This position undoubtedly represents the majority opinion, though it is not unanimous. For the opposing opinion, see especially \cite{wacholder_kampen-etal1997}.}
Of course, the reality is that we do not know for certain what kinds of categories ancient readers used to classify their literature; most likely, however, they were not static nor consistent across time and differed by social group. All the same, the special place that the Torah had for a number of Jewish sects---even in antiquity---seems to me to preclude the idea that \psgraphical texts such as \jub would be placed on-par with the Pentateuch, even if a work carried a potent practical authority (see below). What \emph{can} be said about the book of \jub, however, is that \emph{it presents itself} as a unique revelation that claims for itself the same kind of divine source as the Torah.%
    \footnote{As a matter of clarification, I am assuming a distinction between 1) the author's intent, 2) the way the work presents itself, and 3) the way the work was understood by its readers. Thus the text may present itself as ``on-par'' with the Torah without either the author or audience treating it as such.}

\vanderkam has offered a concise summary and analysis of this ``\psgraphical paradox'' and comes to the conclusion that the book of \jub functions as a vehicle for its author to proffer his own interpretation of Gen 1--Exod 12. \vanderkam addresses the problem of \jub's author both acknowledging the existence and authority of the Torah while simultaneously offering his own original material, writing:

\begin{quote}
    [W]e could say differences in interpreting the Pentateuch had arisen by his time and that the author wanted to defend his own reading as the correct one. But he wished to find a way to package his case more forcefully than that, presumably within the limits of what was acceptable in his society.\autocite[28]{vanderkam_metso-etal2010}
\end{quote}

\noindent
According to \vanderkam, therefore, the project of the author of \jub was primarily one of \emph{exegesis}. The book of \jub is an expression of the author's understanding of Gen 1--Exod 12; it offers explicit teachings about specific ambiguities and difficulties in the text of Genesis and Exodus. He had a particular understanding of how the Pentateuch should be understood, and he used the common rhetorical technique of \psy to ``more forcefully'' get his point across.%
    \autocite[28]{vanderkam_metso-etal2010}

\vanderkam argues that the author of \jub intentionally located the setting of his work in the Exod 24:12 ascent for a rhetorical advantage. He argues for three such advantages: first, by locating the story during Moses's ascent, he is able to draw on the \emph{character} of Moses. The author, therefore was able to imbue his work with the gravitas of Israel's most famous lawgiver. Second, setting the work as a part of the first forty-day period that Moses was on Mt. Sinai grounds the author's interpretation of Torah in the original revelation of the Law (prior to even Deuteronomy). These events putatively took place at the same time that Moses received the first set of stone tablets from God. While the stone tablets were broken and had to be rewritten, the account provided in \jub is prior even to those ``copies'' of the decalogue. Any subsequent interpretation of the Torah is secondary by virtue of its relative lateness. Finally, because Moses himself is presented as the author of \jub, there is no question of the chain of transmission. God revealed the contents of \jub to Moses by having the \ap dictate to him the contents of the \heavenlytablets. God is supreme, the tablets are eternal, and Moses is reliable. Moses, therefore, received more from God on Mt. Sinai than is recorded in the Torah. The claim made by \jub is that it contains the additional information given to Moses, and that the subject of this additional revelation is the sacred history of Israel schematized according to the absolute heavenly reckoning of time (364-day years, weeks of years, and jubilees).%
    \footnote{See also 4 Ezra 14:12,37--48, which describes a similar revelatory process for the figure Ezra. This passage describes the production fo ninety-four books; twenty-four of which were meant to be public, and seventy of which were to be kept secret. The twenty-four books presumably refer to the Tanak. The other seventy, it seems, were kept secret, but the division between exoteric and esoteric revelations provides a compelling analog to \jub. Thanks to Prof. Fraade for pointing me to this reference.}

The tradition that God told Moses more on Mt. Sinai than he recorded in the Torah is not unique to the book of \jub. \vanderkam points toward the later rabbinic tradition that Moses received the Oral Torah during his time atop Mt. Sinai.%
    \footnote{\Cite[28--31]{vanderkam_metso-etal2010}.}
For example, \vanderkam cites b. Berakot 5a, which references the specific time during which \jub is set (Exod 24:12):%
    \footnote{Translations of all rabbinic texts are my own.}

\begin{aramaictext}
    מאי דכתיב ואתנה לך את לחת האבן והתורה והמצוה אשר כתבתי להורתם לחת אלו עשרת הדברות תורה זה מקרא והמצוה זו משנה אשר כתבתי אלו נביאים וכתובים להרתם זה תלמוד מלמד שכולם נתנו למשה מסיני: 
\end{aramaictext}

\begin{translation}
    What is [the meaning where] it is written, \emph{I will give you the tablets of stone and the Torah and the commandments which I have written so that you might teach them} (Exod 24:12)?\\
    \-\hspace{2em}`the tablets' --- these are the ten commandments\\
    \-\hspace{2em}`the Torah' --- this is scripture\\
    \-\hspace{2em}`the commandments' --- this is Mishnah\\
    \-\hspace{2em}`that which I have written' --- these are the Prophets and the Writings\\
    \-\hspace{2em}`that you might teach them' --- this is Talmud\\\~
    [This] teaches that all of them were given to Moses on at Sinai.
\end{translation}

\noindent
The tradition here, therefore, asserts that the decalogue, the full Torah, its interpretation, the rest of the Tanakh, and the Talmud were all revealed to Moses on Sinai. Similarly, Sifra Beḥuqqotay 8, citing Lev 26:46:
\begin{aramaictext}
    אלה החקים והמשפטים והתורת: החוקים אלו המדרשות והמשפטים אלו הדינים והתורות מלמד ששתי תורות ניתנו להם לישראל אחד בכתב ואחד בעל פה
\end{aramaictext}
\begin{translation}
    \emph{These are the statutes and ordinances and Torahs} (Lev 26:46):\\
    \-\hspace{2em} `the statutes' --- this is midrash.\\
    \-\hspace{2em} `and the judgments' --- this is the legal rulings.\\
    \-\hspace{2em} `and the Torahs' --- [this] teaches that two Torahs were given to Israel: one in writing, the other by mouth.
\end{translation}

\noindent
The rhetorical function of asserting that later interpretive material was revealed to Moses is essentially the same as it is for \jub.

Thus, for \vanderkam, the book of \jub upholds the authority of the Torah by offering its own interpretation of its contents in a similar fashion to the way that the oral Torah, too, rooted its authenticity in the Sinai revelation. \jub, therefore asserts itself as a correct and authoritative interpretation of the Torah by claiming that it is the interpretation that Moses himself received from God; as \vanderkam puts it, according to the book of \jub, ``[t]he message of \jub is verbally inerrant.''%
    \footnote{
        \cite[33]{vanderkam_metso-etal2010}.
        Although the book of \jub is not generally thought to be the product of the Qumran community (it likely predates the settlement), it is worth noting that within the community, it was accepted that the community not only possessed the correct interpretation of its scriptures, but also that the community received a special revelation which the rest of Israel did not receive. As Fraade notes, this idea is quite different than supposing that additional material had been revealed \emph{to Moses}. See 
        \cite[67]{fraade_jjs1993}.}

While \vanderkam makes a number of useful observations, his characterization of \jub as exegesis, I think, ignores the question of how readers would have understood the work. This is where the analogy to the Oral Torah breaks down. While rabbinic claims that the Oral Torah was revealed to Moses, rabbinic discourse self-consciously acknowledges its work as exegetical---the rabbis offer explanations and instruction on how to understand the texts that they are commenting on.    % TODO: Yes, but in case of the Mishnah presents its laws without exegetical grounding in Scripture.
Although the rabbis may claim that an interpretation goes back to Moses, it is not the same as claiming to speak \emph{for} Moses or \emph{as} Moses. Thus \vanderkam's assertion that the purpose of writing pseudonymously and claiming that a work is the result of direct divine revelation goes beyond simply advocating for one's own interpretation ``more forcefully.'' The fact that \vanderkam leaves the particulars of this phrase ambiguous, I think, indicates ambiguity in his own thinking about \emph{how specifically} ancient readers may have understood \jub \visavis other so-called authoritative works, in particular, the Torah.

A more nuanced approach to this topic has been offered by Hindy Najman who, similarly has argued that the author of \jub utilized several ``modes of self-authorization'' in order to bolster its audience's perception of the work's authority.%
    \footnote{\Cite[380]{najman_jsj1999}.}
Building on the work of Florentino García Martínez,\autocite{martinez_najman-tigchelaar2012} Najman argues that the book of \jub utilized (at least) four such ``authority conferring strategies,'' which I have reproduced in full:

    \begin{quote}
        1. \jub repeatedly claims that it reproduces material that had been written long before the ``\heavenlytablets,'' a great corpus of divine teachings kept in heaven.

        2. The entire content of the book of \jub was dictated by the angel of the presence at God's own command. Hence, it is itself the product of divine revelation.

        3. \jub was dictated to Moses, the same Moses to whom the Torah had been given on Mount Sinai. Thus the book of \jub is the co-equal accompaniment of the Torah; both were transmitted by the same true prophet.

        4. \jub claims that its teachings are the true interpretation of the Torah. thus, its teachings also derive their authority from that of the Torah; that its interpretations match the Torah's words resolve all interpretive problems further substantiates its veracity.%
        \autocite[380]{najman_jsj1999}
    \end{quote}
\noindent
Her ultimate conclusion is that texts such as \jub which interpret and rewrite portions of the Bible do so to ``[respond] to both the demand for interpretation and the demand for demonstration of authority.''\autocite[408]{najman_jsj1999} Thus the purpose of the book of \jub, according to Najman, is to provide an ``interpretive context'' for reading the Torah---to make explicit a particular tradition of interpretation that guides the Torah-reader away from spurious or otherwise heterodox readings. 

This idea is similar to, but importantly distinct from \vanderkam's understanding of \jub. Whereas \vanderkam envisioned \jub as an exegetical \emph{product} of Gen 1--Exod 12, Najman understands \jub as a kind of ``background'' text which is meant as an aid \emph{for reading} Torah. The difference is subtle, but significant, especially for our understanding of \jub within the framework of cultural memory. \vanderkam's characterization of \jub as a sort of ``official'' interpretation of the Torah is problematic because it does not leave room for Torah going forward: if \jub gives a comprehensive and authoritative interpretation of Gen 1--Exod 12, why read the ``biblical'' text? If \jub portrays itself as \emph{the} meaning of Gen 1--Exod 12---an ``inerrant'' interpretation of this portion of Torah---what need is there for the Torah? Najman's model, on the other hand, assumes that readers are cued into the genre. Rather than  characterizing \jub as an authoritative, but idiosyncratic, interpretation of Torah, Najman's approach understands \jub as something that could be read \emph{before} the Torah in order to quash potentially errant readings of Torah when the reader finally reaches them.%
    \autocite[408]{najman_jsj1999} 
To restate Najman's thesis using the language of memory studies, the book of \jub provides a framework of interpretation \emph{into which} the Torah can be read. It provides boundaries of interpretation and details to the narrative world of the biblical text which support interpretations of the biblical text which align with the ideology of the author of \jub. In other words, \jub is a narrative that provides a kind of etiology for the author's understanding of the biblical text. 

In her subsequent book, Najman builds on her thesis by introducing the idea of ``Mosaic Discourse'' into the discussion of Early Jewish and Christian literary production. She traces the practice of pseudonymous engagement with the Mosaic legal tradition through literary production back to the book of Deuteronomy.\autocite[48]{najman2003} She identifies four features of Mosaic discourse, which she extrapolates from the way that Deuteronomy draws from, augments, and affirms earlier legal traditions (such as the Covenant Code). The way that the author of Deuteronomy was able to both modify/reinterpret the legal tradition of the Covenant Code while retaining the traditions of the Covenant Code served as a model for later tradants (such as the author of \jub, but also the \templescroll and others) to repeat the process by engaging with and developing both the message of Moses and the idea of Moses as an author. This process of cultivating the Moses tradition is what she refers to as ``Mosaic Discourse.'' With this term, Najman builds on a Foucauldian understanding of the Author which is neither static, nor bound by any historical or literary factors. She writes:

\begin{quote}
    As Foucault reminds us, it is not only \emph{texts} that develop over time. The connected \emph{concepts} of the authority and authorship of texts \emph{also} have long and complex histories. Both models of anonymity and of pseudonymity can be found in the Hebrew Bible and in the extra-biblical texts of the \secondtemple period. But even when an author is identified in a biblical text, it is unclear if that identification is to be considered \emph{the same} as what moderns would characterize as \emph{the author function}.%
        \footnote{%
            \cite[9--10]{najman2003}. Here she is referencing
            \cite[213]{foucault_essential-foucault_2}.}
\end{quote}
\noindent
Najman suggests that when ancient writers participated in pseudonymous writing, the purpose was not to deceive their readers so much as to honor the tradition of the Author under whose name they wrote.%
    \footnote{Najman notes a number of classical authors who seem to have practiced a form of pseudonymity where a student writes in the name of their master. In particular, she cites Iamblichus the Pythagorean who claims that it was ``more honorable and praiseworthy'' to use Pythagorus' name, rather than one's own name when publishing (De Vita Pythagorica 98). She also quotes Tertullian who suggests that certain New Testament works ought to be ascribed to Paul and Peter because the works in question were written by their disciples (Marc. 6.5). Likewise, she notes that Plato wrote under the name of his master, Socrates. See \cite[13]{najman2003}.}
Historically speaking, of course, unless one posits that a real figure named Moses established the legal tradition of Israel, \emph{all} Mosaic attribution is, in effect, \psgraphical and an expansion of Moses the Author. The tradition of Moses the Author grew in step with the ``writings'' of Moses. The book of \jub, therefore, can be understood as participating within this tradition of Mosaic attribution which serves to faithfully augment the body of Mosaic teaching through the use of \psy. The interpretation of the Torah by the writer of \jub is not meant to be understood as the ``actual words'' of Moses, but as a representation of ``authentic teaching'' which aligns with the function of Moses as an Author as an aide to reading the Torah.\autocite[13]{najman2003}

% \subsection{Memory, Mosaic Discourse, and Practice}
% Reframe mosaic discourse in terms of Memory

Unsurprisingly, Najman's approach to \jub and Mosaic Discourse dovetails quite well with the idea of social and cultural memory theory. The way that Najman describes the growth and development of the Author extending beyond the historical and literary bounds of the ``real'' author is evocative of the kinds of memory construction that we discussed with respect to the book of Chronicles. In fact, what Najman describes as Mosaic Discourse is conceptually very similar to thinking of Moses as a ``site'' of memory. Just as we discussed David as a site of memory in Chronicles (see \autoref{chap:chronicles}), here, it is Moses whose identity and function has been augmented by the patina of cultural memory. We can easily identify additional sites of memory (``discourses'') about the figures like Abram (in \ga, for example), Daniel (in the book of Daniel and Greek additions), all of whom are the subjects of expanding bodies of literary production in the \secondtemple period, albeit not all as \psa, and not all with the same foundational significance as that of Moses and David. What she describes as Mosaic Discourse is the same set of processes which enabled the author(s) of the Enochic works to expand on and speculate about the Watchers and the Flood, and which enabled \ga to draw from those traditions in its rewriting. Given the fascination with the figure of Enoch in the \secondtemple period (as evidenced by the plethora of texts which evoke the character), we could just as easily talk about ``Enochic Discourse'' when we discuss the various and sundry texts which draw from, expand, and reframe the enigmatic antedeluvian figure.
% ALL DIZ SHIT IS MEMORY
The ability to talk about discourses as ``sites of memory'' signals to the broader applicability of Najman's ideas. From my perspective, therefore the discussion of character discourses as exemplified by Najman's approach can be augmented by incorporating language of cultural and social memory. Najman's terminology is able to describe \emph{that} these various texts are participating in a particular discourse, but it does not \emph{describe} the discourse nor the \emph{social influences} or \emph{social effects} of the discourses. The language of memory, on the other hand, brings with it a taxonomy for discussing these processes in sociological terms that are significant outside \secondtemple studies. In other words, the value of using memory language is not only that such language is descriptive of these processes, but also that it has value for comparative work by allowing for the identification of similar processes within other related literatures.

% WHAT IS INTERESTING ABOUT JUB is the posture toward MEMORY: PRESCRIPTIVE, PRACTICE
Although \ga and \jub generally participate in different sets of discourses,%
    \footnote{With notable exceptions, such as the division of the world sections (\jub 8:11--9:15 || \q{1}{apGen ar}{} \cols{16}{17}). See especially \cite[105--130]{machiela2009}.}
they also engage with them in qualitatively different ways, despite the fact that both can be characterized as \psa. \GA, although written largely in the first-person, takes a broadly \emph{descriptive} approach to its memory construction through rewriting. Although there are parts of the text which portray its characters in ways that betray the author's own social frameworks (See \autoref{chap:ga}), \ga resists readings which could be characterized as didactic or halakhic. When \ga seems to address halakhic matters, it does so simply by showing good people doing good. \jub, on the other hand, includes a framing narrative which encourages the reader not only to reshape the way that they think about the characters within their rewriting, but also encourage particular kinds of \emph{practices}. \jub utilizes literary devices such as testaments and appeals to the \HT to instruct the reader in proper halakot. In this sense \jub can be thought of as engaging with memory in more \emph{prescriptive} discourses.

This shift from more \emph{descriptive} rewriting (like \ga) to rewritings which incorporate \emph{prescriptive} discourses (like \jub) illustrates the social dimensions of talking about these texts as \emph{memory}. Even if the procedural, technical processes of interpretation and rewriting are identical, the social outcomes and concrete purposes of texts effect memory differently. For example, although \ga utilizes the literary form of a \emph{waṣf} (\col{20}, 2--8), this does not bear meaningfully on the tangible social effects of \ga on memory. Even supposing the readers of \ga believed \ga to be the authentic ``historical''%
    \footnote{Here again, I am referring to the fact that, for the ancient readers of \jub, the character of Lamech/Noah/Abram was likely perceived as a real person from the distant past.}
accounts of Lamech, Noah, and Abram, \ga simply \emph{does not ask} the reader to accept its account over and against any others. Its authenticity may be implied by its first-person rhetoric, but \ga does not exhibit the same kinds of authority conferring strategies that we see in \jub. 

On the other hand, the claim made by the book of \jub---that Moses received more information atop Mt. Sinai than is recorded in the books of the Pentateuch---is characterized as a kind of authoritative, revelatory literature which invites the reader to incorporate this new knowledge into their conception of the past in an act of cultural memory construction. The effect of this change on the remembered past does not remain in the abstract, however, but rather alters the way that the reader perceives the past with real-life, concrete, practical outcomes. 


\section{(Re)structuring the Past}
% Transition from memory/praxis to calendars
The practical effect of memory and memory construction can be seen most clearly within the book of \jub through the somewhat idiosyncratic system by which it orders time. In addition to being a central ordering principle of the halakhic concerns of \jub, the restructuring of the past based on a particular contemporary system of reckoning is a classic means by which contemporary social frameworks shape one's understanding of the past, as I noted in \autoref{chap:memory}. Although the system of weeks of years and jubilees as found in the Hebrew Bible (esp. Lev 25) predates the book of \jub, the application of this system of chronology to the whole of Israelite remembered past is most certainly an anachronism of \jub's author.%
    \footnote{In a forthcoming article, Jonathan Kaplan notes that the function of the jubilee in \jub and the book of Daniel are distinct from their original purpose in the book of Leviticus and instead function as a way of structuring time which imitates the epochal mode of reckoning used by the Seleucid empire. See \cite{kaplan_jaj2019}. See also \cite[137--186]{kosmin2018}.}
Likewise, although it is not clear what yearly calendar system was used by the most ancient Israelites, it was most probably not the 364-day calendar of \jub.%
    \footnote{When or whether a 364-day calendar was in use prior to the Persian or \secondtemple period remains a point of contention among scholars (more on this below). However, the portrayal of the 364-day calendar as the immutable system of the Heavenly Tablets is certainly an innovation of the \secondtemple period.}

The question may be asked, therefore, how did the calendar and chronological system of the book of \jub affect its readers through the construction of memory? In particular, how might the effects of this construction of the past affect the \emph{practice} of early Judaism? In this section I will discuss the calendrical and chronological systems presented in the book of \jub and argue that they function as more than idiosyncratic modes of accounting but had real, practical socio-political effects during the late \secondtemple period.

\subsection{The Structure of Time in \jub}
% 364 Day Year
One of the most notable features of the Book of \jub is its preoccupation with the correct division of time---both with respect to a 364 day year as well as longer units encompassing multiple years. Although neither the 364-day year nor the larger 7 and 49 year units (``weeks'' of years and ``jubilees,'' respectively) are unique to the book of \jub, the proper division of time is into these units provides the central organizing principle for the book's rewriting of Gen 1--Exod 12.

The author of \jub makes it very clear that the proper division of time through a 364 day year is an essential practice for the correct observation of religious feasts and other holidays throughout the year. It is, perhaps, unsurprising that the author is especially concerned with the festival of ``Weeks'' (Shavuot) which, though not explicit in the biblical text, is elsewhere associated with Moses' reception of the Torah.%
    \footnote{Shavuot is determined by counting seven weeks (forty-nine days) from the first grain harvest (Lev 23:15), which is typologically identical to the jubilee cycle of seven ``weeks of years'' followed by the jubilee year.}
The pattern and significance of this 364 day cycle is explained to Moses after the \ap retells the events of the Flood. The \ap explains the division of the year into four seasons, each beginning with a memorial day (\jub 6:23) and consisting of thirteen-weeks. The system as a whole yields a fifty-two week year (\jub. 6:29) and is presented as ``inscribed and ordained on the tablets of heaven'' (6:31; Eth.
    \eth{ተቄርፀ~፡ ወተሠርዐ~፡ ውስተ~፡ ጽላተ~፡ ሰማይ}
        {ta\qw{a}rḍa wa-taśarʕa wəsta ṣəllāta samāy}.
        % taqwarda       'engraved'     tG √qrḍ (Les. 440) 'to lacerate, tear away, etc.'
        % taśarʕa        'established'  tG √śrʕ (Les. 532) 'to establish, ordain'
        % ṣəllāta        'tablet' ṣəlle/ā - pl. ṣəllāt (Les. 554) 

The 364 day year is considered ``complete'' (Eth. \eth{ፍጹም}{fəṣṣum}) by the \ap such that proper observance maintains synchrony year-over-year. In other words, adding or subtracting days from this calendar renders a ``revolving'' calendar \visavis the absolute reference of the heavenly tablets.%
    \footnote{For an overview of calendar systems in the ancient world, including a discussion of ``revolving calendars,'' see \cite[214]{glessmer_flint-vanderkam1999}.}
By comparison, the \ap warns against the use of a lunar calendar because the lunar year is too short. \jub 6:36--37 reads:

% !TEX root = dissertation.tex
\begin{ethiopictext}
    \versenum{6:36}
    ወይከውኑ~፡ እለ~፡ ያስተሐይጹ~፡ ወርኀ~፡
    በሑያጼ~፡ ወርን~። እስመ~፡ ትማስን~፡ ይእቲ~፡ ጊዜያተ~፡ ወትቀድም~፡
    እምዓመታት~፡ ለዓመት~፡ ዐሡረ~፡ ዕለተ~።
    \versenum{37}
    በእንተዝ~፡ ይመጽእ~፡
    ዓመታተ~፡ ሎሙ~፡ እንዘ~፡ ያማስኑ~፡ ወይገብሩ~፡ ዕለተ~፡ ስምዕ~፡
    ምንንተ~። ወዕለተ~፡ ርኵስተ~፡ በዓለ~፡ ወኵሉ~፡ ይዴምር~፡ ወማዋዕላ~፡ 
    ቅዱሳተ~፡ ርኩሰ~፡ ወዕለተ~፡ ርኵስተ~፡ ለዕለት~፡ ቅድስት~። እስመ~፡
    ይስሕቱ~፡ አውራኀ~፡ ወሰንበታተ ወብዓላተ~፡ ወኢዩቤለ~።
\end{ethiopictext}
\begin{transliteration}
    \versenum{6:36}
    wa-yekawwenu ʔella yāstaḥayyeṣu warḥa 
    % yekawwenu     G impf. √kwn 'to be' 
    % yāstaḥayyeṣu  CGt impf. √ḥyṣ 'to perceive, observe (closely)' (Les. 252) 
    % warḫa         warḥ pl. ʔawrāḫa 'moon, month' (Les. 617) 
    ba-ḥuyāṣē warḫ ʔesma temās(s)en yeʔeti gizēyāta wa-teqaddem 
    % ḥuyāṣē        'observation'
    % esma          'because'
    % temās(s)en    L impf. √msn 'decay, corrupt' (Les. 366)
    % yeʔeti        'she, it' (here, the moon)
    % gizēyāta      gizē pl. gizēyāt 'time, season' (Les. 210)
    % teqaddem      G impf. 3fs √qdm 'to precede, go before'
    ʔem-ʕāmatāt la-ʕāmat ʕašur ʕelata
    % ʕāmatāt       ʕām pl. ʕāmatāt 'year' (Les. 62)
    % ʕašur         'the tenth day, ten days' (Les. 73)
    % ʕelata        'day' (Les. 603)
    \versenum{37}
    ba-ʔenta-ze yemaṣṣeʔ 
    % yemaṣṣeʔ          g impf 3ms √mṣʔ 'come, happen, arise, overtake'
    ʕāmatāta lomu ʔenza yāmās(s)enu wa-yegabberu ʕelata semʕ 
    % yāmās(s)enu       CL impf 3mp √msn 'decay, be corrupt' Les. 366
    % yegabberu         g impf 3mp √gbr 'act, do'
    % semʕ              √smʕ 'rumor, news, witness, testimony'
    mennenta wa-ʕelata re\kw{e}sta baʕāla wa-\kw{e}llu yedēmmer wa-māwāʕelā 
    % mennenta           D √mnn 'despise, reject, renounce'
    % rekwest           adj. √rkws 'to be unclean, impure' Les. 470
    % baʕāla            'festival' Les. 83
    % yedēmmer          D impf. √dmr 'to insert, add, mix, mingle, multiply (arithmetic)
    qedusāta rekusā wa-ʕelata re\kw{e}sta laʕlat qedust ʔesma
    % rekusa            adj. √rkws 'to be unclean, impure' Les. 470
    % laʕlat            
    yeseḥetu ʔawrāḫa wa-sanbatāta wa-beʕālāta wa-ʔiyobēla
    % yeseḥetu          √sḥt 'wound, harm, violate'
    % ʔewrāḫa           'months' warḥ op. cit.
    % sanbatāta         'sabbath'
\end{transliteration}
\begin{translation}
    \versenum{6:36}
    There will be those who watch the moon closely with lunar observations
    because it is deficient (concerning) the seasons and is premature from year to year by ten days. 
    \versenum{37}
    Therefore
    years will come about for them when they decay. And they will make a day of
    testimony despised (make) and a profane day a festival. All will mingle holy days
    (with) the profane and a profane day with a holy day, for
    the months will err along with the sabbaths and the festivals and the jubilee.
\end{translation}
\noindent
The contrast drawn to the lunar calendar combined with the fact that a 364 day calendar more closely approximates the actual period of Earth's orbit around the sun (approx. 365.24 days) led most early interpreters of \jub to call the 364 day calendar a ``solar'' calendar.%
    \footnote{Some recent contributions retain this designation such as \cite[10]{stern2001}.}
Because some of the early Israelite festivals were tied to the agricultural year (for example, \emph{Shavuot} was celebrated after the wheat harvest, see Exod 34:22), a solar calendar would indeed keep the calendar from drifting backward every year. Because the lunar (synodic) month%
    \footnote{The synodic month is derived from the length of time it takes the moon to process through its full cycle and is distinct from the period of the moon's \emph{orbit}.}
averages approximately 29.5 days, a lunar year (twelve synodic months) lasts approximately 354 days. Without any intercalation the calendar would drift back 11.24 days per year (a so-called ``revolving year''). Within a matter of only two-or-three years, the correlation between agricultural activity and cultic practice would break down.%
    \footnote{The major advantage of the lunar system is the ability for anybody to make reasonably accurate observations about when months begin and end. By contrast, the solar year requires a more subtle and long-term set of measurements. Most cultures which utilize a lunar calendar account for the discrepancy through the intercalation of an additional month every few years to bring the solar and lunar calendars into alignment. Most ``lunar'' calendars, therefore, are really lunisolar calendars, though exceptions (such as the Islamic calendar) do exist. See
        \cite[214, 238]{glessmer_flint-vanderkam1999}; 
        \cite[37--38]{horowitz_janes1996}.}

Recent treatments of the 364-day calendar, however, have eschewed the ``solar'' label in most cases.%
    \footnote{%
        \cite[231]{glessmer_flint-vanderkam1999};
        \cite[80]{bendov_steele2011};
        \cite[438]{jacobus_brooke-hempel2018}.}
The rationale for doing so is two-fold: first, although a 364-day year is \emph{close} to the actual period of Earth's orbit around the sun, the 1.24 day discrepancy is large enough that after fifty years, the calendar would have floated backward a full two-months.%
    \footnote{Specifically, 62 days. This would be the equivalent of celebrating the new year near Halloween.}
In other words, although a 1.24 day drift may not be noticeable from one year to the next, the difference is significant \emph{enough} to be noticeable within the average lifespan of an individual and would certainly conflict with agriculturally contingent festivals.%
    \footnote{\Cite[28--37]{wacholder-wacholder_huca1995}. This assumes, of course, that the various festivals continued to be connected to the agricultural cycle and not a purely utopian construct as Wacholder and Wacholder suggest.}
Second, while the \ap expresses concern with the ``corruption'' of the yearly cycle, the rationale for the 364-day year is not explicitly connected to the solar year. In other words, when the \ap decries the deficiencies of the lunar year, it does so with respect to the 364-day year and \emph{not} with respect to the solar year. Instead, the problem with a 354-day (lunar) year, according to the \ap is that the holidays, months, sabbaths, festivals, and jubilees will fall on the wrong days \emph{according to the 364-day calendar}. This rationale is, essentially, circular. The 364-day year is an absolute measure of a ``year'' according to the book of \jub based, as we have already seen, on its inscription on the \HT. Because of this fact, the calendar is not contingent or defined with reference to the sun or the moon---it is not a matter of interpretation or measurement, but of fiat. Instead, the author of \jub is more concerned with mathematical properties that allow for the clean, even division of \emph{seasons} (defined as three months) and \emph{weeks} (a so-called heptadic structure) without the need for intercalation.%
    \footnote{\Cite[125]{bendov-saulnier_cbr2008}.}

According to most reconstructions of \jub's 364-day calendar, the year was divided into four seasons consisting of exactly thirteen weeks (91 days). Each season was also divided into three months, though, because 91 does not divide evenly by 30, the third month in each season was counted as 31 days. Thus, each season was composed of two months of 30 days and one month of 31 days. Because these seasons' lengths divide evenly by seven, every season began on the same day of the week and followed an identical structure.%
    \footnote{In other words, every season began on the same day of the week, and the ``nth'' day of any given season was the same day of the week as the nth day of any other season.}
The advantage of such a system is its consistency year-over-year. Because the whole year divides evenly by seven, every day of the year (in every year) implicitly referred to a particular day of the week. Thus any scheduled event would fall on the same day of the week the following year, preventing the undesirable situation where a holiday would accidentally fall on a Sabbath (such as the memorial feasts prescribed in \jub 6:23).%
    \footnote{\Cite[233]{bergsma2007}. So, if a person were born on a Tuesday, every subsequent birthday would also fall on a Tuesday. Likewise, there would be no need to buy a new calendar every year, since every year is the same ``shape.'' See esp. \cite[253]{jaubert_vt1953}.}

Although the mechanics of this calendar are reasonably well understood, its purpose and antiquity remain a matter of debate. The seminal work of Annie Jaubert (building on Barthélemy) during the mid-twentieth century, despite numerous criticisms, remains the \emph{Ausganspunkt} for most discussions of the topic.%    
    \footnote{See especially
        \cite{jaubert_vt1953};
        \cite{jaubert_vt1957};
        \cite{jaubert1957}.
        The final work was translated into English as
        \cite*{jaubert1965}.}
Her thesis took as its point of departure Barthélemy's theory that the Jewish 364-day year began on Wednesday, the day that the sun and moon were created, according to the Priestly creation account in Genesis 1:14--19.%
    \footnote{%
        \cite{barthelemy_rb1952};
        \cite[250]{jaubert_vt1953};
        \cite[24--25]{jaubert1965}.}
To prove this idea, she began by noting that the book of \jub specifically prohibits beginning a journey on the sabbath (50:8, 12) and infers that, therefore, the various travel narratives in \jub ought to obey this rule, e.g, when Abram travelled, he would not have done so on the Sabbath according to \jub. She worked backwards through the descriptions of such journeys in \jub to confirm that, indeed, the only possible situation where the patriarchs would not have traveled on the sabbath, as described in \jub demands that the first day of the year be a Wednesday.%
    \footnote{%
        \cite[252--254]{jaubert_vt1953};
        \cite[25--27]{jaubert1965}.}
Jaubert further hypothesized that the 364-day calendar utilized by the author of \jub was, in fact, quite ancient and reflected the same views of the latest Priestly strata of the Hexateuch by applying the same method to the Hexateuch and yielding an identical result.%
    \footnote{%
        \cite[258]{jaubert_vt1953};
        \cite[33]{jaubert1965}.}
Thus, according to Jaubert, the 364-day calendar was the calendar of \secondtemple Judaism and it was not until later---at the time of Ben Sira---that the lunar modifications known from the Rabbinic period were instituted.%
    \footnote{%
        \cite[254--258; 262--264]{jaubert_vt1953};
        \cite[47--51]{jaubert1965}.}

Jaubert's thesis has been challenged and modified over the past several decades, but the publication of a number of important calendrical texts from Qumran have---at least partially---served to support the broad strokes of her thesis that the 364 day calendar was in use during the late \secondtemple period (though the more specific claims remain controversial).%
    \footnote{Early reactions to her thesis were mixed. In particular, she was critiqued by Baumgarten 
    \cite*{baumgarten_baumgarten1977} and more recently by Wacholder \& Wacholder 
    \cite*{wacholder-wacholder_huca1995} and Ravid 
    \cite*{ravid_dsd2003}. Her thesis was adopted and slightly modified by Morgenstern who suggested that the first month of the quarter was 31 days, rather than the last month; 
    \cite*{morgenstern_vt1955}, at least partially supported by \vanderkam 
    \cite*[410--411]{vanderkam_cbq1979} and still retains broad support generally, if at times (seemingly) by virtue of its ubiquity. See 
    \cite[142]{bendov-saulnier_cbr2008}.}
What seems apparent from the more recently discovered evidence from Qumran is that the system of keeping time during the \secondtemple period was not a monolith. As \vanderkam notes, among the Qumran texts the festivals were generally dated based on the 364-day calendar but there still remain cases where 354-day ``lunar'' year was used for more general purposes.\autocite[1:45]{vanderkam2018} And while the book of \jub clearly participates in a tradition that privileged the 364-day year, the particulars of the \jub calendar and its theological and ideological underpinnings do not necessarily align with other advocates for the 364-day year (such as the Astronomical Book and the other calendrical texts from Qumran).%
    \footnote{See \cite[,159]{bendov-saulnier_cbr2008}. Although the calendar of \jub is distinct from other 364 day calendars inferred from the Qumran texts, many of the more general observations about their function apply to all such calendars and are frequently discussed together. The early discussions of Barthélemy and Jaubert mostly focused on \jub, as most of the Qumran scrolls had either not been discovered or not published at the time of writing. See \cite{barthelemy_rb1952} and \cite{jaubert_vt1957}.}
In other words, one of the major observations from the most recent scholarship on the 364-day calendar tradition is that their commonalities are complimented by significant variation. So, although the Astronomical Book (\firstenoch 72--82), the Aramaic Levi Document, the  \templescroll, MMT, \q{4}{252}{} and other astronomical (e.g., \q{4}{317}{}; \q{4}{318}{}), liturgical (Songs of the Sabbath Sacrifice; \q{11}{Psalms}{a}; \q{4}{503}{}; \q{4}{334}{}) and many formally calendrical texts%
    \footnote{Ben Dov and Saulnier lists several dozen texts and fragments of these calendrical texts in their recent summary. See \cite[132--133]{bendov-saulnier_cbr2008}.}
from Qumran tend to prefer a 364-day calendar, they do not all seem to agree on \emph{why} they follow it.%
    \footnote{For a concise summary of the calendrical issues in these texts, see 
        \cite{vanderkam1998};
        \cite[233--268]{glessmer_flint-vanderkam1999};
        \cite[127--135]{bendov-saulnier_cbr2008}; and 
        \cite{jacobus_brooke-hempel2018}.}
This diversity leaves open the question of what the purpose and significance of the 364-day calendar was for the author of the book of \jub and raises new questions about its polemical underpinnings.

The larger super-annual chronological cycles which concern the author of \jub also follow a heptadic structure. Throughout the work, the author refers to ``weeks'' of years (a seven-year interval) and the length of time known as a ``jubilee'' (seven ``weeks'' of years, or 49 years) both of which are heptadic units which reflect the same concern with sabbath cycles as the intra-annual divisions.%
    \footnote{Indeed, as cited above in the prologue, the work is concerned with the ``the testimony for the event[s] of the years; for their weeks, for their jubilees in all the years of the world.''}
In fact, as \vanderkam has observed, while the calendar (364-day year) is only mentioned in \jub 6, the chronological system (7-year ``weeks'' and jubilees) is a pervasive and first-order literary device for the author's adaptation of Israel's past.%
    \footnote{\Cite[522]{vanderkam-b_vanderkam2000}. He credits Wiesenberg with this observation as well who writes, ``His chronology, not his calendar, is the object of primary interest to the writer of the Book of \jub.'' See \cite[4]{wiesenberg_rev-qumran1961}.}

The heptadic quality of the entire system of \jub's calendar and chronological systems is rooted in the traditions surrounding the sabbath and slave laws, which show considerable development within the Hebrew Bible itself. The sabbath and jubilee legislation of Leviticus 25 likely draws from and adapts the earlier slave and fallow laws from the Covenant Code (Exod 21:1--11 and 23:10--11, respectively) and bears similarities with other ancient Near Eastern practices such as the \translit{mīšarum} and \translit{andurārum} known from Mesopotamia.%
    \footnote{For the ostensible antecedents of the biblical jubilee see \cite[1--51]{bergsma2007}. Other major publications on the idea of the biblical jubilee include 
        \cite{north1954};
        \cite{fager1993} and 
        \cite{lefebvre2003}.}
At the core of the jubilee tradition in Leviticus 25 is an abstraction of the idea of sabbath ``rest'' on the seventh day of the week to longer seven-year units of time: the manumission of slaves, the forgiveness of debts, reallocation of ancestral lands, and letting the land lie fallow all occur in the seventh year, just as people were to rest on the sabbath day. Seven sets of these ``weeks'' completed a full cycle, which was then followed by a jubilee year (year 50).%
    \footnote{\Cite[85--92]{bergsma2007}.}

Within the book of \jub, however, the term jubilee is used to delineate a period of 49 years, rather than to specify the 50th year.%
    \footnote{%
        \cite[524--525]{vanderkam-b_vanderkam2000};
        \cite[234]{bergsma2007}.}
Thus, when the author of \jub describes an event occurring in the \emph{nth} jubilee, he is referring to the event occurring within a particular 49-year span and not in the \emph{nth} ``jubilee year.'' The term ``week'' or ``week of years,'' on the other hand, retains its traditional denotation.

% TODO: Add something about Shavuot here?

\subsection{Time and the Social Frameworks of Memory}
As I have alluded to, frameworks for ordering the past are not neutral and the use of particular systems bears on one's interpretation of the past and understanding of the present. In other words, chronological systems can have a profound impact on processes of memory. Thus, the ordering of time with respect to the 364-day year sabbath and jubilee traditions should be understood not simply as the alignment of the past with an idiosyncratic numbering system, but as a reinterpretation and commemoration of Israel's past within a discrete social and ideological framework. The alignment of Israel's remembered past within the symbolic system(s) of the heptadic sabbath and jubilee cycles helps to reinforce both the schema itself and the particular interpretation of history offered by the author.

The insistence of the author of \jub that the 364-day year be maintained and his sharp rebuke of those who ``closely observe the moon'' (\jub 6:36) point toward the likelihood that calendar conflicts were a point of contention between the author of \jub and some of his contemporaries. This apparently polemical tone used by the author has prompted speculation about the possible causes of such polemic. \vanderkam, for example has suggested that the impetus for the calendar dispute was Antiochus IV Epiphanes' imposition of a Hellenistic luni-solar calendar in-or-around \bce{167}. According to \vanderkam's theory the 364-day calendar was the calendar in use by the \jerusalemtemple in the late Persian and early \secondtemple periods (generally following the argument of Jaubert). As evidence for Antiochus IV's calendrical changes, \vanderkam cites the numerous and infamous decrees made by Antiochus IV recounted in the books of Daniel and 1 \& 2 Maccabees. Although he concedes that none of these texts demand a calendrical change (only that the decrees prohibited certain festivals) \vanderkam reads Dan 7:25 to mean that the Seleucids did not only proscribe certain Jewish practices, but may have imposed a different calendar system.%
    \footnote{\Cite[59--60; 68--69]{vanderkam_jsj1981}}
Daniel 7:23--25 reads:

\begin{aramaictext}
    \versenum{Dan 7:23}
    ‏כֵּן אֲמַר חֵיוְתָא רְבִיעָיְתָא מַלְכוּ רְבִיעָיָא תֶּהֱוֵא בְאַרְעָא דִּי תִשְׁנֵא מִן־כָּל־מַלְכְוָתָא וְתֵאכֻל כָּל־אַרְעָא וּתְדוּשִׁנַּהּ וְתַדְּקִנַּהּ׃ ‎
    \versenum{24}
    ‏ וְקַרְנַיָּא עֲשַׂר מִנַּהּ מַלְכוּתָה עַשְׂרָה מַלְכִין יְקֻמוּן וְאָחֳרָן יְקוּם אַחֲרֵיהוֹן וְהוּא יִשְׁנֵא מִן־קַדְמָיֵא וּתְלָתָה מַלְכִין יְהַשְׁפִּל׃ ‎
    \versenum{25}
    ‏ וּמִלִּין לְצַד עִלָּיָא יְמַלִּל וּלְקַדִּישֵׁי עֶלְיוֹנִין יְבַלֵּא וְיִסְבַּר לְהַשְׁנָיָה זִמְנִין וְדָת וְיִתְיַהֲבוּן בִּידֵהּ עַד־עִדָּן וְעִדָּנִין וּפְלַג עִדָּן׃
\end{aramaictext}

\begin{translation}
    \versenum{Dan 7:23} 
    Thus he said, ``As for the fourth beast, there will be a fourth kingdom on the earth which will be different from all the other kingdoms and it will consume the whole earth and trample it and crush it.
    \versenum{24}
    As for the ten horns---from it [the kingdom] ten kings will rise up and another will rise up after them and that one will be different from the previous ones and will bring down three kings.
    \versenum{25}
    And he will speak words against the Most High and he will wear out the Holy Ones of the Most High and he will try to change the times and the Law and they will be given into his hand for a time, two times, and half a time.''
\end{translation}

\noindent
\vanderkam suggests that the Aramaic term \aramaic{זִמְנִין} in v. 25 may be equivalent to Hebrew \hebrew{מוֹעֲדִים} or \hebrew{עִתִּים} and thus may be referring to particular appointed times and festivals.%
    \footnote{\Cite[59--60]{vanderkam_jsj1981}.}
\vanderkam further argues that 1 Macc 1:59 and 2 Macc 6:7a allude to the practice of celebrating the king's birthday with a sacrifice on a monthly basis (every \emph{nth} day of the month) which would have demanded that the \jerusalemtemple to adopt the Seleucid calendar. Thus, he reasons, this may be the time when the traditional 364-day calendar was replaced by the Hellenistic lunisolar calendar in the \jerusalemtemple. When the Maccabees took power, however, they did not, apparently, revert back to the older calendar. The conservative ``Essene'' group which later formed the Qumran community opposed this innovation and separated themselves from the Jerusalem priesthood. Thus, \vanderkam suggests that the calendar change/crisis may have been one of the major precipitating factors for the schism between the Qumran community and the \jerusalemtemple authorities.\autocite[52]{vanderkam_jsj1981} \vanderkam's theory, however, has been met with some resistance, particularly from scholars such as Philip Davies, among others.%
    \footnote{%
        \cite{davies_cbq1983};
        \cite{wacholder-wacholder_huca1995};
        \cite{stern_lim-etal2000};
        \cite{stern_zpe2000};
        \cite[29 n. 136]{stern2001}.
        The core of these criticisms boil down to the fact that \vanderkam's theory is quite speculative and lacking in concrete \emph{positive} evidence of his historical reconstruction. The theory provides a clean explanation for a pressing historical question, but is perhaps a bit over-simplified. Ben Dov and Saulnier observe that \vanderkam's theory tends to be more popular among scholars who specifically study Essenes, while it is generally rejected by historians of the \secondtemple period more generally. See \cite[142]{bendov-saulnier_cbr2008}.}

The putative calendrical conflict to which \jub alludes, for our purposes, points toward the \emph{significance} of such traditions for everyday practice. For the author of \jub (and, perhaps for the Qumran community) the calendar was not simply a mundane system for bookkeeping, but was intimately tied to liturgical  and cosmological order. Such a system aligns with God's created order which takes the seven-day week as its fundamental unit (as described in Gen 1). Such a system, one presumes, ought to respect the sanctity of the sabbath and prevent the overlap of holidays with the sabbath. The book of \jub does not appeal to observation or ``science'' but instead asserts the absolute fact of the 364-day year, as established by God and recorded on the Heavenly Tablets. 

Although the book of \jub portrays the 364-day year as a principle \emph{predicated on} a seven-day week and related numerical properties, in fact, from the perspective of memory construction and reinforcement, the opposite is the case. By insisting on the utilization of a calendar whose distinguishing characteristic is its protection of sabbath laws (i.e., that no holidays will ever conflict with the sabbath), and the consistency of memorial days \visavis the day of the week, the calendar reinforces the practices of observing the sabbath and the other holidays. In other words, it is a system that not only espouses a particular halakhic agenda, but espouses a system which enables and reinforces that agenda. Sabbath observance is, in some sense, a logical response to a cosmos that is ordered in such a fashion. It is a system which (though not, perhaps, designed for the purpose) reinforces some of the fundamental practices of early Judaism. Whether or not the 364-day calendar was the more traditional calendar or whether it was an innovation is, to some degree beside the point. The author of the book of \jub made explicit what \emph{he} thought was the right calendar and leveraged his text---with all of its authority conferring strategies---to participate in the contemporary cultural conflict over what calendar to use. 

The significance of the 364-day calendar as one centered around the idea of the sabbath cycle gives cultural meaning to a calendar that likely found its origins elsewhere. This significance, which was imputed upon the 364-day year, then became one rationale by which the sabbath ritual practice was bolstered. This circular and iterative process of memory construction is not limited to abstract ideas of timekeeping, but is intimately tied to practice. But, importantly, the practice \emph{itself} then serves to reinforce the potency of the system. 
% The significance of thinking about the past within a framework of Jubilees
The larger cycles of weeks and jubilees likewise carry significance beyond their simple numerical values. The seven and 49-year cycles are drawn from Lev 25. Of course, the author of \jub is \emph{drawing from} Lev 25, though the rhetorical force of \jub suggests that the heptadic system of Lev 25 derives from the Heavenly Tablets. 

% FIXME: Add something about weeks?

More so than the seven-year week, the jubilee carried special significance for the Israelite chronological tradition. It is unimportant whether or not the biblical jubilee as described in Lev 25 was every actually carried out or whether it was a utopian fiction;%
    \footnote{Kaplan, for example, finds the practice ``plausible'' (without affirmative evidence), but he is careful to state his case without implying that it likely \emph{did} happen. See \cite{kaplan_cbq2019}. Cf. \cite[119]{wellhausen1957}. Kaplan has also recently offered a new proposal for the origin of the term \hebrew{יובל}. See \cite{kaplan_biblica2018}.}
what matters is that the during the \secondtemple period, the year of jubilee was a potent idea with broad social and political implications.%
    \footnote{The imagery of the biblical jubilee is even evoked in the Christian New Testament in the gospel of Luke when Jesus quotes from Isa 61:1; 58:6; and 61:2. Although the term \hebrew{יוֹבֵל} ``jubilee'' is not explicitly used in the MT of Isa, the related term \hebrew{דְּרוֹר} ``liberty'' (the term used for what one ``declares'' in the jubilee year, according to Lev 25:10) is used.}
Using the system of jubilees allowed the author of the book of \jub to align his chronological system with significant events and ideas in the history and tradition of Israel. For instance, as Bergsma notes, within the framework of the book of \jub, the history of Israel from creation to the entrance to the land amounts to precisely 50 cycles of jubilees.%
% TODO: "The author of the Jubilees also ties significant events to Shavout (i.e., the giving of the Torah at Sinai), which is also calculated around seven weeks plus 1."
    \footnote{\Cite[234--235]{bergsma2007}. Here Bergsma is building on the work of Wiesenberg and VanderKam. See
        \cite[4]{wiesenberg_rev-qumran1961} and 
        \cite[522]{vanderkam-b_vanderkam2000}.}

According to the book of \jub, the Sinai revelation---where the book of \jub is set---occurs precisely 9 years (or, more accurately, one week and two years) into the 50th jubilee cycle since creation. \jub 50:2-4 reads:

% !TEX root = dissertation.tex

%% COMPLETE
\begin{ethiopictext}
    \versenum{50:2}
    ወሰንበታተ~፡ ምድርኒ~፡ ነገርኩከ~፡ በደብረ~፡ ሲና~፡ ወዓመታተ~፡
    ኢዮቤሌዎኒ~፡ ውስተ~፡ ሰንበታተ~፡ ዓመታት~፡ ወዓመቶሰ~፡ ኢነገርናከ~፡
    እስከ~፡ አመ~፡ ትበውኡ~፡ ውስተ~፡ ምድር~፡ እንተ~፡ ትእኅዙ~፡
    \versenum{3}
    ወታሰነብት~፡ ምድርኒ~፡ ሰንበታቲሃ~፡ በነቢሮቶሙ~፡ ዲቤሃ~፡ 
    ወዓመቶ~፡ ለኢዮቤል~፡ ያእምሩ~። 
    \versenum{4}
    በእንተዝ~፡ ሠራዕኩ~፡ ለከ~፡ 
    ሱባዔ~፡ ዓመታት~፡ ወኢዮቤሌዎታተ~፡ አርብዓ~፡ ወተስዐቱ~፡ ኢዮቤሌዎታት~፡
    እምነ~፡ መዋዕሊሁ~፡ ለአዳም~፡ እስከ~፡ ዛቲ~፡ ዕለት~፡ ወሱባዔ~፡
    አሐዱ~፡ ወክልኤ~፡ ዓመታት~፡ ወዓዲ~፡ አርብዓ~፡ ዓመት~፡ 
    ርሑቅ~፡ ለአእምሮ~፡ ትእዛዛተ~፡ እግዚአብሔር~፡ እስከ~፡ አመ~፡ ያዐድው~፡
    ማዕዶተ~፡ ምድረ~፡ ከናአን~፡ ዐዲዎሙ~፡ ዮርዳኖስ~፡ ገጸ~፡ ዐረቢሁ~፡
\end{ethiopictext}
\begin{transliteration}
    \versenum{50:2}%
    wa-sanbatāta mədr-ni nagarkuka ba-dabra sinā wa-ʕāmatāta
    % sanbatāta         'sabbaths'
    % medre-ni          -ni 'even, that very' Lambdin 51.4
    % nabarku-ka        g pf 1cs √nbr 'to say'
    % sinā              'Sinai'
    % ʕāmatāta          ʕām pl. ʕāmatāt 'year' (Les. 62)
    ʔiyyobelewo-ni wəsta sanbatāta ʕāmatāt wa-ʕāmato-sa ʔi-nagarnāka
    % ʔiyyobelewo        + 3mpl
    % -sa               -sa 'but' =gk δε see Lambdin 51.4
    % nagarnā           g pf 1cp + 2ms 'to say'
    ʔəska ʔama təbawwəʔu wəsta mədr ʔənta təʔəḫḫəzu
    % təbawwəʔu          g impf. 2mp √bwʔ Les. 114
    % ʔama              ʔama prep. 'at the time of'
    % teʔeḫḫezu         g impf. 2mp √ʔḫz 'take, catch, hold' Les. 14
    \versenum{3}%
    wa-tāsanabbət mədr-ni sanbatātihā ba-nabirotomu dibehā
    % tāsanabbet        QC impf (quad) √snbt 'to make sabbath'
    % nabirotomu        G inf. + 3mp √nbr 'to live, dwell'
    wa-ʕāmato la-ʔiyyobel yāʔməru
    % ʕāmato            ʕām sn. 'year'
    % yāʔəmmeru         CG subj √ʔmr 'understand, recognize'
    %                   if this was impf e should be a for I-guttural
    \versenum{4}%
    ba-ʔənta-zə śarāʕku laka
    % sarāʕku           g pf 1cs √srʕ 'establish'
    subāʕe ʕāmatāt wa-ʕiyyobelewotāta ʔarbəʕā wa-tasʕatu ʔiyyobelewotāt
    % subāʕe            'week'
    ʔəmənna mawāʕlihu la-ʔadām ʔəska zāti ʕəlat wa-subāʕe
    % mawāʕel           'period, era, time' √mʕl 'to pass the day' Les. 603
    ʔaḥadu wa-kəlʔe ʕāmatāt wa-ʕādi ʔarbəʕā ʕāmat
    % ʕādi
    rəḥuq la-ʔaʔmro təʔzāzāta ʔəgziʔabḥer ʔəska ʔama yaʕaddəw
    % reḥuq             'distant'
    % teʔzāzāta         'commandment'
    % yaʕaddew          g impf. √ʕdw 'to lead'
    māʕdota mədra kanāʔan ʕadiwomu yordānos gaṣṣa ʕarabihu
    % māʕdota           'accross' √ʕdw
    % ʕadiwomu          g pf ptc √ʕdw +3mp

\end{transliteration}

\begin{translation}
    \versenum{50:2}
    I told you [about] the sabbaths of the land [while] on Mt. Sinai and the years of
    Jubilees in the sabbaths of the years. But we did not tell you its year
    until the time that you would enter the land, which you will possess.
    \versenum{3}
    And the land will observe its sabbath when they dwell upon it
    that they may know the year of Jubilee.
    \versenum{4}
    Because of this I have established for you
    the weeks of years and the Jubilees: 49 Jubilees
    from the time of Adam to this day and one week
    and two years. It is still forty years
    away with regard to the learning the commandments of the Lord until the day he leads [them]
    across [to] the land of Canaan, having crossed the Jordan which is west of it.
\end{translation}

\noindent
Thus, after a 40-year wandering, the Israelites enter the land in the jubilee year of the 50th jubilee cycle. The significance of this juxtaposition should not be missed---the jubilee legislation of Lev 25 instructed that all ancestral lands be returned and redistributed among the tribes of Israel. The entrance of the people of Israel into the Land has been reinterpreted as participating in the jubilee cycle. The author of the book of \jub has, therefore, made an explicit connection between the entrance into the land and to the rightful (re)possession of inalienable ancestral landholdings. \jub's restructuring the remembered past, therefore, reinforced the socio-political ideology of Israel's right to the land.%
    \footnote{As Bergsma notes, the descendants of Shem were allotted the land of Israel in the division of the world (specifically, 8:12--21), but that the descendants of Ham (the Canaanites) seized Shem's portion (10:27--34). The passage offers an etiology for the fact that the name of the land was ``Canaan,'' but also serves the ideology that the conquest was a \emph{re}possession, and thus could be thought of as a part of the jubilee/release cycle. See \cite[234--35]{bergsma2007}.}

% Jubilee cycles of the past provides a framework for understanding the relationship of the *present* to the distant past. When is the next Jubilee? What happened in earlier jubilees?
Such an ideology, however, is not confined only to the abstract \emph{idea} of the past, but carries with it concrete political implications. It is significant, I think, that the book of \jub was likely composed near the time of the Maccabees when discourses about national sovereignty and messianic renewal were widespread. Of course, I am not suggesting that \jub is specifically referencing the Maccabean revolt, only that discourses of national sovereignty and idealized social renewal were potent at this time. Carrying this idea forward a bit more, because the jubilee cycle was a predictable one, it stands to reason that one of the underlying subtexts of the book of \jub is that another jubilee year is coming. The connection of the jubilee to the (re)possession of the Land further invites the reader to infer that the coming jubilee might bring with it not only release for those in debt-bondage, but release from the bondage of foreign occupation and a renewed Israelite state. Because the narrative setting of the book is in the distant past, the author does not plainly tell the reader when the next jubilee would arrive and instead links the ultimate restoration of the land to Israel to the successful purgation of sinful behavior from the people. \jub 50:5 reads:

% !TEX root = dissertation.tex

\begin{ethiopictext}
    ወኢዮቤሌዎታት~፡ የኀልፉ~፡ እስከ~፡ እመ~፡ ይነጽሕ~፡ እስራኤል~፡
    እምኵሉ~፡ አበሳ~፡ ዝሙት~፡ ወርኵስ~፡ ወግማኔ~፡ ወኀጢአት~፡ ወጌጋይ~፡ 
    ወየኀድር~፡ ውስተ~፡ ኵሉ~፡ ምድር~፡ እንዘ~፡ ይትአመን~፡ ወአልቦ~፡
    እንከ~፡ ሎቱ~፡ መነሂ~፡ ሰይጣነ~፡ ወአልቦ~፡ መነሂ~፡ እኩየ~፡ ወትነጽሕ~፡ 
    ምድር~፡ እምውእቱ~፡ ጊዜ~፡ እስከ~፡ ኵሉ~፡ መዋዕል~።
\end{ethiopictext}

\begin{transliteration}
    wa-ʔiyyobēlēwotāt yaḫalləfu ʔəska ʔama yenaṣṣəḥ ʔəsrāʔēl
    % ʔiyyobēlēwotāt a gross spelling of Jubilee
    % yaḫḫalefu         g impf √xlf 'to pass by' ye > ya for first gutteral
    % eska              'up to'
    % ʔama              when
    % yenaṣṣeḥ          g impf √nṣḥ 'to be pure' Les. 405
    ʔəm \kw{ə}llu ʔabbasā zəmmut wa-rə\kw{}s wa-gəmānē wa-ḫaṭiʔat wa-gēgāy
    % ʔabbasā           √ʔbs 'sin' Les. 5
    % zəmmut            √zmw 'fornication, adultery' Les. 640
    % rəkws             √rkws 'filth, impurity' Les. 470
    % gəmānē            √gmn 'profanation, pollution' Les. 194
    % ḫaṭiʔat           √ḫṭʔ 'sin, lack, fault, offense' Les 268
    % gēgāy             √ggy 'iniquity, sin' Les. 185
    wa-yaḫaddər wəsta \kw{ə}llu mədr ʔənza yətʔamman wa-ʔalbo
    % yaḫaddər          √ḫdr 'reside, dwell, inhabit'
    % yətʔamman         gt impf √ʔmn 'to be confident' with ʔenza see Lambdin §32.3
    % ʔalbo             'it does not have' negation of possessive 'ba-'
    ʔənka lotu manna-hi sayṭāna wa-ʔalbo manna-hi ʔəkuya wa-tənaṣṣeḥ
    % ʔənka             with negation 'no longer' Les. 29--30
    % manna-hi          'anyone' Les. 347-348
    % sayṭāna           satan
    % ʔekuya            √ʾky 'evil'
    % tənaṣṣeḥ          g impf √nṣḥ 'to be pure' Les. 405
    mədr ʔəm-wəʔətu gizē ʔəska \kw{ə}llu mawāʕel
    % mədr              'land'
    % mawāʕel           'period, era, time' √mʕl 'to pass the day' Les. 603
\end{transliteration}

\begin{translation}
    \versenum{50:5}
    The Jubilees will pass until [that time] when Israel is pure
    from all sin of fornication, filth, pollution, offense, and iniquity
    and it will dwell in the whole land having confidence. And it will no longer have
    any satan nor evil person. The land will be pure from that time until eternity.
\end{translation}
\noindent
Evoking the memory of the biblical jubilee year not only ties the shifting independent political landscape to divinely ordained cycles of renewal, but ties the restoration of ancestral lands to this cycle predicated on proper living and the purgation of evil from the people. 

% Some kind of Concluding statement before conclusion 
Thus the calendrical (364-day) system and the larger chronological ordering (weeks and jubilees) of the book of \jub functioned as a means by which the story of Israel's remembered past could be restructured and recontextualized for its \secondtemple period audience. The systems of time both reinforced the significance of sabbath laws and keeping the festivals and the super-yearly cycles of weeks and jubilees carried with them ideologies of liberty and renewal of the land. Rewriting Israel's past within this system brought new interpretations of Israel's entrance into the land as and expression of the \jub year, and invites the reader expect the \emph{next} jubilee. This expectation carried with it concrete social and political expectations.

\section{Conclusion}

The book of \jub, therefore, can be understood from the perspective of social and cultural memory theory as a participant in the continued process of building up the figure Moses (and the Sinai tradition more generally) as a site of memory. Unlike \ga which primarily engaged its mnemonic subjects (Lamech, Noah, and Abram) descriptively, the book of \jub explicitly engages in memory construction for the purpose of affecting social practice (halakah). While \ga may be read as \emph{implicitly} prescriptive, the book of \jub makes efforts to embed halakhic material within the narrative in such a way so as to clearly and \emph{explicitly} instruct the reader. The book \emph{as a whole} is characterized as a kind of authoritative revelation which invites the reader to incorporate this new knowledge into their conception of the past in an act of cultural memory construction. The effect of this change on the remembered past is not intended to simply change the reader's intellectual picture of Moses and the Sinai tradition, but to impact how the reader and their society behaved.

Nowhere in \jub are the practical effects of memory construction so acutely felt than with its highly-schematic calendrical and epochal systems. The 364-day calendar---presented as an immutable cosmic absolute---reinforces the idea that the 7-day week and sabbath system form an integral part of the cosmic order. %
    % TODO: First introduced with respect to creation of the cosmos.
Its ability to form a calendar which did not change year-over-year reinforced sabbath observance by never allowing a holiday to fall on the sabbath (except, of course, those holidays that last more than seven days). Likewise the system of weeks and jubilees (also built on a heptadic principle) was reinterpreted typologically to connect with the entrance of the people into the land and invites the reader to infer that the coming jubilee might bring with it not only release for those in debt-bondage, but release from the bondage of foreign occupation and a renewed Israelite state.

The mnemonic processes identifiable in the book of \jub are the same as those found in the books of \chronicles, as well as those in \ga. In \jub we can see how Moses and the Sinai traditions were productive sites of memory during the \secondtemple period which invited innovation and exerted magnetic effects on related sites of memory such as the jubilee and sabbath cycles, as well as the festival of Shavuot. Moreover, the book of \jub participates in \psgraphical discourses which qualitatively changed the way that the work was read by its audience. These processes have been identified and analyzed within the book of \jub previously, but considering them within the framework of cultural memory provides a lens through which we can identify not just the abstract discursive qualities of the such texts, but also consider how those discourses may have affected the \emph{practice} of \secondtemple Judaism.