\hypertarget{genesis-pseudepocryphon}{%
\chapter{Genesis Pseudepocryphon}\label{genesis-pseudepocryphon}}

This chapter examines the Genesis Apocryphon through \ldots{}. and it
will do so by discussing the Genesis Apocryphon at three distinct levels
of discourse: structurally at the macro level as an example of
pseudepigraphy, in terms of keying and framing at the level of narrative
discourse, and finally from the perspective of biblical discourse.

\hypertarget{pseudepigraphy-and-1st-person-narrative}{%
\section{Pseudepigraphy and 1st Person
Narrative}\label{pseudepigraphy-and-1st-person-narrative}}

The vast majority of the Hebrew Bible is narrated in the third-person
omniscient. There are, of course, exceptions to this generalization,
most notably within the prophetic corpus (such as Isa 6--8) and the
so-called Nehemiah Memoir (Neh 11--13), but for the lion's share of the
biblical text, the the implied author operates invisibly. The rhetorical
force of this particular authorial voice, as observed by Erhard Blum, is
significant for the function of the Hebrew Bible's participation in the
collective memory of the communities that claim it as their own.
Although the implied author does occasionally engage directly with the
reader by offering explanatory observations (for example where the
author inserts phrases like ``this is why\ldots{}'' or ``\ldots{}until
this day''), for all intents and purposes, the author presents himself
as both \emph{reliable} and \emph{authoritative} without a hint of
subjectivity. As Blum puts it, ``In this sense the narrative does not
distinguish the depiction from the
depicted.''\autocite[33]{blum_barton-etal2007} Put another way, the text
does not acknowledge that it \emph{has} an author, it simply \emph{is}.
The rhetorical effect of this invisible, omniscient author is to
collapse the knowledge gap between the reader and the events narrated by
removing the author from view. This move, according to Blum, allows the
text to convey ``an unmediated truth claim which is not based on the
author's distinguishable critical judgments and
convictions.''\autocite[33]{blum_barton-etal2007} The effectiveness of
this implied author, according to Blum is tied to the pragmatics of the
text. Because the implied audience of the biblical narratives
by-and-large can be understood as insiders for whom the biblical text
worked to reinforce group identity, such ``unmediated truth claims''
\emph{were} in fact mediated and reinforced by those who (orally or
otherwise) transmitted the tradition from one generation to
another.\autocite[33]{blum_barton-etal2007} In other words, one might
say that the voice of the biblical text is the voice of the collective
memory.\autocite[Blum writes, ``If we assume that the traditional
literature was primarily transmitted through oral means, than the
narrator who is speaking supplies the material with a personal presence;
he is not present as an author who judges and evaluates his sources from
a critical distance, but as a `transmitter' who participates in the
tradition itself and is able to lend it credence through his own
personality, his standing, and/or his
office.''][33]{blum_barton-etal2007}

In contrast to the omniscient implied biblical author, the
Genesis Apocryphon frames itself as a collection of first-person
accounts which formally fall into the category of pseudepigraphy. The
extant material from Genesis Apocryphon is made up of a series of three
(mostly) first-person narratives by the early biblical figures of
Lamech, Noah, and Abram. Because the beginning and end of the scroll is
missing, we do not have a way to determine whether some kind of framing
narrative may have unified the first-person accounts of Lamech, Noah,
and Abram or whether each account should be read as a standalone
pseudepigraphical work. The beginning of both the Lamech and Abram
accounts are missing, however, Genesis Apocryphon col. 5.29 seems to
offer a superscription for the account of Noah, reconstructed as
\emph{{[}pršgn{]} ktb mly nwḥ} ``{[}a copy of{]} the book of the words
of Noah.'' This may suggest that the other first-person accounts, too,
had such superscriptions. If this is the case, we can, at least, suppose
that each narrative account was \emph{presented} by Genesis Apocryphon
as an independent written account given by one of these three
patriarchs, whether or not they were ``genuine'' independent
pseudepigraphical works. At the moment, the trends seem to point toward
the authorial unity of Genesis Apocryphon, led by Moshe Bernstein. This
is based on a number of shared terms and themes found throughout
Genesis Apocryphon.\autocite{bernstein_jbl2009}

\hypertarget{keying-and-framing-abraham}{%
\section{Keying and Framing Abraham}\label{keying-and-framing-abraham}}

\hypertarget{inner-biblical-interpretation}{%
\section{Inner-biblical
Interpretation}\label{inner-biblical-interpretation}}
