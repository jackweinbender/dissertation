
\chapter{Genesis Pseudepocryphon}

<<<<<<< HEAD
% !TEX root = dissertation.tex

\nocite{dillamnn_jbw_kleine}
\nocite{ewald_zkm1844}

%%%%%%%%%%%%%%%%%%%%%%%%%%%%%%%%%%%%%%%%%%%%%%%%%%%%%%%%%
% MEMORY CONSTRUCTION is the key Idea for this chapter. %
%%%%%%%%%%%%%%%%%%%%%%%%%%%%%%%%%%%%%%%%%%%%%%%%%%%%%%%%%

Like the \ga, the book of \jub engages in a form of rewriting which participates in the construction of (biblical) memory through \psgraphical discourse. \jub builds clearly from the biblical material (Gen--Exod 12; with the kinds of adaptations, harmonizations, and emendations we expect of \rwb) and bears clear influences from other \secondtemple traditions such as the Astronomical Book of Enoch (\firstenoch 72--82). In this respect, \jub bears many of the same kinds of qualities that I worked with in the previous chapter such as the sources of tradition, generic features, and narrative framing.%
    \footnote{One might argue, for example, that the genre of \jub is that  of Apocalypse. In at least the most formal sense, this is the case, although, notably Todd Hanneken has recently advanced a thesis which argues that \jub is a \emph{subversion} of Apocalypse. See \cite{hanneken2012}. Regarding the formal characteristics of Apocalypse, see John Collins's work on the topic, esp. \cite{collins_mason-etal2012} and \cite{collins_semeia1979}.}
Although not framed as a first-person account, \jub also portrays itself as the product of first-hand experience: the author presents his work as the result of God's repeated command to Moses to ``write down all that you hear'' (1:5, 7, 26) and to record the content of the Heavenly Tablets dictated to him by the chief angelic being (2:1). Thus the author takes on the persona of Moses and proffers his work as a faithful record of Moses's experience atop Mt. Sinai and is therefore also counted among the \psa. The rewritten account of ``biblical history'' from Gen 1--Exod 12 is, like \ga, \emph{drawn from} biblical memory and \emph{speaks back into} biblical memory through the process of rewriting. 

Taking the book of \jub as my point of departure, in this chapter I will attempt to differentiate the \emph{manner} that \rwb texts may have engaged with the cultural memory. In the previous chapter, I suggested that the \psgraphical quality of \ga engaged with the cultural memory differently than other non-\psgraphical texts, but left open the question of how, specifically, readers were intended to understand the ``authority'' or ``authenticity'' of the account. In this chapter I will focus on the ways that \jub portrayed itself as an authoritative revelation and how that affected concrete \emph{practice}. I will argue that the book of \jub engages with cultural memory in a distinct fashion from other texts, such as the \ga, in part through rhetorical means (so-called ``authority conferring strategies''). I will further argue that this distinction is significant because it illustrates the way that memory not only affects the intellectual conceptions of the past, but also carries with it \emph{concrete practical effects which can be concretely observed}. To accomplish this, I will draw on Hindy Najman's work on Mosaic Discourse and will discuss the ways that \jub portrays itself as authoritative literature, how this portrayal may have been understood in antiquity and how it could, in some sense, both authorize and rewrite the Torah with halakhic implications. Then, to illustrate the point, I will discuss the calendrical and chronological system of the book of \jub as an example of the concrete ways that constructing memory can impact one's understanding of the past and present with concrete practical implications.

%%%%%%%%%%%%%%%%%%%%%%%%%%%%%
% Discovery and Publication %
%%%%%%%%%%%%%%%%%%%%%%%%%%%%%
\section{Jubilees: Discovery and Publication}
The work now referred to as the book of \jub was believed to have been lost forever by European scholars prior to the mid nineteenth Century. The work was ``rediscovered,'' however, in 1844 when Heinrich Ewald published a description of an Ethiopian (\geez) manuscript under the title ``the Book of the Division'' \eth{መጽሐፈ~፡ ኩፋሌ}{masḥafa kufālē}.%
        \footnote{All translations are my own. \geez citations are from \vanderkam's critical edition, \cite*{vanderkam1989}.}
Because the name followed the common convention using a work's first few (key) words as its title (in this case, \eth{ዝንቱ~፡ ነገረ~፡ ኩፋሌ}{zentu nabara kufālē}), Ewald suggested that this manuscript may have been a copy of the work known from antiquity as both \greek{τά Ἰωβηλαϊα}, ``the Jubilee,'' and \greek{Λεπτὴ Γίνεσις}, the ``Little Genesis.''\autocite[176--179]{ewald_zkm1844} Although the work had been in continuous use within Ethiopian Christianity since antiquity, prior to Ewald's publication, European scholarship only knew of the work through secondary references in a few classical sources.%
        \footnote{\vanderkam offers a concise summary of the various late-antique citations and allusions in his commentary, most notably in the works of Epiphanius (\emph{Panarion}, \emph{Measures and Weights}) and Syncellus (\emph{Chronography}).
                \cite[1:10--14]{vanderkam2018}. See also 
                \cite{reed_kister-etal2015} and 
                \cite{kreps_ch2018}.
        It is also probable that more recently discovered text, such as the Damascus Document (CD) refer to the book of \jub as 
        ``the Book of the Divisions of the Times into their Jubilees and Weeks'' Heb. \hebrew{ספר מחלקות העתים ליובליהם ובשבועותיהם}. It seems almost unimaginable that CD was not referring to \jub, though, some have questioned the notion. See \cite[242--248]{dimant_vanderkam-etal2006}.}
The work was published (supplemented with a second manuscript) by August Dillmann in 1859\autocite{dillmann1859} and by R.~H. Charles in 1895, who included two additional manuscripts in his edition (totaling four).\autocite{charles1895} More recently, \vanderkam's 1989 edition utilized twenty-seven copies of the text\autocite[1:xiv--xvi]{vanderkam1989} and since its publication over twenty more copies have been cataloged and imaged.%
        \footnote{%
                \cite{erho_bsoas2013}.
                \vanderkam helpfully lists the twenty-seven manuscripts he used for his critical edition in the introduction of his commentary where he also notes the additional manuscripts photographed since its publication. See 
                \cite[1:14--16]{vanderkam2018}.}

With the exception of the ``rediscovery'' of the text for European scholarship, the most significant find for the study of \jub was the discovery of several Hebrew fragments among the \dss. These fragments  attest to the work's antiquity and confirmed that the original language of \jub was Hebrew and not Aramaic, as Dillmann originally supposed.%
        \footnote{\cite[90]{dillamnn_jbw1850}. Though, as \vanderkam notes, he seems to have changed his mind later and supposed a Hebrew original. \cite[324]{dillmann_spaw1883}; \cite[1:1 n. 1]{vanderkam2018}.

        The Ethiopic text is a granddaughter translation of the Hebrew through Greek, though no Greek manuscripts of the text have been found. See especially \vanderkam's treatment of the textual history of \jub in \cite*[1--18]{vanderkam1977}. This fact was convincingly demonstrated by Dillmann who observed several Greek forms preserved as transliterations in the Ethiopic text, specifically: \greek{δρῦς}, \greek{βάλανος}, \greek{λίψ}, \greek{σχῖνος}, and \greek{φάραγξ}. See, \cite[88]{dillamnn_jbw1850}. Charles later added \greek{ἡλιου} to the list. \cite[xxx]{charles1902}.
        
        By the end of the nineteenth century, however, partial copies of \jub had also been uncovered in Latin, which similarly appear to have come through the Greek. Finally, although no direct manuscript evidence has been found, \jub scholars posit that a Syriac translation of the Hebrew was made in antiquity. This suggestion is tenuous, but is based on a number of Syriac citations of \jub which do not show any linguistic influence (loan words, etc.) from Greek.
        See especially
                \cite[231--232]{tisserant_rb1921} and 
                \cite[xxix]{charles1902} but also 
                \cite[2:ix--x]{ceriani1861} and 
                \cite[x]{charles1895}.}
Despite all of these finds, however, the Ethiopic text remains the only tradition to preserve \jub in its entirety. Thus, in my treatment of \jub, I will be relying primarily on the Ethiopic text, supplemented by the Hebrew and other versions when available.

% \subsection{Content and Character}
The book of \jub offers a rewriting of the book of Genesis and the first part of Exodus (Gen 1--Exod 12).\autocite[1:17]{vanderkam2018} The bulk of the book (2:1--50:13) is dedicated to the recounting these ``biblical'' events in the form of a revelation given to Moses by \yahweh with special concern for halakhic matters and the division of time according to ``weeks'' of years (7-year units) and ``Jubilees'' (49-year units). The particulars of the revelation are mediated by the ``\ap'' (1:27; Eth. \ethiopic{መልአከ~፡ ገጽ} [\translit{mal'aka gaṣṣ}]) who dictates the content of the ``heavenly tablets'' (4:5; Eth. \ethiopic{ጽላተ~፡ ሰማይ} [\translit{ṣəllāta samāy}]) to Moses to record what they revealed about the structure and terminus of the cosmos.\autocite{martinez_najman-tigchelaar2012} The treatment of Moses as a scribe places him within a chain of tradition---along with Enoch and Noah---which emphasizes writing and written works as essential sources of tradition and revelation.%
        \footnote{See especially
                \cite[381--388]{najman_jsj1999}.} 

The main body of \jub is framed by a brief prologue and an even briefer epilogue. The prologue offers a short description of the work as an account concerned with the division of time into units of years, weeks, and jubilees given to Moses when he ascended Mt. Sinai to receive the ``stone tablets'':

% !TEX root = dissertation.tex
\begin{ethiopictext}
        \versenum{Prologue}
        ዝንቱ ፡ ነገረ ፡ ኩፋሌ ፡
        መዋዕላተ ፡ ሕግ ፡ ወለስምዕ ፡
        ለግብረ ፡ ዓመታት ፡ ለተሳብዖቶሙ ፡ 
        ለኢዮቤልውሳቲሆሙ ፡ ውስተ ፡ ኲሉ ፡ ዓመታተ ፡ ዓለም ፡
        በከመ ፡ ተናገሮ ፡ ለሙሴ ፡ በደብረ ፡ ሲና ፡
        አመ ፡ ዐርገ ፡ ይንሣእ ፡ ጽላተ ፡ እብን ፡ ሕግ ፡ ወትእዛዝ ፡ 
        በቃለ ፡ አግዚአብሔር ፡ በከመ ፡ ይቤሎ ፡ ይዕርግ ውስተ ፡ ርእሰ ፡ ደብር ።
\end{ethiopictext}

\begin{transliteration}
        \versenum{Prologue}
        zəntu nagara kufālē
        % kufālē                division
        mawāʕəlāta [la-]ḥegg wa-la-səmʕ
        % mawāʕel           'period, era, time' √mʕl 'to pass the day' Les. 603
        % səmʿ               testimony
        la-gəbra ʕāmatāt la-tasābəʕotomu
        % tasābeʿot             tGL perf from √sbʕ; not in the dictionary, but √sbʿ is seven, so… weeks
        la-ʔiyyobēləwəsātihomu wəsta \kw{ə}llu ʕāmatāta ʕālam
        % ˀiyyobēlwelātihomu    ʾiyyobēl is Jubilee, the rest (-welāt) some extended plural?
        ba-kama tanāgaro la-Musē ba-dabra Sinā
        % ba-kama               Just as
        % tanāgaro              Glt perf 3ms + 3ms
        ʔama ʕarga yenšāʔ ṣəllāta ʔəbn---ḥəgg wa-təʔzāz---%
        % ʕarga                 √ʕrg G pf 3ms 'go up'
        % yenšāʔ                √nšʔ G subj 3ms 'raise, accept, receive*' 
        % ṣellē                 pl. ṣellāt    'tablet'
        ba-qāla ʔagziʔabḥēr ba-kama yəbēlo yəʕrəg wəsta rəʔsa dabr.
        % yebēlo                G perf + 3ms
        % yeˤreg                G subj 
\end{transliteration}

\begin{translation}
        \versenum{Prologue}
        These are the words%
        \footnote{Lit. ``This is the word.'' I've chosen to follow VanderKam and others by rendering this construction in the plural based on the probable underlying Hebrew \he{אלה הדברים}. See \cite[125]{vanderkam2018}}
        of the division 
        of the days for the law and for the testimony
        for the event[s] of the years; for their weeks,
        for their Jubilees in all the years of the world
        just as he spoke (them) to Moses on Mount Sinai 
        when he went up to receive the tablets of stone---the law and the commandment---%
        at the command of God, as he had said to him 
        that he should ascend to the top of the mountain.
\end{translation}%
\noindent
The work closes with a terse statement declaring ``Here the account of the division of time is ended'' (\jub 50:13; Eth. 
    \eth{ተፈጸመ~፡ በዝየ~፡ ነገር~፡ ዘኩፋሌ~፡ መዋዕል~።}
        {tafaṣṣama ba-zeyya nagar za-kufālē mawāʕel}).
        % tafaṣṣama     tD fṣm 'to complete'
        % ba-zeyya      here
        % nagar         account, speech, etc.
        % kufālē        division(s)
        % mawāʕel       √mʕl 'to pass the day' here: period, era, time Les. 603
\noindent
It is important to note that this prologue as well as the first chapter of the book of \jub are preserved among the Qumran fragments (specifically \q{4}{216}{}), which represent some of the oldest extant \jub fragments. Thus, this early narrative frame was almost certainly a part of the work in its earliest form and cannot be attributed to a later editor; it is an integral part of the literary shape of the book of \jub. Although superscriptions were often added much later, in this case, we have no reason to doubt that the prologue/superscription and framing narrative of the work were not a part of the most ancient versions.%
        \footnote{See 
                \cite[1:125]{vanderkam2018};
                \cite[25]{vanderkam_metso-etal2010}.}

Thus the work as a whole is presented as a revelation given to Moses by \yahweh, framed by a brief prologue and epilogue which situates the story during Moses's first 40-days atop Mt. Sinai when Moses receives the Tablets of Stone (Exod 24:12).\autocite[1:129]{vanderkam2018}

% !TeX root = ../dissertation.tex

\section{\ga and Biblical Memory}

% Section introduction

Although it is anachronistic to suggest that the ``Bible''  existed during the late \secondtemple period, insofar as the texts and traditions that were later formalized as the ``Bible''---especially those contained in the Pentateuch---were certainly present in a reasonably stable and even privileged state, I think it is a mistake to jettison any discussion of \rwb texts as they relate to the texts that would later become the Hebrew Bible. On the other hand, restricting our discussion to those later biblical texts would likewise not do justice to the wide variety of texts and traditions in existence during the \secondtemple period which undoubtedly influenced \ga. In an effort to strike a middle ground, therefore, I have opted to refer to ``biblical memory,'' by which I simply mean the confluence of stories and traditions which relate to those later formalized in the Hebrew Bible.%
%
\footnote{I would like to emphasize that I am not suggesting that ``biblical memory'' represents a qualitatively unique form of memory, only that the scope of the traditions under consideration relate to texts that later became the Bible, and, in all likelihood, held at least some sort of special privilege within the memory of many \secondtemple Jews.}
%
In this section, therefore, I would like to discuss the ways that the \ga participated in biblical memory.

% What was GA rewriting: Sources for the parts of GS
\subsection{What was the \ga Rewriting?}

Although the \ga is generally touted as one of the more clear-cut examples of the \rwb, it is noteworthy that its relationship to the biblical text is not, in fact, entirely uniform.\footnote{\cite[333]{bernstein_berthelot-etal2010}.}


% First Columns (Lamech Memoir: Summary and relation to 1 Enoch)
\subsubsection{The Lamech Memoir (Cols. 0--5)}

The earliest columns of the \ga (cols. 0--5), which are narrated from the perspective of Lamech (the ``Lamech Memoir'' by my terminology), Noah's father, essentially offer a rewriting of 1 \enoch 106--107.\footnote{\cite[174]{nickelsburg2005}. The birth of Noah seems to have been a matter of some interest; a number of other texts likewise discuss the exceptional qualities of Noah at his birth. See \q{4}{534}{}[\q{4}{BNoah}{a-d}], \q{1}{Noah} as well as \cite{vanderkam_kapera1992}. Note also \cite{stuckenbruck_berthelot-etal2010}.} In this section, Lamech, recounts the birth of Noah and Lamech's fear that his wife, Bitenosh, had conceived Noah by means of the \aram{עירין} "Watchers." Despite Bitenosh's assurances, Lamech petitions his father, Methuselah, to ask \emph{his} father, \enoch, for further assurance, which he ultimately gives. Although this section is fragmentary, its close resemblance to 1 \enoch 106--107 makes the scholarly reconstruction of the missing sections quite plausible. While it may be tempting to suggest that this section of \ga represents a variant edition of 1 \enoch 106--107, rather than a rewriting, the fact that the version of the story preserved in \ga is told in the first-person from the point of view of Lamech, while 1 \enoch 106--107 is told in the third-person, makes this suggestion highly unlikely. Moreover, because both 1 \enoch and \ga were composed in Aramaic, the differences between the two tellings cannot be attributed to translational issues. In other words, although cols. 0--5 deal, nominally, with events in Genesis 5:28--29, for all intents and purposes, the story recounted in these columns is a retelling of events known from the Enochic tradition.%
%
\footnote{It is not clear what the precise relationship between the Enochic traditions and the \ga actually were. Here I have more-or-less assumed the priority of 1 \enoch, but I wish to leave ambiguous whether \ga represents a rewriting of the \emph{text} of 1 \enoch, or wether they simply draw on a common tradition. Thus, I have chosen to refer to the tradition ``known from'' 1 \enoch, rather than 1 \enoch itself. See Stuckenbruck's treatment of these traditions in \cite*{stuckenbruck_berthelot-etal2010}; Nickelsburg's concise but thorough treatment of the similarities and differences in of these texts is also quite helpful. See \cite[173--174]{nickelsburg2005} as well as \cite[122--123]{fitzmyer2004}.} 


% Second Section (Noah Memoir)
\subsubsection{The Noah Memoir (Cols. 5--17)}
The second major section of \ga begins with a superscription identifying What follows as a \aram{[פרשגן] כתב מלי נוח} or ``[A copy of] the Book of the Words of Noah" (5.29) and continues through col 17 (and, likely, onto the beginning of 18).%
%
\footnote{\cite[174--175]{nickelsburg2005}; Regarding the superscription, see \cite{steiner_dsd1995}. On the topic of the existence of a so-called ``book of Noah'' see \cite{dimant_vanderkam-etal2006} and \cite{werman_chazon-etal1999}.}

% Summary
Although this section accounts for the bulk of the scroll, significant portions are missing or unreadable. This ``Noah Memoir'' begins with a description of Noah's righteousness\footnote{vanderkam:righteousness-of-noah} (affirmed even in-utero) and his early family life (5.29--6.9), followed by a vision predicting the flood (6.9--7.9) which comes about due to the evil behavior of the Nephilim. Cols. 7--8 are highly fragmentary, but most likely described the events of the flood, while cols. 9--12 (which are slightly less fragmentary) describe the ark's putting in on Mt. Ararat, God's instructions to  and blessing of Noah (including the prohibition of consuming blood), and Noah's subsequent interest in viticulture. Cols 13--15 recount a dream-vision in which Noah is depicted as a cedar tree with shoots representing his sons, including a fragmentary explanation of the dream. Finally, cols. 16--17 describe the division of the land by Noah to his sons.

% "Sources" and Relationship of NM to Jubilees & 1 Enoch?
As with the Lamech Memoir, the Noah Memoir clearly draws from traditions outside of those preserved in Genesis. This fact was acknowledged even from the scroll's initial publication.\autocite[38]{avigad-yadin1956} Although the flood account in Gen 6:9--9:17 is a longer and more developed story in its own right than is the account of Noah's birth (which the Lamech Memoir takes as its point of departure), characterizing cols 6--17 of \ga as \emph{primarily} a rewriting of the Genesis flood story does not give due consideration to the additional traditions which influenced its composition. The mention of the Watchers (Aram: \aram{עירין}) and the Nephilim in cols. 6--7 especially bear a thematic resemblance to the Book of Watchers in 1 \enoch 6--11.\footnote{\cite[174]{nickelsburg2005}.} and the explicit reference to the ``the [Book] of the Words of Enoch'' in col. 19.25 suggests that the \ga was familiar with 1 \enoch, or at the very least a tradition of enochic writings.%
%
\footnote{It is worth noting, of course, that this reference occurs in the latter Abram section which some have argued originates in a different source than the first two memoirs. See esp. \cite{bernstein_berthelot-etal2010} and \cite{bernstein_as2010}.}

More plain, however, is the Noah Memoir's connection to the book of \jub, which seems to offer a consistent point of contact with this section of the \ga.\autocite[20]{fitzmyer2004} In fact, it was the explicit identification of Lamech's wife Bitenosh which first prompted Trevor's initial identification of the (unopened) scroll with the so-called Book of Lamech.\autocite{trevor_basor1949} Although an exhaustive treatment of the parallels between \jub and \ga is outside the scope of this chapter, it will suffice to note a few of the most significant points of contact between the Noah Memoir and \jub. James VanderKam has recently offered a detailed, yet concise, summary of these similarities and differences, which, while too long to reproduced in full, can be summarized as follows:%
\footnote{See \cite[374--376]{vanderkam_feldman-etal2017}. For additional treatments of this topic, see also \cite{machiela2009} and \cite[305--342]{kugel2012} previously published as \cite{kugel_roitman-etal2011}} 

\begin{enumerate}
    \item Several personal and geographic\footnote{%
        Mahaq Sea (16.9; Jub. 8.22), Tina River (16.15; Jub. 8.12), Mount Lubar (12.13; Jub. 5.28), Erythrean/Red Sea (17.7; Jub. 8.21), and Gadeira (16.11; Jub 8.26).}%
        %
        names which are never mentioned in the Bible show up in both \ga and \jub (including Bitenosh, which is a part of the Lamech Memoir).
    \item Both \jub and \ga utilize ``Jubilees'' as significant chronological unit (\ga to a lesser degree than \jub).
    \item Several shared stories, themes, and phrases such as 1) ``in the days of Jared,'' 2) Enoch remaining accessible after his departure from normal terrestrial life, 3) Noah making atonement for the ``whole earth,'' and 4) stories recounting Noah and his vineyard.
    \item The ``division of the earth,'' while different in several specifics are strikingly similar and offer, perhaps, the most compelling case for a direct, genetic relationship between the two texts.\footnote{See also Machiela's extensive treatment of this section where he argues for the theory that both texts could be drawing from a shared cartographical source in \cite*[105--130]{machiela2009}. See also \cite{alexander_jjs1982}.}
\end{enumerate}

The striking similarities between the Noah Memoir and \jub  (and to a lesser degree, 1 \enoch) over and against the biblical text, again complicates the characterization of \ga as \rwb or strictly exegetical in nature. In other words if \ga drew from \jub (or if they drew from some common source) I think it is fair to scrutinize whether this section of \ga should be considered a rewriting of \emph{Genesis} or of some other set of traditions.\footnote{Of course, if \ga is the earliest (as Avigad and Yadin as well as \vermes supposed), we would simply be asking the same questions about the book of \jub with the same basic implications.}

% Third Section (Abram) Description
\subsubsection{The Abram Memoir (Cols. 19--22)}

The final surviving columns of the scroll, cols. 19--22, represent the longest and most complete sustained narrative preserved in \ga, here referred to as the ``Abram Memoir.'' More so than the previous sections, the Abram Memoir maps very closely onto the events narrated in Genesis. These columns parallel Genesis 12:10--15:14, retelling the stories of Abram and Sarai's sojourn in Egypt (||~Gen 12:10--20), Abram's subsequent conflict with Lot (||~Gen 13:1--18), the Elamite campaign (||~Gen 14:1--24), and the beginning of Abram's vision (||~Gen 15:1--4). \ga's retelling of these stories follows the chronology of Gen 12--15 very closely, but embellishes and augments the narrative throughout. Like the Lamech and Noah Memoirs, this section of the \ga is largely written as a first-person narrative, this time in Abram's voice. The transition between the Noah Memoir and the Abram memoir is missing, so there is no superscription or title for this section, however, the phrase ``I, Abram'' shows up a number of times, making it clear who the narrator is. This fact is complicated, however, by the fact that, although the narrative begins the in the first-person, beginning in 21.23 the narrator transitions to the third person and remains so through the end of the surviving portion of the scroll.\footnote{It is worth pointing out that the final surviving sheet of parchment was not the final sheet of the scroll originally. Avigad and Yadin note that although only four sheets of the work were present, the seem between the fourth and (what would be) the fifth sheets is visible on the edge of the fourth sheet. \cite*[14]{avigad-yadin1956}.} This inconsistency, perhaps more than any other feature of \ga, has complicated its generic classification.

% First Half (Midrash)
The earlier portions of the Abram Memoir strike a balance between fidelity and innovation with regard to the \emph{biblical} text that the other sections lack. For example, the narrative of Abram and Sarai's descent into Egypt is clearly and recognizably built from the story preserved in the Hebrew Bible. The events and chronology of the story map directly onto Gen 12:10--20, but the \ga offers---in addition to the first-person point of view---a number of expansions that seem plainly to be innovative or, as \vermes would put it and example or prototype of ``midrash.''%
%
\footnote{\cite[124]{vermes1961} Notably, the characterization of \ga as \rwb is typically based on an analysis of the Abram Memoir. Although the earlier portions of the scroll were known, \vermes's treatment of \ga only dealt with cols. 19--22. Together with the fact that these are the best-preserved and most complete columns, this fact has, I think, impacted the characterization of \ga as a whole, perhaps unfairly. On the characterization pre-rabbinic texts as ``midrash,'' see \cite[GET PAGE RANGE]{mandel2017}; \cite{mandel_bakhos2006}.}
% 
Numerous small additions and emendations occur throughout the retelling such as making explicit how long Sarai and Abram lived in Egypt prior to Sarai's notice by Pharoah's princes, how long Sarai was with Pharoah, numerous geographical and personal names, etc. A number of these details, as with earlier sections of \ga, are also found in \jub, though, again, the direction of dependence is not clear (if present). More noticeable are the larger expansions present in the \ga such as Abrams portentous dream (19.14--17), the \emph{waṣf} put on the lips of Pharoah's princes about Sarai (20.2--8), Abram's prayer following Sarai's abduction (20.12--16), the details of Pharoah's afflictions(20.16--21), Harkenosh's discussion with Lot (20.21--20.24), and Abram's intervention on Pharoah's behalf (20.24--32).%
%
\footnote{Other changes from later in the memoir include a description of Abram walking the length and width of the land as well as a notable abbreviation of Abram and Lot's conflict in Gen 13:5--12.}

The explanation of these expansions, according to \vermes---which has been adopted by most treatments of \ga---is as a means of ``correcting'' or otherwise supplementing the biblical text in order to engage the reader and to \emph{explain} the biblical text.\autocite[126]{vermes1961} \vermes writes:

\begin{quote}
The author of GA does indeed try, by every means at his disposal, to make the biblical story more attractive, more real, more edifying, and above all more intelligible. Geographic data are inserted to complete biblical lacunae or to identify altered  place names, and various descriptive touches are added to give the story substance\dots To this work of expansion and development Genesis Apocryphon adds another, namely, the reconciliation of unexplained or apparently conflicting statements in the biblical text in order to allay doubt and worry.\autocite[125]{vermes1961}
\end{quote}

% TODO: More Here about the conflation with the Abimelech story?


% Latter Half (Targum)
By contrast, the latter portion of the Abram Memoir, beginning at 21.23 at times borders on a word-for-word translation of Genesis into Aramaic with comparatively few significant changes. This quality provided occasion for a number of (especially early) scholars to compare \ga with the Targums.\footnote{\cite[193]{black1983}. Though, he notably amended his opinion later \cite*{black_black-fohrer1968}.} Although the change from first-person to third-person is, perhaps, the most significant literary shift that occurs in the \ga, other literary features of the Abram Memoir agree against the Lamech and Noah Memoirs in such a way that gives reason to suppose the Abram Memoir makes up a literary unit.\footnote{Specifically, Moshe Bernstein has noted based on the divine names that are use throughout the work that the primary division is between the Lamech/Noah Memoirs and the Abram Memoir; the earlier sections utilizing a specific set of divine titles and the latter section(s) using a different set. See \cite{bernstein_jbl2009}; See also \cite[97]{falk2007}. Regarding the genre(s) and unity of \ga more generally see Bernstein's later work \cite*{bernstein_berthelot-etal2010} and \cite*{bernstein_as2010}.} It is not clear, however, why there seems to be such a dramatic difference in narrative voice beginning in 21.23.

\subsection{Exegesis and Memory}

Thus, modern treatments of the \ga have tended to speak about the work as ``Rewritten Bible'' as a third category somewhere between Targum and Midrash, with a preference to the latter.%
%
\footnote{\cite{evans_revq1988}; \cite[19]{fitzmyer2004}. Esther Eshel has proposed the term ``narrative midrash,'' but I am in agreement with Harrington and Bernstein in eschewing later categories such as ``midrash'' for these pre-rabbinic sources. See \cite[182]{eshel_roitman-etal2011}; Cf. \cite[242]{harrington_kraft-nickelsburg1986}; \cite[327 n. 33; 328--329]{bernstein_berthelot-etal2010}.}

Yet, as I have illustrated, although portions of the \ga relate clearly to the text of Genesis (notably, the Abram Memoir), much of the earlier portions of the scroll only nominally relate to Genesis, and instead show an affinity to the traditions associated with 1 \enoch and \jub. Thus, characterizing the work as a whole as focused primarily on the explanation of Genesis (as \vermes suggests), seems to me to be ill-founded. Indeed, the disjunction between the various parts of \ga have been observed by numerous scholars, even by those who broadly accept the \ga to be a literary unity, but such discussions still seem to focus on generic classification, which, I think is a methodological dead-end for thinking about \ga.\footnote{Notably \cite{bernstein_as2010} and \cite{falk2007}. Cf. \cite{eshel_roitman-etal2011}.}

To illustrate this difficulty, I would like to focus on Moshe Bernstein's treatment of the ``Genre(s)'' of the \ga.\autocite[As argued in][]{bernstein_berthelot-etal2010} Bernstein's basic thesis is to note that the \ga, as a composite work, must be treated as multi-generic, rather than simply as ``rewritten Bible'' or ``parabiblical'' or the like because, as noted above, the \ga does not relate uniformly to the biblical text. The difficulty, for Bernstein, comes when one must decide how to characterize the work as a whole. While works such as \jub and Pseudo-Philo could be viewed as works that have been uniformly ``rewritten'' (that is, that the entirety of the work is a single rewriting), works such as \ga (he also includes the \templescroll) could be viewed as ``a series of mini-rewritings of limited scope.''%
%
\footnote{\cite[336]{bernstein_berthelot-etal2010}. I am reminded here of Nickelsburg's similar sentiment regarding the ways that 1 \enoch rewrites the flood story several times, arguing that the phenomenon of rewriting moved from smaller units of rewriting to larger, more systematic rewritings. See \cite[TODO: Get Pages]{nickelsburg_stone1984}.}
%
According to such a characterization, Bernstein writes, ``we have no choice but to refer to Part I [the Lamech and Noah Memoirs] as `parabiblical' and Part II [the Abram Memoir] as `rewritten Bible''' based on the fact that, while the Abram Memoir rewrites portions of Genesis, the Lamech and Noah Memoirs really only take Genesis as a point of departure for their stories (and may, in fact, be rewriting other texts).\autocite[337]{bernstein_berthelot-etal2010} To refer to the entirety of \ga as \rwb or as two different kinds of \rwb is, according to Bernstein, unacceptably imprecise. While I am happy to accept a multigeneric characterization of \ga (and any number of other texts), I think Bernstein has sidestepped a more fundamental question by suggesting that the relationship between the \ga and its sources is best addressed as an issue of genre. The assumption made by Bernstein is that there was a qualitative difference between the sources utilized by \ga\footnote{While I am sympathetic to viewing \ga as secondary to \jub and 1 \enoch, here, I am simply stating this as Bernstein's position.} which forms the basis of his characterization of \ga as ``multigeneric.'' This pluriformity is in tension with his larger assertion affirming the unity of the work. 

However, it seems to me that the situation may be better analysed in reverse, namely that the genre of \ga is consistent and it its the assumed qualitative distinction between its sources that should be interrogated. After all, formally speaking \ga is composed of three (broadly) first-person accounts told from the perspective of three significant patriarchs. In other words, rather than characterizing \ga as a work that utilized both ``biblical'' and ``non-biblical'' sources, it is just as reasonable to begin with the assumption that \ga's method is consistent and that the use of ``non-biblical'' sources actually points to the possibility that \jub and 1 \enoch were just as legitimate of sources as Genesis. One possible inference from this observation could be that these other works may have been on equal footing as Genesis and enjoyed some special ``scriptural'' (or otherwise authoritative) position for the author of \ga or that such categories were not operative at this time.\footnote{SOMETHING, SOMETHING Eva Mrozeck.} To be clear, the terminology of ``\rwb'' is not what is at stake here, but rather the way that we imagine the relationship(s) between the \ga and the traditions that surround it.

Although the scholarly consensus since the initial publication of \ga has been that 1 \enoch, Jubilees, and \ga all participate in overlapping or adjacent traditions,\footnote{\cite[38]{avigad-yadin1956}; \cite[20--22]{fitzmyer2004}; \cite[110--116]{crawford2008}; \cite[8--19]{machiela2009}.} what remains unclear is the nature and directionality (if any) of these relationships. While Avigad and Yadin suspected that \ga was a source for 1 \enoch and \jub,\autocite[38]{avigad-yadin1956} it is now widely acknowledged that no definitive evidence has yet been assembled to argue one way or another.\footnote{At the risk of over-simplifying the issue, Fitzmyer, Kugel, VanderKam, and Nickelsburg tend to see \ga as secondary to \jub and \ga, while Machiela and Segal have argued the reverse. See \cite{vanderkam_feldman-etal2017}, \cite[]{fitzmyer2004}, \cite[174]{nickelsburg2005}, \cite[305--342]{kugel2012}. Cf. \cite{segal_as2010}, \cite[140--142]{machiela2009}.} Thinking about \ga in terms of cultural memory means thinking about its composition not simply in source-critical terms, but rather as the synthesis of traditions which, regardless of whether they were considered religiously ``authoritative,'' were operative within the \emph{cultural discourse} of late \secondtemple Judaism. In other words, viewing \ga as the product of cultural memory means taking seriously the idea that the combination of traditions in \ga should not primarily be understood as the genius of an author/editor, but rather that the author/editor should be viewed as the instrument by which cultural memory was codified as text. Of course, we must allow for singular, creative contributions of the author/editor of \ga, but even those original contributions should not be treated as if they arose out of a vacuum. In some sense, then, it does not matter which \emph{text} came first. What is clear is that the cultural memory that surrounded the book of Genesis---the biblical memory of Genesis---was more broad than  the text of Genesis and included traditions that we know from \jub and 1 \enoch whether or not they were directly informed by the \emph{texts} of \jub and 1 \enoch.


%% What are the points of contact with the Bible and other contemporary sources? (Jubilees, Enoch, etc.)
%% What is GA's relationship *to* the Bible
    %% Conflating Pharoah and Abimelek is "interpretation" collapsing to a single story
    %% Cedar and Date Palm? Dafuck is going on here?

% (re)frame this example as 'memory.'
    %% A synthesis of traditions about Abe and Sarai
    %% Likely some innovative stuff
    %% But it's not beholden to the tradition. That is, it is participating in the creation of tradition and that tradition may be able to be traced as continuing into the Gen. Rab. with the Cedar and DatePalm stuff (?)
% !TeX root = ../dissertation.tex

\section{Abram in the Diaspora: The Literary Frameworks of \GA}

Having dealt with the \ga as the product of cultural memory in terms of its relationship to its inherited biblical memory (including the traditions which were ancillary to Genesis proper), we may now turn our attention to the ways that the \ga addressed its audience at the level of \emph{social} memory. In this section I will address the way that the \ga \emph{speaks to} its audience and the ways that the \ga changes and adapts its cultural memory into a meaningful piece of literature for \secondtemple Judaism.

As I have already noted, the narrative of the \ga is not simply a straight-forward retelling of Genesis from the perspectives of Lamech, Noah, and Abram, but participates more broadly in the ``biblical memory'' of Genesis. However, what is most compelling about \rwb texts very often is the ways that they adapt biblical memory. These adaptations can come at the level of story---by adding, removing, or rearranging events---or at the level of narrative discourse by describing events differently or with different emphases. In the case of \ga, and in particular in the account of Abram in cols. 19--22, the biblical narrative has been recast as a (first-person) Hellenistic novella in a similar vein to other well-known Second Temple Jewish works such as the narrative portions of Daniel (including the Greek additions), Esther, Tobit, and (arguably) the so-called Joseph novella of Genesis 37 and 39--50.\footnote{See especially Lawrence Wills work on the Jewish novels and novellas in antiquity: \cite*{wills2002} as well as his important earlier works \cite*{wills1995} and \cite{wills1990}.}

The reading of \ga 19--20 as a Hellenistic Jewish novella has recently been very thoroughly explicated by Blake Jurgens, who has further argued that the utilization of Hellenistic literary motifs and structures in \ga altered the overall purpose the the pericope for the purpose of edifying Jews living in the Hellenistic world in the shadow of empire.\autocite{jurgens_jsj2018} Although much of Jurgen's paper is based on long-established observations about the literary influences on \ga, he makes the important discursive turn toward the audience by claiming that the \ga was meant to be useful to readers:

\begin{quote}
By imbuing its story with literary tropes and techniques similar to those found in Dan 1--6, Esther, and other Jewish texts arising out of the Hellenistic period, the author successfully attends to the narratival ambiguities of Gen 12:10--20 through interpretive expansion upon the latent exegetical links of the text while concurrently modifying the narrative to appeal to contemporary literary expectations.\autocite[27]{jurgens_jsj2018} \end{quote}

The process of this transcription, which he terms ``fictionalization,'' is described by Jurgens in six distinct narrative units within cols. 19--20 of \ga which describe Abram and Sarai's sojourn in Egypt: the decent into Egypt, Abram's dream vision, the banquet scene, praise of Sarai's beauty, Abram's prayer, and the final court contest. In each section Jurgens notes the ways that the \ga utilizes literary structures common to its broader Hellenistic mileu to rewrite the the events of this story. Jurgens offers a very thorough description of the ways that the \ga utilizes these literary structures and makes a plausible claim that these changes were meant to engage readers in familiar style. 

Thinking in terms of social memory, however, we can appreciate the way that the story of Abram and Sarai in Egypt is ``remembered into'' the social context of Hellenistic Judaism and is fitted into the contemporary social frameworks (read: literary conventions) therein. In other words, the changes which Jurgens identifies as intended to engage with readers can also be framed as changes which social conventions---by which we mean literary conventions---have affected the way that the author/editor of \ga presents the story, its plot, and characters.

\subsection{Abram in the Diaspora}

One of the primary features of Jewish hellenistic novellas is their setting. Jurgens notes that, typically, these Jewish novellas are set in the diaspora, which invariably place the Jewish (or, in Tobit and Judith's case, Israelite) protagonist under the hegemony of a foreign power. In the case of \ga, although not properly ``diaspora,'' Abram is a sojourner in a foreign land and is under foreign hegemony. Moreover, from a modern perspective, these stories have a tendency to commit rather egregious factual errors about certain historical particulars such as the names of rulers (Judith 1:1; Dan 4; Tobit) and geographic items (Tobit 5:6). Likewise, \ga seems to utilize details which almost certainly were inventions of the author (or an earlier tradant) such as referring to ``Pharaoh Zoan'' (we know of no such figure) and Herqanos, a name popular in the Ptolemaic period, but not attested otherwise as well as referring to the ``Karmon River'' (probably the Kharma canal), as the one of the seven heads of the Nile river, which it is not.\autocites[7]{jurgens_jsj2018}[See also][50--59]{machiela_as2010}[197--199]{fitzmyer2004} These details, according to Jurgens, are meant to create a sense of verisimilitude and authenticity within the narrative. Thus, although the story of Abram's sojourn in Egypt as narrated in the biblical text engages with discourses of the \emph{foundation} of Israel, the narrative of the \ga seems to be turning the story to engage with the contemporary discourses around the idea of \emph{diaspora}. In other words the way that Abram's sojourn in Egypt was remembered in the \secondtemple period, at least in part, took on new meaning for those sojourning in the diaspora and for those living in the land under foreign hegemony.

\subsection{Abram in the Court of a Foreign King: Literary Genre as Social Framework}

If we place the pericope of Abram's journey into Egypt  in \ga under the rubric of diaspora literature, the account bears a striking resemblance to the so called court contest narratives well-known from (especially) the book of Daniel.\footnote{Other court contest narratives include the Joseph Cycle (Gen 41)} Such narratives, as observed by Collins and others, follow particular narrative progressions with common elements, stock characters, tensions, and resolutions.\footnote{\cite[TODO: pages]{collins1993}; \cite{humphreys_jbl1973}; \cite{collins_jbl1975}; \cite{wills1990}. See also \cite{niditch-doran_jbl1977}.}



\subsection{Abram the Sage}


\subsection{Abram the Oracle}
% !TeX root = ../dissertation.tex

\section{\ga as \psa}

% Intro about third level of discourse
While the \ga can be seen engaging with its received cultural memory through its sources and engaging with its contemporary social memory at the level of literary form and genre, the \ga also participates in the construction of cultural memory going forward. Although \emph{all} literary and cultural products can participate in constructing cultural memory, in this section, I will argue that \ga's \psgraphic form participates in this constructive act differently than other forms of literature, in particular the biblical text.\footnote{I continue to reiterate that although the term ``biblical'' is anachronistic for the late \secondtemple period, it is a usefully concise term for my purposes.}

\subsection{The Hebrew Bible as a Baseline}

The vast majority of the Hebrew Bible is narrated in the third-person omniscient and is formally anonymous. There are, of course, exceptions to this generalization, most notably within the prophetic corpus (such as Isa 6--8), the so-called Nehemiah Memoir (Neh 11--13), and perhaps works such as Deuteronomy and Song of Songs. But for the lion's share of the biblical text, the the author (and narrator) operates invisibly.

The rhetorical force of this particular authorial voice, as observed by Erhard Blum, is significant for the function of the Hebrew Bible's participation in the collective memory of the communities that claim it as their own. Although the implied author does occasionally engage directly with the reader by offering explanatory observations (for example where the author inserts phrases like ``this is why\ldots{}'' or ``\ldots{}until this day''), for all intents and purposes, the author presents as both \emph{reliable} and \emph{authoritative} without a hint of subjectivity. As Blum puts it, ``In this sense the narrative does not distinguish the depiction from the depicted.''\autocite[33]{blum_barton-etal2007} Put another way, the text does not acknowledge that it \emph{has} an author, it simply \emph{is}. The rhetorical effect of this invisible, omniscient author is to collapse the knowledge gap between the reader and the events narrated by removing the author from view. This move, according to Blum, allows the text to convey ``an unmediated truth claim which is not based on the author's distinguishable critical judgments and convictions.''\autocite[33]{blum_barton-etal2007} The effectiveness of this implied author, according to Blum, is tied to the pragmatics of the text, that is, tied to the context of the biblical narratives as scripture (though, Blum does not refer to ``scripture'' \emph{per se}). The implied audience of the biblical narratives by-and-large can be understood as group-insiders for whom the biblical text worked to reinforce group identity.

Of course, the ``unmediated truth claims'' of the biblical text \emph{were}, in fact, mediated and reinforced by those who (orally or otherwise) transmitted the tradition from one generation to another.\autocite[33]{blum_barton-etal2007} Individuals within the community---teachers and religious leaders and even parents---become the voice of the biblical text as it is passed on. In other words, one might say that the narrator of the biblical text is the community itself---its collective memory. Blum writes:

\begin{quote}
If we assume that the traditional literature was primarily transmitted through oral means, than the narrator who is speaking supplies the material with a personal presence; he is not present as an author who judges and evaluates his sources from a critical distance, but as a `transmitter' who participates in the tradition itself and is able to lend it credence through his own personality, his standing, and/or his office.\cite[33]{blum_barton-etal2007}
\end{quote}

In other words the authoritative claims of ``biblical'' texts are actually made by their communities and not by the text itself. Thus the way biblical texts participate in the collective memory is determined by their \emph{use}---how their \emph{readers} frame their function and how the text relates to the collective memory. 

\subsection{On \Psy and the \Psa}

Because significant portions of the \ga are written in the first person as though written by Lamech, Noah, and Abram, \ga may be formally included in the literary category of \psy. Before moving on, however, it is worth taking a moment to clearly define what is meant by ``\psy,'' ``\psa,'' and related terms.\autocites[The topic of \psy has received a large amount of very sophisticated attention in recent years. See especially][]{mroczek2016}{tigchelaar_tigchelaar2014}{reed_towsend-moulie2011}{reed_jts2009}{reed_ditomasso-turcescu2008}{najman_hilhorst-puech2007}{najman2003} In the simplest terms, \psa are texts which are fictively purported to be written by figures (typically) from the ancient past. 

The ancient use of the term \psa denoted spurious texts which Church leaders believed to be intentionally misleading about their authorship.\footnote{See esp.~Hist. Eccl. 6.12.2 where the Bishop of Antioch, Serapion, refers to the \emph{Gospel of Peter} among the a number of works ``falsely attributed'': \gk{γάρ, ἀδελφοί, καὶ Πέτρον καὶ τοὺς ἄλλους ἀποστόλους ἀποδεχόμεθα ὡς Χριστόν, τὰ δὲ ὀνόματι αὐτῶν   ψευδεπίγραφα ὡς ἔμπειροι παραιτούμεθα, γινώσκοντες ὅτι τὰ τοιαῦτα οὐ   παρελάβομεν}. ``For we, brothers, accept both Peter and the other apostles as Christ, but we skillfully reject those falsely ascribed writings, knowing that they were not handed down to us.''} Thus, the term has tended to carry a somewhat negative connotation, even when such a connotation is not warranted. Implicit in the negative use of the term is the assumtion that ``false'' attribution was malicious, or at the very least intentionally misleading. Yet, the number of (esp.~Jewish) \psgraphical texts discovered within the past century provide good reason to question the assumption that pseudonymous authors's intentions were to deceive their readers.\autocites[53--58]{mroczek2016}[See also][]{reed_jts2009} On the contrary, the sheer number of \psgraphical works now known to us suggests that the historical reality and social function of \psgraphical works was not simply a matter of being ``falsely attributed.''

At the other end of the spectrum, because so many early Jewish texts seem to fall into the category of \psa, in some scholarly discourse, the term ``\psa'' has become generalized to encompass any text written in around the turn of the era which did not make it into the canon of rabbinic Judaism or early Christianity. Bernstein observes, for example, that although the first volume of James Charlesworth's two-volume \emph{Old Testament Pseudepigrapha} contains a number of formally \psgraphic works, the second volume includes many which do not meet the formal definition of \psa.\autocites[2]{bernstein_chazon-etal1999}{charlesworth_OTP} This expansive practice is not particularly helpful for clarifying the term and so I will attempt to restrict my useage to a more clearly defined set of criteria.

Moshe Bernstein, in his discussion of the phenomenon of \psy distinguishes between ``authoritative'' \psy and ``decorative'' \psy.\footnote{He also identifies a third form, ``convenient'' \psy which is located somewhere between the two. \cite[3--7]{bernstein_chazon-etal1999}.} By ``authoritative'' \psy, Bernstein refers to texts that \emph{portray themselves} as being written by a particular figure. Portions of 1 Enoch (in particular the latter three books, Astronomical Writings [72--82], Dream Visions [83--90], and the Epistle of Enoch [91--107]), which present themselves as if they were written by Enoch himself, are prime examples of ``authoritative'' \psy. Psalm 23, on the other hand, although attributed to David, was presumably not \emph{actually} written by David. Moreover, whoever did write Ps 23, (again, presumably) did not intend to write it \emph{as if} it had been written by David. Rather, the Psalm was simply \emph{attributed} to David, along with many others, in part due to the tradition od David being a musician.\footnote{CITATION NEEDED} Thus, Ps 23 could be classified as ``decorative'' \psy. Thus the difference between ``authoritative'' and ``decorative'' \psy can, in some sense, be boiled down to the notoriously difficult issue of authorial intent---whether a text was \emph{intended} to be read as \psa or whether the work was anonymous, and later attributed to an explicit author.

Less clear-cut examples, however, require a more nuanced treatment. For example, Deuteronomy is not generally referred to as among the \psa, yet, from a literary perspective, it is framed as \he{הדברִם אשׁר דבר משׁה אל־כל־ישׂראל} ``the words which Moses spoke to all Israel'' (Deut 1:1a). Although the whole narrative is not written in the first person, long sections of the book are treated as verbatim recountings of Moses' speech. Was Moses the author of Deuteronomy? Traditionally, most critical scholars have dated Deuteronomy to the late monarchic period and thus have eschewed the traditional attribution. But whether Deuteronomy was \emph{written} as \psa or just attributed to Moses after the fact is difficult to say with certainly and the matter is further complicated by the editorial processes that the book likely underwent through the centuries.\autocite[143--172]{toorn2007} What we \emph{can} say is that there are concrete literary cues within Deuteronomy which make the attribution to Moses easier. Framing Deuteronomy as ``the words which Moses spoke,'' while not formally ``\psa'' participates in the construction of memory in a similar fashion as \psa proper.

\subsection{Pseudepigrapha, \ga, and Memory Construction}

If we take seriously Blum's characterization of the way that the anonymous, third-person omniscient biblical text may have engaged with the collective memory of Israel based on formal, narratological features within the text, it stands to reason that the \ga as first-person \psy would engage that collective memory in a different way, despite the fact that the stories within the \ga are found in the book of Genesis. In other words the literary form of the \ga affects how it relates \emph{back} to the biblical memory, and how it can be used in the further \emph{construction of} that memory.

The \psgraphic quality of \ga shapes the way that the text engages with the remembered past by describing the biblical story through the mouths of important figures.%
%
\footnote{Here ``story'' refers to the abstract sequence of actions which the narrative describes. The \emph{way} a story is recounted, on the other hand, is referred to by narratologists as \emph{narrative discourse.} Thus the \ga's change from third-person omniscient to a \psgraphical first-person narrative can be understood as a change in \emph{narrative discourse} which, broadly, retains the same \emph{story} as that of the biblical text. See \cite[13--27, esp. 18--19]{abbott2008}.}
%
This explicitness changes the way that the reader understands how the text fits into the collective memory by shifting the locus of authenticity onto the text's putative author and away from the mediating figures within the community. In other words, as an example of \psy, the \ga can be thought of as a set of fictional \emph{primary sources} that bypass the received tradition. As these sources are used and enter into the discourse of the broader biblical memory they are able to function not simply as ``alternate'' versions of events but as qualitatively distinct contributions to the tradition as it is passed on to the next generation.\footnote{On analogy to Hindy Najman's notion of ``Mosaic Discourse,'' here I am saying that the \ga is participating in a broader ``biblical'' discourse insofar as it participates in discourses surrounding Lamech, Noah, and Abram. See \cite[GET PAGE]{najman2003}.}

Of course, referring to \psy as ``fictional primary sources'' may overstate these texts' importance or otherwise misunderstand how ``authentic'' these texts were thought to be by various and sundry religious groups in antiquity. On the one hand, it could be that readers understood that such novel fictional adaptations took certain artistic license with their biblicial \emph{Vorlagen}. By way of analogy, modern adaptations of biblical narratives into film are expected to deviate to a certain degree from their source material, despite the fact that the Hebrew Bible remains a sacred, authoritative text for many modern Jews and Christians. Such adaptations are not, typically, understood to be superceding the Bible because viewers understand intuitively that there is a qualitative difference between their scriptures and a movie. On the other hand, there certainly are examples of \psgraphical texts which ultimately \emph{did} become authoritative for certain religious groups.%
%
\footnote{For example, the Ethiopian Orthodox Church includes 1 Enoch among its scriptures. Tobit, too may, under certain rubrics, be considered \psa, which is included within the Roman Catholic and Eastern Orthodox deuterocanon. Insofar as deutero- and trito- Isaiah were penned as is written by Isaiah, they too could be considered \psa. And, of course, a number of the so called ``disputed'' pauline letters within the Christian New Testament likely were not penned by Paul and are properly \psgraphical.}
%
My point here is not to suggest that there were multiple ways to understand \psgraphical writing in antiquity so much as to point out that discussions of ``false'' or ``authentic'' attribution are generally from later periods and do not tell us anything meaningful about \emph{why} such a text was written or \emph{how} it would have been understood by its original readers.

The \ga, of course, was never considered ``scripture'' so far as we know, but that does not mean that it did not participate in the broader biblical memory, even if only in the popular imagination. But even at the level of the popular imagination---even as an entertaining fiction---the \ga participated in how its society conceived of the Genesis narratives. Regardless of whether the memoirs in \ga were thought to be ``authentic,'' they represent both an interpretive understanding of biblical memory and an original contribution to that memory.
=======
 This chapter examines the \ga through \ldots{} and it will do so by discussing the \ga at three distinct levels of discourse: structurally at the macro level as an example of \psy, in terms of keying and framing at the level of narrative discourse, and finally from the perspective of biblical discourse.

 \section{\GA, First Person Narrative, and \Psy}

 One of the most striking features of the \ga when compared to other \rwb texts is its pervasive use of the first-person voice to narrate the events of its story. This quality sets \ga apart from the majority of narrative material in the Hebrew Bible which, with notable exceptions, usually maintains an omniscient third-person voice. The \ga's use of first person must be nuanced, however, by the fact that the work presents itself as a collection of first person ``memoirs''\footnote{I will use the term ``memoir'' throughout this chapter as a way of referring to the distinct (mostly) first-person narratives found in the \ga. This is simply a convenience term that highlights the formal characteristic of being written in the first person voice without any reference to the authenticity of the work and in alignment with the convention of referring to first-person narratives in the Bible as ``memoirs'' (e.g., the ``Nehemiah Memoir'' or the ``Isaiah Memoir'').} from three of the Patriarchs from Genesis (Lamech, Noah, and Abram). So, while each memoir does indeed utilize the first person, the narrator itself changes throughout the course of the text.
 
Speaking of the \ga as a single text, however, should not be taken for granted. Therefore, it is necessary for the moment to consider whether the \ga should be considered ``a'' text or whether instead it should be treated as a collection of ``texts.''  Although the composite character of the \ga was noted in the \emph{editio princeps} by Avigad and Yadin, they maintained that \ga functioned as a single literary unit though allowing that it was made up of several literary sources.\autocite[38]{avigad_yadin1956} Since then, however, the unity of the work has been further interrogated and analyzed from a number of perspectives and the unity of \ga is more tenuous than ever.

Perhaps the most compelling argument for the genuinely composite nature of \ga has been offered by Moshe Bernstein, who has pointed out that discrete units of the \ga utilize distinct titles and epithets for \yahweh.\footnote{\cite{bernstein_jbl2009}. It is delightfully reminiscent of the classical formulations of the Documentary Hypothesis. Attributed to Graf and Wellhausen.  See also~\cite{bernstein_as2010} and~\cite{weigold_as2010}} He notes, for example, that...

% TODO: Give details of Bernstein's argument, 
% TODO: Caveat Bernsteins arguement; note Bernstein's own "other" approach

From a structural standpoint, it is not at all clear whether these three ``memoirs'' are meaningfully related and any thematic consistency is readily explained through the collating process itself.\footnote{In other words, we could easily suppose that the reason that three ``texts'' such as these would be grouped together on a single scroll was that they shared certain common themes or formal characteristics.} One could certainly imagine the \ga as a collection of fictional patriarchal memoirs collected onto a single scroll, and organized roughly by each text's correspondence to the chronology of the biblical narrative. 

In fact, this is precisely how the \ga presents itself. Although the beginnings of the Lamech and Abram accounts are lost due to damage to the scroll, col. 5.29 seems to offer a superscription for the account of Noah, reconstructed as [\emph{pršgn}] \emph{ktb mly nwḥ} ``[a copy of] the book of the words of Noah.'' Based on this superscription, it is reasonable to suppose that the accounts of Lamech and Abram, too, had such superscriptions, although at least two complicating factors should be taken into account. First, while the extant portions of \ga present themselves as distinct units, both the very beginning and the very end of the scroll have been lost. From a structural standpoint, this fact should elicit caution because it is at the beginning and end of texts where such features as ``framing narratives'' and other explanatory material is often located. Without definitive evidence of the presence or absence of such features one must be extra cautious when making observations about the rhetorical purpose of the macro structure of a particular text. Second, although all three accounts are \emph{generally} written in the first-person, as noted above, none of them are rigorously committed to maintaining the voice. As Loren Stuckenbruck notes, each of the three ``documents,'' at one point or another, falls into some kind of third-person voice: Lamech in 5.24--25, Noah in 16.14--17.19, and Abram in 21.23--22.34. Curiously, Stuckenbruck includes the superscription(s) as examples of this inconsistency and does not distinguish between instances where the narrator moves into the third person \emph{within} the narrative and cases where one might suppose the presence of an editorial voice.

The problem of whether to understand the superscription(s) as ``internal'' to the work is a good example of how the macro-structure of the work is important for this kind of analysis. By treating the superscription as a contribution of the ``author'' of a unified \ga, Stuckenbruck understands the superscription to be ``in the third person'' and would (apparently) treat each first-person account as an embedded narrative within a larger framing narrative (of which the superscription would be a part). For example, if the beginning of the scroll gave a brief framing narrative, describing a young man who discovered three scrolls in the desert and thus proceeded to provide ``a copy of the book of the words of X,'' Stuckenbruck would be absolutely correct. However, if one understands the superscription to be an editorial insertion, it does not make sense to include it as an example of third-person discourse for the same reasons it does not make sense to say that Ps 23 uses third-person discourse by beginning with \emph{mizmôr lə-dāwid}.\footnote{\cite[315--316]{stuckenbruck_roitman-etal2011}. See also \cite[15--16]{bernstein_chazon-etal1999}. Even supposing a single author for \ga, as Stuckenbruck and others imply, I am still inclined to consider the superscriptions separately from the former examples because they would exist outside the frame of each embedded narrative.}
In other words, the way that the \ga presents itself, I believe, should best be understood as a \emph{collection} of memoirs compiled by an editor who would have, putatively, supplied a set of paratextual superscriptions.

The fact that \ga presents itself as a collection of disparate texts, however, does not mean that we must treat it as such. Historical and literary disjunction aside, the fact that the \ga was preserved as a ``collection'' tells us something about how its early readers may have understood the relationship between its (putative) constituant parts. Although early readers may not have understood \ga to be a unity, placing each ``part'' of the \ga into a collection queues the reader to engage each part with the same (or similar) reading strategies. 
% TODO: Get something on canonization/collection as it relates to genre?
That means that regardless of the original intent of each of \ga's constituant parts---and therefore whether or not one considers it to be a literary unity---we can still fruitfully consider \ga as a whole with respect to how it may have functioned for its readers and, therefore, as an object of social memory.

 
 \subsection{Pseudepigraphy and the Implied Author}

 Regardless of \ga was composed by a single person or several, the fact that much of the text was written in the first person voice of early biblical figures formally places \ga into the literary category of \psy. Because the terms \psy and (perhaps moreso) \psa are commonly used  in the field of \secondtemple studies to speak generally about ``non-canonical'' and ``fictional'' works I would like to take a moment and make explicit I mean by these and their related terms.\autocites[The topic of \psy has received a large amount of very sophisticated attention in recent years. See especially][]{mroczek2016}{tigchelaar_tigchelaar2014}{reed_towsend-moulie2011}{reed_jts2009}{reed_ditomasso-turcescu2008}{najman_hilhorst-puech2007}{najman2003}

 In the simplest terms, \psa are texts which fictively purport to be written by figures (typically) from the ancient past. For our purposes, I would like to further distinguish between texts which \emph{portray themselves} and texts which were latter \emph{attributed to} ancient figures. Bernstein helpfully distinguishes between these two phenomena by labeling the former ``authoritative'' \psy and the latter ``decorative'' \psy.\footnote{He also identifies a third form, ``convenient'' \psy which is located somewhere between the two. I do not find this category as helpful. \autocite[3--7]{bernstein_chazon-etal1999}.} While the two phenomena are no doubt related, it is the act of writing in the name of another figure which interests me. Thus, Ps 23, again, although attributed to David, I assume was not \emph{actually} written by him, nor was it written \emph{as if} it had been written by him. Thus, Ps 23 should probably be considered ``decorative'' \psy. Major portions of 1 Enoch, on the other hand (in particular the latter three books, Astronomical Writings [72--82], Dream Visions [83--90], and the Epistle of Enoch [91--107]), were \emph{written as though} they were written by Enoch himself, and therefore fall into the category of ``authoritative'' \psy. Less clear-cut examples, however, require a more nuanced definition. For example, Deuteronomy is not generally referred to as among the \psa (see below), yet, from a literary perspective, it is framed as \emph{had-dəbārîm ʾăšer dibber mōšeh ʾel-kol-yiśrāʾēl} ``the words which Moses spoke to all Israel'' (Deut 1:1a). Was Moses the author? Many Jews and Christians from antiquity up to (and for some, including) the modern era, of course, believed so. But whether Deuteronomy was \emph{written} as \psa or just attributed to Moses is difficult to say with certainly. What we \emph{can} say is that there are concrete literary cues within Deuteronomy which suggest Mosaic authorship more strongly than, say Genesis, which was also attributed to Moses in antiquity.

 The ancient use of the term \psa denoted spurious texts which Church leaders believed to be intentionally misleading about their authorship.\footnote{See esp.~Hist. Eccl. 6.12.2 where the Bishop of   Antioch, Serapion, refers to the * Gospel of Peter* among the a number   of works ``falsely attributed'': γάρ, ἀδελφοί, καὶ Πέτρον καὶ τοὺς   ἄλλους ἀποστόλους ἀποδεχόμεθα ὡς Χριστόν, τὰ δὲ ὀνόματι αὐτῶν   ψευδεπίγραφα ὡς ἔμπειροι παραιτούμεθα, γινώσκοντες ὅτι τὰ τοιαῦτα οὐ   παρελάβομεν. ``For we, brothers, accept both Peter and the other   apostles as Christ, but we skillfully reject those falsely ascribed   writings, knowing that they were not handed down to us.''} The number of (esp.~Jewish) \psgraphical texts discovered within the past century provide good reason to question the assumption that pseudonymous authors's intentions were to deceive their readers.\autocites[53--58]{mroczek2016}[See also][]{reed_jts2009} Thus, I wish to eschew the value judgments of this ancient usage. At the other end of the spectrum, in some scholarly discourse, the term ``\psa'' has become generalized to encompass any text written in around the turn of the era which did not make it into the canon of rabbinic Judaism or early Christianity. Bernstein observes, for example, that although the first volume of James Charlesworth's two-volume \emph{Old Testament Pseudepigrapha} contains formally \psgraphic works, the second volume includes many which do not meet the formal definition of \psa.\autocites[2]{bernstein_chazon-etal1999}{charlesworth_OTP} This expansive practice, likewise, is not particularly helpful for clarifying the term and so I will attempt to restrict my useage to a more clearly defined set of criteria.

 Thus I will use the terms \psy and \psa to refer to texts (or practices) which seem to actively push the reader to construct an implied author whose identity would have been well-known and meaningful to its reader and who (typically) would have lived in the distant past.

 The term ``implied author'' also deserves a clear definition; I have adopted that of H. Porter Abbott:

 \begin{quote} Neither the real \emph{author} nor the \emph{narrator}, the implied author is the idea of the author constructed by the reader as she or he reads the \emph{narrative}. In an \emph{intentional reading}, the implied author is that sensibility and moral intelligence that the reader gradually constructs to infer the intended meanings and effects of the narrative. The implied author might as easily (and with greater justice) be called the ``inferred author'' \autocite[235]{abbott2008} \end{quote}

 Because the implied author is a construction of the reader, it is frequently not desirable to talk about this construct as a literary feature so much as a heuristic for intentional reading. Part of the advantage of basing the definition of \psy on the idea of the implied author is that it mitigates any prejudice toward the intention of the author to deceive (maliciously or otherwise) his reader. Thus, the question of ``who'' the implied author \emph{is} generally misses the point. In the case of \psy, however, the central formal characteristic of the work seems to be the \emph{intentional construction of a known implied author}. Therefore, one could make the case that what is characteristic about \psgraphic texts is the intentionality on the part of the real author to shape the processes by which their readers's construct an implied author. Where Abbott notes that the term ``inferred author'' may be a better term than ``implied author'' (the connotation shifts the active role to the reader) when discussing \psy, ``implied'' still fits quite well (since we can attribute some intentionality to the real author). The implied author, therefore, provides an important point of contact between the reader and the (real) author.

Taking the above ideas together, and with respect to the \ga, we can see that the homogenization of the (putative) sources of the \ga brought about by their presence in a ``collection'' (whether truly composite or whether as a literary device of a single author), allows for readers to traverse \ga with what one might consider to be a single implied author.\footnote{This assertion relies on Abbott's description of the Implied author as a ``sensibility and moral intelligence that the reader gradually constructs to infer the intended meanings and effects of the narrative'' (235). Insofar as this is the case, I think it's fair to say that the entirty of \ga can be read with the same ``sensability and moral intellegence,'' regardless of who the narrator is portrayed to be at any given moment.\autocite[235]{abbott2008}.} What we have not addressed, however, is how these literary features mayhave functioned with respect to social memory.

 \subsection{Scripture, \Psy and Memory Construction}

 The vast majority of the Hebrew Bible is narrated in the third-person omniscient and is formally anonymous. There are, of course, exceptions to this generalization, most notably within the prophetic corpus (such as Isa 6--8), the so-called Nehemiah Memoir (Neh 11--13), and perhaps works such as Deuteronomy and Song of Songs. But for the lion's share of the biblical text, the the implied author seems to operate invisibly.

 The rhetorical force of this particular authorial voice, as observed by Erhard Blum, is significant for the function of the Hebrew Bible's participation in the collective memory of the communities that claim it as their own. Although the implied author does occasionally engage directly with the reader by offering explanatory observations (for example where the author inserts phrases like ``this is why\ldots{}'' or ``\ldots{}until this day''), for all intents and purposes, the implied author presents as both \emph{reliable} and \emph{authoritative} without a hint of subjectivity. As Blum puts it, ``In this sense the narrative does not distinguish the depiction from the depicted.''\autocite[33]{blum_barton-etal2007} Put another way, the text does not acknowledge that it \emph{has} an author, it simply \emph{is}. The rhetorical effect of this invisible, omniscient author is to collapse the knowledge gap between the reader and the events narrated by removing the author from view. This move, according to Blum, allows the text to convey ``an unmediated truth claim which is not based on the author's distinguishable critical judgments and convictions.''\autocite[33]{blum_barton-etal2007} The effectiveness of this implied author, according to Blum, is tied to the pragmatics of the text, that is, tied to the context of the biblical narratives as scripture (though, Blum does not refer to ``scripture'' \emph{per se}). The implied audience of the biblical narratives by-and-large can be understood as group-insiders for whom the biblical text worked to reinforce group identity. Such ``unmediated truth claims'' \emph{were} in fact mediated and reinforced by those who (orally or otherwise) transmitted the tradition from one generation to another.\autocite[33]{blum_barton-etal2007} In other words, one might say that the implied author of the biblical text is the community's collective memory.\autocite[Blum writes, ``If we assume that the traditional literature was primarily transmitted through oral means, than the narrator who is speaking supplies the material with a personal presence; he is not present as an author who judges and evaluates his sources from a critical distance, but as a `transmitter' who participates in the tradition itself and is able to lend it credence through his own personality, his standing, and/or his office.''][33]{blum_barton-etal2007}

 In contrast to the omniscient implied biblical author, the \ga frames itself as a collection of first-person accounts which formally fall into the category of \psy. If we take seriously Blum's characterization of the way that the biblical text may have engaged with the collective memory of Israel based on the formal, narratological features of the text, it stands to reason that the \ga as first-person \psy would engage that collective memory in a different way despite the relative similarity of the textual content. The \psgraphic quality of \ga shapes the way that the text engages with the remembered past by describing the biblical story through the mouthes of important figures. Here ``story'' refers to the abstract sequence of actions which the narrative describes. The \emph{way} a story is recounted, on the other hand, is referred to by narratologists as \emph{narrative discourse.} Thus the \ga's change from third-person omniscient to a \psgraphical first-person narrative can be understood as a change in \emph{narrative discourse} which, broadly, retains the same \emph{story} as that of the biblical text.

 Approaching these questions from the perspective of Social Memory Studies asks us to think about the way that differing social frameworks and cognitive contexts may have allowed for or demanded presenting this material in a different form than its \emph{Vorlage}.

 One of the difficulties in dealing with \psy is the apparently divergent ways that the authors and original readers may have understood \psy as compared to the way that later groups (e.g., Church Fathers, modern scholars) treat it. The crux of the issue, it seems to me, is less to do with whether the author intended to ``deceive'' his audience, and more to do with whether the readers understood themselves to be reading something ``authentic'' or were willing participants in an authorial fiction. Yet, even language of ``authenticity'' or ``fiction'' presupposes that such terms were a meaningful part of the discourse surrounding ``scripture'' during the late \secondtemple period.

 \section{Abram in the Diaspora: Keying and Framing in \GA}
 While retelling portions of Genesis as first-person narrative reorients the way that the story engages with the received tradition and collective memory at the macro-level the narrative of the \ga is not simply a straight-forward retelling of Genesis from the perspectives of Lamech, Noah, and Abram. Indeed, what is most compelling about \rwb texts very often is the ways that they depart from the biblical narrative. These departures can come at the level of story by adding, removing, or rearranging events or at the level of narrative discourse by describing events differently or with different emphases. In the case of \ga, and in particular in the account of Abram in cols. 19--22, the biblical narrative has been recast as a (first-person) Hellenistic novella in a similar vein to other well-known Second Temple Jewish works such as the narrative portions of Daniel (including the Greek additions), Esther, Tobit, and (arguably) the so-called Joseph novella of Genesis 37 and 39--50.\autocites*[See especially Lawrence Wills work on the Jewish novels and novellas in antiquity:][]{wills2002}[as well as his important earlier works:][]{wills1995}[and][]{wills1990}

 The reading of \ga 19--20 as a Hellenistic Jewish novella has recently been very thoroughly explicated by Blake Jurgens, who has further argued that the utilization of Hellenistic literary motifs and structures in \ga altered the overall purpose the the pericope for the purpose of edifying Jews living in the Hellenistic world in the shadow of empire.\autocite{jurgens_jsj2018} Although much of Jurgen's paper is based on long-established observations about the literary influences on \ga, he makes the important discursive turn toward the audience by claiming that the \ga was meant to be useful to readers:

 \begin{quote} By imbuing its story with literary tropes and techniques similar to those found in Dan 1--6, Esther, and other Jewish texts arising out of the Hellenistic period, the author successfully attends to the narratival ambiguities of Gen 12:10--20 through interpretive expansion upon the latent exegetical links of the text while concurrently modifying the narrative to appeal to contemporary literary expectations.\autocite[27]{jurgens_jsj2018} \end{quote}

 The process of this transcription, which he terms ``fictionalization,'' is described by Jurgens in the six distinct narrative units within cols. 19-20 of \ga which describe Abram and Sarai's sojourn in Egypt: the decent into Egypt, Abram's dream vision, the banquet Scene, praise of Sarai's beauty, Abram's prayer, and the final court contest. In each section Jurgens notes the ways that the \ga utilizes literary structures common to it broader Hellenistic mileu to rewrite the the events of this story. Jurgens offers a very thorough description of the ways that the \ga utilizes these literary structures and makes a plausible claim that these changes were meant to engage readers in familiar style. Thinking in terms of social memory, however, we can appreciate the way that the story of Abram and Sarai in Egypt is ``remembered into'' the social context of Hellenistic Judaism and is fitted into the contemporary social frameworks (read: literary conventions) therein.

 While \halbwachs addressed the fact that memories are shaped by the present, recently Barry Schwartz has attempted to more clearly articulate this process. One of the most important contributions of Schwartz's work in this area is his conviction that the interactions between the remembered past and the present are not unidirectional. Where \halbwachs limits his discussion to the describing the ways that memories of the past are shaped by the present, Schwartz sees the past as a potent force in the present as well. In other words, not only does the present influence the way the past is remembered, but the past \emph{itself} (that is, both the remembered and ``actual'' past) also effects the present.

 Schwartz employs two terms, ``keying'' and ``framing,'' to describe this additional dimension to the way that the past impacts the present. On the one hand the idea of ``keying'' can be understood as way of


 \subsection{Setting}

 One of the primary features of these novellas is their setting. Jurgens notes that, typically, these Jewish novellas are set in the diaspora, which invariably place the Jewish (or, in Tobit and Judith's case, Israelite) protagonist under the hegemony of a foreign power. In the case of \ga, although not properly ``diaspora,'' Abram is a sojourner in a foreign land and is under foreign hegemony. Moreover, from a modern perspective, these stories have a tendency to commit rather egregious factual errors about certain historical particulars such as the names of rulers (Judith 1:1; Dan 4; Tobit) and geographic items (Tobit 5:6). Likewise, \ga seems to utilize details which almost certainly were inventions of the author (or an earlier tradant) such as referring to ``Pharaoh Zoan'' (we know of no such figure) and Herqanos, a name popular in the Ptolemaic period, but not attested otherwise as well as referring to the ``Karmon River'' (probably the Kharma canal), as the one of the seven heads of the Nile river, which it is not.\autocites[7]{jurgens_jsj2018}[See also][50--59]{machiela_as2010}[197--199]{fitzmyer2004} These details, according to Jurgens, are meant to create a sense of verisimilitude and authenticity within the narrative. Thus, although the story of Abram's sojourn in Egypt as narrated in the biblical text engages with discourses of the \emph{foundation} of Israel, the narrative of the \ga seems to be turning the story to engage with the contemporary discourses around the idea of \emph{diaspora}.


 \subsection{Abram in the Court of a Foreign King}
 \subsection{Abram the Sage}
 \subsection{Abram the Oracle}
 \section{Inner-biblical Interpretation}
>>>>>>> school
