\hypertarget{memory-and-the-rwb}{%
\chapter{Memory and the Rewritten Bible}\label{memory-and-the-rwb}}

The past several decades has seen a dramatic increase in interest toward
the topic of ``memory'' throughout the social sciences.

As Olick and Robbins note, although the topic of ``memory'' has been of
interest philosophers and thinkers since the ancient Greeks, the modern
social-scientific approaches which concern this chapter almost
exclusively trace their genealogy to the early 20th century work of
sociologist Maurice Halbwachs.\autocite[106]{olick-robbins_ars1998}
Although Halbwachs's work was not limited to exploring the topic of
social memory (he also made contributions to statictics and probability
theory, as well as sociological work on the topic of suicide and social
morphology\autocite[13--20]{coser_halbwachs1992}), his work in this area
has managed to endure and continues to find new applications and to
inspire new approaches in ways that his other contributions have not.

Halbwachs's seminal work on the topic of memory, published in 1925 under
the title \emph{Les cadres sociaxu de la mémoire} (hereafter refered to
in translation as ``The Social Frameworks of Memory'') was excerpted and
translated into English in 1992 by Lewis Coser along with the final
chapter of his short book on the remembrance of the Holy Land (entitled
\emph{la topographie légendaire des évangiles en terre sinte: Etude de
mémoire collective}, hereafter referred to in translation as ``The
Legendary Topography of the Holy Land'') in a single volume under the
title \emph{On Collective Memory}.\autocite{halbwachs1992} These two
works, in some sense, form the core of Halbwachs's work. A thrid volume,
however, entitled \emph{La mémoire collective} (hereafter, \emph{The
Collective Memory}) was published posthumously in 1950 and translated
into English in 1980 by Mary Douglass. This work simultaneously
represents some of Halbwachs's most nuanced ideas (while responding to
critics) and evinces an incompleteness which posthumous publications
often suffer.\autocite[Coser observes, ``One may doubt that the author
himself would have been willing to publish it in what seems to be an
unfinished state. The book nevertheless contains many further
developments of Halbwachs's thought in regard to such matters as the
relation of space and time to collective memory as well as fruitful
definitions and applications of the differences between individual,
collective, and historical memory.''][2]{coser_halbwachs1992}

The central contribution of Halbwachs's work was the notion that human
memory is intrinsically and inextricably tied to its social context.
Humans are social beings and the recollection of memeories can only be
completed within the context of a society. Halbwachs, of course, locates
what we might call the biologocal processes of memory within the
individual, but he, argues, memories divorced from society yeild an
incomplete picture.{[}halbwachs1992, 38{]} To illustrate this point,
Halbwachs begins \emph{The Social Frameworks of Memory} with a
discussion of dreams wherein he argues that the dream state approximates
human isolation from society in a particularly potent way. He writes:

\begin{quote}
It is at this moment {[}in the dream state{]} that he is no longer
capable---nor has need---of relying on frames of collective memory. It
is then possible to measure the operation of these frameworks by
observing what becomes of individual memory when this operation is no
longer present.\autocite[39]{halbwachs1992}
\end{quote}

Contra Freud \emph{et, al.}, Halbwachs argues that it is not in
\emph{memory} that the human mind is most isolated, but rather in the
dream state. Memory, on the other hand \emph{cannot} be understood in
isolation from its social framework. This is why, according to
Halbwachs, ``{[}no{]} real and complete memory ever appears in our
dreams as it appears in our waking state,''\autocite[41]{halbwachs1992}
Halbwach is able to argue that it is the social frameworks of memory
which anchor human memory.
