\hypertarget{memory-and-the-rwb}{%
\chapter{Memory and the Rewritten Bible}\label{memory-and-the-rwb}}

The past several decades has seen a dramatic increase in interest toward
the topic of ``memory'' throughout the social sciences.

As Olick and Robbins note, although the topic of ``memory'' has been of
interest philosophers and thinkers since the ancient Greeks, the modern
social-scientific approaches which concern this chapter almost
exclusively trace their genealogy to the early 20th century the work of
sociologist Maurice Halbwachs.\autocite[106]{olick-robbins_ars1998}
Although Halbwachs' work was not limited to exploring the topic of
social memory (he also made contributions to statictics and probability
theory, as well as sociological work on the topic of suicide and social
morphology\autocite[13--20]{coser_halbwachs1992}), his work in this area
has managed to endure and continues to find new applications and to
inspire new approaches in ways that his other contributions have not.

Halbwachs' seminal work on the topic of memory, published in 1925 under
the title \emph{Les cadres sociaxu de la mémoire} (hereafter refered to
in translation as ``The Social Frameworks of Memory'') was excerpted and
translated into English in 1992 by Lewis Coser along with the final
chapter of his short book on the remembrance of the Holy Land (entitled
\emph{la topographie légendaire des évangiles en terre sinte: Etude de
mémoire collective}, hereafter referred to in translation as ``The
Legendary Topography of the Holy Land'') in a single volume under the
title \emph{On Collective Memory}.\autocite{halbwachs1992} These two
works, in some sense, form the core of Halbwachs' work. A thrid volume,
however, entitled \emph{La mémoire collective} (hereafter, \emph{The
Collective Memory}) was published posthumously in 1950 and translated
into English in 1980 by Mary Douglass. This work simultaneously
represents some of Halbwachs' most nuanced ideas (while responding to
critics) and evinces an incompleteness which posthumous publications
often suffer.\autocite[\^{} Coser observes, ``One may doubt that the
author himself would have been willing to publish it in what seems to be
an unfinished state. The book nevertheless contains many further
developments of Halbwachs's thought in regard to such matters as the
relation of space and time to collective memory as well as fruitful
definitions and applications of the differences between individual,
collective, and historical memory.''][2]{coser_halbwachs1992}
