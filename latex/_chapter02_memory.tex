\hypertarget{social-memory-studies-and-the-rwb}{%
\chapter{Social Memory Studies and the
Rewritten Bible}\label{social-memory-studies-and-the-rwb}}

The past several decades, beginning in the late 1970's, a dramatic
increase in scholarly interest toward the topic of ``memory'' has swept
throughout the social sciences. The precipitous rise in scholarly
literature dealing with topics of memory coupled with its proliferation
in popular media discourses has prompted some in the field to refer to
the ``memory industry'' and to describe the ubiquity of memory
discourses as a ``boom'' fast-approaching a
bust.\autocites{rosenfeld_jmh2009}{winter2006}{berliner_aq2005} Yet, as
Olick et al.~make very clear in their Introduction to \emph{The
Collective Memory Reader}, there remain a significant number of scholars
throughout the social sciences who continue to find ``memory'' to a
useful heuristic and a compelling theoretical basis for their various
and sundry analytical applications.\autocite[3--6]{olick_olick-etal2011}

Although the topic of ``memory'' has been of interest philosophers and
thinkers since the ancient Greeks, as Olick and Robbins note, the modern
social-scientific approaches which concern this chapter almost
exclusively trace their genealogy to the early 20th century work of
sociologist Maurice Halbwachs.\autocites[106]{olick-robbins_ars1998}[It
should be noted, however, that Halbwachs was not the first or only
person to do work on memory or the impact of society thereon, despite
most recent literature's preoccupation with him,
see][8--36]{olick_olick-etal2011} Although Halbwachs's work was not
limited to exploring the topic of social memory (he also made
contributions to statictics and probability theory, as well as
sociological work on the topic of suicide and social
morphology\autocite[13--20]{coser_halbwachs1992}), his work in this area
has managed to endure and continues to find new applications and to
inspire new approaches in ways that his other contributions have not.

\hypertarget{the-work-of-maurice-halbwachs-a-very-brief-overview}{%
\section{The Work of Maurice Halbwachs: A Very Brief
Overview}\label{the-work-of-maurice-halbwachs-a-very-brief-overview}}

Halbwachs published three primary works on the topic of memory, the
first of which appeared in 1925 under the title \emph{The Social
Frameworks of Memory} (\emph{Les cadres sociaxu de la mémoire}). This
monograph, along with a short essay dealing with the remembered
geography the Holy Land, entitled \emph{The Legendary Topography of the
Holy Land} (\emph{la topographie légendaire des évangiles en terre
sinte: Etude de mémoire collective}) was excerpted and translated by
Lewis Coser in a single volume under the title \emph{On Collective
Memory} in 1992.\autocite[Several of the most important chapters of
\emph{The Social Frameowrks of Memory} were included in full. Likewise,
the entirety of the conclusion of \emph{The Legendary Topography} was
included][]{halbwachs1992} His final contribution, entitled \emph{The
Collective Memory} (\emph{La mémoire collective}), was published
posthumously in 1950 and translated into English in 1980 with an
editorial introduction by Mary Douglas.\autocite{halbwachs1980} This
work simultaneously represents some of Halbwachs's most developed ideas
(responding to critics such as Charles Blondel) and evinces an
incompleteness which posthumous publications often suffer. As Coser
observes, ``One may doubt that the author himself would have been
willing to publish it in what seems to be an unfinished state. The book
nevertheless contains many further developments of Halbwachs's thought
in regard to such matters as the relation of space and time to
collective memory as well as fruitful definitions and applications of
the differences between individual, collective, and historical
memory.''\autocite{coser_halbwachs1992}

The central contribution of Halbwachs's work was the notion that human
memory is intrinsically and inextricably tied to social frameworks.
Humans are social beings and as such human processes, such as memory,
can only be completed within the context of a society. This focus on the
\emph{social} dimensions of memory betrays the deep influence that Émile
Durkheim's work had on Halbwachs. Unlike Durkheim, however, Halbwachs's
approach was tempered by his desire to identify the physical (or,
perhaps ``biological'') location of memory within the individual.
Although the term ``collective memory'' evokes an ethereal or
metaphysical idea, Halbwachs's use of the term is entirely concrete.
Individual memories are kept by \emph{individuals}. However, an
individual's ability to retrieve and utilize a particular memory is
intrinsically tied to the individual's social context. Memories require
social frameworks to function.\autocite[38]{halbwachs1992}

To illustrate this point, Halbwachs begins \emph{The Social Frameworks
of Memory} by attempting to prove the negative. \emph{Without} a social
framework, he argues, memories are always incomplete. Because
humans---for all intents and purposes---always exist within a society,
it is the dream state, he avers, that most closely approximates the
complete isolation of memory from society. Therefor the way that the
human brain deals with memories while dreaming can illustrate the
(dis)function of memories lacking a social framework. Thus, he observes
that ``dreams are composed of fragments of memory too mutilated and
mixed up with others to allow us to recognize them.'' Because in dream
states the mind lacks the ability to ``check'' itself against anything
external to itself, dreams do not contain ``true
memories.''\autocite[41]{halbwachs1992} This assertion is set against
the ``purely individual psychology'' (read: Bergson, et al.), which
viewed \emph{memory} as a location of social isolation. Regarding the
incompleteness of the dream state, he writes:

\begin{quote}
Almost completely detached from the system of social representations,
{[}the dream state's{]} images are nothing more than raw materials,
capable of entering into all sorts of combinations. They establish only
random relations among each other---relations based on the disordered
play of corporal modification.\autocite[42]{halbwachs1992}
\end{quote}

The ``system of social representations'' that Halbwachs refers to is not
limited, however, to macro structures such as familial, religious, or
class groups. Although these structures certainly \emph{do} make up an
important stratum of social frameworks, Halbwachs envisions something
much more fundamental which betrays his broadly structuralist
perspective. Halbwachs uses the phrase ``social representations'' to
refer to a system of shared ``signs'' that encompassed not only
language, but every aspect of a group's social framework---a sort of
``cultural \emph{langue}.'' Although, Halbwachs does not use the
language of semiotics, the analogy is helpful. Just as Saussurian
linguistics argues that the concrete arbitrary sign is given meaning
only by participating in the broader, shared \emph{langue}, so too
memories (read: ``signs'') require a framework to convey meaning, as do
the concrete, individual expressions of remembrance (read:
\emph{parole}). Halbwachs writes:

\begin{quote}
It is in this sense that there exists a collective memory and social
frameworks for memory; it is to the degree that our individual thought
places itself in these frameworks and participates in this memory that
it is capable of the act of recollection.\autocite[PAGE]{halbwachs1992}.
\end{quote}

Memories, therefore, cannot be understood in isolation from their social
framework and therefore should not be analyzed without consideration to
the social context of the rememberer.

Of course, people participate in a plurality of social groups, often
simultaneously, and the experiences that are later to be recalled, too,
must be situated within these frameworks. In order to bring these
autobiographical memories to mind, according to Halbwachs, an individual
attempts to mentally situate herself within those same frameworks. For
instance, I find it much easier to recall whether a particular course I
have taken occurred in the Fall or Spring semester, rather than which
month or even year it occurred. The social framework that is the
``academic year'' remains a potent framework for my own memories, while
my wife---who had the good sense to stop after one degree---stopped
remembering in terms of ``semesters'' years ago. Yet, when remembering
events during her time at the University, the semester once again
becomes a useful framework for memory. It is for this reason that recent
memories are more easy to call to mind: because the social frameworks
that produced the memory (the people, places, customs, etc.) remain in
close proximity for the rememberer and the effort required to situate
the memory within the social frameworks that produced it is
minimal.\autocite[52]{halbwachs1992}

This overview is representative of his central thesis and provides a
point of departure for his more in-depth studies of collective memory in
the family, religion, and social classes. Thus, by ``social'' memory,
Halbwachs refers to individual memory and how it is recalled within
externally defined social frameworks, i.e.~how society provides the
framework that makes individual memory
possible.\autocite[180]{hubenthal_carstens-hasselbalch2012} On the other
hand, ``collective'' memory refers to shared memory, ``the shared
cultural past to which individuals contribute and upon which they call;
but ultimately a past that transcends individual
memory.''\autocites[360]{keith_ec2015}[See
also][180]{hubenthal_carstens-hasselbalch2012} The two ideas work
together and mutually influence one another. As Hübenthal puts it, ``The
difference {[}between social and collective memory{]} lies in the
perspective: \emph{social memory} is using the framework,
\emph{collective memory} is establishing
it.''\autocite[180.]{hubenthal_carstens-hasselbalch2012} Hübenthal's use
of the active verb ``establish'' is intentional: for Halbwachs,
collective memory is not a passive social accretion, but an actively
constructed part of the group's common identity which \emph{speaks to
the to concerns and needs of the community in the present}. Social
frameworks shape the way that people remember. The retrieval of memories
is shaped by those same frameworks, and as those frameworks shift, so
too do the memories that are recalled in those societies.

\hypertarget{memory-and-history}{%
\section{Memory and History}\label{memory-and-history}}

In his later work, Halbwachs distinguished between two kinds of memory
which were distinguished by the experiential-relation of the individual
to the memory: autobiographical and historical
memory.\autocite[52]{halbwachs1980} Autobiographical memory refers to
the sort of memories which are the result of individual, subjective
experience, while historical memory refers to memories which fall
outside the experience of the individual. Elsewhere Halbwachs refers to
these as ``internal'' and ``external'' memory. Autobiographical memory
is rooted in the individual, sensory, experiences of the individual and
provides a full, ``thick'' memory (to borrow from Geertz) while
historical memory, necessarily, offers only a thin, schematic overview
and by definition is never ``experienced'' by the rememberer.

Although Halbwachs distinguished between these two forms of memory, he
nevertheless emphasized their interrelatedness. In particular, Halbwachs
notes that autobiographical memory necessarily is dependent upon
historical memory, insofar as our lives participate in ``general
history.''\autocite[52]{halbwachs1980} For example, memories of a more
indirect nature are able to shape autobiographical memory by shaping the
social frameworks which produced them and the frameworks into which they
are recalled. The quintessential example of this sort of indirect
influence for Americans of roughly my age would be the events of
September 11, 2001. Although comparatively few people directly witnessed
the events (I was asleep on the West Coast when the first plane
crashed), the impact that those events had (and continue to have) on the
orientation of American national memory is unquestionably a part of many
people's lived experience, including my own and would therefore
constitute a part of America's current ``collective memory.'' Although
the incoming undergraduate class at the University of Texas at Austin,
many of whom will have been born after 2001, have \emph{no}
autobiographical memory of these events, it is, nonetheless, a part of
the collective memory of which they are a part. On the other hand the
War of 1812 is not a part of any living person's autobiographical memory
and its impact on the collective memory of most Americans is likely
restricted to a few popular media references,{[}\^{}Such as Jimmy
Driftwood's \emph{The Battle of New Orleans}, best known as performed by
Johnny Horton which topped \emph{Billboard} charts in the US, Canada,
and Australia in 1959 and was recently acknowledged to be on of the Top
100 Western songs of all time. See,
https://en.wikipedia.org/wiki/The\_Battle\_of\_New\_Orleans.{]} or
localized to specific geographical regions with a close connection to
major events in the conflict (e.g., New Orleans).

The memories of historical events, however, are shaped by the social
frameworks of the rememberer. The events of September 11, 2001 in the
memory of most Americans are now shaped by the social discourses
surrounding the United States's continued military presence in the
region and its controversial pretexts for engagement in the region, in
particular with the Iraq war. Likewise, although no living person has an
autobiographical memory of the American Civil War, the construction of
certain confederate monuments on the campus of the University of Texas
at Austin during the Jim Crow era, and their subsequent removal in
August of 2017, illustrates how historical memory can be (consciously,
in this case) reshaped and restructured as the remembering society
changes.

It is the way that these remembered events change over time that makes
social memory studies so interesting for the historian. Halbwachs's own
work in the area of history is best seen in his work on \emph{The
Legendary Topography of the Holy Land}, where he focuses on the ways
that memories relate to particular geographic sites. In particular,
Halbwachs's study focuses on the way that the sites in and around the
Galilee and Jerusalem have been imbued with significance based on their
putative connection with some significant event relating to Jesus, the
Apostles, and early Christian communities. Halbwachs makes a number
observations about the way that memories are formed and the ways that
they interact which will become important for the current study.

Halbwachs's first observation comes in contrasting the portrayal of
Jesus within the Gospels with what must have been the lived experience
of the Apostles.\autocite[193--198]{halbwachs1992} The involvement of
the apostles in the day-to-day life of Jesus in some sense would have
prohibited them from achieving the kind of ``necessary detachment'' to
write something like the the Gospels. In other words (and to use
Halbwachs's later terminology), the memory of Jesus as protrayed in the
Gospels is almost necessarily informed by \emph{historical} rather than
autobiographical memory.\autocite[194]{halbwachs1992} Indeed, Halbwachs
rightly observes that the Gospels present Jesus and his ministry ``as if
Jesus's whole life was but a preparation for his death, as if this was
what he had announced in advance.''\autocite[198]{halbwachs1992} Surely
Jesus's mother remembered the death of her son differently than the way
the Church later commemorated it.\autocites[It is worth noting that
Halbachs himself compares his basic sentiment toward the historicity of
the Gospels as similar to that of Ernst Renan. Of course, the field of
New Testament studies has progressed considerably in content, method,
and nuance beyond Renan, whatever the merits of his work may have been.
While Halbachs seems to take seriously that Jesus did exist, he does not
accept the Gospels at face value as historically accurate. That said,
throughout the work, Halbwachs does talk about the ``actual'' past and
allows for the possibility that the Gospels do refer, at least
partially, to these real events. Which is to say, he is not making the
argument that the Gospels were fabricated of whole-cloth, but that the
``actual past'' is irretrievable and unknowable and that historical
memory has no obligation to align with the actual past as such.
Regardless of whether Halbwachs's conception of Early Christianity would
be considered sound today, the idea that the Gospels represent several
collective remembrances of Jesus's life, ministry and death each bearing
marks from their own \emph{Sitz im Leben} (to borrow from the form
critics) seems relatively uncontroversial. A number of studies on the
Jesus and early Christian memory have come about in the past several
years. See][]{ledonne2009}{rodriguez2010}[For an overview of the modern
impact of Halbwachs (and memory studies more generally) on the field of
Historical Jesus studies, see][]{keith_ec2015}[and][]{keith_ec2015b}

Halbwachs, drawing on the Pauline epistles, observes that the earliest
recollections of Jesus, however, make no mention of the location of his
death (Jerusalem) nor of his ministry (Galilee).

\begin{quote}
In the authentic epistles of Paul, we are told only that the son of God
has come to earth, that he died for our sins, and that he was brought
back to life again. There is no allusion to the circumstances of his
life, except for the Lord's Supper, which, Paul says, appeared to him in
a vision (and not through witnesses). There is no indication of
locality, no question of Galilee, or of the preachings of Jesus on the
shores of the lake of Gennesaret.\autocite[209]{halbwachs1992}
\end{quote}

Halbwachs's point is that within the narrative of the Gospels, the
location of Jesus's death---by virtue of the social and political
reality of the day---\emph{had} to occur in
Jerusalem.\autocite[211]{halbwachs1992} Whether or not it actually did,
or whether or not that information was explicitly handed down to the
authors of the Gospels is irrelevant for the purposes of collective
memory. Sacred places become sacred through the process of memory
\emph{construction}, not simply through the transmission of
autobiographical experience. According to Halbwachs, they are spaces
where significant ideas within the collective memory of a group can take
concrete form. He writes, ``Sacred places thus commemorate not facts
certified by contemporary witnesses but rather beliefs born perhaps not
far from these places and strengthened by taking root in this
environment.''\autocite[199]{halbwachs1992} Thus, localizing historical
memory functions as a way to move abstract ideas into the real world and
thus reinforce fundamental components of the group's collective memory.

Perhaps more interesting is Halbwachs's treatment of the ability for
memories to coalesce and split over time. Halbwachs makes the
observation that, according to tradition (i.e., the collective memory of
the Church), certain places in the Holy Land mark the location of
\emph{several} significant events. From an historical perspective
Halbwachs, obviously, doubts that assertions are accurate---even
assuming the events may have taken place one place or another---but,
finds the clustering of these events to be more than just coincidence.
For example, he writes:

\begin{quote}
One is surprised to find on the shores of the lake Gennesaret, near the
Seven Fountains, the place where apostles were chosen, the Sermon on the
Mount, the appearance of Jesus on the waters after the
Resurrection---all in the same place.\autocite[220]{halbwachs1992}
\end{quote}

Halbwachs's assumption is that there was something about the location
\emph{itself}, some ``earlier
consecration,''\autocite[220]{halbwachs1992} which attracted these
memories to particular locales. Extending this rationale further, we can
appreciate the fact that for Christianity, the significance of Jerusalem
is not limited to the significance of the city as the location of
Jesus's death, but rather by the prior significance of the city for
Judaism. Within the collective memory of Christian tradition, one might
say that Jerusalem is not significant because it is the location of the
Passion and resurrection of Jesus, but that the Passion and resurrection
of Jesus happened in Jerusalem because Jerusalem was significant.
Halbwachs writes:

\begin{quote}
The Christian collective memory could annex a part of Jewish collective
memory only by appropriating part of the latter's local remembrance
while at the same time transforming its entire perspective of historical
space.\autocite[215]{halbwachs1992}
\end{quote}

One might object to this suggestion by noting that, supposing Jesus
\emph{actually was} crucified in Jerusalem, one hardly needs to
re-appropriate Jewish tradition or attribute this remembrance to some
special process. Yet, it is worth pointing out in cases where the
historical data are lacking (or, perhaps, where eyewitness accounts did
not exist), this same basic phenomenon occurred. For example, Halbwachs
points to the birth narratives of the Gospels, in particular that of
Matthew, where Jesus is described as being born in Bethlehem, the ``city
of David.'' Although there is no reason to think that Jesus was
\emph{actually} born in Bethlehem, Halbwachs, rightly observes, ``the
authors of the gospels seem entirely to have invented this poetic
history which has occupied a considerable place in Christian
History.''\autocite[214]{halbwachs1992} In fact, Jesus's entire
portrayal in the Gospel of Matthew is an exercise in collective
remembrance which is structured on the foundational narratives of the
Hebrew Bible (e.g., the slaughter of innocents, and Jesus's portrayal as
a lawgiver ``on the mount''), and framed as the fulfillment of Jewish
prophecy.

The inverse of this phenomenon is also observable. According to
Halbwachs while some events converge to particular locations, other
events diverge among several sites. One expression of this process is by
dividing the \emph{event} and localizing each part of the event to a
particular place or object. For example, Halbwachs notes how the memory
of specific important events, such as the Passion, may be split and
localized at a very fine level of detail:

\begin{quote}
Around Golgotha and the Holy Sepulchre, for example, we find the rock of
anointing, the rock of the angel, the rock of the gardener, the place
where Jesus was stripped, etc.\autocite[220]{halbwachs1992}
\end{quote}

The proliferation of these micro-sites of memory, according to
Halbwachs, aide and reinforce the collective memory through repetition.
Furthermore, the added detail serves in ``renewing and rejuvenating an
ancient image.''\autocite[220]{halbwachs1992}

The same event may also be localized in multiple places. Halbwachs
describes several traditional locations of the Cenacle (the ``Upper
Room'' from the Gospels), including the Mount of Olives, Gethsemane, and
the Grotto of Jesus's teaching. These traditions coexisted into the
fourth century, yet, later, the site was moved to the Christian hill of
Zion. Likewise, Halbachs notes there were two locations for Emmaus and
two different mountains on which Jesus appeared in Galilee after his
resurrection. While it runs counter to our modern historiographical
sensibilities, the ability of seemingly contradictory traditions to
coexist is well documented.

Halbwachs points out that autobiographical memory does not allow for
this kind of fragmentation. We all realize that the same event from our
own past can not have happened in two locations simultaneously. Yet,
Halbwachs points out that should that same person belong to two groups
who disagree on a particular remembered event (one that the individual
did not personally witness), the individual is able to hold the
disagreement without the need assert one or the other. the same is true
of complex social entities, such as religious groups, who are composed
of several subgroups with their own collective memories. Halbwachs
writes:

\begin{quote}
A community must often accommodate itself to contradictions introduced
by diverse groups so long as none of these groups prevails, or so long
as the community itself does not find a new reason for decisively
settling the issue. This is especially true when the community faces a
controversy about its rites, which are an anchor for its component
groups.\autocite[224]{halbwachs1992}
\end{quote}

\hypertarget{modern-theories-of-memory}{%
\subsection{Modern Theories of Memory}\label{modern-theories-of-memory}}

\hypertarget{collective-memory-and-cultural-memory-and-social-memory-a-terminological-assessment}{%
\section{Collective Memory, and Cultural Memory, and Social Memory: A
Terminological
Assessment}\label{collective-memory-and-cultural-memory-and-social-memory-a-terminological-assessment}}

Although Halbwachs used the term ``collective memory'' to talk about the
interaction between memory at the individual, psychological level and
the social frameworks which enable the act of recollection, recent work
in memory studies has seen a proliferation of terminology which has both
clarified and (at times) obfuscated the discussion.\footnote{In fact,
  there is a grand tradition of imprecise and overlapping terminologies
  within memory studies going back to Halbwachs himself. For example, on
  page 40 of \emph{On Collective Memory}, Halbwachs uses the terms
  ``collective memory,'' ``social memory,'' ``social frameworks of
  memory,'' and ``collective frameworks of memory'' and it is not
  entirely clear what how Halbwachs is distinguishing between them. The
  way that he is able to use the terms almost interchangeably has led
  some in the current discussion to treat them as synonyms. As Anthony
  Le Donne observes, ``In fact, {[}``social'' and ``collective''
  memory{]} are currently used synonymously with such frequency that
  their nuances vary from author to author \autocite[42
  n.8]{ledonne2009}. Yet, Le Donne points out, Halbwachs actually uses
  these terms in with slightly different nuances. On the one hand,
  Halbwachs uses the term''sociaux" when he is describing the way social
  structures affect memory, while on the other hand ``Collective
  Memory'' tends to refer to the content of memories which are
  transmitted between individuals \autocite[42 n.8]{ledonne2009}. In
  other words, when Halbawachs uses the term ``social'' he is usually
  talking about \emph{frameworks} of memory, whereas ``collective
  memory'' might be cautiously glossed as ``common'' or ``shared''
  memory. Sandra Hübenthal puts it well, ``The difference {[}between
  social and collective memory{]} lies in the perspective: \emph{social
  memory} is using the framework, \emph{collective memory} is
  establishing it'' \autocite[180]{hubenthal_carstens-hasselbalch2012}.}
