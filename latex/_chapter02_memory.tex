\hypertarget{social-memory-studies-and-the-rwb}{%
\chapter{Social Memory Studies and the
Rewritten Bible}\label{social-memory-studies-and-the-rwb}}

The past several decades, beginning in the late 1970's, a dramatic
increase in scholarly interest toward the topic of ``memory'' has swept
throughout the social sciences. The precipitous rise in scholarly
literature dealing with topics of memory coupled with its proliferation
in popular media discourses has prompted some in the field to refer to
the ``memory industry'' and to describe the ubiquity of memory
discourses as a ``boom'' fast-approaching a
bust.\autocites{rosenfeld_jmh2009}{winter2006}{berliner_aq2005} Yet, as
Olick et al.~make very clear in their Introduction to \emph{The
Collective Memory Reader}, there remain a significant number of scholars
throughout the social sciences who continue to find ``memory'' to a
useful heuristic and a compelling theoretical basis for their various
and sundry analytical applications.\autocite[3--6]{olick_olick-etal2011}

Although the topic of ``memory'' has been of interest philosophers and
thinkers since the ancient Greeks, as Olick and Robbins note, the modern
social-scientific approaches which concern this chapter almost
exclusively trace their genealogy to the early 20th century work of
sociologist Maurice Halbwachs.\autocites[106]{olick-robbins_ars1998}[It
should be noted, however, that Halbwachs was not the first or only
person to do work on memory or the impact of society thereon, despite
most recent literature's preoccupation with him,
see][8--36]{olick_olick-etal2011} Although Halbwachs's work was not
limited to exploring the topic of social memory (he also made
contributions to statictics and probability theory, as well as
sociological work on the topic of suicide and social
morphology\autocite[13--20]{coser_halbwachs1992}), his work in this area
has managed to endure and continues to find new applications and to
inspire new approaches in ways that his other contributions have
not.\footnote{Halbwachs's seminal work on the topic of memory, published
  in 1925 under the title \emph{The Social Frameworks of Memory}
  (\emph{Les cadres sociaxu de la mémoire}) was excerpted and translated
  into English in 1992 by Lewis Coser along with the final chapter of
  his short book on the remembrance of the Holy Land, entitled \emph{The
  Legendary Topography of the Holy Land} (\emph{la topographie
  légendaire des évangiles en terre sinte: Etude de mémoire collective})
  in a single volume under the title \emph{On Collective
  Memory}.\autocite{halbwachs1992} These two works, in some sense, form
  the core of Halbwachs's work. A third volume, however, entitled
  \emph{The Collective Memory} (\emph{La mémoire collective}) was
  published posthumously in 1950 and translated into English in 1980
  with an editorial introduction by Mary
  Douglas.\autocite{halbwachs1980} This work simultaneously represents
  some of Halbwachs's most developed ideas (while responding to critics)
  and evinces an incompleteness which posthumous publications often
  suffer. As Coser observes, ``One may doubt that the author himself
  would have been willing to publish it in what seems to be an
  unfinished state. The book nevertheless contains many further
  developments of Halbwachs's thought in regard to such matters as the
  relation of space and time to collective memory as well as fruitful
  definitions and applications of the differences between individual,
  collective, and historical memory.''\autocite{coser_halbwachs1992}}

The central contribution of Halbwachs's work was the notion that human
memory is intrinsically and inextricably tied to its social context.
This focus on the \emph{social} dimensions of memory betrays the deep
influence that Émile Durkheim's work had on Halbwachs. Humans are social
beings and as such human processes, such as memory, can only be
completed within the context of a society. Unlike Durkheim, however,
Halbwachs's approach was tempered by his desire to identify the physical
(or, perhaps ``biological'') location of memory within the individual.
Although the term ``collective memory'' evokes an ethereal or
metaphysical idea, Halbwachs's use of the term is entirely concrete.
Individual memories are kept by \emph{individuals}. However, an
individuals' ability to retrieve and utilize a particular memory is
intrinsically tied to the individual's social context. Memories require
social frameworks to function.\autocite[38]{halbwachs1992}

To illustrate this point, Halbwachs begins \emph{The Social Frameworks
of Memory} by attempting to prove the negative. \emph{Without} a social
framework, he argues, memories are always incomplete. Because
humans---for all intents and purposes---always exist within a society,
it is the dream state, he avers, that most closely approximates the
complete isolation of memory from society. Therefor the way that the
human brain deals with memories while dreaming can illustrate the
(dis)function of memories lacking a social framework. Thus, he observes
that ``dreams are composed of fragments of memory too mutilated and
mixed up with others to allow us to recognize them.'' Because in dream
states the mind lacks the ability to ``check'' itself against anything
external to itself, dreams do not contain ``true
memories.''\autocite[41]{halbwachs1992} This assertion is set against
the ``purely individual psychology'' (read: Bergson, et al.), which
viewed \emph{memory} as a location of social isolation. Regarding the
incompleteness of the dream state, he writes:

\begin{quote}
Almost completely detached from the system of social representations,
{[}the dream state's{]} images are nothing more than raw materials,
capable of entering into all sorts of combinations. They establish only
random relations among each other---relations based on the disordered
play of corporal modification.\autocite[42]{halbwachs1992}
\end{quote}

The ``system of social representations'' that Halbwachs refers to is not
limited, however, to macro structures such as familial, religious, or
class groups. Although these structures certainly \emph{do} make up an
important stratum of social frameworks, Halbwachs envisions something
much more fundamental. Although he does not use the language of
semiotics, Halbwachs seems to use ``social representations'' in a way
which could be fruitfully thought of as a shared system of ``signs''
that encompassed not only language, but every aspect of a group's
collective ``cultural \emph{Langue}.''{[}\^{} langue{]}

{[}\^{}{]}: USing a Saussurian comparison is actually quite
neat.\textless{}\textgreater{}hrom

net

Memory, therefore, cannot be understood in isolation from its social
framework and therefore should not be analyzed without consideration to
the social context of the rememberer.

While this example makes up only a small part of Halbwachs's work, it is
representative of his central thesis and provides a point of departure
for his more in-depth studies of collective memory in the family,
religion, and social classes. Social frameworks shape the way that
people remember. The retrieval of memories is shaped by those same
frameworks, and as those frameworks shift, so too do the memories that
are recalled in those societies.

Halbwachs also distinguished between two kinds of memory which were
distinguished by the relation of the individual to the experience of the
memory: autobiographical and historical. Autobiographical memory refers
to the sort of memories which are the result of individual, subjective
experience, while historical memory refers to memories which fall
outside the experience of the individual. Autobiographical memory
naturally includes memories of events that the rememberer personally
participated in, or of people directly known. But it also includes
memories of a more indirect nature that nevertheless shape the memory of
an individual. The quintessential example of this sort of indirect
autobiographical memory for Americans of roughly my age would be the
events of September 11, 2001. Although comparatively few people directly
witnessed the events (I was asleep on the West Coast when the first
plane crashed), the impact that those events had (and continue to have)
on the orientation of American national memory is unquestionably a part
of many people's lived experience. On the other hand, the events of the
American Civil War, although its impact continues to be seen in
contemporary American discourse, is not a part of any living person's
autobiographical memory.\autocite[19]{olick_olick-etal2011} Both of
these, however, are shaped by the social frameworks of the rememberer.
Thus, the events of September 11, 2001, now, are shaped by the social
discourses surrounding the United States's military presence in the
region and the controversial pretexts for engagement in the region, in
particular with the Iraq war. Likewise the construction (and recent
removal) of certain confederate monuments on the campus of the
University of Texas at Austin illustrates how historical memory can be
(consciously, in this case) reshaped and restructured as the remembering
society changes.

Of course, people participate in a plurality of social groups, often
simultaneously, and the experiences that are later to be recalled, too,
must be situated within these frameworks. In order to bring these
autobiographical memories to mind, according to Halbwachs, an individual
attempts to mentally situate herself within those same frameworks. For
instance, I find it much easier to recall whether a particular course I
have taken occurred in the Fall or Spring semester, rather than which
month or even year it occurred. The social framework that is the
``academic year'' remains a potent framework for my own memories, while
my wife---who had the good sense to stop after one degree---stopped
remembering in terms of ``semesters'' years ago. Yet, when remembering
events during her time at the University, the semester once again
becomes a useful framework for memory. It is for this reason that recent
memories are more easy to call to mind: because the social frameworks
that produced the memory (the people, places, customs, etc.) remain in
close proximity for the rememberer and the effort required to situate
the memory within the social frameworks that produced it is
minimal.\autocite[52]{halbwachs1992}

\hypertarget{collective-memory-and-cultural-memory-and-social-memory-a-terminological-assessment}{%
\section{Collective Memory, and Cultural Memory, and Social Memory: A
Terminological
Assessment}\label{collective-memory-and-cultural-memory-and-social-memory-a-terminological-assessment}}

Although Halbwachs used the term ``collective memory'' to talk about the
interaction between memory at the individual, psychological level and
the social frameworks which enable the act of recollection, recent work
in memory studies has seen a proliferation of terminology which has both
clarified and (at times) obfuscated the discussion.\footnote{In fact,
  there is a grand tradition of imprecise and overlapping terminologies
  within memory studies going back to Halbwachs himself. For example, on
  page 40 of \emph{On Collective Memory}, Halbwachs uses the terms
  ``collective memory,'' ``social memory,'' ``social frameworks of
  memory,'' and ``collective frameworks of memory'' and it is not
  entirely clear what how Halbwachs is distinguishing between them. The
  way that he is able to use the terms almost interchangeably has led
  some in the current discussion to treat them as synonyms. As Anthony
  Le Donne observes, ``In fact, {[}``social'' and ``collective''
  memory{]} are currently used synonymously with such frequency that
  their nuances vary from author to author \autocite[42
  n.8]{ledonne2009}. Yet, Le Donne points out, Halbwachs actually uses
  these terms in with slightly different nuances. On the one hand,
  Halbwachs uses the term''sociaux" when he is describing the way social
  structures affect memory, while on the other hand ``Collective
  Memory'' tends to refer to the content of memories which are
  transmitted between individuals \autocite[42 n.8]{ledonne2009}. In
  other words, when Halbawachs uses the term ``social'' he is usually
  talking about \emph{frameworks} of memory, whereas ``collective
  memory'' might be cautiously glossed as ``common'' or ``shared''
  memory. Sandra Hübenthal puts it well, ``The difference {[}between
  social and collective memory{]} lies in the perspective: \emph{social
  memory} is using the framework, \emph{collective memory} is
  establishing it'' \autocite[180]{hubenthal_carstens-hasselbalch2012}.}
