% !TEX root = dissertation.tex

\section{Constructing Authority in Jubilees}

The primary difference between the ways that the \ga and the book of \jub engages with cultural memory is that the book of \jub engages more directly in what I will refer to as ``prescriptive discourses.'' This is not to say, of course, that the author of the \ga did not or could not have had halakhic intentions behind his writing, but only that such intentions are not, generally, able to be identified. The \ga functions ``prescriptively'' only insofar as its characters \emph{demonstrate} good practice. Within this framework, the \ga participates in prescriptive discourses in a comparatively \emph{indirect} manner. The \ga portrays its main characters as good, Torah-following, Jews but we cannot say for certain whether these portrayals are intended to be novel contributions to the image of the characters, or whether they merely \emph{reflect} an understanding of how the righteous patriarchs \emph{would have} conducted themselves.%
    \footnote{For example, Noah, in \col{10}, 13--18 makes sacrifices in accordance with the legal tradition from Lev 4--6. Thus, Noah---the righteous patriarch---followed Torah even before it was given to Moses. This particular incident, as it happens, also is seen in \jub. See \cite[112]{crawford2008}. See also \cite[419]{reeves_revq1986}.}
In other words, the question of whether the author of \ga was proffering a novel portrait of the patriarchs or was more passively projecting his understanding of the patriarchs goes to the question of authorial intent, which, lacking more concrete evidence one way or the other, is a dead end. What \emph{is} certain, however, is that the \emph{participation} of the author within a particular site of memory affected the broader cultural memory, either by injecting new ideas, interpretations, or literary color into the tradition, or by reinforcing a set of inherited traditions.

To be sure, the book of \jub engages in similar kinds of positive portrayals of the patriarchs. And in fact, one of the tacit assertions of the book of \jub seems to have been that---at least certain aspects of---the Torah had not only been revealed to the pre-Mosaic patriarchs, but was practiced by them. Nickelsburg writes:
    \begin{quote}
        Noah offered a proper sacrifice (7:3--5), and Levi discharged the office of priest (32:4--9). Major holidays were observed by the patriarchs: First fruits by Noah, Abraham, Isaac, Jacob, and Ishmael (6:18, 15:1--2; 22:1--5); Tabernacles by Abraham (16:200--31); and the Day of Atonement by Jacob (34:12--20). Special prescriptions are given for Passover, the Jubilee Year, and the Sabbath (chaps. 49--50).%
        \footnote{\cite[69]{nickelsburg2005};
            See also \cite[70]{crawford2008}.}
    \end{quote}
\noindent
As with the \ga, these otherwise anachronistic displays of religious piety could be viewed as engaging with biblical memory in multiple ways---as either an innovation or a preservation of cultural memory. Although the intention of the author certainly \emph{may} have been to advocate for a particular halakhic point of view, such intentions are simply not knowable with any degree of certainty. We can, however, assert that the portrayal of the patriarchs as being observant of particular halakhic practices \emph{reinforced} such practices as normative and thus may have had a prescriptive \emph{function}. In other words, although it is fraught to speculate about what an author \emph{intended}, we \emph{can} reasonably speculate that the repetition and the positive portrayal of these practices would have helped to buttress the cultural support for of particular practices for contemporary readers.
    \footnote{The degree to which such practices would serve as an example to be followed, of course, assumes that the text was a trusted source of tradition. Within social contexts where a particular text was \emph{not} a trusted source of tradition, of course, the text would not function in this way.}

What is unique about the The book of \jub, by comparison to both Chronicles and \ga, is that \jub goes beyond \emph{demonstrating} good practice and engages in more direct forms of prescriptive discourses. It stops short, however, of taking the form of traditional legal material. Although there are examples of \rwb text which may be categorized formally as legal writing (e.g., the \templescroll\autocite{fraade_goldstein-etal2017}), the book of \jub does not, formally, include such material. 

Instead, the book of \jub includes explicit halakhic instruction embedded within the narrative. These explicit instructions take the literary form of, for example, testaments or as references to precepts recorded on the ``\heavenlytablets.'' In both cases, although the instructions are not meta-discursively directed toward the reader, this is precisely the effect.%
    \footnote{One might question how this is different than the legal material within the Hebrew Bible, which is also embedded within various narratives. I would argue, however, that even embedded within the narrative, the \emph{form} of these sections is legal. Importantly, these sections are \emph{portrayed within the narrative} as legal material. Thus, although embedded within a narrative, it remains (generically speaking) ``legal'' in form.}

%% Testament %%
One of the best examples of the testamental form of halakhic instruction falls within \jub 20--22. In this section, a series of three testamental speeches are given by Abraham: the first to Ishmael (and his children), Isaac (and his children), and Keturah's children (20:1--20:13), the second to Isaac (21:1--22:8), and the third to Jacob (22:16--24). Common among the speeches is an admonishment to follow God's commands (20:2, 21:5; 21:21--24) and to abstain from idolatry (20:7; 21:5; 22:17--18) and sexual impurities (including especially intermarriage with Canaanites; 20:3--6; 22:20). The testament directed toward Isaac (21:1--22:8), which is the longest of the three, includes a lengthy digression on several specific sacrificial practices including the proper exsanguination and butchering of the sacrificial animal (21:7--8), what flour and oil to use (21:7), which woods are acceptable for the fire (21:12--14), and ritual washing (21:16).%
    \footnote{One cannot help but notice the conspicuousness of the fact that it is \emph{Isaac} to whom Abraham gives these specific instructions. VanderKam notes that the Aramaic Levi Document, which shares some considerable overlap with this particular testament (including a concern for which woods are to be used for sacrifice), describes Isaac teaching his grandson Levi proper sacrificial procedure. Thus, it could be that Isaac's connection to the sacrificial cult is more a way to get the instructions to Levi than an allusion to the Aqedah. On the relationship between the Aramaic Levi Document and \jub and the kinds of wood allowed for sacrifice, see \cite[625, 636--639]{vanderkam2018}.}
Notably, although many of the specifics align with biblical instructions on sacrifice (), the sum-total of these instructions cannot be accounted for through purely exegetical moves. The instruction about which woods are acceptable, in particular, does not have any biblical precedent. Yet, within the Hebrew Bible, traces of the practice can be seen in the book Nehemiah.%
    \footnote{\cite[]{vanderkam2018}.}
Nehemiah 10:33--35 (Eng. 10:32--34) reads:

\begin{hebrewtext}
    \versenum{Neh 10:33}
    וְהֶעֱמַדְנוּ עָלֵינוּ מִצְוֹת לָתֵת עָלֵינוּ שְׁלִשִׁית הַשֶּׁקֶל בַּשָּׁנָה לַעֲבֹדַת בֵּית אֱלֹהֵינוּ׃ 
    \versenum{34}
    לְלֶחֶם הַמַּעֲרֶכֶת וּמִנְחַת הַתָּמִיד וּלְעוֹלַת הַתָּמִיד הַשַּׁבָּתוֹת הֶחֳדָשִׁים לַמּוֹעֲדִים וְלַקֳּדָשִׁים וְלַחַטָּאוֹת לְכַפֵּר עַל־יִשְׂרָאֵל וְכֹל מְלֶאכֶת בֵּית־אֱלֹהֵינוּ׃
    \versenum{35}
    וְהַגּוֹרָלוֹת הִפַּלְנוּ עַל־קֻרְבַּן הָעֵצִים הַכֹּהֲנִים הַלְוִיִּם וְהָעָם לְהָבִיא לְבֵית אֱלֹהֵינוּ לְבֵית־אֲבֹתֵינוּ לְעִתִּים מְזֻמָּנִים שָׁנָה בְשָׁנָה לְבַעֵר עַל־מִזְבַּח יְהוָה אֱלֹהֵינוּ כַּכָּתוּב בַּתּוֹרָה׃
\end{hebrewtext}
\begin{translation}
    \versenum{Neh 10:32}
    And we have obligated ourselves with a command to give a third of a shekel yearly for the service of the temple of our God:
    \versenum{33}
    for the showbread, the regular offering and for the regular burnt offering, the sabbaths, the new moons, for the appointed festivals and for the sacred things and for the sin offerings and for the atonement of Israel and all the work of the temple of our God.
    \versenum{34}
    Now, we have cast lots---the priests, the Levites, and the people---concerning the gift of wood, to bring (it) to the temple of our God by the ancestral houses for the appointed times, year by year to burn upon the altar of \yahweh our God \emph{as it is written in the law}. (emphasis added)
\end{translation}
\noindent
Thus the book of Nehemiah at least hints that there was some ritual associated with the wood for the altar during the (nascent) \secondtemple period. The fact that such a ritual is described as ``written in the law'' is a conspicuous difficulty given the absence of any such instruction within the Torah.%
    \footnote{Blenkinsopp states that the offering of wood was ``implicit'' in the command for the priests to keep the fire burning at all times (Lev. 6:2, 5--6). He further notes that the practice is described in Josephus' \emph{Jewish War} 2.425. See \cite[317]{blenkinsopp1988}. VanderKam notes further that a similar practice is described in the Mishnah (m. Taʿan 4:5, m. Tamid 2:3). See, \cite[636]{vanderkam2018}.}
Whatever the case, what is important for our purposes is that plainly halakhic material (for which we have evidence of practice elsewhere, but which does not show up in the Hebrew Bible) is clearly presented as \emph{instructive} for the reader, despite being embedded within a narrative.

%% Heavenly Tablets %%
Another way that the author(s) of \jub expresses these more direct forms of prescriptive discourses is through references to ``the \heavenlytablets'' (\HT; Eth. \eth{ጽላተ~፡ ሰማይ}{ṣəllāta samāy}). Florentino García Martinez has noted several functional valences to \jub's literary use of these tablets, including  the ``divine, pre-existing archetype of the Torah,'' a register of good and evil deeds, the ``book of destiny,'' a calendar for the proper observation of sabbaths and feasts, and a source for ``new halakot.''\autocite{martinez_najman-tigchelaar2012} It is the last function of the \HT which concerns us.

The author of \jub uses the \HT as a literary device to assert the absoluteness of a given halakhic practice. For instance, \jub 30:7 prohibits a man from giving his daughter or sister to a foreigner in marriage upon penalty of death by stoning. Moreover, the woman given is also to be killed by burning. Martinez notes that the prohibition of foreign marriage is an extremely important theme in \jub, however, there is no such prescription in the Torah (though not entirely foreign to the Hebrew Bible). Martinez writes:
\begin{quote}
    The biblical basis of the penalty imposed, that is, stoning, is deduced from the equivalence between delivering one's offspring to Moloch and delivering them to foreigners. The punishment of burning a woman can only be explained by a comparison with the Israelite woman who marries a foreigner to the daughter of a priest who prostitutes herself. In both cases the interpretation of the biblical text supposes, as does the halakah itself, that it can be accepted only by virtue of the authority that is conferred upon it by its inscription upon the ``heavenly tablets.''\autocite[67]{martinez_najman-tigchelaar2012}
\end{quote}
\noindent
In other words, the halakah that the author proffers is not presented as an interpretation of the Torah, but as an absolute precept known from an immutable and divine source. Martinez notes that the formula of referencing halakot as deriving from the \HT (with minor variations) can be found in at least six instances where the halakot cannot be directly linked to the Torah.%
    \footnote{The references are \jub 3:31; 4:32; 15:25; 28:6; 30:9; and 32:10--15. See \cite[64--68]{martinez_najman-tigchelaar2012}. Placing ``Torah backed'' halakot in a different category than ``new'' halakot, I think is a methodological problem with Martinez's study. Martinez goes so far as to say that the \HT ``do not represent a single notion,'' but I think this begs the question. The \HT \emph{are} presented as a single idea (with multiple functions). Nickelsberg, on the other hand, includes 6:17--22 and 33:10--20 in his discussion of the phrase and I think it helps to unify what the \HT were imagined to be.}

Thus, where \ga certainly offered \emph{examples} of good practice, \jub presents itself as an official, correct, understanding of the cosmic order which is divinely ordained, at times speaking literally in the imperative mood. \jub deals with legal and halakhic matters directly---it gives instructions about how and when to celebrate the sabbath, festivals, how to observe the yearly calendar, who to marry, and directly critiques the sinful behavior of Israel. While, the \ga may have \emph{implicitly} endorsed particular ideologies and halakhic practices through linking them with the foundational figures of Genesis (Lamech, Noah, and Abram), the book of \jub at times engages in direct imperative and presents itself as an authoritative text whose content comes directly from God, incised in the \heavenlytablets, mediated by God's chief angelic being (the \ap), and ultimately recorded by Israel's most authoritative legal figure, Moses. It stands to reason, therefore, that the purpose of \jub was not simply religious entertainment or vaguely edifying storytelling but was intended to be read, understood, and (importantly) to affect the behavior of its readers.

At times, this places \jub at odds with a plain reading of the Torah and in such cases, the author of \jub insists to the reader that its characterizations are absolute and divinely sourced. But, of course, such a characterization is simply a rhetorical means of bolstering the author's particular claims about proper Jewish praxis. Such claims, of course, must be framed not as innovations \emph{over and against} the Torah, but as the \emph{originally intended} or \emph{omitted} precepts from God. Thus, the author still holds up the Torah as the divine standard.

% 
For example, God tells Moses in \jub 1:9--10 that the people will stray from the covenant in part by ``forgetting'' God's commandments and neglecting proper cultic activities. Furthermore, the persecution of those who study the law is included in a catena of evil deeds that Israel will perpetrate:

% !TEX root = dissertation.tex
\begin{ethiopictext}
    \versenum{Jubilees 1:9}
    እስመ~፡ ይረስዑ~፡ ኵሎ~፡ ትእዛዝየ~፡ 
    ኵሎ~፡ ዘአነ~፡ እኤዝዞሙ~፡ ወየሐውሩ~፡ ድኅረ~፡ አሕዛብ~፡ ወድኅረ~፡
    ርኵሶሙ~፡ ወድኅረ~፡ ኀሳሮሙ~፡ ወይትቀነዩ~፡ ለአማልክቲሆሙ~፡ 
    ወይከውንዎሙ~፡ ማዕቀፈ~፡ ወለምንዳቤ~፡ ወለፃዕር~፡ ወለመሥገርት~፡
    \versenum{10}
    ወይትሐጐሉ~፡ ብዙኃን~፡ ወይትአኀዙ~፡ ወይወድቁ~፡ ውስተ~፡
    እደ~፡ ፀር~፡ እስመ~፡ ኀደጉ~፡ ሥርዓትየ~፡ ወትእዛዝየ~፡ ወበዓላተ~፡
    ኪዳንየ~፡ ወሰንበታትየ~፡ ወቅድሳትየ~፡ ዘቀደስኩ~፡ ሊተ~፡ በማእከሎሙ~።
    ወደብተራየ~፡ ወመቅደስየ~፡ ዘቀደስኩ~፡ ሊተ~፡ በማእከለ~፡
    ምድር~፡ ከመ~፡ እሢም~፡ ስምየ~፡ ሳዕሌሁ~፡ ወይኅድር~።
    \versenum{11}
    ወገብሩ~፡
    ሎሙ~፡ ፍሥሐታተ~፡ ወኦመ~፡ ወግልፎ : ወሰገዱ~፡ ዘዘ~፡ ዚአሆሙ~፡ 
    ለስሒት~፡ ወይዘብሑ~፡ ውሉዶሙ~፡ ለአጋንንት~፡ ወለኵሉ~፡ ግብረ~፡
    ስሕተተ~፡ ልቦሙ~።
    \versenum{12}
    ወእፌኑ~፡ ኀቤሆሙ~፡ ሰማዕተ~፡ ከመ~፡
    አስምዕ~፡ ሎሙ~፡ ወኢይሰምዑ~፡ ወሰምዕተኒ~፡ ይቀትሉ~፡ ወለእለሂ~፡
    የኀሥሡ~፡ ሕገ~፡ ይሰድድዎሙ~፡ ወኵሎ~፡ ያፀርዑ~፡ ወይዌጥኑ~፡ ለገቢረ~፡
    እኩይ~፡ በቅድመ~፡ አዕይንትየ~፡
    \versenum{13}
    ወአኀብእ~፡ ገጽየ~፡
    እምኔሆሙ~፡ ወእሜጥዎሙ~፡ ውስተ~፡ እደ~፡ አሕዛብ~፡ ለፂዋዌ~፡
    ወለሕብል~፡ ወለተበልዖ~። ወአሴስሎሙ~፡ እማእከለ~፡ ምድር~፡
    ወእዘርዎሙ~፡ ማእከለ~፡ አሕዛብ~፡
    \versenum{14}
    ወይረስዑ~፡ ኵሎ~፡ ሕግየ~፡
    ወኵሎ~፡ ትእዛዝየ~፡ ወኵሎ~፡ ፍትሕየ~፡ ወይስሕቱ~፡ ሠርቀ~፡ ወሰንበተ~፡
    ወበዓለ~፡ ወኢዮቤለ~፡ ወሥርዓተ~።
\end{ethiopictext}

\begin{transliteration}
    \versenum{Jubilees 1:9}
    ʔəsma yərassəʕu \kw{ə}llo təʔzāzəya
    % yərassəʕu         g impf 3mp √rsʕ 'to forget' Les. 473
    % təʔzāzya          təʔzāz + 1cs 'law'
    \kw{ə}llo za-ʔana ʔəʔēzzəzomu wa-yaḥawwəru dəḫra ʔaḥzāb wa-dəḫra
    % ʔəʔēzzezomu       d impf 1cs +3mp √ ʔzz 'to command' Les. 53
    % yaḥawwəru         g impf 3mp √ḥwr 'to go' Les. 249
    % dəḫra             'back, past, after, then' Les. 129
    % ʔaḥzāb            pl. of ḥəzb 'nation, people, tribe' Les. 253
    rə\kw{ə}somu wa-dəḫra ḫasāromu wa-yətqan\-nayu la-ʔamāləktihomu
    % rəkwsomu          n. √rkws 'to be unclean, impure' Les. 470
    % ḫasārom           n. √ḫsr 'dishonor' Les. 265
    % yətqannayu        tG impf. 3mp 'serve' √qny Les. 437
    % ʕamālektihomu     pl. of ʔamlāk + 3mp '(false) gods, idols' Les. 344
    wa-yəkawwənəwwomu māʕəqafa wa-la-məndābē wa-la-ḍāʕr [ṣāʕr] wa-la-maśgart
    % yəkawwənəwwomu    g impf. 3mp (ənu- > ənəww- Lambd. 64) √kwn 'become'
    % māʕəqafa          √ʕqf 'to trip up' Les 67
    % məndābē           'tribulation, affliction' Les. 348
    % ḍāʕər             √ṣʕr 'anguish, trouble' Les. 544
    % maśgart           √śgr II 'snare, trap' Les 527
    \versenum{10}
    wa-yətḥag\gw{a}lu [yəthag\gw{a}lu] bəzuḫān wa-yətʔaḫḫazu wa-yəwaddəqu wəsta
    % yətḥagolu         tG impf √hgwl 'destroy'
    % bəzaḫān           'many' √bzḫ 'be numerous' Les 117
    % yətʔaḫḫazu        tG impf. 3mp √ʔḫz 'take, catch' Les. 14
    % yəwaddəqu         G impf 3mp √wdq 'collapse, go to ruin' Les.
    ʔəda ḍarr ʔəsma ḫadagu śərʕātəya wa-təʔzāzəya wa-baʕālāta
    % ʔəda              'hand' pl. ʔədaw Les. 7
    % ḍar               'enemy' pl. ʔaḍrār Les. 152
    % ḫadagu             g pf. 3mp √ḫdg 'to abandon' Les. 258
    % śərʕāteya          pl. 'ordinances' √śrʕ Les. 533
    % baʕālāta          pl. 'festivals' Les. 83
    kidānǝya wa-sanbatātǝya wa-qəddǝsātǝya za-qaddasku lita ba-māʔkalomu
    % kidāneya           'covenant' √kyd Les 301
    % sanbatāteya        'sabbaths'
    % qəddesāteya        'holy things' Les 422
    % qaddasku           d pf. 1cs √qds 'to make holy' Les. 422
    % ba-māʔkal          'amongst'
    wa-dabtarāya wa-maqdasǝya za-qaddasku lita ba-māʔkala
    % dabtarāya         'tabernacle'
    % maqdasya          'temple'
    % qadasku           g pf. 1cs √qds 'to make holy' Les. 422
    mədr kama ʔəśim səməya lāʕlēhu wa-yəḫdər
    % ʔəśim         g. subj. 1cs √śym 'put, place' Les. 539
    % səmya         'name' +1cs Les. 504
    % lāʕlēhu       prep. 'upon'
    % yəḫdər        g subj 3ms √ḫdr 'to dwell'  Les. 258
    \versenum{11}
    wa-gabru
    % gabru             g pf. 3mp √gbr 'to make'
    lomu fəśḥatāta wa-ʔoma [ʕoma] wa-gəlfo wa-sagadu zazza ziʔahomu
    % fəśḥatāta         pl. 'high place' √fśḥ II Les. 168
    % ʔoma              ʕoma 'grove' Les. 62
    % gəlfo             adj. 'carved' √glf 'to carve' Les. 190
    % sagadu            g pf 3mp √sgd 'bow down, prostrate'
    % zaza              each
    % ziʔahomu          each of them (probably zazazaiʔahomu)
    la-səḥit wa-yəzabbəḥu wəludomu la-ʔəgānənt wa-la-\kw{ə}llu gəbra
    % səḥit             'sin, error' √sḥt 'to stray, err' Les. 494
    % yəzabbəḥu         g impf 3mp √zbḥ 'to sacrifice' Les. 631
    % ʔəgānənt          'demons' sn. gānən Les 198
    % gəbra             'thing' √gbr 'to do, make' Les 178
    səḥtata ləbbomu
    % saḥtata           'sinful' √sḥt 'to stray, err' Les. 494
    % ləbbomu           'hear, mind'
    \versenum{12}
    wa-ʔəfēnnu ḫabēhomu samāʕta kama
    % ʔəfēnnu           d. impf 1cs √fnw 'to send' Les. 163
    % ḫabē-             'toward, near, to'          
    % samāʕta           'witnessses' samāʕi √smʕ 'hear'
    ʔasməʕ [ʔāsməʕ] lomu wa-ʔiyyəsamməʕu  wa-samāʕta-ni yəqattəlu wa-la-ʔəlla-hi
    % ʔasməʕ            CG Subj 1cs √smʕ 'to bear witness'
    % ʔiyyəsammeʕu      g impf 3mp √smʕ 'to listen' Les. 501
    % samāʕta           'witnessses' samāʕi √smʕ 'hear' 
    yaḫaśśəśu ḥəgga yəsaddədəwwomu wa-\kw{ə}llo yāḍarrəʕu wa-yəwēṭṭənu la-gabira
    % yaḫaśśəśu         g impf √ḫśś 'to seek' Les. 266
    % yəsaddədəwwomu    g impf 3mp √sdd 'persecute, drive away'
    % yāḍarrəʕu         CG impf 3mp √ḍrʕ 'to annul, leave aside'
    % yəwēṭṭənu         D impf 3mp √wṭn Les. 623
    % gabira            G inf √gbr 'to do'
    ʔəkuya ba-qədma ʔaʕəyyəntəya
    % ʔəkuya            'evil'
    % ba-qədmu          'before'
    % ʔaʕəyyəntəya      'eyes' sn. ʕayn +1cs Les. 79
    \versenum{13}
    wa-ʔaḫabbəʔ gaṣṣəya
    % ʔaḫabbəʔ          g impf 1cs √ḫbʔ 'hide, conceal' Les. 253
    % gaṣṣəya           'face' Les. 205
    ʔəmmənēhomu wa-ʔəmēṭṭəwomu wəsta ʔəda ʔaḥzāb la-ḍiwāwē
    % ʔəmēṭəwwomu       d impf 1cs √mṭw 'hand over, deliver' Les. 374
    % ʔaḥzāb            pl. of ḥəzb 'nation, people, tribe' Les. 253
    % ḍiwāwē            'captivity' √ḍww 'take prisoner' Les. 153         
    wa-la-ḥəbl wa-la-tabalʕā wa-ʔasēssəlomu [ʔāsēssəlomu] ʔəm-māʔkala mədr
    % ḥəbl              'spoils' Les. 223
    % tabalʕā           tG pf 3fs? 'devouring' √blʕ 'to eat' Les. 94
    % ʔasēssəlomu       CD impf 1cs +3mp 'remove' Les. 516
    wa-ʔəzarrəwomu māʔəkla ʔaḥzāb
    % ʔəzarrəwomu       G impf 1cs +3mp √zrw 'scatter, disperse' Les. 644
    % ʔaḥzāb            pl. of ḥəzb 'nation, people, tribe' Les. 253
    \versenum{14}
    wa-yərassəʕu \kw{ə}llo ḥəggəya
    % yərassəʕu         g impf 3mp √rsʕ 'to forget' Les. 473
    wa-\kw{ə}llo təʔzāzəya wa-\kw{ə}llo fətḥəya wa-yəsəḥḥətu śarqa wa-sanbata
    % təʔzāzya          təʔzāz + 1cs 'law'
    % fətḥəya           fətḥ 'precept' √ftḥ Les. 170
    % yəsəḥḥətu         G impf 3mp √sḥt 'to err, make a mistake' Les 494
    % śarqa             'rising' here 'beginning of the month' √śrq Les. 534
    % sanbata           'sabbath' Les. 505
    wa-baʕāla wa-ʔiyyobēla wa-śərʕāta
    % śərʕāta           'ordinance' √śrʕ Les. 533
\end{transliteration}

\begin{translation}
    \versenum{Jubilees 1:9}
    Therefore they will forget my law---%
    all that I am commanding them---and they will go after the nations and after
    their impurity and after their dishonor. And they will serve their (false) gods
    and they will become a hindrance and an affliction and a snare for them.
    \versenum{10}
    Many will be destroyed and they will be captured and will fall into
    the hand of the enemy because they abandoned my ordinances and my laws and festivals of 
    my covenant and my sabbaths and my sacred items which I sanctified for myself amongst them.
    Also my tabernacle and my \temple which I sanctified for myself in midst of
    the land that I might establish my name upon it and it might dwell (there).
    \versenum{11}
    And they made
    for themselves high places and a grove, and a carved image and they bowed down, each one of them,
    to error. And they will sacrifice their children to demons and to every thing
    of their sinful mind.
    \versenum{12}
    And I will send witnesses to them that
    I may testify to them, but they will not listen and (instead) they will kill the witnesses and even
    those (who) seek after the law, they will persecute. And they will leave aside everything and they will begin to do
    evil before my eyes.
    \versenum{13}
    And I will hide my face
    from them and I will deliver them into the hand of the nations for captivity,
    for spoils, and for their devouring. I will remove them from the midst of the land
    and I will scatter them among the nations.
    \versenum{14}
    They will forget my whole law,
    and all my commandments, and all my precepts. And they will err regarding the new moon and the sabbath
    and the festival and the jubilee and the ordinances.
\end{translation}

\noindent
%% TODO: Tease this out—explain what you mean more clearly
Thus, \jub at once affirms the centrality of the Torah, while, in some sense, circumventing it by providing its own idiosyncratic account of Gen 1--Exod 12. The juxtaposition of deference toward Torah while simultaneously circumventing its claim to primacy yields a sort of ``\psgraphical paradox.'' It is not immediately clear how a \psgraphical author, knowingly writing under a false name, can simultaneously endorse one text, while offering novel embellishments and changes to its interpretation. At least to the modern reader, this practice seems foreign and disingenuous by the \psgraphical author. The question should be raised, therefore, whether \jub \emph{was in fact} intending to supersede or circumvent the authority of the Torah (as some scholars suggest) or whether some other relationship existed between the texts.%
    \footnote{Wacholder, for example, understands \jub and the \templescroll to be a single unit and a work which was meant to supersede the Pentateuch. See \cite{wacholder_kampen-etal1997}. His theory has not been widely accepted.}

Although there is some question whether the book of \jub attained the status of ``scripture'' in antiquity, it is generally agreed that \psgraphical texts such as \jub were not intended as replacements for the more well-known scriptures (especially the Torah).%
    \footnote{This position undoubtedly represents the majority opinion, though it is not unanimous. For the opposing opinion, see especially \cite{wacholder_kampen-etal1997}.}
Of course, the reality is that we do not know for certain what kinds of categories ancient readers used to classify their literature; most likely, however, they were not static nor consistent across time and differed by social group. All the same, the special place that the Torah had for a number of Jewish sects---even in antiquity---seems to me to preclude the idea that \psgraphical texts such as \jub would be placed on-par with the Pentateuch, even if a work carried a potent practical authority (see below). What \emph{can} be said about the book of \jub, however, is that \emph{it presents itself} as a unique revelation that claims for itself the same kind of divine source as the Torah.%
    \footnote{As a matter of clarification, I am assuming a distinction between 1) the author's intent, 2) the way the work presents itself, and 3) the way the work was understood by its readers. Thus the text may present itself as ``on-par'' with the Torah without either the author or audience treating it as such.}

\vanderkam has offered a concise summary and analysis of this ``\psgraphical paradox'' and comes to the conclusion that the book of \jub functions as a vehicle for its author to proffer his own interpretation of Gen 1--Exod 12. \vanderkam addresses the problem of \jub's author both acknowledging the existence and authority of the Torah while simultaneously offering his own original material, writing:

\begin{quote}
    [W]e could say differences in interpreting the Pentateuch had arisen by his time and that the author wanted to defend his own reading as the correct one. But he wished to find a way to package his case more forcefully than that, presumably within the limits of what was acceptable in his society.\autocite[28]{vanderkam_metso-etal2010}
\end{quote}

\noindent
According to \vanderkam, therefore, the project of the author of \jub was primarily one of \emph{exegesis}. The book of \jub is an expression of the author's understanding of Gen 1--Exod 12; it offers explicit teachings about specific ambiguities and difficulties in the text of Genesis and Exodus. He had a particular understanding of how the Pentateuch should be understood, and he used the common rhetorical technique of \psy to ``more forcefully'' get his point across.%
    \autocite[28]{vanderkam_metso-etal2010}

\vanderkam argues that the author of \jub intentionally located the setting of his work in the Exod 24:12 ascent for a rhetorical advantage. He argues for three such advantages: first, by locating the story during Moses's ascent, he is able to draw on the \emph{character} of Moses. The author, therefore was able to imbue his work with the gravitas of Israel's most famous lawgiver. Second, setting the work as a part of the first forty-day period that Moses was on Mt. Sinai grounds the author's interpretation of Torah in the original revelation of the Law (prior to even Deuteronomy). These events putatively took place at the same time that Moses received the first set of stone tablets from God. While the stone tablets were broken and had to be rewritten, the account provided in \jub is prior even to those ``copies'' of the decalogue. Any subsequent interpretation of the Torah is secondary by virtue of its relative lateness. Finally, because Moses himself is presented as the author of \jub, there is no question of the chain of transmission. God revealed the contents of \jub to Moses by having the \ap dictate to him the contents of the \heavenlytablets. God is supreme, the tablets are eternal, and Moses is reliable.

Moses, therefore, received more from God on Mt. Sinai than is recorded in the Torah. The claim made by \jub is that it contains the additional information given to Moses, and that the subject of this additional revelation is the sacred history of Israel schematized according to the absolute heavenly reckoning of time (364-day years, weeks of years, and jubilees).

The tradition that God told Moses more on Mt. Sinai than he recorded in the Torah is not unique to the book of \jub. \vanderkam points toward the later rabbinic tradition that Moses received the Oral Torah during his time atop Mt. Sinai.%
    \footnote{\cite[28--31]{vanderkam_metso-etal2010}.}
For example, \vanderkam cites b. Berakot 5a, which references the specific time during which \jub is set (Exod 24:12):%
    \footnote{Translations of all rabbinic texts are my own.}

\begin{aramaictext}
    מאי דכתיב ואתנה לך את לחת האבן והתורה והמצוה אשר כתבתי להורתם לחת אלו עשרת הדברות תורה זה מקרא והמצוה זו משנה אשר כתבתי אלו נביאים וכתובים להרתם זה תלמוד מלמד שכולם נתנו למשה מסיני: 
\end{aramaictext}

\begin{translation}
    What is [the meaning where] it is written, \emph{I will give you the tablets of stone and the Torah and the commandments which I have written so that you might teach them} (Exod 24:12)?\\
    \-\hspace{2em}`the tablets' --- these are the ten commandments\\
    \-\hspace{2em}`the Torah' --- this is scripture\\
    \-\hspace{2em}`the commandments' --- this is Mishnah\\
    \-\hspace{2em}`that which I have written' --- these are the Prophets and the Writings\\
    \-\hspace{2em}`that you might teach them' --- this is Talmud\\\~
    [This] teaches that all of them were given to Moses on at Sinai.
\end{translation}

\noindent
The tradition here, therefore, asserts that the decalogue, the full Torah, its interpretation, the rest of the Tanakh, and the Talmud were all revealed to Moses on Sinai. Similarly, Sifra Beḥuqqotay 8, citing Lev 26:46:
\begin{aramaictext}
    אלה החקים והמשפטים והתורֹת: החוקים אלו המדרשות והמשפטים אלו הדינים והתורות מלמד ששתי תורות ניתנו להם לישראל אחד בכתב ואחד בעל פה
\end{aramaictext}
\begin{translation}
    \emph{These are the statutes and ordinances and Torahs} (Lev 26:46):\\
    \-\hspace{2em} `the statutes' --- this is midrash.\\
    \-\hspace{2em} `and the judgments' --- this is the legal rulings.\\
    \-\hspace{2em} `and the Torahs' --- [this] teaches that two Torahs were given to Israel: one in writing, the other by mouth.
\end{translation}

\noindent
The rhetorical function of asserting that later interpretive material was revealed to Moses is essentially the same as it is for \jub.

Thus, for \vanderkam, the book of \jub upholds the authority of the Torah by offering its own interpretation of its contents in a similar fashion to the way that the oral Torah, too, rooted its authenticity in the Sinai revelation. \jub, therefore asserts itself as a correct and authoritative interpretation of the Torah by claiming that it is the interpretation that Moses himself received from God; as \vanderkam puts it, according to the book of \jub, ``[t]he message of \jub is verbally inerrant.''%
    \footnote{
        \cite[33]{vanderkam_metso-etal2010}.
        Although the book of \jub is not generally thought to be the product of the Qumran community (it likely predates the settlement), it is worth noting that within the community, it was accepted that the community not only possessed the correct interpretation of its scriptures, but also that the community received a special revelation which the rest of Israel did not receive. As Fraade notes, this idea is quite different than supposing that additional material had been revealed \emph{to Moses}. See 
        \cite[67]{fraade_jjs1993}.}

While \vanderkam makes a number of useful observations, his characterization of \jub as exegesis, I think, ignores the question of how readers would have understood the work. This is where the analogy to the Oral Torah breaks down. While rabbinic claims that the Oral Torah was revealed to Moses, rabbinic discourse self-consciously acknowledges its work as exegetical---the rabbis offer explanations and instruction on how to understand the texts that they are commenting on. Although the rabbis may claim that an interpretation goes back to Moses, it is not the same as claiming to speak \emph{for} Moses or \emph{as} Moses. Thus \vanderkam's assertion that the purpose of writing pseudonymously and claiming that a work is the result of direct divine revelation goes beyond simply advocating for one's own interpretation ``more forcefully.'' The fact that \vanderkam leaves the particulars of this phrase ambiguous, I think, indicates ambiguity in his own thinking about \emph{how specifically} ancient readers may have understood \jub \visavis other so-called authoritative works, in particular, the Torah.

A more nuanced approach to this topic has been offered by Hindy Najman who, similarly has argued that the author of \jub utilized several ``modes of self-authorization'' in order to bolster its audience's perception of the work's authority.%
    \footnote{\cite[380]{najman_jsj1999}.}
Building on the work of Florentino García Martínez,\autocite{martinez_najman-tigchelaar2012} Najman argues that the book of \jub utilized (at least) four such ``authority conferring strategies,'' which I have reproduced in full:
    % FIXME: Paraphrase this?
    \begin{quote}
        1. \jub repeatedly claims that it reproduces material that had been written long before the ``\heavenlytablets,'' a great corpus of divine teachings kept in heaven.

        2. The entire content of the book of \jub was dictated by the angel of the presence at God's own command. Hence, it is itself the product of divine revelation.

        3. \jub was dictated to Moses, the same Moses to whom the Torah had been given on Mount Sinai. Thus the book of \jub is the co-equal accompaniment of the Torah; both were transmitted by the same true prophet.

        4. \jub claims that its teachings are the true interpretation of the Torah. thus, its teachings also derive their authority from that of the Torah; that its interpretations match the Torah's words resolve all interpretive problems further substantiates its veracity.%
        \autocite[380]{najman_jsj1999}
    \end{quote}
\noindent
Her ultimate conclusion is that texts such as \jub which interpret and rewrite portions of the Bible do so to ``[respond] to both the demand for interpretation and the demand for demonstration of authority.''\autocite[408]{najman_jsj1999} Thus the purpose of the book of \jub, according to Najman, is to provide an ``interpretive context'' for reading the Torah---to make explicit a particular tradition of interpretation that guides the Torah-reader away from spurious or otherwise heterodox readings. 

This idea is similar to, but importantly distinct from \vanderkam's understanding of \jub. Whereas \vanderkam envisioned \jub as an exegetical \emph{product} of Gen 1--Exod 12, Najman understands \jub as a kind of ``background'' text which is meant as an aid \emph{for reading} Torah. The difference is subtle, but significant, especially for our understanding of \jub within the framework of cultural memory. \vanderkam's characterization of \jub as a sort of ``official'' interpretation of the Torah is problematic because it does not leave room for Torah going forward. If \jub portrays itself as \emph{the} meaning of Gen 1--Exod 12---the inerrant interpretation of this portion of Torah---what need is there for the Torah? Najman's model, on the other hand, assumes that readers are cued into the genre. Rather than  characterizing \jub as an authoritative, but idiosyncratic, interpretation of Torah, Najman's approach understands \jub as something that could be read \emph{before} the Torah in order to quash potentially errant readings of Torah when the reader finally reaches them.\autocite[408]{najman_jsj1999}

In her subsequent book, Najman builds on this thesis by introducing the idea of ``Mosaic Discourse'' into the discussion of Early Jewish and Christian literary production. She traces the practice of pseudonymous engagement with the Mosaic legal tradition through literary production back to the book of Deuteronomy.\autocite[48]{najman2003} She identifies four features of Mosaic discourse, which she extrapolates from the way that Deuteronomy draws from, augments, and affirms earlier legal traditions (such as the Covenant Code). The way that the author of Deuteronomy was able to both modify/reinterpret the legal tradition of the Covenant Code while retaining the traditions of the Covenant Code served as a model for later tradants (such as the author of \jub, but also the \templescroll and others) to repeat the process by engaging with and developing both the message of Moses and the idea of Moses as an author. This is what she refers to as ``Mosaic Discourse.'' With this term, Najman builds on a Foucauldian understanding of the Author which is neither static, nor bound by any historical or literary factors. She writes:

\begin{quote}
    As Foucault reminds us, it is not only \emph{texts} that develop over time. The connected \emph{concepts} of the authority and authorship of texts \emph{also} have long and complex histories. Both models of anonymity and of pseudonymity can be found in the Hebrew Bible and in the extra-biblical texts of the Second Temple period. But even when an author is identified in a biblical text, it is unclear if that identification is to be considered \emph{the same} as what moderns would characterize as \emph{the author function}.%
        \footnote{%
            \cite[9--10]{najman2003}. Here she is referencing
            \cite[213]{foucault_essential-foucault_2}.}
\end{quote}
\noindent
Najman suggests that when ancient writers participated in pseudonymous writing, the purpose was not to deceive their readers so much as to honor the tradition of the Author under whose name they wrote.%
    \footnote{Najman notes a number of classical authors who seem to have practiced a form of pseudonymity where a student writes in the name of their master. In particular, she cites Iamblichus the Pythagorean who claims that it was ``more honorable and praiseworthy'' to use Pythagorus' name, rather than one's own name when publishing (De Vita Pythagorica 98). She also quotes Tertullian who suggests that certain New Testament work sought to be ascribed to Paul and Peter because the works in question were written by their disciples (Marc. 6.5). Likewise, she notes that Plato wrote under the name of his master, Socrates. See \cite[13]{najman2003}.}
Historically speaking, of course, unless one posits that a real figure named Moses established the legal tradition of Israel, \emph{all} Mosaic attribution is, in effect, \psgraphical and an expansion of Moses the Author. The tradition of Moses the Author grew in step with the ``writings'' of Moses.

The book of \jub, therefore, can be understood as participating within this tradition of Mosaic attribution which serves to faithfully augment the body of Mosaic teaching through the use of \psy. The interpretation of the Torah by the writer of \jub is not meant to be understood as the ``actual words'' of Moses, but as a representation of ``authentic teaching'' which aligns with the function of Moses as an Author as an aide to reading the Torah.\autocite[13]{najman2003}

% \subsection{Memory, Mosaic Discourse, and Practice}
% Reframe mosaic discourse in terms of Memory
Unsurprisingly, Najman's approach to \jub and Mosaic Discourse dovetails quite well with the idea of social and cultural memory theory. The way that Najman describes the growth and development of the Author extending beyond the historical and literary bounds of the ``real'' author is evocative of the process of memory construction. In fact, from my perspective, what Najman describes as Mosaic Discourse \emph{is} a process of memory construction, though she does not use the terminology. What she describes as Mosaic Discourse is the same set of processes which enabled the author(s) of the Enochic works to expand on and speculate about the Watchers and the Flood, and which enabled the \ga to draw from those traditions in its rewriting.%
    \footnote{The \ga, of course, may have also drawn from the book of \jub, which only goes to further this point.}
Given the fascination with the figure of Enoch in the \secondtemple period (as evidenced by the plethora of texts which evoke the character), we could just as easily talk about ``Enochic Discourse'' when we discuss the various and sundry texts which draw from, expand, and reframe the enigmatic antedeluvian figure. Furthermore, we can easily identify additional Discourses about the figures of Abram, Daniel, and David, all of whom are the subjects of expanding bodies of literary production in the \secondtemple period, albeit not all as \psa, and not all with the same foundational significance as that of Moses.

% ALL DIZ SHIT IS MEMORY
The ability to also talk about these other discourses, I think, signals to the broader applicability of Najman's ideas. From my perspective, the discussion can be further augmented by including language of cultural and social memory which brings with it a taxonomy for discussing the processes in sociological terms. Najman's terminology is able to describe \emph{that} these various texts are participating in a particular discourse, but it does not \emph{describe} the discourse nor the \emph{social influences} or \emph{social effects} of the discourses. 

% FIXME: You need to clarify the above and figure out whether she actually does any of this. She's usually talking about "reading strategies" I think, so that will be an angle from which to depart.

% WHAT IS INTERESTING ABOUT JUB is the posture toward MEMORY: PRESCRIPTIVE, PRACTICE
Although the \ga and \jub generally participate in different sets of discourses (with notable exceptions, such as the division of the world sections), they engage with them in qualitatively different ways, despite the fact that both can be characterized as \psa. The \ga, although written largely in the first-person, takes a broadly \emph{de}scriptive approach to its memory construction through rewriting. Although there are parts of the text which portray its characters in ways that betray the author's own social frameworks (See \ref{ch:ga}), the \ga resists readings which could be characterized as didactic or halakhic. \jub, on the other hand, includes a framing narrative which encourages the reader to not only reshape the way that they think about the characters within their rewriting, but also encourage particular kinds of \emph{practices}. In this sense \jub can be thought of as engaging with memory in more \emph{pre}scriptive discourses.

This shift from more \emph{descriptive} rewriting (like the \ga) to rewritings which incorporate \emph{prescriptive} discourses (like \jub) illustrates the social dimensions of talking about these texts as \emph{memory}. Even if the procedural, technical processes of interpretation and rewriting are identical, the social outcomes and concrete purposes of texts effect memory differently. For example, although the \ga utilizes the literary form of a \emph{waṣf}, this does not bear meaningfully on the concrete social effects of the \ga on memory. Even supposing the readers of \ga believed \ga to be the authentic ``historical''%
    \footnote{Here again, I am referring to the fact that, for the ancient readers of \jub, the character of Lamech/Noah/Abram was likely perceived as a real person from the distant past.}
accounts of Lamech, Noah, and Abram, the \ga simply \emph{does not ask} the reader to accept its account over and against any others. Its authenticity may be implied by its first-person rhetoric, but \ga does not exhibit the same kinds of authority conferring strategies that we see in \jub. 

On the other hand, the claim made by the book of \jub---that Moses received more information atop Mt. Sinai than is recorded in the books of the Pentateuch---is characterized as a kind of authoritative, revelatory literature which invites the reader to incorporate this new knowledge into their conception of the past in an act of cultural memory construction. The effect of this change on the remembered past does not remain in the abstract, however, but rather alters the way that the reader perceives the past with real-life, concrete, practical outcomes. 
