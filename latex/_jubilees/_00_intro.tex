% !TEX root = dissertation.tex

\nocite{dillamnn_jbw_kleine}
\nocite{ewald_zkm1844}

%%%%%%%%%%%%%%%%%%%%%%%%%%%%%%%%%%%%%%%%%%%%%%%%%%%%%%%%%
% MEMORY CONSTRUCTION is the key Idea for this chapter. %
%%%%%%%%%%%%%%%%%%%%%%%%%%%%%%%%%%%%%%%%%%%%%%%%%%%%%%%%%

Like the \ga, the book of \jub engages in a form of rewriting which participates in the construction of memory through \psgraphical discourse. \jub builds clearly from the biblical material (Gen--Exod 12) with the kinds of adaptations, harmonizations, and emendations we expect of \rwb and bears clear influences from other \secondtemple traditions such as the Astronomical Book of Enoch (\firstenoch 72--82). In this respect, \jub bears many of the same kinds of qualities that I considered in the previous chapter such as the sources of tradition, generic features, and narrative framing.%
    \footnote{One might argue, for example, that the genre of \jub is that  of Apocalypse. But Cf. \cite{hanneken2012}. Regarding the formal characteristics of Apocalypse, see John Collins's work on the topic, esp. 
        \cite{collins_mason-etal2012} and 
        \cite{collins_semeia1979}.}
Although not framed as a first-person account, \jub also portrays itself as the product of first-hand experience. The author presents his work as the result of God's repeated command to Moses to ``write down all that you hear'' (1:5, 7, 26) and to record the content of the \heavenlytablets dictated to him by the chief angelic being (2:1). Thus the author takes on the persona of Moses and proffers his work as a faithful record of Moses's experience atop Mt. Sinai and is therefore also counted among the \psa. The rewritten account of ``biblical history'' from Gen 1--Exod 12 is, like \ga, \emph{drawn from} biblical memory and \emph{speaks back into} biblical memory through the process of rewriting. 

Taking the book of \jub as my point of departure, in this chapter I will attempt to differentiate the \emph{manner} that \rwb texts may have engaged with cultural memory. In the previous chapter, I argued that the \psgraphical quality of the \ga engaged with the cultural memory \emph{differently} than other non-\psgraphical texts. However, I left open the question of how, specifically, readers were intended to understand the ``authority'' or ``authenticity'' of the account. Were these \psgraphical texts believed to be ``genuine'' first-hand accounts from Lamech, Noah, and Moses? Or were readers otherwise queued into the genre, understanding them to be ``historical fiction,'' as it were? Or perhaps both, or neither? In this chapter I will focus on the ways that \jub portrayed itself as an authoritative revelation which invited the reader to incorporate new knowledge into their conception of the past in an effort to affect the behavior of its readers and to reinforce the practices of its remembering community. I will argue that the book of \jub engages with cultural memory in a distinct fashion from other texts such as the \ga in part through rhetorical means (so-called ``authority conferring strategies''). This distinct form of engagement is significant because it illustrates the way that memory not only affects the intellectual conceptions of the past, but also carries with it \emph{concrete practical effects which can be concretely observed}. To accomplish this, I will draw on Hindy Najman's work on Mosaic Discourse and will discuss the ways that \jub portrays itself as authoritative literature, how this portrayal may have been understood in antiquity and how it could, in some sense, both authorize and rewrite the Torah with halakhic implications. Then, to illustrate the point, I will discuss the calendrical and chronological system of the book of \jub as an example of how memory construction extends beyond abstract or intellectual processes and into the realm of concrete social practice.


%%%%%%%%%%%%%%%%%%%%%%%%%%%%%
% Discovery and Publication %
%%%%%%%%%%%%%%%%%%%%%%%%%%%%%
\section{Jubilees: Discovery and Publication}
The work now referred to as the book of \jub was believed to have been lost forever by European scholars prior to the mid nineteenth Century. The work was ``rediscovered,'' however, in 1844 when Heinrich Ewald published a description of an Ethiopian (\geez) manuscript under the title ``the Book of the Division'' \eth{መጽሐፈ~፡ ኩፋሌ}{masḥafa kufālē}.%
        \footnote{All translations are my own. \geez citations are from \vanderkam's critical edition, \cite*{vanderkam1989}.}
Because the name followed the common convention using a work's first few (key) words as its title (in this case, \eth{ዝንቱ~፡ ነገረ~፡ ኩፋሌ}{zentu nabara kufālē}), Ewald suggested that this manuscript may have been a copy of the work known from antiquity as both \greek{τά Ἰωβηλαϊα}, ``the Jubilee,'' and \greek{Λεπτὴ Γίνεσις}, the ``Little Genesis.''\autocite[176--179]{ewald_zkm1844} Although the work had been in continuous use within Ethiopian Christianity since antiquity, prior to Ewald's publication, European scholarship only knew of the work through secondary references in a few classical sources.%
        \footnote{\vanderkam offers a concise summary of the various late-antique citations and allusions in his commentary, most notably in the works of Epiphanius (\emph{Panarion}, \emph{Measures and Weights}) and Syncellus (\emph{Chronography}).
                \cite[1:10--14]{vanderkam2018}. See also 
                \cite{reed_kister-etal2015} and 
                \cite{kreps_ch2018}.
        It is also probable that more recently discovered texts, such as the Damascus Document (CD), refer to the book of \jub as 
        ``the Book of the Divisions of the Times into their Jubilees and Weeks'' Heb. \hebrew{ספר מחלקות העתים ליובליהם ובשבועותיהם}. It seems almost unimaginable that CD was not referring to \jub, though, some have questioned the notion. See \cite[242--248]{dimant_vanderkam-etal2006}.}
The work was published (supplemented with a second manuscript) by August Dillmann in 1859\autocite{dillmann1859} and by R.~H. Charles in 1895, who included two additional manuscripts in his edition (totaling four).\autocite{charles1895} More recently, \vanderkam's 1989 edition utilized twenty-seven copies of the text\autocite[1:xiv--xvi]{vanderkam1989} and since its publication over twenty more copies have been cataloged and imaged.%
        \footnote{%
                \cite{erho_bsoas2013}.
                \vanderkam helpfully lists the twenty-seven manuscripts he used for his critical edition in the introduction of his commentary where he also notes the additional manuscripts photographed since its publication. See 
                \cite[1:14--16]{vanderkam2018}.}

With the exception of the ``rediscovery'' of the text for European scholarship, the most significant find for the study of \jub was the discovery of several Hebrew fragments among the \dss. These fragments  attest to the work's antiquity and confirmed that the original language of \jub was Hebrew and not Aramaic, as Dillmann originally supposed.%
        \footnote{\cite[90]{dillamnn_jbw1850}. Though, as \vanderkam notes, he seems to have changed his mind later and supposed a Hebrew original. \cite[324]{dillmann_spaw1883}; \cite[1:1 n. 1]{vanderkam2018}.

        The Ethiopic text is a granddaughter translation of the Hebrew through Greek, though no Greek manuscripts of the text have been found. See especially \vanderkam's treatment of the textual history of \jub in \cite*[1--18]{vanderkam1977}. This fact was convincingly demonstrated by Dillmann who observed several Greek forms preserved as transliterations in the Ethiopic text, specifically: \greek{δρῦς}, \greek{βάλανος}, \greek{λίψ}, \greek{σχῖνος}, and \greek{φάραγξ}. See, \cite[88]{dillamnn_jbw1850}. Charles later added \greek{ἡλιου} to the list. \cite[xxx]{charles1902}.

        By the end of the nineteenth century, however, partial copies of \jub had also been uncovered in Latin, which similarly appear to have come through the Greek.
        See
                \cite[15--54]{ceriani1861} and
                \cite{ronsch1874}.
        See also the work of Todd Hanneken and the Jubilees Palimpsest Project (\href{http://jubilees.stmarytx.edu}{jubilees.stmarytx.edu}).

        Finally, although no direct manuscript evidence has been found, \jub scholars posit that a Syriac translation of the Hebrew was made in antiquity. This suggestion is tenuous, but is based on a number of Syriac citations of \jub which do not show any linguistic influence (loan words, etc.) from Greek.
        See especially
                \cite[231--232]{tisserant_rb1921} and 
                \cite[xxix]{charles1902} but also 
                \cite[2:ix--x]{ceriani1861} and 
                \cite[x]{charles1895}.}
Despite all of these finds, however, the Ethiopic text remains the only tradition to preserve \jub in its entirety. Thus, in my treatment of \jub, I will be relying primarily on the Ethiopic text, supplemented by the Hebrew and other versions when available.

% \subsection{Content and Character}
The book of \jub offers a rewriting of the book of Genesis and the first part of Exodus (Gen 1--Exod 12).\autocite[1:17]{vanderkam2018} The bulk of the book (2:1--50:13) is dedicated to recounting these ``biblical'' events in the form of a revelation given to Moses by \yahweh with special concern for halakhic matters and the division of time according to ``weeks'' of years (7-year units) and ``jubilees'' (49-year units). The particulars of the revelation are mediated by the ``\ap'' (1:27; Eth. \ethiopic{መልአከ~፡ ገጽ} [\translit{mal'aka gaṣṣ}]) who dictates the content of the ``heavenly tablets'' (4:5; Eth. \ethiopic{ጽላተ~፡ ሰማይ} [\translit{ṣəllāta samāy}]) to Moses to record what they revealed about the structure and terminus of the cosmos.\autocite{martinez_najman-tigchelaar2012} The treatment of Moses as a scribe places him within a chain of tradition---along with Enoch and Noah---that emphasizes writing and written works as essential sources of tradition and revelation.%
        \footnote{See especially
                \cite[381--388]{najman_jsj1999}.} 

The main body of \jub is framed by a brief prologue and an even briefer epilogue. The prologue offers a short description of the work as an account concerned with the division of time into units of years, weeks, and jubilees given to Moses when he ascended Mt. Sinai to receive the ``stone tablets'':

% !TEX root = dissertation.tex
\begin{ethiopictext}
        \versenum{Prologue}
        ዝንቱ ፡ ነገረ ፡ ኩፋሌ ፡
        መዋዕላተ ፡ ሕግ ፡ ወለስምዕ ፡
        ለግብረ ፡ ዓመታት ፡ ለተሳብዖቶሙ ፡ 
        ለኢዮቤልውሳቲሆሙ ፡ ውስተ ፡ ኲሉ ፡ ዓመታተ ፡ ዓለም ፡
        በከመ ፡ ተናገሮ ፡ ለሙሴ ፡ በደብረ ፡ ሲና ፡
        አመ ፡ ዐርገ ፡ ይንሣእ ፡ ጽላተ ፡ እብን ፡ ሕግ ፡ ወትእዛዝ ፡ 
        በቃለ ፡ አግዚአብሔር ፡ በከመ ፡ ይቤሎ ፡ ይዕርግ ውስተ ፡ ርእሰ ፡ ደብር ።
\end{ethiopictext}

\begin{transliteration}
        \versenum{Prologue}
        zəntu nagara kufālē
        % kufālē                division
        mawāʕəlāta [la-]ḥegg wa-la-səmʕ
        % mawāʕel           'period, era, time' √mʕl 'to pass the day' Les. 603
        % səmʿ               testimony
        la-gəbra ʕāmatāt la-tasābəʕotomu
        % tasābeʿot             tGL perf from √sbʕ; not in the dictionary, but √sbʿ is seven, so… weeks
        la-ʔiyyobēləwəsātihomu wəsta \kw{ə}llu ʕāmatāta ʕālam
        % ˀiyyobēlwelātihomu    ʾiyyobēl is Jubilee, the rest (-welāt) some extended plural?
        ba-kama tanāgaro la-Musē ba-dabra Sinā
        % ba-kama               Just as
        % tanāgaro              Glt perf 3ms + 3ms
        ʔama ʕarga yenšāʔ ṣəllāta ʔəbn---ḥəgg wa-təʔzāz---%
        % ʕarga                 √ʕrg G pf 3ms 'go up'
        % yenšāʔ                √nšʔ G subj 3ms 'raise, accept, receive*' 
        % ṣellē                 pl. ṣellāt    'tablet'
        ba-qāla ʔagziʔabḥēr ba-kama yəbēlo yəʕrəg wəsta rəʔsa dabr.
        % yebēlo                G perf + 3ms
        % yeˤreg                G subj 
\end{transliteration}

\begin{translation}
        \versenum{Prologue}
        These are the words%
        \footnote{Lit. ``This is the word.'' I've chosen to follow VanderKam and others by rendering this construction in the plural based on the probable underlying Hebrew \he{אלה הדברים}. See \cite[125]{vanderkam2018}}
        of the division 
        of the days for the law and for the testimony
        for the event[s] of the years; for their weeks,
        for their Jubilees in all the years of the world
        just as he spoke (them) to Moses on Mount Sinai 
        when he went up to receive the tablets of stone---the law and the commandment---%
        at the command of God, as he had said to him 
        that he should ascend to the top of the mountain.
\end{translation}%
\noindent
The work closes with a terse statement declaring ``Here the account of the division of time is ended'' (\jub 50:13; Eth. 
    \eth{ተፈጸመ~፡ በዝየ~፡ ነገር~፡ ዘኩፋሌ~፡ መዋዕል~።}
        {tafaṣṣama ba-zeyya nagar za-kufālē mawāʕel}).
        % tafaṣṣama     tD fṣm 'to complete'
        % ba-zeyya      here
        % nagar         account, speech, etc.
        % kufālē        division(s)
        % mawāʕel       √mʕl 'to pass the day' here: period, era, time Les. 603
\noindent
It is important to note that this prologue as well as the first chapter of the book of \jub are preserved among the Qumran fragments (specifically \q{4}{216}{}), which represent some of the oldest extant \jub fragments. Thus, this early narrative frame was almost certainly a part of the work in its earliest form and cannot be attributed to a later editor; it is an integral part of the literary shape of the book of \jub. Although superscriptions were often added much later, in this case, we have no reason to doubt that the prologue/superscription and framing narrative of the work were not a part of the most ancient versions.%
        \footnote{See 
                \cite[1:125]{vanderkam2018};
                \cite[25]{vanderkam_metso-etal2010}.}

Thus the work as a whole is presented as a revelation given to Moses by \yahweh, framed by a brief prologue and epilogue which situates the story during Moses's first 40-days atop Mt. Sinai when Moses receives the Tablets of Stone (Exod 24:12).\autocite[1:129]{vanderkam2018}
