% !TEX root = dissertation.tex

\section{Conclusion}

The book of \jub, therefore, can be understood from the perspective of social and cultural memory theory as a participant in the continued process of building up the figure Moses (and the Sinai tradition more generally) as a site of memory. Unlike the \ga which primarily engaged its mnemonic subjects (Lamech, Noah, and Abram) descriptively, the book of \jub explicitly engages in memory construction for the purpose of affecting social practice (halakah). While the \ga may be read as \emph{implicitly} prescriptive, the book of \jub makes efforts to embed halakhic material within the narrative in such a way so as to clearly and \emph{explicitly} instruct the reader, though it stops short of participating in traditional legal discourse, \emph{per se}. Instead the book \emph{as a whole} is characterized as a kind of authoritative revelation which invites the reader to incorporate this new knowledge into their conception of the past in an act of cultural memory construction. The effect of this change on the remembered past is not intended to simply change the reader's intellectual picture of Moses and the Sinai tradition, but to impact how the reader and their society behaved.

Nowhere in \jub are the practical effects of memory construction so acutely felt than with its highly-schematic calendrical and epochal systems. The 364-day calendar---presented as an immutable cosmic absolute---reinforces the idea that the 7-day week and sabbath system form an integral part of the cosmic order. Its ability to form a calendar which did not change year-over-year reinforced sabbath observance by never allowing a holiday to fall on the sabbath. Likewise the system of weeks and jubilees (also built on a heptadic principle) was reinterpreted typologically to connect with the entrance of the people into the Land and invites the reader to infer that the coming jubilee might bring with it not only release for those in debt-bondage, but release from the bondage of foreign occupation and a renewed Israelite state.

The mnemonic processes identifiable in the book of \jub are the same as those found in the books of \chronicles, as well as those in the \ga. In \jub we can see how Moses and the Sinai traditions were productive sites of memory during the \secondtemple period which invited innovation and exerted magnetic effects on related sites of memory such as the jubilee and sabbath cycles, as well as the festival of Shavuot. Moreover, the book of \jub participates in \psgraphical discourses which qualitatively changed the way that the work was read by its audience. These processes have been identified and analyzed within the book of \jub previously, but considering them within the framework of cultural memory provides a lens through which we can identify not just the abstract discursive qualities of the such texts, but also consider how those discourses may have affected the \emph{practice} of \secondtemple Judaism.