\chapter{Chronicles}\label{chronicles}

Scholars of the Hebrew Bible have long observed that the book of
Chronicles is a derivative work of Samuel--Kings which reflects the
concerns and religious sensibilities of the late Persian or Early
Hellenistic Periods. For example, in his classic work on the history of
ancient Israel, Wellhausen begins his treatment of the history of
traditions within Judaism with a lengthy discussion of Chronicles which,
he observes, differs from the history of Israel portrayed in
Samuel--Kings due to the chronological distance of the works and the
intervention of the Priestly Code into the theology of Second Temple
Judaism.\autocite[171--172]{wellhausen1957} Chronicles, according to
Wellhausen, provides a clear example of the way that ancient Israel's
traditions evolved over time. Just as the legal material of the
Hexateuch developed over the centuries, so too the traditions of the
historical books were subject to the changing theologies of later
centuries. While the particulars of the relationship of Samuel--Kings to
Chronicles and the nuances of Priestly influences on the Hebrew Bible
remain subject to scholarly debate, the broad consensus remains that
Chronicles was written sometime in the late Persian or early Hellenistic
periods, heavily utilizing Samuel--Kings as a literary source, and
bearing an ideological \emph{Tendenz} influenced by (though not
identical to) the final layers of the Pentateuch.\autocites[For a
thorough and reasonably recent summary of the \emph{status questionis},
see][72--89]{knoppers2003}[See
also][]{japhet1993}{japhet2009}{braun1986}[and][]{coggins1976}

The scholarly discourse surrounding the formation of the Hebrew Bible
has increasingly turned to memory studies as a robust framework for
describing the processes by which the biblical traditions were produced
and
transmitted.\autocites{wright2014}{blenkinsopp2013}{rogerson2010}{davies2008}{hendel2005}{smith_cbq2002}
And, it is for the same reasons that Wellhausen chose to begin his
treatment of the history of traditions with Chronicles, namely, the
relative security with which scholars have dated Samuel--Kings and the
major Pentateuchal strata\textasciitilde{}vis-a-vis Chronicles, that the
work has likewise played an important role as memory studies has become
an increasingly influential lens tough which to view the biblical
traditions.\autocites{benzvi_st2017}{benzvi-a_evans-williams2013}{benzvi-b_evans-williams2013}
Thus, while Chronicles is not the \emph{latest} book in the Hebrew
Bible, it is uniquely situated at the end of the traditioning process
preserved in the Hebrew Bible. In some ways, therefore, Chronicles gets
the last word on a certain set of traditions surrounding the monarchic
period, in particular those of David, Solomon, and the kings of Judah.

Although it is broadly agreed upon that Chronicles exhibits a
hypertextual\autocite[5]{genette1997} relationship to Samuel--Kings,
treating Chronicles as an exemplar of RwB is somewhat less
common.\autocites[Though, not particularly
\emph{un}common:][]{campbell_zsengeller2014}{zahn_lim-collins2010}{bernstein_textus2005}[Alexander
considers Chronicles to be a ``prototype'' of RwB,
see][100]{alexander_carson-williamson1988} The ambivalence of scholars
towards treating Chronicles as RwB is rooted, unsurprisingly, in the
confusion surrounding the definition of the term (see Chapter 1).
Knoppers, for example, takes special care to treat the question of
whether Chronicles should be understood as RwB in the introduction of
his commentary and notes, from the very beginning, that he will answer
the question based on what he understands to be the essential elements
of RwB.\autocite[ , 129--134]{knoppers2003} He writes:

\begin{quote}
They {[}RwB{]} select from, interpret, comment on , and expand protions
of a particular biblical book (or group of books), addressing
obscurities, contradictions, and other percieved problems with the
source text. Rewritten Bible texts normally emulate the form of the
source text and follow it sequentially. The major intention of such
works seems to be to provide a coherant interpretive reading of the
biblical text.\autocite[130]{knoppers2003}
\end{quote}

Knoppers observes that, while Chronicles exhibits most of the specific
literary moves which RwB is known for (expanding, harmonizing, etc. its
\emph{Vorlage}), the presence of material which is entirely unique to
Chronicles, cannot be attributed to a purely exegetical or explanatory
impulse. In addition to the narrative additions within Chronicles, the
genealogies of Chr 1--9 have no corollary in Samuel--Kings, and can
hardly be considered a rewritten form of the
Penatateuch.\autocite[132]{knoppers2003} Thus, while Knoppers affirms
that certain portions of Chronicles ``may be profitably compared with a
number of rewritten Bible texts,''\autocite[131]{knoppers2003}
ultimately he concludes that Chronicles ``is more than a paraphrase or
literary elaboration of the primary
history''\autocite[134]{knoppers2003} and thus RwB cannot can account
for Chronicles as a whole instead suggesting that it should be treated
as its own work.\autocite[131--134]{knoppers2003}

While Knoppers' assessment is fair given the definition that he supplies
for RwB, I have adopted a broader definition of RwB that accommodates
for Knoppers' reservations. In fact, the rationale for why RwB cannot
account for the complexity of the whole book of Chronicles, according to
Knoppers, is the same basic argument that I have made for why an
\emph{exegetically} focused definition of RwB is insufficient to account
for the complexity even of the literature that scholars
\emph{traditionally} consider to be RwB. Knoppers' criticisms,
therefore, could just as easily be applied to Josephus'
Jewish Antiquities or to Jubilees (or, I would argue, even to
Genesis Apocryphon) and why I have suggested that memory studies may
provide a useful model for discussing RwB.

This chapter will focus on the ways that social and cultural memory
theory can be used to explain the processes through which Chronicles
emended, harmonized, augmented, and omitted traditions in Samuel--Kings
and the way that those traditions shaped and were shaped by the social
location of their author(s). In particular, I will focus on the mnemonic
site of the Temple in Jerusalem and the discourse around its foundation.
\textless{}\textless{} CLARIFY THIS WHEN YOU FINISH THE CHAPTER.

\section{David's Census and Araunah/Ornan's Threshing
Floor}\label{davids-census-and-araunahornans-threshing-floor}

The account of David's census in 2 Samuel and its hypertext in 1
Chronicles provides a clear example of the way that present ideologies
play a significant role in the reception of cultural memory by a society
and its (re)construction in the social memory. Both 2 Samuel 24 and 1
Chronicles 21 recount the story of David's census, the punishment that
Yahweh inflicts on David for doing so, and David's penitential offerings
at the threshing floor of Araunah/Ornan. \textless{}\textless{} More
here. Perhaps talk about ``Nodes'' \textgreater{}\textgreater{}

\subsection{2 Samuel 24}\label{samuel-24}

The narrative in 2 Sam 24 begins by stating that Yahweh had become angry
with Israel and ``incited David against them'' (\emph{wayyāseṯ ʾeṯ-dāwiḏ
bāhem}). David orders the census and is stricken with guilt (it is not
clear in the text why this was a bad thing). Yahweh offers him a choice
through the prophet Gad between three years of famine, three months of
pursuit, or three days of pestilence. David consents to the pestilence,
and, after seeing its destructive force in the form of an angel, pleads
with Yahweh to punish him personally, and not to continue harming the
people. The prophet Gad instructs David to make offerings to Yahweh
where he saw the angel, at the threshing floor of Araunah, the Jebusite.
David purchases the threshing floor and Araunah's livestock, builds an
altar to Yahweh, and makes his offerings.

The purpose of this pericope within Samuel is not clear and its
ostensible connection to 2 Sam 21:1--14 is merely
thematic.\autocite[509]{mccarter1984} The internal logic of the
narrative is no less problematic: the text offer no explanation for
Yahweh's anger, nor does it explain the connection between the census
and the pestilence. While Yahweh's initial anger can be accounted for
theologically (gods do not need reasons to be angry), the connection
between the census and the pestilence seems to be built on some tacit
correlation between the two. McCarter, following a number of earlier
studies, for example, has suggested that the danger in taking a census
is found in the connection of between the census and the ritual failure.
Drawing comparisons to Num 1:2--3 and Exod 30:11--14, McCarter notes
that censuses in the ancient world were generally performed for military
purposes (Num 1:2--3) or to levy funds (Exod 30:11--14). Since the
purpose of David's census seems to be a measurement of military might,
it likely would have involved enrolling fighting-aged men in the
military (Cf. Num 1:2--3), thereby forcing those enrolled to remain
ritual purity in accordance with military standards. The likelihood that
not all of the soldiery would be able to maintain this level of purity
would have been high and may have been thought to invite divine
retribution. On the other hand, while the census instructions in Exod
30:11--14 were not for the purposes of the military, they did require
that each person ``give a ransom for his life to Yahweh'' (\emph{nāṯənû
ʾı̂š kōp̄er nap̄šô la-yhwh}), ``lest a plague come upon them at their
registration'' (\emph{lōʾ-yihyeh ḇāhem neḡep̄ bip̄qōḏ ʾōṯām}). The
explicit connection between registration (√\emph{pqd}) and plague
(\emph{neḡep̄}) again passes without explanation, but again, the
stipulation that those registered must provide a ransom (a half-shekel
to the sanctuary and a half-shekel as ``an offering to Yahweh''
{[}\emph{tərûmāh la-yhwh}{]}) opens the possibility for ritual failure.
In both cases (maintaining ritual purity and making ransoms), the
near-certain failure of some percentage of the population provide a
plausible explanation for the association of people-counting and plague.

At the literary level, then, the function of the pericope could---as
with much of the so-called History of David's Rise (HDR)---be understood
as an \emph{apologia} for David's putative decision to ask for a census
(against Joab's advice) and a subsequent calamity perceived to be caused
by it. If the story originated from a time when it was well-known that
David had commanded the census, the story provides an explanation for
why David made the decision (at Yahweh's prompting) despite the
(apparent) risk of plague that accompanied numbering the
people.\autocite[518]{mccarter1984} Furthermore, David's
self-sacrificial posture on display in his willingness to take on the
punishment personally and by his willingness to purchase Araunah's
threshing floor (at full price) and make the appropriate offerings to
Yahweh.

\subsection{1 Chronicles 21}\label{chronicles-21}

While the broad strokes of the 2 Sam 24 and 1 Chron 21 remain quite
similar, a few relatively small modifications to the account preserved
in 1 Chron 21 dramatically alters the literary and theological
significance of the tale.

\subsubsection{Yahweh or (a) Satan?}\label{yahweh-or-a-satan}

The Chronicler's narrative most noticable change is seen in the shift in
explaining who incited David to take the census in the first place.
While 2 Sam 24 attributes this act to Yahweh, 1 Chron 21 introduces a
new figure to the story referred to as \emph{śāṭān}. 1 Chron 21:1 reads:

\begin{quote}
\emph{way-yaʿămōḏ śāṭān ʿal-yiśrāʾēl way-yāseṯ ʾeṯ-dāwı̂ḏ limnôṯ
ʾeṯ-yiśrāʾēl}

{[}a{]} \emph{śāṭān} stood up against Israel and he incited David to
count Israel.
\end{quote}

Scholars remain divided over whether \emph{śāṭān} should be understood
as a simple indefinite noun ``an adversary,''
\autocites{stokes_jbl2009}[114--117]{japhet2009}[370--390]{japhet1993}
or whether the absence of the definite article indicates that by the
time of the Chronicler, Satan referred to a malevolent spirit which
prefigured the more developed, personified ``Satan'' found in the New
Testament.\autocite[4--5]{rollston_keith-stuckenbruck2016} The most
common usage of the term \emph{śāṭān} in the Hebrew Bible refers to
human adversaries and accusers generally (see, Num 22:22, 32; 1 Sam
29:4, 2 Sam 19:23; 1 Kgs 5:18, 11:14, 23, 25; Ps 38:21, 71:13, 109:4, 6,
20, 29). However, the figure \emph{haś-śāṭān} (wth definite article) in
both the prologue to Job (chs. 1--2) and Zech 3:1--2 appears as a
celestial figure to whom Yahweh speaks directly.\footnote{This notion is
  more clear in Job, where haś-śāṭān is described in the heavenly courts
  and is described as having supernatural powers over the health and
  prosperity of those on the Earth. On the other hand, the reference in
  Zechariah is somewhat ambiguous. Zech 3:1 reads: \emph{way-yarʾēnı̂
  ʾeṯ-yəhôšuaʿ hak-kōhēn hag-gāḏôl ʿōmēḏ lip̄nê malʾaḵ yhwh wə-haś-śāṭān
  ʿōmēḏ ʿal-yəmı̂nô ləśiṭnô}, ``And he showed me Joshua, the high priest
  standing before the angel of Yahweh, and \emph{haś-śāṭān} was standing
  on his right (side) to accuse him.'' The antecedent of ``his'' in
  ``his right(side)'' is unclear. If ``his'' refers to the \emph{malʾaḵ
  yhwh}, then haś-śāṭān likely refers to some kind of spiritual being.
  However, it is possible that ``his'' refers to Joshua, and that
  haś-śāṭān should be understood as a human adversary.} Proponents of
reading \emph{śāṭān} as the personal name of a malevolent spirit argue
that the absence of the definite article indicates that the idea of
\emph{the śāṭān} of Job and Zechariah had evolved into a fully
personified Satan by the time of the
Chronicler.\autocites[107]{coggins1976}[Rollston also finds this reading
compelling, though, not without difficulties.
See,][4--5]{rollston_keith-stuckenbruck2016} Additionally, while the LXX
hails from a slightly later chronological horizon than Chronicles, it is
worth noting that the translator used the indefinite substantive
\emph{diabolos} to translate \emph{śāṭān}---the same term used in Job
and Zechariah (also, Ps 108:6) \emph{with} a definite article---which
gives some indication that, in the mind of the translator, these
passages likely referred to the same entity.\footnote{Elsewhere the LXX
  renders the nominal forms of \emph{śāṭān} with the feminine
  \emph{diabolē} or, in the case of 1 Kgs 11:14, simply in
  transliteration as \emph{satan}. It should be noted, however, that
  Esther 7:4 and 8:1 render the Hebrew √ṣrr as the masculine
  \emph{diabolos} as well.}

Critics of this view, however, have pointed to the fact that in other
cases in the Hebrew Bible, generic nouns that are treated as personal
names or titles often \emph{do} retain the definite
article.\autocites[114--117]{japhet2009}[370--390]{japhet1993} Japhet,
for example, notes that direct references to the Canaanite deity Baʿal
are always accompanied by the definite article. In every instance, the
name/title \emph{baʿal} is made grammatically definite whether by adding
the definite article, through the use of pronominal suffixes or being in
construct with an explicitly definite
noun.\autocites[115]{japhet2009}[citing][§126d]{geseniuskautzsch1910} In
such a case, \emph{śāṭān} should simply be understood as an indefinite
noun, ``an accuser'' and may be understood as a human antagonist of
David.\autocites{stokes_jbl2009}[114--117]{japhet2009}[370--390]{japhet1993}

Regardless of how one understands \emph{śāṭān} to be functioning, the
author of Chronicles plainly understood the mechanisms at work---whether
supernatural or interpersonal---differently than the author of 2 Sam 24.
Chronicles, therefore, shifts the incitement away from Yahweh and
removes any reference to the deity's anger prior to the census, thereby
simplifying the rationale for David's guilt and Yahweh's retribution.
The resulting narrative removes the problematic ``entrapment'' of David
and absolves Yahweh from seeming so ``mercurial''
\autocite[4]{rollston_keith-stuckenbruck2016}.

\subsubsection{The Threshing Floor of Araunah/Ornan and the Temple of
Solomon}\label{the-threshing-floor-of-araunahornan-and-the-temple-of-solomon}

The pericope in 1 Chron 21 likewise ends with a significant deviation
from its hypotext preserved in 2 Sam 24. While both accounts describe
the angel of Yahweh relenting from his destructive activities at the
threshing floor of Araunah (=Ornan in 1 Chron 21) the Jebusite, the
significance of the site as the future location of the Temple of Yahweh
in Jerusalem is made explicit only in the Chronicler's account. The
Chronicler makes the explicit assertion that the threshing floor of
Ornan the Jebusite would become the spot for the construction of the
temple of Solomon by putting it on the lips of David himself in 1 Chr
22:1, stating :

\begin{quote}
\emph{wayyōʾmer dāwı̂ḏ zeh hûʾ bêṯ yhwh hāʾĕlōhı̂m wəzeh-mizbēaḥ ləʿōlāh
ləyiśrāʾēl}

Then David said, ``This is the Temple of Yahweh, God, and this altar for
burnt offerings is for Israel''
\end{quote}

The declaration is fulfilled in 2 Chr 3:1 when Solomon begins to
construct the temple:

\begin{quote}
\emph{wayyāḥel šəlōmōh liḇnôṯ ʾeṯ-bêṯ-yhwh bı̂rûšālaim bəhar hammôrı̂ā
ʾăšer nirʾāh ləḏāwı̂ḏ ʾāḇı̂hû ʾăšer hēḵı̂n bimqôm dāwı̂ḏ bəḡōren ʾornān
ha-yəḇûsı̂}

Solomon began to build the temple of Yahweh in Jerusalem on mount
Moriah, where Yahweh appeared to David, his father, where David had
designated, at the threshing floor of Ornan the Jebusite.
\end{quote}

The question of whether the author of the account in 2 Sam 24 understood
this story to be an etiological account of the founding of the Jerusalem
temple or whether this detail was the invention of the Chronicler,
however, is not clear from the biblical text. In fact, unlike the book
of Chronicles which gives the precise location of the temple of Solomon,
Isaac Kalimi has suggested that the reason for its neglect in the
Deuteronomistic History can be attributed to the assertion that the
location of the temple in Jerusalem (as well as all the other temples in
the Hebrew Bible, whose locations are not specified) was common
knowledge and thus such that ``there was no need to mention
them.''\autocite[355--356]{kalimi_htr1990}
