\chapter{The Chronicler's Temple: Social Memory and Magnetism in
Chronicles}\label{the-chroniclers-temple-social-memory-and-magnetism-in-chronicles}

Scholars of the Hebrew Bible have long observed that the book of
Chronicles is a derivative work of Samuel--Kings reflecting the concerns
and religious sensibilities of the late Persian or Early Hellenistic
Periods. For example, in his classic work on the history of ancient
Israel, Julius Wellhausen begins his treatment of the history of
traditions within Judaism with a lengthy discussion of the book of
Chronicles. In his treatment of the book, Wellhausen argues that the
history of Israel as portrayed in Chronicles differs from that portrayed
in Samuel--Kings due to the chronological distance of the works and the
intervention of the Priestly Code into the theology of Second Temple
Judaism.\autocites[171--172]{wellhausen1957}[See
also][]{wright_ulrich-wright1992} Chronicles, according to Wellhausen,
provides a clear example of the way that ancient Israel's traditions
evolved over time. Just as the legal material of the Hexateuch developed
over the centuries, so too the traditions of the historical books were
subject to the changing theologies of later centuries. While the
particulars of the relationship of Samuel--Kings to Chronicles and the
nuances of priestly influences on the Hebrew Bible remain subject to
scholarly debate, the broad consensus remains that 1) Chronicles was
written sometime in the late Persian or early Hellenistic periods, 2) it
heavily utilized Samuel--Kings as a literary source, and 3) it bears an
ideological \emph{Tendenz} influenced by (though not identical to) the
final layers of the Pentateuch.\autocites[For a thorough and reasonably
recent summary of the \emph{status questionis},
see][72--89]{knoppers2003}[See
also][]{japhet1993}{japhet2009}{braun1986}[and][]{coggins1976}

The scholarly discourse surrounding the formation of the Hebrew Bible
has increasingly turned to memory studies as a robust framework for
describing the processes by which the biblical traditions were produced
and
transmitted.\autocites{wright2014}{blenkinsopp2013}{rogerson2010}{davies2008}{hendel2005}{smith_cbq2002}
Wellhausen chose to begin his treatment of the history of traditions
with Chronicles because of the relative security with which scholars are
able to date Samuel--Kings and the major Pentateuchal strata vis-à-vis
Chronicles. For the very same reason, Chronicles likewise has played an
important role in early applications of memory theory within biblical
studies.\autocites{benzvi_st2017}{benzvi-a_evans-williams2013}{benzvi-b_evans-williams2013}.
While Chronicles is not the \emph{latest} book in the Hebrew Bible, it
is uniquely situated at the end of the traditioning process preserved in
the Hebrew Bible. In some ways, therefore, Chronicles gets the last word
on a certain set of traditions surrounding the monarchic period, in
particular those of David, Solomon, and the kings of Judah.

Although it is broadly agreed upon that Chronicles exhibits a
hypertextual relationship to Samuel--Kings, treating Chronicles as an
exemplar of Rewritten Bible (RwB) is somewhat less
common.\autocites[Though, not particularly
\emph{un}common:][]{campbell_zsengeller2014}{zahn_lim-collins2010}{bernstein_textus2005}[Alexander
considers Chronicles to be a ``prototype'' of RwB,
see][100]{alexander_carson-williamson1988}[I have adopted the
terminology of hyper-/hypotext from Genette. In this case, to say that
Chronicles is a ``hypertext'' of Samuel--Kings is to say that it is
derivative, but not a commentary on Samuel--Kings. See][5]{genette1997}
The ambivalence of scholars towards treating Chronicles as RwB is
rooted, unsurprisingly, in the confusion surrounding the definition of
the term (see Chapter 1). Knoppers, for example, takes special care to
treat the question of whether Chronicles should be understood as RwB in
the introduction of his commentary and notes, from the very beginning,
that he will answer the question based on what he understands to be the
essential elements of RwB.\autocite[ , 129--134]{knoppers2003} He
writes:

\begin{quote}
They {[}RwB{]} select from, interpret, comment on, and expand portions
of a particular biblical book (or group of books), addressing
obscurities, contradictions, and other perceived problems with the
source text. Rewritten Bible texts normally emulate the form of the
source text and follow it sequentially. The major intention of such
works seems to be to provide a coherant interpretive reading of the
biblical text.\autocite[130]{knoppers2003}
\end{quote}

Knoppers observes that, while Chronicles exhibits most of the specific
literary moves which RwB is known for (expanding, harmonizing, and
augmenting its \emph{Vorlage}), the presence of material which is
entirely unique to Chronicles cannot be attributed to a purely
exegetical or explanatory impulse. In addition to the narrative
additions within Chronicles, the genealogies of 1 Chr 1--9 have no
corollary in Samuel--Kings, and can hardly be considered a rewritten
form of the Penatateuch.\autocite[132]{knoppers2003} Thus, while
Knoppers affirms that certain portions of Chronicles ``may be profitably
compared with a number of rewritten Bible
texts,''\autocite[131]{knoppers2003} ultimately he concludes that
Chronicles ``is more than a paraphrase or literary elaboration of the
primary history''\autocite[134]{knoppers2003} and thus RwB cannot can
account for Chronicles as a whole, instead suggesting that it should be
treated as its own work.\autocite[131--134]{knoppers2003}

While Knoppers' assessment is fair given the definition that he supplies
for RwB, I have adopted a broader definition of RwB that accommodates
for Knoppers' reservations. The rationale for why RwB cannot account for
the complexity of the whole book of Chronicles, according to Knoppers,
is the same basic argument that I have made for why an
\emph{exegetically} focused definition of RwB is insufficient to account
for the complexity even of the literature that scholars
\emph{traditionally} consider to be RwB. Knoppers' criticisms,
therefore, could just as easily be applied to Josephus'
Jewish Antiquities or to Jubilees (or, I would argue, even to
Genesis Apocryphon) and why I have suggested that memory studies may
provide a useful model for discussing RwB.

This chapter will focus on the ways that social and cultural memory
theory can be applied as a model for explaining the processes by which
the Chronicler\footnote{Throughout this chapter I will refer to ``the
  Chronicler'' to refer simply to the the author or authors who are
  responsible for the book of Chronicles and not, as others have used
  the term, to refer to the common author of Chronicles, Ezra and
  Nehemiah.} utilized traditions in Samuel--Kings and the way that those
traditions shaped and were shaped by the social location of their
author(s). In particular, I will focus on the memory of the Temple in
Jerusalem and the way that the figure David---as portrayed in
Chronicles---functioned together with the temple as ``magnetic'' sites
of memory within Second Temple Judaism.\autocite[73]{benzvi_st2017}

\section{David's Census and Araunah/Ornan's Threshing
Floor}\label{davids-census-and-araunahornans-threshing-floor}

The account of David's census in 2 Samuel and its hypertext in 1
Chronicles provides a clear example of the way that present ideologies
play a significant role in the reception of cultural memory by a society
and its (re)construction in the social memory. Both 2 Samuel 24 and 1
Chronicles 21 recount the story of David's census, the punishment that
Yahweh inflicts on David for doing so, and David's penitential offerings
at the threshing floor of Araunah/Ornan.

\subsection{2 Samuel 24}\label{samuel-24}

The narrative in 2 Sam 24 begins by stating that Yahweh had become angry
with Israel and ``incited David against them'' (\emph{wayyāseṯ ʾeṯ-dāwiḏ
bāhem}). David orders the census and is stricken with remorse (it is not
clear in the text why this was a bad thing). Yahweh offers him a choice
through the prophet Gad between three years of famine, three months of
pursuit, or three days of pestilence. David consents to the pestilence,
and, after seeing its destructive force in the form of an angel, pleads
with Yahweh to punish him personally, and not to continue harming the
people. The prophet Gad instructs David to make offerings to Yahweh
where he saw the angel, at the threshing floor of Araunah, the Jebusite.
David purchases the threshing floor and Araunah's livestock, builds an
altar to Yahweh, and makes his offerings.

The purpose of this pericope within Samuel is not clear and its
ostensible connection to 2 Sam 21:1--14 is merely
thematic.\autocite[509]{mccarter1984} The internal logic of the
narrative is no less problematic: the text offer no explanation for
Yahweh's anger, nor does it explain the connection between the census
and the pestilence. While Yahweh's initial anger can be accounted for
theologically (gods do not need reasons to be angry), the connection
between the census and the pestilence seems to be built on some tacit
correlation between the two. McCarter, following a number of earlier
studies, for example, has suggested that the danger in taking a census
is found in the connection between the census and ritual failure.
Drawing comparisons to Num 1:2--3 and Exod 30:11--14, McCarter notes
that censuses in the ancient world were generally performed for military
purposes (Num 1:2--3) or to levy funds (Exod 30:11--14). Since the
purpose of David's census seems to be a measurement of military might,
it likely would have involved enrolling fighting-aged men in the
military (Cf. Num 1:2--3), thereby forcing those enrolled to remain
ritually pure in accordance with military standards. The likelihood that
not all of the soldiery would be able to maintain this level of purity
would have been high and may have been thought to invite divine
retribution. On the other hand, while the census instructions in Exod
30:11--14 were not for the purposes of the military, they did require
that each person ``give a ransom for his life to Yahweh'' (\emph{nāṯənû
ʾı̂š kōp̄er nap̄šô la-yhwh}), ``lest a plague come upon them at their
registration'' (\emph{lōʾ-yihyeh ḇāhem neḡep̄ bip̄qōḏ ʾōṯām}). The
explicit connection between registration (√\emph{pqd}) and plague
(\emph{neḡep̄}) again passes without explanation, but again, the
stipulation that those registered must provide a ransom (a half-shekel
to the sanctuary and a half-shekel as ``an offering to Yahweh''
{[}\emph{tərûmāh la-yhwh}{]}) opens the possibility for ritual failure.
In both cases (maintaining ritual purity and making ransoms), the
near-certain failure of some percentage of the population provides a
plausible explanation for the association of people-counting and plague.

At the literary level, then, the function of the pericope could---as
with much of the so-called History of David's Rise (HDR)---be understood
as an \emph{apologia} for David's putative decision to ask for a census
(against Joab's advice) and a subsequent calamity perceived to be caused
by it. If the story originated from a time when it was well-known that
David had commanded the census, the story provides an explanation for
why David made the decision (at Yahweh's prompting) despite the
(apparent) risk of plague that accompanied numbering the
people.\autocite[518]{mccarter1984} Furthermore, David's
self-sacrificial posture is put on display through his willingness to
take on the punishment personally and by his willingness to purchase
Araunah's threshing floor (at full price) and make the appropriate
offerings to Yahweh.

\subsection{1 Chronicles 21}\label{chronicles-21}

While the broad strokes of 2 Sam 24 and 1 Chr 21 remain quite similar, a
few relatively small modifications found in 1 Chr 21 dramatically alter
the literary and theological significance of the tale.

\subsubsection{Yahweh or (a) Satan?}\label{yahweh-or-a-satan}

The most noticeable change made in the Chronicler's narrative is seen in
who incited David to take the census in the first place. While 2 Sam 24
attributes this act to Yahweh, 1 Chr 21 introduces a new figure to the
story referred to as \emph{śāṭān}. 1 Chr 21:1 reads:

\begin{quote}
\emph{way-yaʿămōḏ śāṭān ʿal-yiśrāʾēl way-yāseṯ ʾeṯ-dāwı̂ḏ limnôṯ
ʾeṯ-yiśrāʾēl}

{[}a{]} \emph{śāṭān} stood up against Israel and he incited David to
count Israel.
\end{quote}

Scholars remain divided over whether \emph{śāṭān} should be understood
as a simple indefinite noun ``an adversary,''
\autocites{stokes_jbl2009}[114--117]{japhet2009}[370--390]{japhet1993}
or whether the absence of the definite article indicates that by the
time of the Chronicler, Satan referred to a malevolent spirit which
prefigured the more developed, personified ``Satan'' found in the New
Testament.\autocite[4--5]{rollston_keith-stuckenbruck2016} The most
common usage of the term \emph{śāṭān} in the Hebrew Bible refers to
human adversaries and accusers generally (see, Num 22:22, 32; 1 Sam
29:4, 2 Sam 19:23; 1 Kgs 5:18, 11:14, 23, 25; Ps 38:21, 71:13, 109:4, 6,
20, 29). However, the figure \emph{haś-śāṭān} (with definite article) in
both the prologue to Job (chs. 1--2) and Zech 3:1--2 appears as a
celestial figure to whom Yahweh speaks directly.\footnote{This notion is
  more clear in Job, where \emph{haś-śāṭān} is described in the heavenly
  courts and is described as having supernatural powers over the health
  and prosperity of those on the Earth. On the other hand, the reference
  in Zechariah is somewhat ambiguous. Zech 3:1 reads: \emph{way-yarʾēnı̂
  ʾeṯ-yəhôšuaʿ hak-kōhēn hag-gāḏôl ʿōmēḏ lip̄nê malʾaḵ yhwh wə-haś-śāṭān
  ʿōmēḏ ʿal-yəmı̂nô ləśiṭnô}, ``And he showed me Joshua, the high priest
  standing before the angel of Yahweh, and \emph{haś-śāṭān} was standing
  on his right (side) to accuse him.'' The antecedent of ``his'' in
  ``his right(side)'' is unclear. If ``his'' refers to the \emph{malʾaḵ}
  Yahweh, then \emph{haś-śāṭān} likely refers to some kind of spiritual
  being. However, it is possible that ``his'' refers to Joshua, and that
  \emph{haś-śāṭān} should be understood as a human adversary.}
Proponents of reading \emph{śāṭān} as the personal name of a malevolent
spirit argue that the absence of the definite article indicates that the
idea of \emph{the śāṭān} of Job and Zechariah had evolved into a fully
personified Satan by the time of the
Chronicler.\autocites[216--217]{braun1986}[107]{coggins1976}[Rollston
also finds this reading compelling, though, not without difficulties.
See,][4--5]{rollston_keith-stuckenbruck2016} Additionally, while the LXX
hails from a slightly later chronological horizon than Chronicles, it is
worth noting that the translator used the indefinite substantive
\emph{diabolos} to translate \emph{śāṭān}---the same term used in Job
and Zechariah (also, Ps 108:6) \emph{with} a definite article---which
gives some indication that, in the mind of the translator, these
passages likely referred to the same entity.\footnote{Elsewhere the LXX
  renders the nominal forms of \emph{śāṭān} with the feminine
  \emph{diabolē} or, in the case of 1 Kgs 11:14, simply in
  transliteration as \emph{satan}. It should be noted, however, that
  Esther 7:4 and 8:1 render the Hebrew √ṣrr as the masculine
  \emph{diabolos} as well.}

Critics of this view, however, have pointed to the fact that in other
cases in the Hebrew Bible, generic nouns that are treated as personal
names or titles often \emph{do} retain the definite
article.\autocites[114--117]{japhet2009}[370--390]{japhet1993} Japhet,
for example, notes that direct references to the Canaanite deity Baʿal
are always accompanied by the definite article. In every instance, the
name/title \emph{baʿal} is made grammatically definite whether by adding
the definite article, pronominal suffixes, or being in construct with an
explicitly definite noun.\autocites[115]{japhet2009}[citing][§126d]{gkc}
In such a case, \emph{śāṭān} should simply be understood as an
indefinite noun, ``an accuser'' and may be understood as a human
antagonist of
David.\autocites{stokes_jbl2009}[114--117]{japhet2009}[370--390]{japhet1993}

Regardless of how one understands \emph{śāṭān} to be functioning, the
author of Chronicles plainly understood the mechanisms at work---whether
supernatural or interpersonal---differently than the author of 2 Sam 24.
Chronicles, therefore, shifts the incitement away from Yahweh and
removes any reference to the deity's anger prior to the census, thereby
simplifying the rationale for David's guilt and Yahweh's retribution.
The resulting narrative removes the problematic ``entrapment'' of David
and absolves Yahweh from seeming so ``mercurial.''
\autocite[4]{rollston_keith-stuckenbruck2016}

\subsubsection{The Threshing Floor of Araunah/Ornan and the Temple of
Solomon}\label{the-threshing-floor-of-araunahornan-and-the-temple-of-solomon}

The pericope in 1 Chr 21 likewise ends with a significant deviation from
its hypotext preserved in 2 Sam 24. While both accounts describe the
angel of Yahweh relenting from his destructive activities at the
threshing floor of Araunah (=Ornan in 1 Chr 21) the Jebusite, the
significance of the site as the future location of the Temple of Yahweh
in Jerusalem is made explicit only in the Chronicler's account. The
Chronicler asserts that the threshing floor of Ornan the Jebusite would
become the spot for the construction of the Solomon's Temple by putting
it on the lips of David himself in 1 Chr 22:1, stating :

\begin{quote}
\emph{wayyōʾmer dāwı̂ḏ zeh hûʾ bêṯ yhwh hāʾĕlōhı̂m wəzeh-mizbēaḥ ləʿōlāh
ləyiśrāʾēl}

Then David said, ``This is the Temple of Yahweh God, and this altar for
burnt offerings is for Israel''
\end{quote}

The declaration is fulfilled in 2 Chr 3:1 when Solomon begins to
construct the temple:

\begin{quote}
\emph{wayyāḥel šəlōmōh liḇnôṯ ʾeṯ-bêṯ-yhwh bı̂rûšālaim bəhar hammôrı̂ā
ʾăšer nirʾāh ləḏāwı̂ḏ ʾāḇı̂hû ʾăšer hēḵı̂n bimqôm dāwı̂ḏ bəḡōren ʾornān
ha-yəḇûsı̂}

Solomon began to build the temple of Yahweh in Jerusalem on mount
Moriah, where Yahweh appeared to David, his father, where David had
designated, at the threshing floor of Ornan the Jebusite.
\end{quote}

The question of whether the author of the account in 2 Sam 24 understood
this story to be an etiological account of the founding of the Jerusalem
temple or whether this detail was the invention of the Chronicler,
however, is not clear from the biblical text. Unlike the book of
Chronicles which gives the precise location of the temple of Solomon,
Samuel--Kings makes no specific claims about the location of the temple.
Mordechai Cogan has argued that this silence is was rooted in the desire
of Deuteronomistic school to forfend the suggestion that the site of the
temple was determined by an historical event, rather than the divine
edict.\autocite[307]{cogan_tarbiz1986} Thus Cogan seems to imagine that
reference to Araunah's threshing floor must have been removed from the
account of the temple's construction in 1 Kgs 6--7 by the
Deuteronomistic editor in order to maintain this
principle.\autocite[307]{cogan_tarbiz1986} On the other hand, Isaac
Kalimi has suggested that the reason for this neglect in 1 Kgs 6--7 can
be attributed to the assertion that the location of the temple in
Jerusalem (as well as all the other temples in the Hebrew Bible, whose
locations are not specified) was so well known that ``there was no need
to mention them.''\autocite[I would, however, make the observation that,
depending on how one dates the account in 1 Kng 6--7, it may be the case
that an \emph{exilic} author genuinely did not know the precise location
of the temple. However, Kalimi also points out that other ANE temple
building texts often neglect to specify the precise location of their
subjects. On this point, I would also hasten to add that monumental
inscriptions should be treated separately, since the location of the
inscription, ostensibly, \emph{would be} the location of the
temple.][355--356]{kalimi_htr1990} Furthermore, Kalimi suggests that the
selection of Araunah's threshing floor as the future location of the
temple was made sufficiently clear by the account of 2 Sam 24 thereby
obviating the need to repeat the location in 1
Kings.\autocite[357]{kalimi_htr1990} The Chronicler, therefore, merely
makes explicit that which the Deuteronomistic Historian meant to imply.

I am not at all convinced by either of the above arguments. Instead, I
suspect that the construction account in 1 Kings makes no mention of
Araunah's threshing floor because, at the time of its composition, the
connection had not been made. The significance of threshing floors as
preferred locations for theophanies has been documented
elsewhere\autocites[McCarter points to Jdgs 6:37 among other, less
clear, examples from the Hebrew Bible. He also notes their significance
in the Ugaritic literature (KTU 1 17.5.4ff;
19.1.19ff)][511--512]{mccarter1984}[See also][]{waters2015} and the fact
that 2 Sam 24 recounts a sort of theophany may be enough to account for
its mention in the narrative.

Although it is not clear whether the connection was made originally by
the Chronicler himself or whether it was an inherited tradition, the
rationale for making Ornan's threshing floor the site of the temple
likely grew from an uneasiness with the notion that David built and
utilized an altar at a secondary location (viz., a place that was not
the tabernacle). At the same time, the Chronicler understood that the
future location of the temple would be in Jerusalem and had to deal with
the fact that David not only made the offerings, but did so at the
command of Yahweh. To resolve these tensions, the Chronicler was able to
assert that David, indeed, made sacrifices outside the tabernacle, but
with his offerings, he laid the groundwork for the construction of the
temple.

\section{Constructing the Memory of the Chronicler's
Temple}\label{constructing-the-memory-of-the-chroniclers-temple}

Having addressed the particular ways and specific rationales that the
Chronicler likely employed while composing 1 Chr 21, we may now turn to
the topic of social memory and the ways that this theoretical framework
can be utilized to explore the phenomenon of biblical rewriting. While
the individual changes and additions made by the Chronicler may, on
their own, provide some insight into the ideology of the Chronicler and
his position toward a few theological topics, treating the Chronicler's
work as social memory attempts to take this a step further. Beyond
questioning the intention of the author, social memory theory seeks to
explore the ways that received traditions may have functioned at the
social level as a part of a community's essential project of identity
building and maintenance.\autocite[Ben Zvi puts it well: ``Everything is
supposed to be forgotten except that which is supposed to be remembered
by a group. Social, shared remembering within a group involves and
requires a social effort by the relevant group, but without such an
effort and shared memory, the group itself would cease to
exist.''][70]{benzvi_st2017}

\subsection{Magnetism}\label{magnetism}

In the context of social memory theory, magnetism refers to the quality
of certain core sites of a community's social memory which tend to
accumulate additional significances over time as more ideas and sites of
memory become ``attracted'' to it.\autocites[I have borrowed this term
from Ben Zvi, who has used the term in numerous contexts.
See][]{benzvi_st2017}[also][]{benzvi_edelman-benzvi2013} The analogy
could be extended further by thinking of this magnetism in the context
of the interaction of gravitational forces between bodies in space.
Larger mnemonic nodes, given their greater ``mass,'' attract smaller
bodies to themselves forming orbital systems akin to the way that the
planets of our solar system are held in orbit by the sun and how the
moon is held by the gravitational forces of Earth. As more nodes attach
themselves the relative mass of the core mnemonic node (that is, the
larger ``central'' node) continues to grow in a sort of ``positive
feedback loop.''\autocites[6]{benzvi_edelman-benzvi2013}[Alternatively,
this idea could be modeled using graph theory as a scale-free network
and treating central mnemonic sites as hubs and peripheral sites as
sparse nodes. See,][]{barabasi_science2009} Within the biblical
tradition, perhaps the premier example of such a node within the Hebrew
Bible is Moses. Ben Zvi writes:

\begin{quote}
Not only does he bring together Torah/the bestowal of Torah with the
Exodus, but he also serves as the foundational prophet, scribe, lawgiver
and so on. Even the story of Israel's rejection of YHWH and subsequent
punishment and future reconciliation is directly associated with
him.\autocite[73--74]{benzvi_st2017}
\end{quote}

Using this model, the book of Deuteronomy functions as one means by
which traditions become attached to Moses through the retelling process
and the ability of the author to synthesize lesser sites of memory with
central ideas and figures creating newer, even more significant and
highly connected mnemonic sites.

Both the Jerusalem temple and king David loom large in biblical memory,
so it is no surprise that both of these mnemonic sites function
magnetically. In the case of David, for example, it is commonly argued
that one Elhanan, and not David, was the historical defeater of Goliath
the Gittite, as recorded in 2 Sam 21:19b:

\begin{quote}
\emph{wayyaḵ ʾelḥānān ben-yaʿrê ʾōrəḡı̂m bêṯ hallaḥmı̂ ʾēṯ golyāṯ haggittı̂
wəʿēṣ ḥănı̂ṯô kimnôr ʾōrəḡı̂m}

Elhanan, son of Yaʿre-Oregim (the Bethlehemite), killed Goliath the
Gittite, the shaft of whose spear was like a weaver's beam.
\end{quote}

In this case, it is supposed that the more potent mnemonic site (David)
is connected to the well-known slaying of the Philistine giant at the
expense of Elhanan. The extended narrative of David and Goliath from 1
Sam 17, likewise bears the signs of magnetism. In fact, the name
``Goliath'' only occurs twice in the extended narrative, in vv. 4 and
23, which has caused some scholars to question whether the
identification of the ``giant'' with Goliath was, like David, a
secondary addition. In every other instance throughout the narrative,
the man is referred to simply as ``the Philistine.'' Thus, it is
supposed that the story originally may have been entirely anonymous, and
only later were these figures identified with David and
Goliath.\autocites[For a fuller account of the textual issues
surrounding the main narrative about David and Goliath,
see,][280--309]{mccarter1980}[and][69--77]{mckenzie2000} Thus
McCarter's, observation that ``{[}d{]}eeds of obscure heroes tend to
attach themselves to famous heroes,''\autocite[450]{mccarter1984} though
not rooted in memory theory \emph{per se}, fits our model for
``magnetic'' figures.

Within the story of David's census and the threshing floor of Oruna in 1
Chr 21, several mnemonic sites can be understood as exhibiting magnetic
qualities. First, although the identity of \emph{śāṭān} is not clear in
the text, later treatments of the story, including in the LXX, treat the
character as a malevolent supernatural being. The potency---at least in
the interpretation of later periods---of Satan was able to don this
particular mischief and take credit for the incitement of David to sin.
One might imagine, given \emph{haśśāṭān}'s subordinate position
vis-à-vis Yahweh in the prologue to Job, that the two passages were not
seen to be in direct conflict with one another, whatever the original
intent of the Chronicler might have been. Shifting the incitement of
David away from Yahweh can be explained by the ostensible discomfort (or
confusion) created by the story in 2 Sam 24, but, supposing \emph{śāṭān}
was imagined by the Chronicler to be a supernatural being, we can
account for the Chronicler's choice as an example of magnetism on the
part of Satan.

Second, the identification of Ornan's threshing floor with the future
site of the Solomonic Temple creates an explicit connection between two
very potent mnemonic sites: David and the Temple. The discourse
surrounding the fact that it was Solomon and not David who constructed
the Temple of Yahweh predates even the DtrH, as indicated by 2 Sam 7,
but the degree to which David involves himself in making preparations
for the construction of the temple in 1 Chr 22:2--29:22 indicates that
DtrH did not get the last word on the matter. While 2 Sam 7 provides an
apologia for why it was Solomon, and not David, who built the temple,
Chronicles takes it a step further by attributing the planning and
preparation of the temple's construction to David, leaving only the most
nominal tasks for the temple's ``builder,'' Solomon. The inability of
David's mnemonic gravitas to fully absorb the temple's construction
illustrates the immutability of certain mnemonic sites, particularly
those such as buildings and geographic features, despite the relative
significance of David for the continued maintenance of Judean identity
continued into the Second Temple Period vis-à-vis Solomon.

Finally, after identifying the site of the temple of Solomon with
Ornan's threshing floor, and drawing the mnemonic sites of the temple
and king David even closer together, the Chronicler includes one
additional piece of information to the reader in the description of the
temple's construction in 2 Chr 3:1:

\begin{quote}
\emph{wayyāḥel šəlōmōh liḇnôṯ ʾeṯ-bêṯ-yhwh bı̂rûšālaim bəhar hammôrı̂ā
ʾăšer nirʾāh ləḏāwı̂ḏ ʾāḇı̂hû ʾăšer hēḵı̂n bimqôm dāwı̂ḏ bəḡōren ʾornān
hayəḇûsı̂}

Solomon began to build the temple of Yahweh in Jerusalem \emph{on the
mountain of Moriah} where he appeared to David, his father, the place
which David designated, on the threshing floor of Ornan, the Jebusite.
\end{quote}

Although this is the only reference to the \emph{mountain} of Moriah in
the Hebrew Bible, the \emph{land} of Moriah is mentioned only in Gen 22,
the Aqeda, as the land to which Abraham was to bring Isaac for sacrifice
(on a mountain!).\autocite[358--359]{kalimi_htr1990} The reference to
Moriah appears to be another example of the magnetic quality of core
events in a community's identity. In the same way that David's
sacrificial acts at the threshing floor of Ornan---in the memory of the
Chronicler---prefigured and made acceptable the offerings made by David
there, so too the near-sacrifice of Isaac, by its geographic association
with the foundation of the temple in Jerusalem, becomes a prototype for
the sacrificial cult.\autocite[In fact, Vermes makes this point explicit
and traces the tradition into early Christianity.
See][204--211]{vermes1961}

\section{Conclusions}\label{conclusions}
