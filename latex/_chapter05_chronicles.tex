\chapter{Chronicles}\label{chronicles}

Scholars or the Hebrew Bible have long observed that the book of
Chronicles is a derivative work of Samuel--Kings which reflects the
concerns and religious sensibilities of the late Persian or Early
Hellenistic Periods. For example, in his classic work on the history of
ancient Israel, Wellhausen begins his treatment of the history of
traditions within Judaism with a lengthy discussion of Chronicles which,
he observes, differs from the history of Israel portrayed in
Samuel--Kings due to the chronological distance of the works and the
intervention of the Priestly Code into the theology of Second Temple
Judaism\autocite[171--172]{wellhausen1957} While the particulars of the
relationship of Sam--Kings to Chronicles and the nuances of Priestly
influences on the Hebrew Bible remain subject to scholarly debate, the
broad consensus remains that the book of Chronicles was written sometime
in the late Persian or early Hellenistic periods, heavily utilizing the
books of Samuel and Kings as source material, and bearing an ideological
\emph{Tendenz} influenced by (though not identical to) the final layers
of the Pentateuch.\autocites[For a thorough and reasonably recent
summary of the \emph{status questionis}, see][72--89]{knoppers2003}[See
also][]{japhet1993}{japhet2009}{braun1986}[and][]{coggins1976}

It is unsurprising, therefore, that memory studies has become an
influential lens with which to study the book of Chronicles. Over the
past few decades, the scholarly discourse surrounding the composition of
Chronicles has increasingly turned to memory studies as a robust
framework for describing the processes by which Chronicles and other
biblical traditions were produced and
transmitted.\autocites{benzvi2017}[148--166]{wright2014}{blenkinsopp2013}{benzvi-a_evans-williams2013}{benzvi-b_evans-williams2013}
