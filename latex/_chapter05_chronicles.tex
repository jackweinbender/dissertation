\chapter{Chronicles}\label{chronicles}

Scholars of the Hebrew Bible have long observed that the book of
Chronicles is a derivative work of Samuel--Kings which reflects the
concerns and religious sensibilities of the late Persian or Early
Hellenistic Periods. For example, in his classic work on the history of
ancient Israel, Wellhausen begins his treatment of the history of
traditions within Judaism with a lengthy discussion of Chronicles which,
he observes, differs from the history of Israel portrayed in
Samuel--Kings due to the chronological distance of the works and the
intervention of the Priestly Code into the theology of Second Temple
Judaism.\autocite[171--172]{wellhausen1957} Chronicles, according to
Wellhausen, provides a clear example of the way that ancient Israel's
traditions evolved over time. Just as the legal material of the
Hexateuch developed over the centuries, so too the traditions of the
historical books were subject to the changing theologies of later
centuries. While the particulars of the relationship of Samuel--Kings to
Chronicles and the nuances of Priestly influences on the Hebrew Bible
remain subject to scholarly debate, the broad consensus remains that
Chronicles was written sometime in the late Persian or early Hellenistic
periods, heavily utilizing Samuel--Kings as a literary source, and
bearing an ideological \emph{Tendenz} influenced by (though not
identical to) the final layers of the Pentateuch.\autocites[For a
thorough and reasonably recent summary of the \emph{status questionis},
see][72--89]{knoppers2003}[See
also][]{japhet1993}{japhet2009}{braun1986}[and][]{coggins1976}

The scholarly discourse surrounding the formation of the Hebrew Bible
has increasingly turned to memory studies as a robust framework for
describing the processes by which the biblical traditions were produced
and
transmitted.\autocites{wright2014}{blenkinsopp2013}{rogerson2010}{davies2008}{hendel2005}{smith_cbq2002}
And, it is for the same reasons that Wellhausen chose to begin his
treatment of the history of traditions with Chronicles, namely, the
relative security with which scholars have dated Samuel--Kings and the
major Pentateuchal strata\textasciitilde{}vis-a-vis Chronicles, that the
work has likewise played an important role as memory studies has become
an increasingly influential lens tough which to view the biblical
traditions.\autocites{benzvi_st2017}{benzvi-a_evans-williams2013}{benzvi-b_evans-williams2013}
Thus, while Chronicles is not the \emph{latest} book in the Hebrew
Bible, it is uniquely situated at the end of the traditioning process
preserved in the Hebrew Bible. In some ways, therefore, Chronicles gets
the last word on a certain set of traditions surrounding the monarchic
period, in particular those of David, Solomon, and the kings of Judah.

Although it is broadly agreed upon that Chronicles exhibits a
hypertextual\autocite[5]{genette1997} relationship to Samuel--Kings,
treating Chronicles as an exemplar of RwB is somewhat less
common.\autocites[Though, not particularly
\emph{un}common:][]{campbell_zsengeller2014}{zahn_lim-collins2010}{bernstein_textus2005}[Alexander
considers Chronicles to be a ``prototype'' of RwB,
see][100]{alexander_carson-williamson1988} The ambivalence of scholars
towards treating Chronicles as RwB is rooted, unsurprisingly, in the
confusion surrounding the definition of the term (see Chapter 1).
Knoppers, for example, takes special care to treat the question of
whether Chronicles should be understood as RwB in the introduction of
his commentary and notes, from the very beginning, that he will answer
the question based on what he understands to be the essential elements
of RwB.\autocite[ , 129--134]{knoppers2003} He writes:

\begin{quote}
They {[}RwB{]} select from, interpret, comment on , and expand protions
of a particular biblical book (or group of books), addressing
obscurities, contradictions, and other percieved problems with the
source text. Rewritten Bible texts normally emulate the form of the
source text and follow it sequentially. The major intention of such
works seems to be to provide a coherant interpretive reading of the
biblical text.\autocite[130]{knoppers2003}
\end{quote}

Knoppers observes that, while Chronicles exhibits most of the specific
literary moves which RwB is known for (expanding, harmonizing, etc. its
\emph{Vorlage}), the presence of material which is entirely unique to
Chronicles, cannot be attributed to a purely exegetical or explanatory
impulse. In addition to the narrative additions within Chronicles, the
genealogies of Chr 1--9 have no corollary in Samuel--Kings, and can
hardly be considered a rewritten form of the
Penatateuch.\autocite[132]{knoppers2003} Thus, while Knoppers affirms
that certain portions of Chronicles ``may be profitably compared with a
number of rewritten Bible texts,''\autocite[131]{knoppers2003}
ultimately he concludes that Chronicles ``is more than a paraphrase or
literary elaboration of the primary
history''\autocite[134]{knoppers2003} and thus RwB cannot can account
for Chronicles as a whole instead suggesting that it should be treated
as its own work.\autocite[131--134]{knoppers2003}

While Knoppers' assessment is fair given the definition that he supplies
for RwB, I have adopted a broader definition of RwB that accommodates
for Knoppers' reservations. In fact, the rationale for why RwB cannot
account for the complexity of the whole book of Chronicles, according to
Knoppers, is the same basic argument that I have made for why an
\emph{exegetically} focused definition of RwB is insufficient to account
for the complexity even of the literature that scholars
\emph{traditionally} consider to be RwB. Knoppers' criticisms,
therefore, could just as easily be applied to Josephus'
Jewish Antiquities or to Jubilees (or, I would argue, even to
Genesis Apocryphon) and why I have suggested that memory studies may
provide a useful model for discussing RwB.

This chapter will focus on the ways that social and cultural memory
theory can be used to explain the processes through which Chronicles
emended, harmonized, augmented, and omitted traditions in Samuel--Kings
and the way that those traditions shaped and were shaped by the social
location of their author(s). In particular, I will focus on the mnemonic
site of the Temple in Jerusalem and the discourse around its foundation.
\textless{}\textless{} CLARIFY THIS WHEN YOU FINISH THE CHAPTER.

\section{David's Census and Araunah/Ornan's Threshing
Floor}\label{davids-census-and-araunahornans-threshing-floor}

The account of David's census in 2 Samuel and its hypertext in 1
Chronicles provides a clear example of the way that present ideologies
play a significant role in the reception of cultural memory by a society
and its (re)construction in the social memory. Both 2 Samuel 24 and 1
Chronicles 21 recount the story of David's census, the punishment that
Yahweh inflicts on David for doing so, and David's penitential offerings
at the threshing floor of Araunah/Ornan.

The narrative in 2 Sam 24 begins by stating that Yahweh had become angry
with Israel and ``incited David against them'' (\emph{wayyāseṯ ʾeṯ-dāwiḏ
bāhem}). David orders the census and is stricken with guilt (it is not
clear why David, apparently, was not supposed to do this). Yahweh offers
him a choice through the prophet Gad between three years of famine,
three months of pursuit, or three days of pestilence. David consents to
the pestilence, and, after seeing its destructive force in the form of
an angel, pleads with Yahweh to punish him personally, and not to
continue harming the people. The prophet Gad instructs David to make
offerings to Yahweh where he saw the angel, at the threshing floor of
Araunah, the Jebusite. David purchases the threshing floor and Araunah's
livestock, builds an altar to Yahweh, and makes his offerings.

moriah = here and Gen 22

Modern readers of 2 Samuel are understandably perplexed by its narrative
logic. Yahweh becomes angry, prompts David to do something bad, then
punishes the people for David's mistake---a sort of divine entrapment.
It is no wonder, then, that we see a different explanation for David's
actions within Chronicles. The account in 1 Chron 21 follows closely
that of 2 Sam 24, with one notable exception, it is Satan (or, possibly
`an adversary') who incites David against Israel: \emph{wayyaʕămoḏ śāṭān
ʕal-yiśrāʔēl wayyāseṯ ʔeṯ-dāwîd limnôṯ ʔeṯ yiśrāʔēl} ``Satan stood up
against Israel and incited David to count Israel.'' The results are, for
our purposes, the same: Yahweh becomes angry with David who bears the
guilt for the census and Yahweh's death-angel kills some
seventy-thousand people.

It is not clear what the author of Chronicles had in mind when using the
term \emph{śāṭān}. It is commonly held that in this case the author
\emph{was} in fact using the term as a proper name in reference to a
particular, malevolent spirit, ``Satan'' who functions as a sort of
precursor to the more developed ``Satan'' of post-biblical and New
Testament theologies \autocite[4--5]{rollston_keith-stuckenbruck2016}.
This figure is to be distinguished from other ``satans'' in the Hebrew
Bible, found in Job and Zechariah, due to its lack of definite article.
Proper names, in Hebrew, do not take the definite article, therefore,
those references in Job and Zechariah must not be proper names. However,
there is not a scholarly consensus on this matter, and several scholars
have, in fact, argued the exact opposite, citing the fact that article
may be used ``when terms applying to whole classes are restricted
(simply by useage) to particular individuals \ldots{} or things.''
\autocites[§126d]{geseniuskautzsch1910}[as argued by][115]{japhet2009}
In such a case, \emph{śāṭān} should simply be understood as an
indefinite noun, ``an accuser'' and may be understood as a human
antagonist of David
\autocites{stokes_jbl2009}[114--117]{japhet2009}[370--390]{japhet1993}.

Regardless of how one understands \emph{śāṭān} to be functioning, it is
clear that the author of Chronicles understands the mechanisms at
work---whether supernatural or interpersonal---differently than the
author of 2 Sam 24. As Japhet notes, Chronicles shifts the incitement
away from Yahweh and removes any reference to the deity's anger prior to
the census. The result is that the problematic ``entrapment'' of David
no longer exists and Yahweh no longer seems so ``mercurial''
\autocite[4]{rollston_keith-stuckenbruck2016}. Whatever the specific
point of contention the author of 1 Chronicles had with 2 Sam 24, his
theological toolbox was conspicuously different than the author of 2
Sam. Whether that toolbox was more restrictive (\emph{śāṭān} as a
person) or more expansive (\emph{śāṭān} as a celestial being) is a
matter of interpretation.

As a process of memory, we can see that a theological shift---whatever
it may have been---meant that the same set of actions and consequences
required a different explanation than the original context. Therefore, I
understand Chronicle's account of David's census to be an example of a
theological reanalysis of the tradition preserved in 2 Samuel. Yahweh's
actions as described in 2 Samuel were non-sensical in what one might
call the ``grammar'' of the Chronicler's theology. The narrative
required another active agent whom could instigate David's actions in
order that Yahweh could punish the Israelites. The Chronicler used
\emph{śāṭān} as the malevolent actor to achieve his narrative goals.
