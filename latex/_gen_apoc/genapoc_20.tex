% !TeX root = ../dissertation.tex
\begin{aramaictext}
    \versenum{XX 1}
    ◦◦◦[\hspace{3em}]◦◦◦
    \versenum{2}
    ◦◦◦[\hspace{9em}] כמ̇ה ◦◦◦ ושפיר לה צלם אנפיהא וכמא
    \versenum{3}
    [נ]ע֯י֯ם̇ ו֯כ֯מ֯א֯ רקיק לה שער ראישה כמא יאין להון לה עיניהא ומא רגג הוא
    לה אנפהא וכול נץ
    \versenum{4}
    אנפיהא ◦◦◦ כמא יאא לה הדיה וכמא שפיר לה כול לבנהא דרעיהא מא
    שפירן וידיהא כמא
    \versenum{5}
    כלילן וחמיד כול מחזה יד̇[י]הא כמא יאין כפיהא ומא אריכן וקטינן כול
    אצבעת ידיהא רגליהא
    \versenum{6}
    כמא שפירן וכמא שלמא להן לה שקיהא וכל בתולן וכלאן די יעלן לגנון
    לא ישפרן מנהא ועל כול
    \versenum{7}
    נשין שופר שפר̇ה ועליא שפרהא לעלא מן כולהן ועם כול שפרא דן חכמא
    שגיא עמהא ודלידיהא
    \versenum{8}
    יאא וכדי שמע מלכא מלי חרקנוש ומלי תרין חברוהי די פם חד תלתהון
    ממללין שגי רחמה ושלח
    \versenum{9}
    לעובע דברהא וחזהא ואתמה על כול שפרהא ונסבהא לה לאנתא ובעא
    למקטלני ואמרת שרי
    \versenum{10}
    למלכא דאחי הוא כדי הוית מתגר על דילהא ושביקת אנה אברם בדילהא
    ולא קטילת ובכית אנה
    \versenum{11}
    אברם בכי תקיף אנה ולוט בר אחי עמי בליליא כדי דבירת מני שרי באונס
    \vacat
\end{aramaictext}

\begin{translation}
\end{translation}

\begin{aramaictext}
    \versenum{12}
    בליליא דן צלית ובעית ואתחננת ואמרת באתעצבא ודמעי נחתן בריך אנתה
    אל עליון מרי לכול
    \versenum{13}
    עלמים די אנתה מרה ושליט על כולא ובכול מלכי ארעא אנתה שליט
    למעבד בכולהון דין וכען
    \versenum{14}
    קבלתך מרי על פרעו צען מלך מצרין די דברת אנתתי מני בתוקף עבד לי
    דין מנה ואחזי ידך רבתא
    \versenum{15}
    בה ובכול ביתה ואל ישלט בליליא דן לטמיא אנתתי מני וי\textsuperscript{נ}דעוך מרי די
    אנתה מרה לכול מלכי
    \versenum{16}
    ארעא ובכית וחשית בליליא דן שלח לה אל עליון רוח מכדש למכתשה
    ולכול אנש ביתה רוח
    \versenum{17}
    באישא והואת כתשא לה ולכול אנש ביתה ולא יכל למקרב בהא ואף לא
    ידעהא והוא עמה
    \versenum{18}
    תרתין שנין ולסוף תרתין שנין תקפו וגברו עלוהי מכתשיא ונגדיא ועל כול
    אנש ביתה ושלח
    \versenum{19}
    קרא לכול ח֯כ֯י֯מ̇[י] מצרין ולכול אשפיא עם כול אסי מצרין הן יכולון
    לאסי֯ו֯תה מן מכתשה דן ולאנש
    \versenum{20}
    ביתה ול֯א֯ י֯כ֯לו֯ כ֯ול א̇ס̇יא֯ ואשפיא וכול חכימיא למקם לאסיותה ארי הוא
    רוחא כתש לכולהון
    \versenum{21}
    וערקו \vacat
\end{aramaictext}

\begin{translation}
\end{translation}

\begin{aramaictext}
    באדין אתה עלי חרקנוש ובעא מני די אתה ואצלה על
    \versenum{22}
    מלכא֯ ו֯אסמ֯וך ידי עלוהי ויהה ארי ב[ח]לם ח֯ז[ני] ואמר֯ לה לו֯ט לא יכ֯ו֯ל
    אברם דדי לצלי֯א על
    \versenum{23}
    מלכא ושרי אנ֯ת֯ת֯ה֯ ע̇מ̇ה וכען אזל אמר למלכא וישלח אנתתה מנה לבעלהא
    ויצלה עלוהו ויחה
\end{aramaictext}

\begin{translation}
\end{translation}

\begin{aramaictext}
    \versenum{24}
    \vacat
    ו֯כ֯די שמע חרקנוש מלי לוט אזל אמר למלכא כול מכתשיא ונגדיא
    \versenum{25}
    אלן די מתכתש ומתנגד מר̇י מלכא בדיל שרי אנתת אברם י֯ת֯יבו נה לשרי
    לא̇ב֯רם בעלה
    \versenum{26}
    ויתוך מ̇נכה מ̇כ̇ת̇ש̇א̇ דן ורוח שחלניא וקרא [מ]ל[כ]א לי ואמר לי מ̇א עבדתה
    לי בדיל [שר]י ותאמר
    \versenum{27}
    לי די אחתי היא והיא הואת אנתתך ונסבתהא לי לאנתה הא אנתתך ד̇ב֯ר̇ה֯
    א̇זל ועדי לך מן
    \versenum{28}
    כול מדינת מצרין וכען צ֯לי ע֯לי ו֯ע֯ל ביתי ותתגער מננה רוחא דא באיש֯ת̇א
    וצ̇לית֯ עלו֯ה֯י֯ מ֯ג֯דפא
    \versenum{29}
    הו וסמכת ידי ע֯ל֯ [ראי]שה ואתפלי מנה מכתשא ואתגערת [מנה רוחא]
    ב֯אישתא וחי ו֯ק֯ם ו֯י֯הב
    \versenum{30}
    ל̇י֯ מלכא ב[יומא] ד̇נא̇ מנתנ[ן] ש̇גיא̇ן וימ֯א לי מלכא במומה די לא֯ ◦◦◦[  
    ]הא ואת̇יב̇ לי
    \versenum{31}
    לש̇ר̇י ויהב לה מלכא̇ [כסף וד]הב ש̇גיא ולבוש שגי די בוץ וארגואן ו[\hspace{1em}]
    \versenum{32}
    קודמיהא ואף להגר וא[ש]למה לי ומני עמי אנוש די ינפקו֯נני֯ ול◦◦◦ מן
    מ̇צ̇רין \vacat
\end{aramaictext}

\begin{translation}
\end{translation}