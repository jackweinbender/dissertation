% !TEX root = dissertation.tex


Chronicles is Rwb
    Not everyone agrees
    I think it checks off enough boxes
BUT, Chronicles does *seem* different, especially for us modern scholars, because it is *in* the Hebrew Bible. up until this point, we have largely sidestepped the problem of textual authority and (later) canonical status. For both GenApoc and (to a lesser degree) Jubilees, it has been easy for (especially modern, western) scholars to corelate the relative authority of these rwb texts with their percieved faithfulness to their putative Vorlagen. In other words, becasue neither GA or Jub became canonical in the Hebrew Bible or the (Western) Christian Old Testament, it is tempting to correlate the inferiority of GA and Jub (and other such texts) with the fact that they are explicitly *derivative.* 

I want to be clear about this line of reasoning. I don't think anyone is making these arguments, but I think this correlation does inform the ways that we tend to treat these texts---they are pseudepigrapha; they are not canon. The fact that they "rewrite" the big traditions the Hebrew Bible does not intersect with or challenge modern notions of canonicity or textual authority. It is okay for GA to rewrite portions of Genesis because the social implication sof ';s.fgb';df.g;'b.dfg

What I'm getting at is that the line of reasoning that goes ``Well, even if some people thouht they were `authoritative,' they weren't canonized, so they couldn't have been that important falls appart when we look at Chronicles. Chronicles forces the issue. It is both a rewriting of Sam-Kings as well as a text which was incorperated into the body of canonical literature of the Hebrew Bible. The *fact* of the plurality of parallel traditions cannot be explained by heirarchies of authority.