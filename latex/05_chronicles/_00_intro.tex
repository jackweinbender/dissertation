\section{Introduction}

Although \chronicles meets the formal criteria of \rwb by almost every measure, it is rarely given more than a passing nod within the scholarly literature on \rwb.%
    \footnote{TODO: Get Refs}
This bracketing of \chronicles as \rwb is, I think, largely a function of the ways that conversations within particular scholarly sub-disciplines often are quite insular and resistant to traversing disciplinary boundaries. In this case, the book of \chronicles falls squarely within the field of ``biblical studies'' and ``Hebrew Bible'' while most of the scholarly work surrounding \rwb falls within the related, but distinct, fields of Early Judaism, Qumran Studies, and Second Temple Studies. Bridging such disciplinary gaps, even when the sub-disciplines are adjacent, can be difficult. Thus, although some commentators on \chronicles note the family resemblance, there is no commentary-length treatment of \chronicles which utilizes \rwb as the primary literary framework for reading \chronicles.

By way of example, Klein's recent commentary on \chronicles (2006) mentions the \rwb only in a footnote, stating:

\begin{quote}
    Perhaps \chronicles could also be compared with the genre called ``rewritten Bible,'' known from Qumran and in the works of Josephus. Such works retell some portion of the Bible while interpreting it through paraphrase, elaboration, allusion to other texts, expansion, conflation, rearrangement, and other techniques. In this case, of course, the ``rewritten Bible'' also became part of the Bible itself.\autocite[17 n.157]{klein2006}
\end{quote}

Setting aside Klein's rather anemic description of \rwb, it is striking---if not terribly surprising---to me that the comparison is not taken more seriously. Instead, Klein characterizes \rwb as a genre ``known from Qumran and ... Josephus'' and not as a phenomenon of Jewish literary production during the \secondtemple period in which \chronicles may also have been participating. Knoppers on the other hand, in his slightly earlier commentary (2003), takes up the issue in a special section of his introduction and offers his thoughts on whether \chronicles is \rwb.\autocite[129--134]{knoppers2003} His conclusions, however, are not much more satisfying. According to Knoppers, \rwb is an interesting angle from which to approach \chronicles, but ultimately, ``\chronicles needs to be understood as its own work.''\autocite[134]{knoppers2003} 

Many of Knoppers' observations are actually quite good. And although, I ultimately disagree with his conclusions about \chronicles \visavis \rwb, the root of this disagreement is only about how to define \rwb. In fact, the objections which Knoppers makes toward characterizing \chronicles as \rwb should sound somewhat familiar to the reader. First, Knoppers notes that the genealogies of 1 Chr 1--9 are not meaningfully ``rewriting'' the Pentateuch in genre or content (though it, in some sense, accounts for the chronology from Adam to David) but rather represents a reworking of earlier material with an incorporation of other traditions know from the Bible and, presumably, a number of the Chronicler's own innovations.%
    \footnote{Throughout this chapter I will refer to the author/editor of Chronicles as ``the Chronicler.'' In doing so, I do not have in mind a particular theory about such a person, or indeed, whether there were multiple ``Chroniclers.'' Nor am I referencing a particular scholarly tradition of including Ezra and Nehemiah within the scope of the work.}
The later, more familiar, account of the monarchy of Judah (which closely follows the narratives of Samuel--Kings), according to Knoppers, exhibits some of these same qualities. Although the narrative structure of this portion of Chronicles in certain ways closely follows Samuel--Kings, the focus and purpose of the work is distinct from that of the \DtrH (\drth)\autocite[132]{knoppers2003} Knoppers writes:

\begin{quote}
    The Chronicler's account of the monarchy follows the broad outline of Samuel--Kings and borrows extensively from this earlier work. Yet one must ask whether the Chronicler's presentation, with its focus of the Davidides and Judahite history, is a literary embellishment of Samuel--Kings or whether it employs Samuel--Kings, among other sources, to create an alternative story of the monarchy.\dots The Chronicler's depiction of the monarchy is not simply a commentary on the \DtrH\dots He employs the older work as a major source, even as he contests some of its central claims.\autocite[132--33]{knoppers2003}
\end{quote}
\noindent
