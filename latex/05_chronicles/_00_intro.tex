\section{Introduction}

Although \chronicles meets the formal criteria of \rwb by almost every measure, it is rarely given more than a passing nod within the scholarly literature on \rwb.%
    \footnote{TODO: Get Refs}
This bracketing of \chronicles as \rwb is, I think, largely a function of the ways that conversations within particular scholarly sub-disciplines often are quite insular and resistant to traversing disciplinary boundaries. In this case, the book of \chronicles falls squarely within the field of ``biblical studies'' and ``Hebrew Bible'' while most of the scholarly work surrounding \rwb falls within the related, but distinct, fields of Early Judaism, Qumran Studies, and Second Temple Studies. Bridging such disciplinary gaps, even when the sub-disciplines are adjacent, can be difficult. Thus, although some commentators on \chronicles note the family resemblance, there is no commentary-length treatment of \chronicles which utilizes \rwb as the primary literary framework for reading \chronicles.

By way of example, Klein's recent commentary on \chronicles (2006) mentions the \rwb only in a footnote, stating:

\begin{quote}
    Perhaps \chronicles could also be compared with the genre called ``rewritten Bible,'' known from Qumran and in the works of Josephus. Such works retell some portion of the Bible while interpreting it through paraphrase, elaboration, allusion to other texts, expansion, conflation, rearrangement, and other techniques. In this case, of course, the ``rewritten Bible'' also became part of the Bible itself.\autocite[17 n.157]{klein2006}
\end{quote}

Setting aside Klein's rather anemic description of \rwb, it is striking---if not terribly surprising---to me that the comparison is not taken more seriously. Instead, Klein characterizes \rwb as a genre ``known from Qumran and ... Josephus'' and not as a phenomenon of Jewish literary production during the \secondtemple period in which \chronicles may also have been participating. Knoppers on the other hand, in his slightly earlier commentary (2003), takes up the issue in a special section of his introduction and offers his thoughts on whether \chronicles is \rwb.\autocite[129--134]{knoppers2003} His conclusions, however, are not much more satisfying. According to Knoppers, \rwb is an interesting angle from which to approach \chronicles, but ultimately, ``\chronicles needs to be understood as its own work.''\autocite[134]{knoppers2003} 

The problem with Knoppers' assessment is not the observations he makes about \chronicles, but the way he frames \rwb. The major objections that Knoppers has toward characterizing \chronicles
