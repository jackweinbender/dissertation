\section{Introduction}

Although Chronicles meets the formal criteria of \rwb by almost every measure, it is rarely given more than a passing nod within the scholarly literature on \rwb.%
    \footnote{TODO: Get Refs}
This bracketing of Chronicles as \rwb is, I think, largely a function of the ways that conversations within particular scholarly sub-disciplines often are quite insular and resistant to traversing disciplinary boundaries. In this case, the book of Chronicles falls squarely within the field of ``biblical studies'' and ``Hebrew Bible'' while most of the scholarly work surrounding \rwb falls within the related, but distinct, fields of Early Judaism, Qumran Studies, and Second Temple Studies. Bridging such disciplinary gaps, even when the sub-disciplines are adjacent, can be difficult. Thus, although some commentators on Chronicles note the family resemblance, there is no commentary-length treatment of Chronicles which utilizes \rwb as the primary literary framework for reading Chronicles.

By way of example, Klein's recent commentary on Chronicles (2006) mentions the \rwb only in a footnote, stating:

\begin{quote}
    Perhaps Chronicles could also be compared with the genre called ``rewritten Bible,'' known from Qumran and in the works of Josephus. Such works retell some portion of the Bible while interpreting it through paraphrase, elaboration, allusion to other texts, expansion, conflation, rearrangement, and other techniques. In this case, of course, the ``rewritten Bible'' also became part of the Bible itself.\autocite[17 n.157]{klein2006}
\end{quote}

Setting aside Klein's rather anemic description of \rwb, it is striking to me that the comparison if not taken more seriously. Instead, Klein characterizes \rwb as a genre ``known from Qumran and ... Josephus'' and not as a phenomenon of Jewish literary production during the \secondtemple period in which Chronicles may also be participating.

Knoppers on the other hand, in his slightly earlier commentary (2003), takes up the issue in a special appendix to his introduction and offers his thoughts on whether Chronicles is \rwb.\autocite[129--134]{knoppers2003} His conclusions, however, are no more satisfying. According to Knoppers, \rwb is an interesting angle from which to approach Chronicles, but ultimately, ``Chronicles needs to be understood as its own work.''\autocite[134]{knoppers2003}