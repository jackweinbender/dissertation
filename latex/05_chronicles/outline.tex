
This chapter is meant to bring together the discussions of RwB and Memory studies by talking about Chronicles—it has both been treated extensively from the perspective of Memory and is widely touted as an example of RwB. However, the move to talk about other RwB texts as Memory has not, generally, occurred.

This chapter is to begin to illustrate how memory studies "works" currently in the field. This is meant to be a fairly run-of-the-mill chapter which illustrates some of the key ideas that are important for the study.

This is where I will be making explicit connections methodologically to the theorists like Halbwachs, Assmann, and Schwartz.

I will focus on; Reception, reformulation, and construction
    % Explain Sites of Memory (add Sites of memory to Memory Cha)
    * Sites of Memory (Magnetism)
        David
        Saul
        the Temple
    
    % This section needs Beef
    * Keying and Framing?
    * Social Frameworks and Recontextualization *of CONTENT*
        * How does David behave?
        * How does the temple work?
        * Who's important in the temple?
        * Who's good and who's bad?

        * This recontextualization is based on *present* social expectations. The purpose of such texts was to convey meaning, and not (necessarily) accuraccy

    * Social Frameworks and Recontextualization *of FORM*
        * As a *text* what does it mean to be recontextualized?
    
    * Memory Construction
        * Construction and Forgetting
        * Construction *for* < See utopia stuff
            * Genealogies
            * National Ideology? 
        * Discuss reception of Chronicles in 2nd Temple per

    The purpose of Chronicles is not to "explain" Sam--Kings. Although it's been called midrash, etc., the history of scholarship of Chroncles has "decided" that it is something different with its own goals. Memory theory has been very successfully applied to Chronicles. There was a socal function to the book of Chronicles. It drew upon traditions about Israel, it's ancestors, and its kings and religious institutions not simply to explain its sacred texts, but to reshape how that percieved past was understood for the purpose of informing the present circumstances of its readers. 

    