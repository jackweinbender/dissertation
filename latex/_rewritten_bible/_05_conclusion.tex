% !TeX root = ../dissertation.tex

\section{Conclusion}

While each of these discussions has worked to expand and refine the study of \rwb, the basic trajectory set by \vermes---the conviction that \rwb reflects an effort by the writer to implicitly comment on the biblical (or some ``scriptural'') text---has remained surprisingly consistent.%
%
\footnote{Most general introductions to the topic treat it in this way. See \cite{crawford_charlesworth2000}; \cite{brooke_schiffman-vanderkam2000}; \cite{zahn_lim-collins2010}, \cite{zahn2011}.}
%
This tendency to focus first and foremost on the exegetical qualities of \rwb reflects \vermes's original purpose for the term quite well, but given the current state of the discussion, I think it is worth reconsidering this central tenet. This is not to say, of course, that there is \emph{no} exegetical purpose to \emph{any} \rwb text, rather that given a broader view of \rwb, it is worth considering that the phenomenon of rewriting, as a literary process, was not primarily concerned with nor necessarily tied to the \emph{explication} of a scriptural \emph{Vorlage}.%
%
\footnote{See the work of Koskenniemi and Linqvist who similarly conclude that \rwb should be treated as rewritten \emph{story}. They ask several of the same questions that I hope to address in this dissertation. See \cite{koskenniemi-lindqvist_laato-ruiten2008}.}
%
Campbell, for example, has recently suggested that the practice of rewriting may have extended beyond works with scriptural \emph{Vorlagen} and may better be understood as a more general literary phenomenon of the late \secondtemple period.\autocite{campbell_zsengeller2014} In his article, Campbell observes that the rewriting of non-scriptural texts has by-and-large been ignored by scholars or \rwb and offers a number compelling examples of \secondtemple texts which rewrite non-scriptural material following the same basic process as \rwb. In particular, Campbell notes that while \ant 1--11 focuses on biblical material, \ant 12--13 offers a rewriting of the \emph{Letter of Aristeas} and portions of 1 Maccabees. These rewritings maintain the ``structure and flow'' of their base texts, just like \rwb, but it is generally agreed that neither 1 Macc nor the \emph{Letter of Aristeas} were viewed as scripture by Josephus, who, notably, provides us with one of the earliest lists of sacred writings from the period.\autocite{mason2002_mcdonald-sanders2002} According to Campbell, ``Josephus handles these compositions in the same way that he treats scriptural material in \emph{Ant.} 1--11.''\autocites[70]{campbell_zsengeller2014}[See also][126]{mason2002_mcdonald-sanders2002} Furthermore, 4 Macc 5--17 retells the story of the martyrdom of seven brothers along with their mother found in 2 Macc 3--7. As with \ant 12--13, 4 Maccabees follows the structure and ordering of the account in 2 Maccabees while augmenting the story and using it to advance the author's thesis as part of a philosophical treatise. 

What these examples lack, as compared with \vermes's understanding of rewriting, is a tradition of \emph{interpretation} (\emph{aggadah} for \vermes) which can account for the changes between the \emph{Vorlage} and the rewritten work. None of these examples is primarily concerned with clarifying their \emph{Vorlage}. At best, we might speculate that the authors refined the stories to better suit their needs (making corrections, emendations, etc.), but we do not imagine that there was any expectation that these ``interpretations'' carry any kind of normative force in later understandings of the original account. In other words, there is no reason for us to assume that the author of 4 Maccabees intended his reworking of the story of seven brothers to affect the way that readers would understand 2 Maccabees.\footnote{I hasten to point out that the later account certainly \emph{would have} affected readers' understanding of the story in 2 Maccabees, but here I am arguing that we do not imagine the \emph{intent} of the author of 4 Maccabees to be affecting that change.} On the other hand, this is precisely what \vermes was arguing for in \citetitle{vermes1961}: that the activity of rewriting was meant not only to be descriptive of \emph{how} the authors understood the biblical \emph{Vorlagen}, but that the \emph{purpose} of these rewritten texts in some way functioned \emph{prescriptively} within the tradition of biblical interpretation, that is, according to \vermes, \rwb texts were, by definition, \emph{about} the texts that they rewrote. 

Although modern genre theory, as presented by Molly Zahn, has offered some avenue for discussing the classification of these texts based on \emph{function}, it provides no means for analyzing the \emph{function itself}. In other words, Zahn's work has opened the possibility that the function of a text may affect its generic profile and that at least one such function could be common to texts that we call \rwb. However, \emph{what} that function was and how it operated within its social context falls outside the field of genre studies. Similar limitations cropped up in discussions of the boundaries between text editions and \rwb as framed by Tov. Thus the social role of \rwb as a literary phenomenon within \secondtemple Judaism remains a central and relatively understudied area.

As away to address this topic, this dissertation will approach the phenomenon of \rwb through the lens of Social and Cultural Memory studies. Although concrete data for the function of \rwb is still lacking, Memory studies offers a set of theoretical tools and models for thinking about the transmission, adaptation, and generation of cultural ``memory'' based on observed anthropological behavior and social theory. This dissertation, therefore, is meant to be an attempt to pivot away from reading \rwb texts as primarily functioning ``exegetically,''---by which I mean focused on the explanation, even if implicitly, of a particular text---toward reading rewritten texts as products of cultural transmission and adaptation through the lens of cultural and social memory theory. To be sure, biblical interpretation played a part in the production of these texts, but other social, cultural, and literary forces were at work behind these texts as well which have largely been ignored in favor of exegetical and interpretive discourses. In the next chapter, I will provide an overview of Memory theory and how it can inform the discussion surrounding \rwb.

