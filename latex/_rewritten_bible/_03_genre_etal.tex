% !TeX root = ../dissertation.tex

\section{\RWB: A Genre, Process, or Something Else?}

 One of the central issues with the term \RwB is whether it should be treated as a ``genre'' or as a ``process'' or ``activity.'' \vermes, is not particularly helpful in clarifying the issue: 

\begin{quote}
    The question has been raised whether the ``Rewritten Bible'' corresponds to a process or a genre? In my view, it verifies both. The person who combined the biblical text with its interpretation was engaged in a process, but when his activity was completed, it resulted in a literary genre.%
    \autocite[8]{vermes_zsengeller2014}
\end{quote} 

Within \vermes's schema of aggadic development, \rwb occupied a liminal space outside the genres of classical Jewish texts such as Targum and Midrash. Because these texts eluded categorization within  established text categories, \vermes's treatment of \rwb as a discrete group was not unreasonable. A number of scholars have since upheld the categorical approach and argued for \rwb as a literary \emph{genre}. 

The parade example of this perspective is Philip Alexander's 1988 article ``Retelling the Old Testament,'' which, although dated, remains the most widely cited exemplar of the ``genre'' perspective.%
    \footnote{\cite{alexander_carson-williamson1988}.
        \vermes himself even put his stamp of approval on it, see
        \cite[4]{vermes_zsengeller2014}.}
Alexander takes up four \rwB texts (\jub, \ga, \lab, and \ant) to determine whether there exists a set of concrete criteria by which scholars can admit or exclude texts from the category. Although I ultimately disagree with his conclusion that \rwb should be treated as a literary genre, his list of nine ``principle characteristics'' make a number of useful observations about the nature of \rwb texts generally and are summarized as follows: 

\begin{enumerate}
    \item \rwb texts are \emph{narratives} which follow the order of the biblical text. 
    \item \rwb texts are ``free standing'' literary works that take on the same form as the text they rewrite. They do not comment explicitly on their \emph{Vorlagen}, but weave interpretation into their seamless retelling. 
    \item \rwb texts are not meant to replace the biblical work. 
    \item \rwb texts cover a large portion of the biblical narrative and exhibit a ``centripetal'' relationship to the biblical text. 
    \item \rwb texts follow the biblical text's narrative ordering, but may omit certain, non-essential elements. 
    \item \rwb texts offer an interpretive reading of scripture which offer, quoting \vermes, ``a fuller, smoother and doctrinally more advanced form of the sacred narrative''\autocite[Citing \vermes in][305]{schurer1986} and implicitly comment on the biblical text. 
    \item \rwb texts are limited by their literary form which only allows a single interpretation of the biblical text that they rewrite. 
    \item \rwb texts are limited by their literary form which does not allow them to explain their exegetical rationale. 
    \item \rwb texts incorporate traditions and material not derived from the biblical text.
\end{enumerate} 

Despite Alexander's emphatic conclusion affirming the genre of \RwB, I find a number of these criteria to be unconvincing.

First, his criterion that the text be a \emph{narrative} strikes me as arbitrary. While \vermes focused on \rwb as a narrative phenomenon, he has since noted that the reason for this was that his focus was on \emph{aggadic} material, that is, non-halakhic interpretation, which by definition is non-legal. Coupled with the first half of his second observation---that \rwb texts take on the same form as the text they rewrite---these observations seem self-fulfilling and suffering from a sort of selection bias.%
    \footnote{Although, all of the texts he surveyed are narratives, this fact illustrates one of the major shortcomings in Alexander's method, specifically, that his conclusions were based on four texts ``normally included in the genre.'' See \cite[99]{alexander_carson-williamson1988}. Therefore the selection of these four texts was the result of a deductive selection, in part, based on their narrative form.}
Alexander states that the author was ``limited'' by the genre of narrative to a single interpretation and could not provide his exegetical rationale illustrates the major, overarching assumption about Alexander's (and \vermes's) approach to these texts---that the essential function of the texts and the purposes of their authors are the same as the later exegetes.

Second, several of his criteria are comments about the intention of the author or purpose of the work, for example that the \rwb text was not ``meant'' to replace its \emph{Vorlage}. Although I believe this to be fundamental to the discussion, as formal characteristics of a genre, Alexander does not address how one is to determine such purposes and intentions.  In particular when discussing texts---as \vermes does---such as the \pTarg, or (now) the so-called Reworked Pentateuch (4QReworkedPent)%
    \footnote{In fairness, 4QReworkedPent was not available to Alexander or \vermes. Yet, one still may wonder why   the \lxx or \sampent are not included.}
Similarly, claiming that \rwb texts ``implicitly comment'' on their \emph{Vorlagen} speaks to the \emph{intention} of the author, which in the edge cases is not clearly demonstrable. Such claims overstep the issue of genre and have entered into speculation about the text's social function. Thus while Alexander does offer some concrete formal characteristics for \rwb, a number of his criteria are actually issues of textual \emph{function}.

Alexander insists that ``Any text admitted to the genre must display \emph{all} the characteristics.''%
    \autocite[119 n. 11]{alexander_carson-williamson1988}
This principle seems needlessly rigid to me. Although these characteristics were inductively identified,  Alexander offers no formal rationale for selecting his sample. The texts that he selects represent the \emph{core} of what is generally accepted to be \rwb, but texts on the periphery of a genre by definition will not display \emph{every} characteristic of the core texts. Thus, Alexander's criteria, from my perspective, should not be treated as prerequisites for inclusion to the category of \rwb (if we are to treat it as such), instead, they should be used to describe a sort of literary \emph{Idealtypus} for \rwb.%
    \footnote{I have borrowed and adapted the well-known term \emph{Idealtypus} from Max Weber. See 
        \cite{weber1978}. For a concise summary of Weber's work, see 
        \cite[12--16]{smith-riley2009}.}

Moshe Bernstein, too, has upheld a Vermesian understanding of \rwb as a literary category and has argued that for the category to be useful to scholars, the boundaries must be clearly demarcated and reasonably narrow.%
    \autocite{bernstein_textus2005}
Notably, Bernstein never clearly articulates what it means for a category to be ``useful.'' All the same, he writes that he set out to: 

\begin{quote}
    examine the definition and descriptions of ``rewritten Bible'' proffered by \vermes and several subsequent scholars, in order to delineate the variety of ways in which the term is currently employed and to make some suggestions for how we might use it more clearly and definitively in the future.%
    \autocite[171--172]{bernstein_textus2005}
\end{quote}

Bernstein begins by addressing the few small modifications that he makes to \vermes's list, namely that Bernstein does not understand the Targums to be examples of \rwb. He excludes Targums from his discussion ``\emph{ab initio},'' as well as ``biblical'' books, (by which he seems to mean ``Chronicles''), and includes legal texts such as the \templescroll. Despite this second exclusion, Bernstein acknowledges that ``One group's rewritten Bible could very well be another's biblical text!''%
    \footnote{\cite[175]{bernstein_textus2005}. This seems particularly odd, since, an Ethiopian Christian may protest that \jub should be excluded as well.}
Thus, Bernstein concedes that ``matters of canon and audience may play a role,'' but does not address the topic further. 

Bernstein critiques scholars such as Nicklesburg,%
    \autocite{nickelsburg_stone1984}
Harrington,%
    \autocite{harrington_kraft-nickelsburg1986}
and Brooke%
    \autocite{brooke_schiffman-vanderkam2000}
for excessively expanding the use of the term \rwb at its ``upper bound'' (my term) to the point that they have weakened the term and have ``not aided in focusing scholarly attention on the unifying vs.~divergent traits of some of these early interpretive works.''%
    \autocite[179]{bernstein_textus2005}
Likewise, Bernstein critiques Tov for including reworked texts (e.g., 4QReworkedPent) and therefore expanding the ``lower bound'' of the category. While Bernstein avers that ``Rearrangement with the goal of interpretation is probably an earlier stage in the development of biblical `commentary' than supplementation with the goal of interpretation,''%
    \footnote{\cite[183]{bernstein_textus2005}. I make special note of the fact that Bernstein places the term ``commentary'' in quotes to indicate that he is not saying that \rwb is formally ``commentary.'' Yet, the overarching principle remains that \rwb is implicitly ``commenting'' on the biblical text. See also Fraade's work in this area: \cite{fraade_bakhos2006} and \cite{fraade_zsengeller2014}.}
he nevertheless distinguishes the former from the category \rwb and declares that ``the definitions of `rewritten Bible' furnished by Tov and \vermes are [not] even remotely compatible, and we need to choose between them simply for the purposes of clarity.''%
    \autocite[185]{bernstein_textus2005}
Bernstein, ultimately, argues that \vermes's category is worth keeping around, and admonishes the reader to maintain a narrow definition of the category, because, in his own words, ``the more specific the implications of the term, the more valuable it is as a measuring device,''%
    \autocite[195]{bernstein_textus2005}
and conversely that ``the looser the definition, the less precisely it classifies those items under its rubric.''%
    \autocite[195]{bernstein_textus2005} 

At the other end of the spectrum, a number of important scholars have treated \rwb as a ``process'' or ``activity'' rather than as a genre or category. These scholars also have tended to be more ``expansive'' when it comes to which texts should be discussed as ``rewritten.'' Harrington, as noted above, is the parade example of those who wish to treat \rwb as a process. He states: 

\begin{quote}
    Nevertheless, establishing that these books are not appropriately described as targums or midrashim is not the same as proving that they all represent a distinctive literary genre called ``rewritten Bible.'' In fact, it seems better to view rewriting the Bible as a kind of activity or process than to see it as a distinctive literary genre of Palestinian Judaism.%
    \autocite[242--243]{harrington_kraft-nickelsburg1986}
\end{quote} 

Instead, he observes that while texts such as \jub and \emph{Assumption of Moses} both constitute a rewriting of the Bible, both ``are formally revelations of apocalypses.''%
    \autocite[243]{harrington_kraft-nickelsburg1986}
This is an important criticism of scholars who see \rwb as a distinct genre. Unlike, for example, the Gospels, which arguably have the same basic ``form,'' the texts typically described as ``rewritten'' come in a variety of ``forms'' such as narratives (\ga), apocalypses (\jub), and legal (\templescroll). In other words, a single \emph{genre}---insofar as the word describes a literary \emph{form}---is not sufficient to subsume the varied \emph{forms} which all can be described as ``rewritten.''%
    \footnote{Anders Petersen has, more recently, attempted to bridge this genre/process gap by arguing that \rwb makes sense as a genre from an \emph{etic}, scholarly, perspective, whether or not (he thinks not) such a genre existed in antiquity (i.e., as an \emph{emic} category). Such distinctions, are useful, and I am in broad agreement insofar as Petersen accepts a classical definition of genre (see below). \cite{petersen_hilhorst-puech2007}. Contra Petersen, see the recent work of \cite{tino_jsj2018}.}

More recently, Molly Zahn has attempted to move the conversation forward by interacting with modern genre theory---which is conspicuously absent from most discussions of ``genre'' and \rwb.%
    \footnote{\cite{zahn_jbl2012}. Daniel Machiela noted the absence of genre theory in his 2010 article, as well, see \cite{machiela_jjs2010}. Brooke is a notable exception. See \cite{brooke_dsd2010}.}
Zahn discusses the difficulty that Harrington addresses by noting that works may participate in multiple genres simultaneously. While older conceptions of genre ``pigeonhole'' texts to specific genres, modern genre theorists---she cites Fowler---now prefer to talk about texts ``participating'' in a genre. Citing Fowler, Zahn notes that ``genres are less like pigeonholes and more like pigeons'' and further augments the metaphor to suggest that genres are ``more like flocks of pigeons.'' She writes: 

\begin{quote}
    Just as a flock of pigeons might change shape, lose and add members, be absorbed into larger flocks of break apart into several smaller flocks, genres and their boundaries are not static.%
    \autocite[277]{zahn_jbl2012}
\end{quote} 

Zahn's contribution is nuanced and deserves to be taken seriously. The implication for \rwb is clear: although \jub is a revelatory text while the \templescroll is a legal text, they can both participate in their respective ``formal'' genres simultaneously with the supposed \rwb genre. Zahn also explores the ``functional'' aspects of genre. She notes that genres are ``not simply systems of classifications developed and used by literary critics, but are fundamental to all human communication.''%
    \autocite[280]{zahn_jbl2012}
Thus, genres manifest as common patterns recognized by both the author and the reader which aid communication and in this way; genre functions as a sort of ``literary body language.''%
    \footnote{%
        \cite[276]{zahn_jbl2012}. See also 
        \cite[199]{newsom_grossman2010} and 
        \cite[37-53]{fowler2002}.}
By way of a modern example, it is quite common in modern films for such generically-mixed works exist. Mel Brooke's films, for example, are well-known for a particular genre of goofball comedy/farce set against the backdrop of some other well-known genre. Here I have in mind films such as \emph{Young Frankenstein} (1974), \emph{Blazing Saddles} (1974), \emph{Spaceballs} (1987), and others, each of which participates in both a specific comedic genre as well as ``Star Wars-esque,'' ``Western,'' or ``Classic Horror'' genres, respectively. Even in these examples, however, there are formal characteristics in \emph{both} genres which the audience can point to. Although the social function of farce is distinct from classical horror movies, the distinction between classical horror films and farces is not only defined by its social function. 

Yet, it is not at all clear to me what we have gained by upholding \rwb as a genre by simply changing what we mean by ``genre,'' however well rooted in theory.%
    \footnote{Machiela critiques Zahn's approach for similar reasons. See \cite{machiela_jjs2010}.}
If by Zahn's definition of ``genre'' we no longer are talking about formal characteristics to \rwb, what we are ultimately left with is a particular kind of relationship (rewriting) matched to a particular kind of biblical \emph{Vorlage}. Although this more sophisticated approach to genre is certainly superior, it seems to me that the kinds of questions that remain to be answered regarding the \emph{purpose} of \rwb fall at the outer limits of generic discourse and may require a different set of analytical and theoretical tools.