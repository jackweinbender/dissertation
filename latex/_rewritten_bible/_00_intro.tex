% !TeX root = ../dissertation.tex

In his seminal work \citetitle*{vermes1961}, \Vermes introduced the term ``\rwB'' into the discussion of \secondtemple Jewish literature as part of a larger project to trace the development of haggadic traditions from the late \secondtemple period into the rabbinic period. \vermes used the term \rwB to describe a number of texts which closely follow portions of the biblical narrative but also augment, elide, and emend the text in ways which produced new literary works in their own right. According to \vermes, through this exegetical process, ``the midrashist inserts haggadic development into the biblical narrative'' in order to ``anticipate questions, and to solve problems in advance.''%
%
\footnote{\cite[95]{vermes1961}; See also \cite{vermes_zsengeller2014}.}
%
\vermes traced these interpretive traditions historically and attempted to demonstrate an interpretive continuity between the \secondtemple period and nascent Rabbinic Judaism and Christianity. Although the formal characteristics of these narratives differed from later midrash, \rwB texts displayed the same kinds of ``midrashic'' tendencies. In \vermes's conception, therefore, the authors of \rwB texts \emph{implicitly} made use of interpretive traditions that later works such as the Talmud and Mishnah expressed \emph{explicitly}. 

Since the publication of \citetitle{vermes1961}, \vermes's concept of \rwB has taken on a life of its own and developed into its own discreet area of study as scholars from various related disciplines have reused, reinterpreted, and redefined the term.%
%
\footnote{See especially the early discussions in \cite{alexander_carson-williamson1988}), \cite{nickelsburg_stone1984}, and \cite{harrington_kraft-nickelsburg1986}.}
%
The discussion of \rwb has become especially fruitful within the field of \qumran studies where new texts from the \secondtemple period continued to be published throughout the late 20th century and where new material discoveries continue to this day. However, the idea of biblical rewriting has also been fruitfully applied to texts that have long been known to scholars such as \jub, Deuteronomy, Chronicles, and even the Synoptic Gospels.%
%
\footnote{On reading the Gospels as \rwb, see \cite{müller_back-kankaanniemi2012}; \cite{malan_hts2014}, and more recently \cite{allen_jsnt2018}}

Although the scope and nuance of of the term \rwB has shifted in the intervening years, the trajectory set by \vermes nearly sixty years ago has remained reasonably consistent. By focusing on the relationships that exist between \rwb texts and their scriptural \emph{Vorlagen}, studies on \rwb texts have tended to discuss the topic primarily through the lens of biblical or scriptural interpretation by focusing on how the authors or editors of \rwb texts retained, emended, or excised material \emph{from the biblical text}. While these treatments are often very good, this preoccupation with the ``biblical'' text (or a particular ``scriptural'' text, using the more common terminology) I think has impeded the study of these texts as participants in a broader cultural discourse that extends beyond ``biblical interpretation.''

In this chapter I will trace the emergence and evolution of the concept of \rwB from \vermes's use in \citetitle{vermes1961} to the present focusing on three key questions and ideas which have shaped the scholarly discourse around \rwB studies: 1) the terminology surrounding \rwB, 2) what works should fall under the rubric of \rwB and 3) whether \rwB constitutes a literary genre, a process, or some combination of the two. Taking these questions into account, I will then offer my own suggestions on how the \rwb conversation can be augmented by the treatment of \rwb texts as participants in a broader cultural discourse through the lens of cultural and social memory studies.