% !TeX root = ../dissertation.tex

\section{Scripture and Tradition}

The primary purpose of \citetitle{vermes1961} was not to offer a clear definition of the term ``\rwB,'' but to lay the groundwork for the historical, diachronic, study of aggadic traditions, of which \rwB makes only a small part.% 
    \autocite[3]{vermes_zsengeller2014}
As \vermes recounts, prior to the mid-twentieth century, the prevailing approach to the study of aggadic exegesis was to treat the aggadah as originating during the Tannaitic period. The aggadah were viewed as ``the result of the adoption, and anonymous repetition, of popular interpretations by favourite preachers,''%
    \autocite[3]{vermes1961}
the earliest of which were from the second century CE and were represented by Targums Onkelos and Jonathan. Furthermore, studies of ancient Jewish literature at this time focused on texts which modern Judaism considered authentic. As a result, a good number of earlier texts---for example, the apocrypha, pseudepigrapha, and sectarian texts---were often categorically excluded from discussions of the origins of aggadic exegesis.%
    \autocite[2]{vermes1961} 

A series of publications and discoveries beginning in the 1930's, however, began to undermine the notion that these early exegetical traditions began in the second century CE. \vermes credits this broadening of aggadic studies to a series of major studies and discoveries such as Rappaport's \emph{Agada und Exegese bei Flavius Josephus},%
    \autocite{rappaport1930}
Paul Kahle's Schweich Lectures at the British Academy on the Cairo Geniza (given in 1941, published 1947),% 
    \autocite{kahle1947}
Kisch's new text edition of Ps. Philo's \lab (1949),%   
    \autocite{kisch1949}
the discovery of the Dead Sea Scrolls (1948) and Codex Neofiti (1956), as well as (and perhaps especially) Renée Bloch's work on midrash.%
    \footnote{%
        \cite{bloch1954};
        \cite{bloch1955_repr};
        \cite[3--7]{vermes1961}.}
The overarching theme among these works was the evidence for continuity between biblical interpretive traditions prior to the second century, and later aggadah. For example, \vermes notes that Rappaport's work on \ant identified substantial overlaps between Josephus's text and Rabbinic aggadah and suggested, therefore, that Josephus had drawn from an already living tradition of interpretation. The implication of his suggestion is that the aggadah of the second century were not novel exegetical works, but were themselves products of earlier exegetical traditions. 

Building on these recent advancements, the explicitly stated purpose of \emph{Scripture and Tradition} was to push the field beyond synchronic analysis of aggadah toward diachronic, historical analyses to trace the development of these exegetical traditions \autocites[1]{vermes1961}[See also][]{bloch1955_repr}. The book is eight chapters long and is divided into four parts. 

The first section of \citetitle{vermes1961}, entitled ``The Symbolism of Words,'' is composed of three chapters which attempt to explain some of the processes by which localized symbolic interpretations were able to affect the interpretation of other, nominally related texts.
%
In the first chapter, \vermes notes the divergent treatment of Gen 44:18--19 among ancient commentators and proceeds through a synoptic study of this passage in the Fragmentary Targum, Targum Neofiti, and the Tosefta of Targum Yerushalmi in order to argue for a relative chronology based on their use of shared interpretive traditions. He concludes that the Fragmentary Targum represents the most primitive work, whose interpretive strategy is essentially inner-biblical. He then argues that the Tosefta of Targum Yerushalmi depends on the Fragmentary Targum but offers a distinct interpretive stance and that the Targum Neofiti represents a later combination of these two traditions.
%
In his second chapter, \vermes examines the symbolic use of the term ``Lebanon'' in the Hebrew Bible and other Jewish literature as a reference to Jerusalem and the Temple and how those symbolic meanings came to be. He identifies the Song of Songs as the intermediary text which helped to establish this tradition within post-exilic Judaism and that it occupies a unique position as the only biblical text which clearly uses the name Lebanon symbolically for the Temple. Importantly, \vermes shows that the symbolic use of Lebanon to represent Jerusalem and the Temple is rooted in \emph{biblical} exegesis. This is a key idea for \vermes because it establishes a continuity between the production of the biblical text and its later interpretation.
%
In chapter three, \vermes builds on his earlier work on the term ``Lebanon'' and examines other words which take on symbolic meaning in later Jewish texts: ``lion,'' ``Damascus,'' ``\emph{Meḥoqeq},'' and ``Man'' and attempts to show that a similar process took place among the \dss texts and the targumic and midrashic materials. 

It is in the second part of \emph{Scripture and Tradition}---entitled ``The \RwB''---that the topic of \RwB is first addressed directly. The section is composed of two chapters (four and five), both of which focus on the figure Abraham and the aggadic traditions surrounding his life. The purpose of these two chapters is to demonstrate a continuity of interpretive traditions from the late \secondtemple period through to the early rabbinic period and beyond. 

\vermes uses these two chapters to approach the topic from both ends of the chronological spectrum in what he refers to as ``retrogressive'' and ``progressive'' historical studies. In chapter four, \vermes embarks on what he calls a ``retrogressive historical study'' by which he means beginning with later, more developed traditions and working back toward their origins. In this case, \vermes begins with the eleventh century CE text \emph{Sefer ha-Yashar} and works backward to identify sections of the text which exhibit earlier traditions, most notably those in the Targums, Josephus, \jub, and (Ps.) Philo. The purpose of this chapter is to demonstrate that even late texts can contain valuable information about earlier methods of exegesis. As \vermes puts it, ``[Sefer ha-Yashar] manifests a direct continuity with the corresponding tradition of the time of the second Temple, but reflects also the influence of the haggadah of the Tannaim and Amoraim.''%
    \autocite[95]{vermes1961}
On the other hand, in chapter five, \vermes proceeds with a ``progressive historical study,'' beginning with the oldest materials and working forward. Still focusing on the figure Abraham, \vermes treats in detail the relationship between Gen 12:8--15:4 and \cols{19}{22} of the \ga. Notably, \vermes treats \ga as ``the most ancient midrash of all''%
    \autocite[124]{vermes1961}
and, rather dramatically declares it to be the ``lost link between the biblical and the Rabbinic midrash.''%
    \autocite[124]{vermes1961}
For \vermes, the \ga occupies a unique position just one step removed from inner-biblical exegesis. Accordingly, \vermes believed that the author of the \ga was attempting ``to make the biblical story more attractive, more real, more edifying, and above all more intelligible'' and to ``[reconcile] unexplained or apparently conflicting statements in the biblical text in order to allay doubt and worry.''%
    \autocite[126]{vermes1961}
According to \vermes, the \ga's interpretation of Genesis was ``organically bound'' to the text of Genesis and the additions that \emph{were} made sprung from the interpretation of the Bible itself and not whole-sale from the mind of the author. Where texts like \jub sought to systematically advance a theological vision, according to \vermes, the author of \ga intended to simply ``explain the biblical text,'' calling it illustrative of ``the unbiased rewriting of the Bible.''%
    \footnote{\cite[126]{vermes1961}. I think this statement is demonstrably false, as I will argue in chapter three. The \ga utilizes traditions tangential to Genesis which are not themselves contained within the biblical work. In fairness to \vermes, the early columns of \ga were not available to him when he published \citetitle{vermes1961} and it is in the earlier columns where this reliance on extra-biblical material is most easily seen.}

The third part of \citetitle{vermes1961} is titled ``Bible and Tradition'' and is composed of a single chapter engaging in a lengthy analysis of the traditions surrounding the seer Balaam from Numbers 22--24. \vermes observes that while the majority of post-biblical texts treat Balaam as a villain, in \emph{LAB} he is treated as a sort of tragic hero%
    \autocite[173]{vermes1961}
The more traditional portrayal of Balaam as a wicked prophet began within the nexus of biblical tradition itself. The various documentary strata of the Balaam story cast the prophet in differing lights, and it is the final stratum, the P layer, which got the final say---within the biblical text---about him.
%
% TODO: as evinced by which texts. You need to elaborate on this point briefly.
%
\vermes points out, however, that ignoring the Priestly additions yields a story somewhat similar to that of \emph{LAB}. Thus, \vermes concludes that the exegetical traditions found in the later Targums and rabbinic works are simply the continuation of the exegetical strategies employed within the Bible itself, which he calls ``biblical midrash or haggadah.''%
    \autocite[176]{vermes1961}

The last two chapters make up the final section of \vermes's study, titled ``Theology and Exegesis,'' and push the discussion to include early Christianity. Chapter seven is entitled ``Circumcision and Exodus 4:24--26'' but offers a subtitle of ``Prelude to the Theology of Baptism,'' which gives some hint at the ultimate, if tacit, goal of the chapter. Discussing the topic of circumcision in Ex 4:24--26 and its treatment among the early exegetes, \vermes's primary observation is simply that the theology of circumcision and the exegetical traditions which surrounded it, were affected by historical forces and theological ideologies. For instance, he claims that \jub omitted the rather odd statement that God was going to kill Moses---who was saved by the circumcision of his son by Zipporah---because ``[i]t was impossible for its author to accept that God tried to kill Moses as it was for him to believe that Moses neglected to circumcise his son on the eighth day after his birth.''%
    \autocite[185]{vermes1961}
Similarly, he notes that after the Bar Kokhba rebellion, the practice of circumcision was outlawed and so, ``it is not surprising, therefore, to find the spiritual authorities of Palestinian Judaism emphasizing the greatness and necessity of this essential rite, and explaining away \ldots{} every possible biblical excuse for delaying the circumcision of their children.''%
    \autocite[189]{vermes1961}
He ends the chapter by suggesting that the early Christian association of baptism with circumcision (citing Rom 4:3--4 and Col 2:11--12) was enabled by the traditional Jewish association of circumcision with blood sacrifice (``the Blood of the Covenant'').%
    \autocite[190]{vermes1961}
That Paul associated baptism with circumcision therefore, was ``not due, therefore, to his own insight, but springs directly from the contemporary Jewish doctrine of circumcision which he adopted and adapted.''%
    \autocite[191]{vermes1961} 

\vermes makes a similar move in chapter eight, entitled ``Redemption and Genesis XXII: The Binding of Isaac and the Sacrifice of Jesus.'' In it, he compares a number of ancient works' treatment of the Akedah and demonstrates how the (near-)sacrifice of Isaac became a prototype for the entire sacrificial system in later Judaism. The sacrifice of animals in the Temple functioned as a ``reminder'' to God of the faithfulness of Abraham. Furthermore, he shows the ways the tradition grew to focus on the willingness of Isaac to be sacrificed and his function as a proto-martyr. Thus, he ends the chapter by addressing the New Testament's portrayal of Jesus as a willing sacrifice to God and its putative relationship to the Akedah. \vermes makes the case that the redemptive theology of the NT---typically attributed to Paul---was not original to him. He writes: 

\begin{quote}
    For although [Paul] is undoubtedly the greatest theologian of the Redemption, he worked with inherited materials and among these was, by his own confession, the tradition that ``Christ dies for us according to the Scriptures.''%
    \autocite[221]{vermes1961}
\end{quote}

\noindent
He then proceeds to push the origin of this theology back further into the first century CE, and, in rather dramatic fashion, suggests that the introduction of the Akedah motif into Christian theology---by means of the Suffering Servant---may have been by Jesus himself.%
    \autocite[223]{vermes1961}

\vermes concludes the chapter by discussing the Akedah and the Eucharist. Just as the whole sacrificial system pointed back toward the binding of Isaac in targumic exegesis, the eucharistic rite likewise was intended---according to \vermes---to point back to Jesus's redemptive sacrifice. Thus he concludes: 

\begin{quote}
    Although it would be inexact to hold that the Eucharistic doctrine of the New Testament, together with the whole Christian doctrine of Redemption, is nothing but a Christian version of the Jewish Akedah theology, it is nevertheless true that in the formation of this doctrine the targumic representation of the Binding of Isaac has played an essential role. 
    
    Indeed, without the help of Jewish exegesis it is impossible to perceive any Christian teaching in its true perspective.%
    \autocite[227]{vermes1961}
\end{quote} 
\noindent
%
The arc of \vermes's study, therefore, is meant to establish a continuity between the earliest traditions of biblical interpretation with the later traditions of both Rabbinic Judaism and Early Christianity and to trace the evolution of those traditions historically. Rather than viewing the early rabbinic interpretations as \emph{sui generis}, \vermes's larger purpose is to establish \emph{continuity} between the earliest examples of biblical interpretation---even innerbiblical interpretation---and the exegetical work of the rabbis. \RwB texts, therefore represent an intermediary phase between innerbiblical interpretation and later explicit commentaries, all of which can be viewed on a single interpretive continuum.

\subsection{\vermes's Use of \RWB} 

The fact that \vermes spent so little time explaining precisely what he meant by the term \rwb bears witness to the fact that \vermes thought the term was self-explanatory. \vermes makes this sentiment clear in his short retrospective on the origins of the term, expressing shock over the debate that his term prompted and the scholarly confusion surrounding it.%
    \footnote{The only other works that \vermes addresses the topic of \rwb (to my knowledge) is in \cite{vermes_eretz-israel1989} and his contributions to Emil Schürer's multi-volume history, \cite{schurer1986}.}
%
He writes: 

\begin{quote}
    The notion [of \rwb], which over fifty years ago I thought was quite clear, seemed to the majority of the more recent practitioners nebulous and confused, and lacked methodological precision.%
    \autocite[3]{vermes_zsengeller2014}
\end{quote} 
\noindent
%
Only a few scholars, according to \vermes, managed to remain true to his original vision.%
    \footnote{He specifically references
    \cite{alexander_carson-williamson1988} and 
    \cite{bernstein_textus2005}.}
%
Instead, many subsequent studies, according to \vermes, ``moved the goalposts'' to better ``suit the interest of their inquiry.''%
    \autocite[4]{vermes_zsengeller2014}

Yet, one cannot help but push back against \vermes here as scholars' desire to narrow the scope of the term is, I think, a reasonable impulse. After all, \vermes's use of \rwb covers texts written in several languages, across centuries, in no particular geographical region, and, while all the texts are ``narratives,'' the formal similarities between \ga, \ant, \jub, and the \pTarg stop there. \vermes specifically laments the narrowing of the term \rwb to focus primarily on the \dss texts. Of course, when \citetitle{vermes1961} was first published in 1961 (\vermes notes that the manuscript, in fact, was submitted for publication in 1959), only a small portion of the scrolls were published or accessible to more than a few specific scholars. But the field's subsequent preoccupation with the \qumran material, he suggests, is misguided.%
    \footnote{I am sympathetic to what \vermes perceived as ``moving the goalposts''---I think the context and purpose of how he used the term \rwb is often ignored---but it is worth pointing out that the reason the term \rwb is so often applied to the \qumran texts likely has less to do with a conscious, scholarly effort, and more to do with the fragmentation of the various fields that deal with the texts in question. A scholar with a background primarily focused on the New Testament or Hebrew Bible may not be as familiar with the texts and traditions of rabbinic Judaism that \vermes discusses in \citetitle{vermes1961}. Perhaps ironically, it was this sort of fragmentation that \citetitle{vermes1961} was written---at least in part---to overcome.} 

This sentiment is---it seems to me---a bit over-blown. On the one hand, \ga and the \templescroll receive a lot of scholarly attention, but \jub and \ant do as well. Even so, whatever narrowing of the discussion of \rwb has occurred toward the \qumran scrolls is likely symptomatic of the ``methodological [im]precision '' attributed to \citetitle{vermes1961} and the fact that \vermes did not clearly state what he meant when he used the term \rwB. For example, \vermes's inclusion of the medieval \sefer muddies the waters for those who wish to discuss \rwb as a process of scriptural interpretation which can be situated historically. On the other hand, his inclusion of the Palestinian Targums makes sense diachronically, but formally, the Targums are translations and not ``new compositions'' in the same sense that \jub or \ga are. Within \citetitle{vermes1961}, of course, \vermes treats these texts with due care and nuance---in the case of \sefer, he endeavors to show that traditions preserved in the text can be traced back to the \secondtemple period---but the fact that \vermes sought to situate haggadic developments diachronically while implementing a category that spanned such broad socio-religious (\qumran, Early Christian, Rabbinic, Medieval), chronological (1st -- 12th centuries CE), and literary (translations, narrative, revelatory/apocalyptic, history?) horizons has given some scholars a reasonable challenge when attempting to use the term in their own work. Thus, simply because \vermes set the ``goalposts'' (to suit his \emph{own} thesis, I might add), does not mean that others cannot or should not move them when appropriate, though hopefully along with a well-reasoned explanation for the change. 
