% !TEX root = dissertation.tex

\section{Sites of Memory in the Book of Chronicles}

We have plenty of language to describe the various processes of individual memory, but one of the main problems we have when talking about social memory and cultural memory is that we lack good language describe the structures and functions of those mnemonic systems at the level of society. As such, memory theorists have adopted a number of analogies and terms to describe how societies remember and how individuals and groups interact with memory at the social level. 

It is important to remember that because social memory is a social construct we must not equate the remembered past with the events, experiences, and individuals which informed it. Where one might refer to an individual person having ``a memory'' of a particular event, there is no central repository---be it material or biological---of social memory.%
    \footnote{See especially \cite{brockmeier_cp2010} and \cite{wertsch_cp2011}.}  
As has been noted by numerous memory theorists, ``there is no such `thing' and social or collective memory.''%
    \footnote{\cite[14]{wilson2017} citing \cite[112]{olick-robbins_ars1998} and \cite[118--24]{wertsch_boyer-wertsch2009}.}
In other words, when we talk about social or cultural ``memory'' we are talking about a complex network of social processes and discourses which make up a society's understanding of the past.

These social processes and discourses tend to center around particular events, places, people, and ideas which the society has imbued with special mnemonic significance. These clusters of discourse are commonly referred to by memory theorists as ``sites'' of memory. The term ``site of memory'' is a translation of the French \emph{lieu de mémoire} was coined by Pierre Nora in the 1970's and has been adopted and adapted by numerous theorists since then.%
    \footnote{%
        The term was originally coined by Nora in the work
        \cite*{nora_goff-etal1978}, and used subsequently in 
        \cite*{nora1984} and 
        \cite*{nora_representations1989}. For a discussion of Nora's use of the term and its reception, see 
        \cite{szpociński_teksty-drugie2016}.}
Although Nora did not clearly define the term, a ``site of memory,'' as used by Nora, might better be translated as a ``place of remembrance,'' or a ``place where people remember.'' For Nora, modern-day ``sites'' of memory existed ``because there are no longer \emph{milieux de mémoire}, real environments of memory.''%
    \autocite[7]{nora_representations1989}
In other words, because modern historical consciousness, by Nora's reckoning, has all but eradicated ``memory,'' the preservation of memory in the modern era has been relegated to particular ``sites'' of memory---monuments, structures, and practices whose purpose is to perpetuate memory. He writes:

\begin{quote}
    \emph{Lieux de mémoire} are simple and ambiguous, natural and artificial, at once immediately available in concrete sensual experience and susceptible to the most abstract elaboration. Indeed they are \emph{lieux} in three senses of the word---material, symbolic, and functional. Even an apparently purely material site, like an archive, becomes a \emph{lieu de mémoire} only if the imagination invests it with a symbolic aura. A purely functional site, like a classroom manual, a testament, or a veterans' reunion belong only inasmuch as it is also the object of a ritual. And the observation of a commemorative minute of silence, an extreme example of a strictly symbolic action, serves as a concentrated appeal to memory by literally breaking the temporal continuity.\autocite[18--19]{nora_representations1989}
\end{quote}

\noindent
Sites of memory, therefore, are not entirely abstract and intellectual, but bear on the practice and materiality of a society in addition to having symbolic significance.

Although Nora's original use of the term tended to focus especially on sites of memory which bear on so-called ``great traditions''%
    \footnote{As coined by Redfield in \cite*[41--42]{redfield1956}.}
of political and ideological importance such as national monuments and archives, the modern use of the term tends to be more abstract and to refer to any ``place'' where memory discourses occur within a society for the purpose of remembering. Such sites of memory may operate within any number of social/cultural spheres such as national memory (war memorials, national holidays, etc.), religious memory (religious holidays, symbolic ritual acts, etc.), or family memory (traditional foods, birthdays, anniversaries) and may be thought of as distinct, but connected ``nodes'' of symbolic meaning within a complex network of cultural symbols---what \halbwachs called the ``social frameworks of memory.''%
    \autocite[38]{halbwachs1992}  

Every edge and node within the graph of a society's collective memory is the product of memory construction. It is an abstraction. In much the same way that historiography offers a schematic narrative of past events which is necessarily selective and intentional about what specific events, people, and ideas are germane to the purpose of the historian, so too social and cultural memory is selective of the particulars which it preserves and constructive in how it presents people, events, and ideas within particular symbolic systems. Thus, sites of memory are social spaces where memory is constructed. For our purposes, and following a number of modern practitioners of memory studies, I will use the term ``site'' of memory to describe any discrete person, place, practice or idea where such discourses of memory occur.%
    \footnote{Within Hebrew Bible studies, see especially the work of Ehud Ben Zvi as well as his student Ian Wilson, esp. \cite[72--74]{benzvi_st2017} and \cite[25--26]{wilson2017}.}

The Hebrew Bible is replete with sites of memory---ideas, people, places, and practices which have been imbued with significance by numerous societies since antiquity and which form a central component to the identities and self-understanding of (especially) Jews and Christians throughout the world. Take, for example, the Exodus from Egypt. Regardless of the historical reality of such an event, the story of the Exodus as recounted in the Hebrew Bible is the central narrative undergirding the biblical rationale for Israel's possession of the Land. Likewise, the Israelites are told to be kind to strangers and sojourners within their community based on the memory of Israel's enslavement in Egypt. Similarly, the Torah could be understood as a distinct (and particularly potent) site of memory found in the Hebrew Bible; the same goes for the figure of Moses. Each of these sites of memory (the Exodus, Torah, and Moses) are distinct but they also exhibit clear relationships within the network of discourses which are found in the Hebrew Bible. And moreover, each site of memory also relates to and bears distinct significance for the various religious communities which hold the Hebrew Bible as a part of their tradition within their distinct systems of symbolic meaning. Remembering these connections and their culturally defined significance is what cultural memory is all about.