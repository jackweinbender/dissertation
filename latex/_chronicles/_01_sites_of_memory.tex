% !TEX root = dissertation.tex

\section{Sites of Memory in the Book of Chronicles}

We have plenty of language to describe the various processes of individual memory, but one of the main problems we have when talking about social memory and cultural memory is that we lack good language describe the structures and functions of those mnemonic systems at the level of society. As such, memory theorists have adopted a number of analogies and terms to describe how societies remember and how individuals and groups interact with memory at the social level. 

It is important to remember that because social memory is a social construct we must not equate the remembered past with the events, experiences, and individuals which informed it. Where one might refer to an individual person having ``a memory'' of a particular event, there is no central repository---be it material or biological---of social memory.%
    \footnote{See especially \cite{brockmeier_cp2010} and \cite{wertsch_cp2011}.}  
As has been noted by numerous memory theorists, ``there is no such `thing' and social or collective memory.''%
    \footnote{\cite[14]{wilson2017} citing \cite[112]{olick-robbins_ars1998} and \cite[118--24]{wertsch_boyer-wertsch2009}.}
In other words, when we talk about social or cultural ``memory'' we are talking about a complex network of social processes and discourses which make up a society's understanding of the past.

These social processes and discourses tend to center around particular events, places, people, and ideas which the society has imbued with special mnemonic significance. These clusters of discourse are commonly referred to by memory theorists as ``sites'' of memory. The term ``site of memory'' is a translation of the French \emph{lieu de mémoire} was coined by Pierre Nora in the 1970's and has been adopted and adapted by numerous theorists since then.%
    \footnote{%
        The term was originally coined by Nora in the work
        \cite*{nora_goff-etal1978}, and used subsequently in 
        \cite*{nora1984} and 
        \cite*{nora_representations1989}. For a discussion of Nora's use of the term and its reception, see 
        \cite{szpociński_teksty-drugie2016}.}
Although Nora did not clearly define the term, a ``site of memory,'' as used by Nora, might better be translated as a ``place of remembrance,'' or a ``place where people remember.'' For Nora, modern-day ``sites'' of memory existed ``because there are no longer \emph{milieux de mémoire}, real environments of memory.''%
    \autocite[7]{nora_representations1989}
In other words, because modern historical consciousness, by Nora's reckoning, has all but eradicated ``memory,'' the preservation of memory in the modern era has been relegated to particular ``sites'' of memory---monuments, structures, and practices whose purpose is to perpetuate memory. He writes:

\begin{quote}
    \emph{Lieux de mémoire} are simple and ambiguous, natural and artificial, at once immediately available in concrete sensual experience and susceptible to the most abstract elaboration. Indeed they are \emph{lieux} in three senses of the word---material, symbolic, and functional. Even an apparently purely material site, like an archive, becomes a \emph{lieu de mémoire} only if the imagination invests it with a symbolic aura. A purely functional site, like a classroom manual, a testament, or a veterans' reunion belong only inasmuch as it is also the object of a ritual. And the observation of a commemorative minute of silence, an extreme example of a strictly symbolic action, serves as a concentrated appeal to memory by literally breaking the temporal continuity.\autocite[18--19]{nora_representations1989}
\end{quote}

\noindent
Sites of memory, therefore, are not entirely abstract and intellectual, but bear on the practice and materiality of a society in addition to having symbolic significance.

Although Nora's original use of the term tended to focus especially on sites of memory which bear on so-called ``great traditions''%
    \footnote{As coined by Redfield in \cite*[41--42]{redfield1956}.}
of political and ideological importance such as national monuments and archives, the modern use of the term tends to be more abstract and to refer to any ``place'' where memory discourses occur within a society for the purpose of remembering. Such sites of memory may operate within any number of social/cultural spheres such as national memory (war memorials, national holidays, etc.), religious memory (religious holidays, symbolic ritual acts, etc.), or family memory (traditional foods, birthdays, anniversaries) and may be thought of as distinct, but connected ``nodes'' of symbolic meaning within a complex network of cultural symbols---what \halbwachs called the ``social frameworks of memory.''%
    \autocite[38]{halbwachs1992}  

Every edge and node within the graph of a society's collective memory is the product of memory construction. It is an abstraction. In much the same way that historiography offers a schematic narrative of past events which is necessarily selective and intentional about what specific events, people, and ideas are germane to the purpose of the historian, so too social and cultural memory is selective of the particulars which it preserves and constructive in how it presents people, events, and ideas within particular symbolic systems. Thus, sites of memory are social spaces where memory is constructed. For our purposes, and following a number of modern practitioners of memory studies, I will use the term ``site'' of memory to describe any discrete person, place, practice or idea where such discourses of memory occur.%
    \footnote{Within Hebrew Bible studies, see especially the work of Ehud Ben Zvi as well as his student Ian Wilson, esp. \cite[72--74]{benzvi_st2017} and \cite[25--26]{wilson2017}.}

The Hebrew Bible is replete with sites of memory---ideas, people, places, and practices which have been imbued with significance by numerous societies since antiquity and which form a central component to the identities and self-understanding of (especially) Jews and Christians throughout the world. Take, for example, the Exodus from Egypt. Regardless of the historical reality of such an event, the story of the Exodus as recounted in the Hebrew Bible is the central narrative undergirding the biblical rationale for Israel's possession of the Land. Likewise, the Israelites are told to be kind to strangers and sojourners within their community based on the memory of Israel's enslavement in Egypt. Similarly, the Torah could be understood as a distinct (and particularly potent) site of memory found in the Hebrew Bible; the same goes for the figure of Moses. Each of these sites of memory (the Exodus, Torah, and Moses) are distinct but they also exhibit clear relationships within the network of discourses which are found in the Hebrew Bible. And moreover, each site of memory also relates to and bears distinct significance for the various religious communities which hold the Hebrew Bible as a part of their tradition within their distinct systems of symbolic meaning. Remembering these connections and their culturally defined significance is what cultural memory is all about.

%%%%%%%%%%%%%%%%%%%%%%%%%%%%%%%
%% DAVID AS A SITE OF MEMORY %%
%%%%%%%%%%%%%%%%%%%%%%%%%%%%%%%
\subsection{King David as a Site of Memory}

% David was important before Chronicles in the Bible
It is important to note that although the book of Chronicles is a work of cultural memory, it is unquestionably the case that the figure David was a prominent site of memory for ancient Israel long before the book of Chronicles was written. Chronicles, more so than Samuel--Kings, is characterized in terms of ``memory'' because it is clear that the Chronicler used Samuel--Kings as a primary source and the differences between the sources and the end-product are demonstrable. In other words, because we know that Chronicles is secondary to Samuel--Kings and we can see where the Chronicler departed from Samuel--Kings, it is easy to characterize those changes as the result of changes in cultural memory. But it is important to remember that even Samuel--Kings is the product of mnemonic construction and the David presented there already functioned as a special site of memory for ancient Israel. In other words, despite the fact that Samuel--Kings functions as a foundational source \emph{for Chronicles}, it should not be treated as if it was the origin of all Davidic traditions.

% David was important before Chronicles in the Ancient World 
Even setting aside the biblical material (e.g., Samuel--Kings, Psalms, et al.), it is demonstrably the case that the Davidic \emph{dynasty}---whatever one might think about David as an historical figure---had symbolic meaning in the ancient world which extended beyond the borders of Israel. For example, we know from the Old Aramaic inscription from Tel Dan that the term \aram{בת דוד} ``house of David'' was used as a dynastic name for the monarchy of the kingdom of Judah in the \bce{late ninth or early eighth centuries}.%
    \footnote{The \emph{editio princeps} were published in two articles: the first find as \cite{biran-naveh_iej1993}, and the subsequent fragments as \cite{biran-naveh_iej1995}.}
Likewise, it has been suggested that the Mesha Stele, too, refers to the ``house of David,'' although this reading is not secure.%
    \footnote{The reading \aram{בת דוד} was proposed by Lemaire, but his reading is not universally accepted. See \cite{lemaire_sel1994} and \cite{lemaire_bar1994}. The Mesha inscription is typically dated to the \bce{mid-ninth century} and thus would be slightly earlier than the reference in the Tel Dan inscription, if Lemaire is correct.}
Although such references have traditionally been used to bolster claims of an historical David, for our purposes it suffices to say that around the turn of the \bce{eight century}, ``David'' existed as a meaningful eponymous symbol and site of memory with respect to the monarchy of Judah. Thus, when we turn to the biblical portrayals of the figure David (which, by most accounts were products of later periods of Israelite history than Tel Dan and Mesha), it is important to keep in mind that those portrayals are participating in established discourses about David. This is all the more important when we consider the book of Chronicles which represents some of the latest strata of memory preserved in the Hebrew Bible. Thus when we discuss the figure of David as a site of memory which the book of Chronicles engages with extensively, I want to emphasize that the processes of constructing the remembered figure of David did not begin with the Chronicler just as it did it end with the Chronicler.

Although the particular relationship between the book of Chronicles and the books of Samuel and Kings is a matter of scholarly debate, it is generally agreed that Samuel--Kings forms the basis for much of the Chronicler's depiction of Israel's history.%
    \footnote{The observation was made as early as de Wette in the early nineteenth century in his \cite*{dewette1806}. More recently, see especially the work of McKenzie
        \cite*{mckenzie1985};
        \cite{mckenzie_graham-mckenzie1999};
        \cite[66--71]{knoppers2003}; and 
        \cite[30--42]{klein2006} as well as that of 
        \cite[74--74]{carr2011}. Notable exceptions, however, do exist. See especially the work of 
        \cite{auld1994}; 
        \cite{auld_graham-mckenzie1999} and 
        \cite{person2010}.}
A great deal of work has been done analyzing the particular literary relationship between Samuel--Kings and Chronicles and the textual processes involved---e.g., what version(s) of Samuel--Kings the Chronicler may have used, etc.---but thinking in terms of social memory requires us to consider the relationship between the texts in \emph{social} terms. In other words, not just to ask \emph{what} the received traditions about David said, but to consider the \emph{role} and \emph{status} of those traditions and to consider why they were (or were not) significant within a particular social context.

% Social context matters; what mattered then might not matter now.
Take, for example, the so-called History of David's Rise (HDR) narrative  and the dramatic family disputes that preoccupied the latter
years of David's reign and those of his son Solomon (the so-called ``Succession Narrative'' [SN]) which form a core set of narratives for Samuel--Kings, but whose intrigue are essentially absent from the book fo Chronicles.%
    \footnote{Although the compositional and redactional history of the Deuteronomistic History is hotly debated---with wildly divergent scholarly opinions---I will take as my point of departure the centrist view of McCarter, Halpern, and specifically Knapp which view the HDR and SN (collectively, the ``Court Narrative'' [CN] or ``Traditions of David's Rise and Reign'' [TDRR] \emph{per} Knapp) as royal apologia. I follow Knapp in his view that these traditions do not represent ``the residue of a single apologetic composition'' (161), but rather a diverse set of traditions. However, because the sources cannot meaningfully be parsed, I will also follow him in ``[dealing] with the early narrative traditions in their entirety'' (161). See 
        \cite{knapp2015};
        \cite{mccarter_interpretation1981};
        \cite{mccarter_jbl1980};
        \cite{mccarter1980};
        \cite{halpern2001}.}
It is widely held that that the HDR and SN should be understood as forms of ancient royal apologia---an effort by the author(s) to legitimize David's actions which might otherwise have been construed as a usurpation of the divinely elected king, Saul. Knapp, for example observes that ``[i]n some ways, [the Traditions of David's Rise and Reign] is the paradigmatic ancient Near Eastern apology.''%
    \autocite[218]{knapp2015}
He elaborates:

\begin{quote}
    The apologist employs nearly every apologetic motif in his effort to legitimize David, including passivity, transcendent non-retaliation, the unworthy predecessor, military prowess, and the entire triad of establishing legitimacy.%
    \autocite{knapp2015}
\end{quote}

Clearly the apologist sought to make a forceful and potent argument in
favor of David's legitimacy. The apologist operated within his social context---using literary devices and forms which were meaningful in his society---and engaged in discourses about David's legitimacy in an attempt to define David's rise and reign in a particular (positive) way. Of course, we know that for ancient Israel David \emph{did} become known as the legitimate king of Israel and Judah \emph{par excellence}---a figure against whom subsequent kings would be measured. In this way, the construction of the apologist's David was ultimately successful.%
    \footnote{It is worth pointing out that such an apologia likely arose in response to accusations of usurpation. Thus, we can imagine that the HDR is representative of the ``last word'' on the matter, which was an attempt to suppress alternative voices in the matter which questioned the legitimacy of David's rule, the means by which he gained the throne, and the manner of his succession. These discourses were not entirely suppressed from the Hebrew Bible, as evidenced by the figure Shimei and his condemnation of David as a usurper, ``Come out, come out, Oh man of blood!, Oh worthless man! Yahweh has repaid you all the blood of the house of Saul, in whose place you reign'' (2~Sam 16:7b).}

For all the potency of these stories, one may wonder why they did not make it into the Chronicler's history. That is to say, why remove such persuasive, and effective material? The answer, I think, is quite simple: the Chronicler was operating within a social milieu which not only accepted the legitimacy of David and his heirs, but celebrated them as foundational figures. In other words in the symbolic world of the Chronicler, David was significant \emph{because} he was king and---his legitimacy was assumed and celebrated. The discourses that HDR participated in had long been resolved and the Davidic dynasty was thoroughly legitimate in the mind of the Chronicler. As such, it was sufficient for the Chronicler to simply recount the death of Saul---which David had no part in---and the subsequent anointing of David. No mess; a thoroughly unremarkable transfer of power. Similarly, the Chronicler makes no mention of the difficult power struggles that occurred near the end of David's life between him and his sons. Instead, it sufficed for the Chronicler to state: 
    \begin{hebrewtext}
        \versenum{1 Chr 23:1}
        וְדָוִיד זָקֵן וְשָׂבַע יָמִים וַיַּמְלֵךְ אֶת־שְׁלֹמֹה בְנוֹ עַל־יִשְׂרָאֵל׃
    \end{hebrewtext}
    \begin{translation}
        When David was old and full of days, he made Solomon, his son, king over Israel
    \end{translation}
\noindent
The struggle between Solomon and Adonijah following David's death is likewise omitted. Instead, opening verse of 2 Chronicles reads simply:
    \begin{hebrewtext}
        \versenum{2 Chr 1:1}
        וַיִּתְחַזֵּק שְׁלֹמֹה בֶן־דָּוִיד עַל־מַלְכוּתוֹ וַיהוָה אֱלֹהָיו עִמּוֹ וַיְגַדְּלֵהוּ לְמָעְלָה׃
    \end{hebrewtext}
    \begin{translation}
        Solomon, the son of David, established himself in his kingdom, and Yahweh his God was with him and made him exceedingly great.   
    \end{translation}
\noindent
It seems, therefore, that the DH was so successful in its apologetic that it precluded the need for continued apologia. The Chronicler had no need to ``legitimize'' the \emph{fact of} the Davidic dynasty, but instead would focus his attention on defining the \emph{significance of} that dynasty for his own readers in a dramatically different social setting.

\subsubsection{The David of Chronicles}
% How is David portrayed in Chr?
    % Rightful King
    % Pious prayer-maker
    % David the musician
    % Temple-builder
    % "After God's own Heart"

\subsubsection{Historicizing the Chronicler's Memory of David}
    % What informs this portrait?
    % Biblical interpretation? perhaps?
    % Biblical Memory? Yes.
        % Talk about Psalms
        % Where does this tradition come from?

\subsection{The Jerusalem Temple as a Site of Memory}
Another potent site of memory in Chronicles is the temple in Jerusalem. As with David, the memory of the Temple in Chronicles is not entirely novel. Already in the book of Deuteronomy the mythology surrounding the divine selection of Jerusalem and the uniquely ordained site of the Solomonic temple had been well-established. This development is easily seen by contrasting the ways that the Covenant Code of \exod 20 in which \yahweh seems to command the Israelites to establish cult sites \hebrew{בְּכָל־הַמָּקוֹם אֲשֶׁר אַזְכִּיר אֶת־שְׁמִי} ``in every place that I commemorate my name'' (\exod 20:24)%
    \footnote{The grammar of 20:24 is a bit unclear and it is hard not to speculate that some of the difficulty with it is due to the implication that \yahweh seemed cool with multiple cult sites compared to its counterpart in \deut 12:5. This discomfort is illustrated in  \sampent's omission of \hebrew{כל} with a result that \hebrew{מָקוֹם} is conceptually singular (in \emph{the} place), while \lxx, Syriac, and the Targums all support the reading ``in every place.'' I am willing to entertain the suggestion made by the Niqqudim, that the clause \hebrew{בְּכָל־הַמָּקוֹם} ``in every place'' modifies only the following clause \hebrew{אָבוֹא אֵלֶיךָ וּבֵרַכְתִּיךָ} ``I will come to you and bless you'' and not completing the action of the preceding \hebrew{מִזְבַּח אֲדָמָה תַּעֲשֶׂה־לִּי וְזָבַחְתָּ עָלָיו} ``you will make an earthen altar for me and make sacrifices upon it.'' Indeed, the first person form \hebrew{אַזְכִּיר} favors the former reading. The Syriac does, however, offer a variant suggesting a possible second person \emph{Vorlage} \hebrew{תַּזְכִּיר*}, which I find intriguing, ``in every place that \emph{you} commemorate my name, I will come....'' Another complimentary explanation for the grammar is to read the imperfect form \hebrew{אָבוֹא} as a volitive ``[in order that] I might come to you and bless you.'' I admit that all of these options are tenuous, but the clear departure from \exod 20 found in \deut 12 at least suggests that this bit of the Covenant Code seemed ambiguous or uncomfortable to the author of \deuteronomy such that he felt the need to forcefully clarify his position.}
with that of Deuteronomy, in which \yahweh commands the Israelites to destroy all cult cites within the land and furthermore that: 

\begin{hebrewtext}
    \versenum{\deut 12:5}
    כִּי אִם־אֶל־הַמָּקוֹם אֲשֶׁר־יִבְחַר יְהוָה אֱלֹהֵיכֶם מִכָּל־שִׁבְטֵיכֶם לָשׂוּם אֶת־\\שְׁמוֹ שָׁם לְשִׁכְנוֹ תִדְרְשׁוּ וּבָאתָ שָׁמָּה׃
    \versenum{6}
    וַהֲבֵאתֶם שָׁמָּה עֹלֹתֵיכֶם וְזִבְחֵיכֶם וְאֵת מַעְשְׂרֹתֵיכֶם וְאֵת תְּרוּמַת יֶדְכֶם וְנִדְרֵיכֶם וְנִדְבֹתֵיכֶם וּבְכֹרֹת בְּקַרְכֶם וְצֹאנְכֶם׃
\end{hebrewtext}
\begin{translation}
    \versenum{\deut 12:5}
    You shall seek the place that \yahweh your God will choose from among all your tribes as his dwelling to put his name there. You shall go there
    \versenum{6}
    and you will bring your burnt offerings there as well as your sacrifices, your tithes and the offerings of your hands, your votive gifts, your freewill offerings, and the firstborn of your cattle and flocks. 
\end{translation}

% The Temple
    % How is the temple portrayed in Chr?
        % Levites ?
    % Is there anything unique about this?
    % What about the Physicality of the temple makes it unique as a site of memory?

\subsection{Magnetism and Convergence of Mnemonic Sites}
% Convergence and Magnetism between Sites of Memory
    % MAGNETISM OF SPECIFIC SITES: 
        % Give basic examples for:
            % David (Goliath?)
            % Temple (Creation?)
    % MAGNETISM BETWEEN SITES
        % Look at the Araunah/Census story of Chr.
            % Cf. with Sam--Kings
            % David and Temple converge (within certain boundaries) 
            % How might external factors affect how this was "remembered"
                    % Was Araunah's threshing floor known?
                    % What about the imagined role of the king?
                        % David as a cult leader
                        % Why not Solomon?
        % Mt. Moriah is an example of further `magnetism' to another potent site of memory (the Aqedah)

\subsection*{Section Conclusions}
% Concluding bit that points toward "Social Frameworks and Recontextualization" These are "active" sites of memory they are received, reimagined, and reconstructed