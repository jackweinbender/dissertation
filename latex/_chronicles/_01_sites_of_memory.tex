% !TEX root = dissertation.tex

\section{Sites of Memory in the Book of Chronicles}
% Sites of Memory

% TODO: I need to explain what sites of memory are
% I need to get from Mosaic discourse to sites of memory


% Talk about the figure David in general terms (whole-book ideas)
Within the memory of ancient Israel, it is unquestionably the case that the figure David was a prominent mnemonic site for ancient Israel and the book of \chronicles engages with that mnemonic site extensively.
    % How is David portrayed in Chr?
        % David the musician
        % Good-guy
        % Prayer-maker
        % Compare with portrayal of other figures like Abram in GA?
    % What informs this portrait?
        % Biblical interpretation? perhaps?
        % Biblical Memory? Yes.
            % Talk about Psalms
            % Where does this tradition come from?

% The Temple
    % How is the temple portrayed in Chr?
        % Levites ?
    % Is there anything unique about this?
    % What about the Physicality of the temple makes it unique as a site of memory?

% Convergence and Magnetism between Sites of Memory
    % Look at the Arunah/Census story of Chr.
        % Cf. with Sam--Kings
        % David and Temple converge (within certain boundaries) 
        % How might external factors affect how this was "remembered"
                % Was arunah's threshing floor known?
                % What about the imagined role of the king?
                    % David as a cult leader
                    % Why not Solomon?
        % Mt. Moriah is an example of further `magnetism' to another potent site of memory (the Aqedah)