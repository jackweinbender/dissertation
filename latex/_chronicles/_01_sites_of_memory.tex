% !TEX root = dissertation.tex

\section{Sites of Memory in the Book of Chronicles}
% Sites of Memory

% nora_representations1989

% TODO: I need to explain what sites of memory are
we have plenty of language to describe the various processes of individual memory, but one of the main problems we have when talking about social memory and cultural memory is that we lack good language describe the structures and functions of those mnemonic systems. As such, memory theorists have adopted a number of analogies and terms to describe how societies remember and how individuals and groups interact with memory at the social level. 

It is important to remember that because social memory is a social construct we must not equate the remembered past with the events, experiences, and individuals which informed it. Rather, social memory should be thought of as a schematic representation of a set of mnemonic ``sites.'' It is an abstraction. In much the same way that historiography offers a narratized schematic of past events which necessarily is selective and intentional about what specific events, people, and ideas are germane to the purpose of the historian, so too social memory is selective of the particulars which it preserves. 

These particulars are commonly referred to by memory theorists as ``sites'' of memory, a term coined by Pierre Nora in the early 1970's which has been adopted and adapted by numerous theorists since then.%
    \footnote{%
        The term was originally coined by Nora in the work
        \cite*{nora_goff-nora1974}, and used subsequently in 
        \cite*{nora1984} and 
        \cite*{nora_representations1989}. For a discussion of Nora's use of the term and its reception, see 
        \cite{szpociński_teksty-drugie2016}.}
Although Nora did not clearly define the term, a ``site of memory'' (\emph{lieux de mémoire}), as used by Nora, might better be translated as a ``place of remembrance.'' For Nora, modern-day ``sites'' of memory existed ``because there are no longer \emph{milieux de mémoire}, real environments of memory.''%
    \autocite[7]{nora_representations1989}
In other words, because modern historical consciousness, by Nora's reckoning, has all but eradicated ``memory,'' the preservation of memory in the modern era has been relegated to particular ``sites'' of memory---monuments, structures, and practices whose purpose is to perpetuate memory. He writes:

\begin{quote}
    \emph{Lieux de mémoire} are simple and ambiguous, natural and artificial, at once immediately available in concrete sensual experience and susceptible to the most abstract elaboration. Indeed they are \emph{lieux} in three senses of the word---material, symbolic, and functional. Even an apparently purely material site, like an archive, becomes a \emph{lieu de mémoire} only if the imagination invests it with a symbolic aura. A purely functional site, like a classroom manual, a testament, or a veterans' reunion belong only inasmuch as it is also the object of a ritual. And the observation of a commemorative minute of silence, an extreme example of a strictly symbolic action, serves as a concentrated appeal to memory by literally breaking the temporal continuity.\autocite[18--19]{nora_representations1989}
\end{quote}

\noindent
Sites of memory, therefore, are not entirely abstract and intellectual but bear on the practice and materiality of a society in addition to its symbolic significance.

Although Nora's original use of the term tended to focus especially on sites of memory which bear on so-called ``great traditions''%
    \footnote{As coined by Redfield in \cite*[41--42]{redfield1956}.}
of political and ideological importance such as national monuments and archives, the modern use of the term tends to be more abstract and to refer to any ``place'' where memory discourses occur within a society. For our purposes, and following a number of modern practitioners of memory studies, I will use the term ``site'' of memory to describe any discrete person, place, practice or idea where such discourses of memory occur.%
    \footnote{Within Hebrew Bible studies, see especially the work of Ehud Ben Zvi as well as his student Ian Wilson, esp. \cite[72--74]{benzvi_st2017} and \cite[25--26]{wilson2017}.}




\subsection{King David as a Site of Memory}
% Talk about the figure David in general terms (whole-book ideas)
Within the memory of ancient Israel, it is unquestionably the case that the figure David was a prominent site of memory for ancient Israel and the book of \chronicles engages with that mnemonic site extensively.

% How is David portrayed in Chr?
    % David the musician
    % Good-guy
    % Prayer-maker
    % Compare with portrayal of other figures like Abram in GA?

    % What informs this portrait?
    % Biblical interpretation? perhaps?
    % Biblical Memory? Yes.
        % Talk about Psalms
        % Where does this tradition come from?


\subsection{The Jerusalem Temple as a Site of Memory}
% The Temple
    % How is the temple portrayed in Chr?
        % Levites ?
    % Is there anything unique about this?
    % What about the Physicality of the temple makes it unique as a site of memory?

\subsection{Magnetism and Convergence of Mnemonic Sites}
% Convergence and Magnetism between Sites of Memory
    % MAGNETISM OF SPECIFIC SITES: 
        % Give basic examples for:
            % David (Goliath?)
            % Temple (Creation?)
    % MAGNETISM BETWEEN SITES
        % Look at the Araunah/Census story of Chr.
            % Cf. with Sam--Kings
            % David and Temple converge (within certain boundaries) 
            % How might external factors affect how this was "remembered"
                    % Was Araunah's threshing floor known?
                    % What about the imagined role of the king?
                        % David as a cult leader
                        % Why not Solomon?
        % Mt. Moriah is an example of further `magnetism' to another potent site of memory (the Aqedah)

\subsection*{Section Conclusions}
% Concluding bit that points toward "Social Frameworks and Recontextualization" These are "active" sites of memory they are received, reimagined, and reconstructed