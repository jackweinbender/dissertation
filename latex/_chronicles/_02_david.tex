% !TEX root = dissertation.tex

%%%%%%%%%%%%%%%%%%%%%%%%%%%%%%%
%% DAVID AS A SITE OF MEMORY %%
%%%%%%%%%%%%%%%%%%%%%%%%%%%%%%%
\section{King David as a Site of Memory}

% David was important before Chronicles in the Bible
It is important to note that although the book of \chronicles is a work of cultural memory, it is unquestionably the case that the figure David was a prominent site of memory for ancient Israel long before the book of \chronicles was written. \chronicles, more so than Samuel--Kings, is characterized in terms of ``memory'' because it is clear that the \chronicler%
    \footnote{My use of the term ``Chronicler'' is meant only to reference the author(s) of the book of \chronicles. Although the term is sometimes associated with a particular theory about the composition of \chronicles, Ezra and Nehemiah, I am not using it as such.}
used Samuel--Kings as a primary source and the differences between the sources and the end-product are demonstrable. In other words, because we know that \chronicles is secondary to Samuel--Kings and we can see where the \chronicler departed from Samuel--Kings, it is easy to characterize those changes as the result of changes in cultural memory. But it is important to remember that even Samuel--Kings is the product of mnemonic construction and the David presented there already functioned as a special site of memory for ancient Israel. In other words, despite the fact that Samuel--Kings functions as a foundational source \emph{for \chronicles}, it should not be treated as if it was the origin of all Davidic traditions.\autocite{frohlich_frohlich2019}

% David was important before Chronicles in the Ancient World 
Even setting aside the biblical material (e.g., Samuel--Kings, Psalms, et al.), it is demonstrably the case that the Davidic \emph{dynasty}---whatever one might think about David as an historical figure---had symbolic meaning in the ancient world which extended beyond the borders of Israel. For example, we know from the Old Aramaic inscription from Tel Dan that the term \aram{בת דוד} ``house of David'' was used as a dynastic name for the monarchy of the kingdom of Judah in the \bce{late ninth or early eighth centuries}.%
    \footnote{The \emph{editio princeps} were published in two articles: the first find as \cite{biran-naveh_iej1993}, and the subsequent fragments as \cite{biran-naveh_iej1995}.}
Likewise, it has been suggested that the Mesha Stele, too, refers to the ``house of David,'' although this reading is not secure.%
    \footnote{The reading \aram{בת דוד} was proposed by Lemaire, but his reading is not universally accepted. See \cite{lemaire_sel1994} and \cite{lemaire_bar1994}. The Mesha inscription is typically dated to the \bce{mid-ninth century} and thus would be slightly earlier than the reference in the Tel Dan inscription, if Lemaire is correct.}
Although such references have traditionally been used to bolster claims of an historical David, for our purposes it suffices to say that around the turn of the \bce{eight century}, ``David'' existed as a meaningful eponymous symbol and site of memory with respect to the monarchy of Judah. Thus, when we turn to the biblical portrayals of the figure David (which, by most accounts were products of later periods of Israelite history than Tel Dan and Mesha), it is important to keep in mind that those portrayals are participating in established discourses about David. This is all the more important when we consider the book of \chronicles which represents some of the latest strata of memory preserved in the Hebrew Bible. Thus when we discuss the figure of David as a site of memory which the book of \chronicles engages with extensively, I want to emphasize that the processes of constructing the remembered figure of David did not begin with the \chronicler just as it did it end with the \chronicler.\autocite{frohlich_frohlich2019}

Although the particular relationship between the book of \chronicles and the books of Samuel and Kings is a matter of scholarly debate, it is generally agreed that Samuel--Kings forms the basis for much of the \chronicler's depiction of Israel's history.%
    \footnote{The observation was made as early as de Wette in the early nineteenth century in his \cite*{dewette1806}. More recently, see especially the work of McKenzie
        \cite*{mckenzie1985};
        \cite{mckenzie_graham-mckenzie1999};
        \cite[66--71]{knoppers2003}; and 
        \cite[30--42]{klein2006} as well as that of 
        \cite[74--74]{carr2011}. Notable exceptions, however, do exist. See especially the work of 
        \cite{auld1994}; 
        \cite{auld_graham-mckenzie1999} and 
        \cite{person2010}.}
A great deal of work has been done analyzing the particular literary relationship between Samuel--Kings and \chronicles and the textual processes involved---e.g., what version(s) of Samuel--Kings the \chronicler may have used, etc.---but thinking in terms of social memory requires us to consider the relationship between the texts in \emph{social} terms. In other words, not just to ask \emph{what} the received traditions about David said, but to consider the \emph{role} and \emph{status} of those traditions and to consider why they were (or were not) significant within a particular social context.

% Social context matters; what mattered then might not matter now.
Take, for example, the so-called History of David's Rise (HDR) narrative  and the dramatic family disputes that preoccupied the latter years of David's reign and those of his son Solomon (the so-called ``Succession Narrative'' [SN]) which form a core set of narratives for Samuel--Kings, but whose intrigue are essentially absent from the book fo \chronicles.%
    \footnote{Although the compositional and redactional history of the Deuteronomistic History is hotly debated---with wildly divergent scholarly opinions---I will take as my point of departure the centrist view of McCarter, Halpern, and specifically Knapp which view the HDR and SN (collectively, the ``Court Narrative'' [CN] or ``Traditions of David's Rise and Reign'' [TDRR] \emph{per} Knapp) as royal apologia. I follow Knapp in his view that these traditions do not represent ``the residue of a single apologetic composition'' (161), but rather a diverse set of traditions. However, because the sources cannot meaningfully be parsed, I will also follow him in ``[dealing] with the early narrative traditions in their entirety'' (161). See 
        \cite{knapp2015};
        \cite{mccarter_interpretation1981};
        \cite{mccarter_jbl1980};
        \cite{mccarter1980};
        \cite{halpern2001}.}
It is widely held that that the HDR and SN should be understood as forms of ancient royal apologia---an effort by the author(s) to legitimize David's actions which might otherwise have been construed as a usurpation of the divinely elected king, Saul. Knapp, for example observes that ``[i]n some ways, [the Traditions of David's Rise and Reign] is the paradigmatic ancient Near Eastern apology.''%
    \autocite[218]{knapp2015}
He elaborates:

\begin{quote}
    The apologist employs nearly every apologetic motif in his effort to legitimize David, including passivity, transcendent non-retaliation, the unworthy predecessor, military prowess, and the entire triad of establishing legitimacy.%
    \autocite{knapp2015}
\end{quote}

Clearly the apologist sought to make a forceful and potent argument in favor of David's legitimacy. The apologist operated within his social context---using literary devices and forms which were meaningful in his society---and engaged in discourses about David's legitimacy in an attempt to define David's rise and reign in a particular (positive) way. Of course, we know that for ancient Israel David \emph{did} become known as the legitimate king of Israel and Judah \emph{par excellence}---a figure against whom subsequent kings would be measured. In this way, the construction of the apologist's David was ultimately successful.%
    \footnote{It is worth pointing out that such an apologia likely arose in response to accusations of usurpation. Thus, we can imagine that the HDR is representative of the ``last word'' on the matter, which was an attempt to suppress alternative voices in the matter which questioned the legitimacy of David's rule, the means by which he gained the throne, and the manner of his succession. These discourses were not entirely suppressed from the Hebrew Bible, as evidenced by the figure Shimei and his condemnation of David as a usurper, ``Come out, come out, Oh man of blood!, Oh worthless man! Yahweh has repaid you all the blood of the house of Saul, in whose place you reign'' (2~Sam 16:7b).}

For all the potency of these stories, one may wonder why they did not make it into the \chronicler's history. That is to say, why remove such persuasive, and effective material? The answer, I think, is quite simple: the \chronicler was operating within a social milieu which not only accepted the legitimacy of David and his heirs, but celebrated them as foundational figures. In other words in the symbolic world of the \chronicler, David was significant \emph{because} he was king and---his legitimacy was assumed and celebrated. The discourses that HDR participated in had long been resolved and the Davidic dynasty was thoroughly legitimate in the mind of the \chronicler. As such, it was sufficient for the \chronicler to simply recount the death of Saul---which David had no part in---and the subsequent anointing of David. No mess; a thoroughly unremarkable transfer of power. Similarly, the \chronicler makes no mention of the difficult power struggles that occurred near the end of David's life between him and his sons. Instead, it sufficed for the \chronicler to state: 
    \begin{hebrewtext}
        \versenum{1 Chr 23:1}
        וְדָוִיד זָקֵן וְשָׂבַע יָמִים וַיַּמְלֵךְ אֶת־שְׁלֹמֹה בְנוֹ עַל־יִשְׂרָאֵל׃
    \end{hebrewtext}
    \begin{translation}
        When David was old and full of days, he made Solomon, his son, king over Israel
    \end{translation}
\noindent
The struggle between Solomon and Adonijah following David's death is likewise omitted. Instead, opening verse of 2~\chronicles reads simply:
    \begin{hebrewtext}
        \versenum{2 Chr 1:1}
        וַיִּתְחַזֵּק שְׁלֹמֹה בֶן־דָּוִיד עַל־מַלְכוּתוֹ וַיהוָה אֱלֹהָיו עִמּוֹ וַיְגַדְּלֵהוּ לְמָעְלָה׃
    \end{hebrewtext}
    \begin{translation}
        Solomon, the son of David, established himself in his kingdom, and Yahweh his God was with him and made him exceedingly great.   
    \end{translation}
\noindent
It seems, therefore, that the DH was so successful in its apologetic that it precluded the need for continued apologia. the \chronicler had no need to ``legitimize'' the \emph{fact of} the Davidic dynasty, but instead would focus his attention on defining the \emph{significance of} that dynasty for his own readers in a dramatically different social setting. 

Thus the process of ``remembering'' David in \chronicles can be viewed from two different angles which map onto the dual valences of the term ``remember'': to ``recall'' and to ``commemorate.'' On the one hand, the \chronicler ``recalls'' stories about David which are adapted to the frameworks of the \chronicler's social situation. the \chronicler is a product of his time and society and as such inherited sets of traditions about David, the past, and the world more broadly, which color how he understands the history of Israel and David in particular. On the other hand, the composition of the book of \chronicles is itself an act of commemoration which (as we've noted) is a conscious, constructive process. It represents the process of memory encoding and the construction of cultural memory from which future rememberers would draw. As a work literature, it also bears the idiosyncrasies of its author(s), however constrained by their social milieu they may have been. In fact, determining which of these processes best accounts for any particular ``innovation'' of \chronicles is quite difficult. Was the \chronicler consciously ``reshaping'' the memory of David? Or was the \chronicler more passively reproducing a composite picture of David that he inherited from his culture? Traditional approaches to the book of \chronicles have tended to attribute a great deal of agency to the \chronicler as an innovator of tradition. But thinking in terms of cultural memory pushes us to consider a fuller picture of how cultural memory is created and calls into question whether every theological or ideological augmentation of the \chronicler should be attributed to his novel understanding of the Israelite past. Such an approach takes into account that textual ``sources'' are not merely copied and ``altered,'' but are read, internalized, believed, understood, and reasoned about, which is to say, \emph{remembered}.

\subsection{The David of \chronicles}
How then was David remembered in \chronicles? This question carries with it the assumption that the author of \chronicles was not simply copying-and-changing Samuel--Kings (or other traditions), but rather was a product of a \emph{remembering community} and participated in memory discourses at various sites within the cultural memory of \secondtemple Judaism. Answering this question requires that we not only consider what sources the \chronicler may have used and how he altered those sources, but also to consider the social frameworks which shaped how those sources were received by the \chronicler and how they affected how the \chronicler presented (or commemorated) his work.

Although the David of \chronicles largely resembles that of the DH (he is recognizably the same figure), his function within the narrative of the book of \chronicles is different than that of the DH and that difference can be seen in how the \chronicler portrays and uses him. In both works David is beloved, but he is noticeably less-flawed in the book fo \chronicles. This is not to say that  David is treated as entirely faultless in the book of \chronicles, but I think it is fair to say that the overall portrait presented by the \chronicler is more willing to overlook (and literally to omit) some of David's more egregious acts, and to highlight his role as a model King. This positive portrayal of David in \chronicles is well documented and oft-repeated, so it will suffice for me to focus on two of the most significant features of the \chronicler's portrayal of David, specifically, his portrayal as a divinely elected king, and his role in the establishment of the Israelite cult in Jerusalem.%
    \footnote{See \cite{jarick_frohlich2019}; \cite[347--383]{japhet2009} \cite{knoppers_biblica1995}; \cite[47--48]{japhet1993}; \cite[44--48]{klein2006}; \cite[80--85]{knoppers2003}.}

\subsubsection{David the Divinely Elected King}
First, as I have just alluded to, in the book of \chronicles, David is portrayed as the quintessential, rightful Israelite ruler, elected by \yahweh (1~Chr 10:14) and anointed by the elders of Israel to lead the people (1 Chr 11:1--3). By comparison to the account in Samuel--Kings, the process by which David becomes the ruler of Israel is somewhat less contentious. The apologetic tone of the HDR narratives is nowhere to be found. The rationale for Saul's demise is, like in the DH, predicated on his supposed infidelity to \yahweh, with special reference to his consultation with a medium (although, the story is not told in \chronicles), however, the election of David as Saul's ``successor'' is described by the \chronicler does not include Saul, aside from a passing reference to his death and infidelities. David himself offers his version of events in 1 Chr 28:4:
\begin{hebrewtext}
    \versenum{1 Chr 28:4}
    וַיִּבְחַר יְהוָה אֱלֹהֵי יִשְׂרָאֵל בִּי מִכֹּל בֵּית־אָבִי לִהְיוֹת לְמֶלֶךְ עַל־יִשְׂרָאֵל לְעוֹלָם כִּי בִיהוּדָה בָּחַר לְנָגִיד וּבְבֵית יְהוּדָה בֵּית אָבִי וּבִבְנֵי אָבִי בִּי רָצָה לְהַמְלִיךְ עַל־כָּל־יִשְׂרָאֵל׃ 
\end{hebrewtext}
\begin{translation}
    \versenum{1 Chr 28:4}
    \yahweh, the God of Israel chose me from among my father's whole house to be king over Israel forever. He chose Judah to be a leader and (from) the house of Judah, the house of my father and (from) the house of my father, he took delight in me to make (me) king over all Israel.
\end{translation}
Conspicuously absent from the \chronicler's narrative and David's summary, are the major conflicts with Saul during David's rise to power. In fact, if one did not know better, simply removing all references to Saul in \chronicles would not meaningfully change how David's election is described.%
    \footnote{This fact raises the question of why the \chronicler \emph{did not} simply omit Saul. I suspect that, although not favored Saul was a useful foil narratively and was a well-enough known figure that omitting him entirely simply did not make sense. Saul was, doubtless, a major figure in the traditions of early Israel.}

 Similarly, the tumult within David's court at the end of his life and the succession of Solomon are omitted by the \chronicler, where 1~Kings begins with a feeble, impotent David and his messy succession by Solomon, 1~Chr 23:1 is content simply to report that:
\begin{hebrewtext}
    \versenum{1 Chr 23:1}
    וְדָוִיד זָקֵן וְשָׂבַע יָמִים וַיַּמְלֵךְ אֶת־שְׁלֹמֹה בְנוֹ עַל־יִשְׂרָאֵל׃
\end{hebrewtext}
\begin{translation}
    When David was old and full of days, he made Solomon, his son, king over Israel.
\end{translation}
\noindent
It went \emph{so} well, in fact, that David saw fit to do it a second time, according to 1 Chr 29:22b--23:
\begin{hebrewtext}
    \versenum{1 Chr 29:22b}
    וַיַּמְלִיכוּ שֵׁנִית לִשְׁלֹמֹה בֶן־דָּוִיד וַיִּמְשְׁחוּ לַיהוָה לְנָגִיד וּלְצָדוֹק לְכֹהֵן׃ 
    \versenum{23}
    וַיֵּשֶׁב שְׁלֹמֹה עַל־כִּסֵּא יְהוָה לְמֶלֶךְ תַּחַת־דָּוִיד אָבִיו וַיַּצְלַח וַיִּשְׁמְעוּ אֵלָיו כָּל־יִשְׂרָאֵל׃
\end{hebrewtext}
\begin{translation}
    \versenum{1 Chr 29:22b}
    Then they made Solomon, son of David, king a second time and they anointed him by \yahweh as a prince as well as Zadok as a priest.
    \versenum{23}
    And Solomon sat on the throne of \yahweh as king in place of David, his father. And he prospered and all Israel obeyed him.
\end{translation}
\noindent
These matter-of-fact descriptions contrast sharply with the events depicted in 1 Kings: Adonijah's self-exaltation (1~Kgs 1:5--53), David's deathbed speech to Solomon (1~Kgs 2:1--9), Solomon's subsequent conflict with Adonijah over Abishag (1~Kgs 2:13--25), with Joab (1~Kgs 2:28--35), and with Shimei (1~Kgs 2:36--46); all of which culminates with the ominous pronouncement of 1~Kgs 2:46b:
\begin{hebrewtext}
    וְהַמַּמְלָכָה נָכוֹנָה בְּיַד־שְׁלֹמֹה׃
\end{hebrewtext}
\begin{translation}
    So the kingdom was established in the hand of Solomon.
\end{translation}
The contrast between the violent establishment of the kingdom ``in the hand of Solomon'' and the popular assent of the people to both the reigns of David and Solomon in \chronicles could not be more clear. On the one hand, the accounts of 1 Kings offer narratives which provide \emph{rationale} for the events that take place---everything that David and Solomon do is framed as a sensible response to wrongdoing. The  descriptions of \chronicles, on the other hand, are not at all interested in providing such rationales, but rather \emph{assume} the premise of 1~Kings which. Instead, \chronicles offers plain, black-and-white, narratives which---by virtue of their declarative rhetoric---help to reinforce the idea that David and his successors were not only elect by \yahweh, but were ``good'' kings whose reigns were not contested, but were supported by the population at large.

% TODO: Note the change in language from "throne of David" to throne of "yahweh" also in 16:14

\subsubsection{David the Temple-builder (Almost)}
While the portrayal of David as the unquestioned founder of the Israelite monarchy in \chronicles is accomplished primarily through omitting details of David's faults, the book of \chronicles makes its most significant \emph{positive} contribution to its picture of David by crediting him as the founder of the Jerusalem Temple. 

David did not build the Temple in Jerusalem, of course, but the picture that the \chroncicler paints of how the Jerusalem Temple came about leaves little doubt about whose idea it \emph{really} was, David's. In the mind of the \chronicler, although Solomon may have been the one to \emph{build} the Temple, David wrote, directed, funded, and produced the project.

In both 2 Sam 7 and 1 Chr 17, David expresses a desire to build a temple for \yahweh and is rebuffed. Instead, the prophet Nathan tells David that \yahweh would establish David's line through his son, Solomon. Although neither account gives a reason for \yahweh's preference toward Solomon, later in \chronicles, David states that the reason was that David was a man of war, while Solomon would be a man of peace:

\begin{hebrewtext}
    \versenum{1 Chr 22:7}
    וַיֹּאמֶר דָּוִיד לִשְׁלֹמֹה בְּנוֹ [בְּנִי] אֲנִי הָיָה עִם־לְבָבִי לִבְנוֹת בַּיִת לְשֵׁם יְהוָה אֱלֹהָי׃ 
    \versenum{8}
    וַיְהִי עָלַי דְּבַר־יְהוָה לֵאמֹר דָּם לָרֹב שָׁפַכְתָּ וּמִלְחָמוֹת גְּדֹלוֹת עָשִׂיתָ לֹא־תִבְנֶה בַיִת לִשְׁמִי כִּי דָּמִים רַבִּים שָׁפַכְתָּ אַרְצָה לְפָנָי׃
    \versenum{9}
    הִנֵּה־בֵן נוֹלָד לָךְ הוּא יִהְיֶה אִישׁ מְנוּחָה וַהֲנִחוֹתִי לוֹ מִכָּל־אוֹיְבָיו מִסָּבִיב כִּי שְׁלֹמֹה יִהְיֶה שְׁמוֹ וְשָׁלוֹם וָשֶׁקֶט אֶתֵּן עַל־יִשְׂרָאֵל בְּיָמָיו׃
\end{hebrewtext}
\begin{translation}
    \versenum{1 Chr 22:7}
    And David said to Solomon, ``My son, my heart desired to build a temple for the name of \yahweh, my God
    \versenum{8}
    but the word of \yahweh came to me saying, `You have spilled much blood and fought in great battles. You shall not build a temple for my name because you have spilled so much blood on the earth before me.
    \versenum{9}
    Rather, a son will be born to you. He will be a man of rest. And I will give him rest from all his enemies who surround him. Thus, Solomon will be his name and I will give peace and quiet to Israel during his days.'''
\end{translation}

\subsection{Historicizing the \chronicler's Memory of David}
    % What informs this portrait?
    % Biblical interpretation? perhaps?
    % Biblical Memory? Yes.
        % Talk about Psalms
        % Where does this tradition come from?