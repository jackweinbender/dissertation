% !TEX root = dissertation.tex

%%%%%%%%%%%%%%%%%%%%%%%%%%%%%%%
%% DAVID AS A SITE OF MEMORY %%
%%%%%%%%%%%%%%%%%%%%%%%%%%%%%%%
\section{King David as a Site of Memory}

% David was important before Chronicles in the Bible
It is important to note that although the book of Chronicles is a work of cultural memory, it is unquestionably the case that the figure David was a prominent site of memory for ancient Israel long before the book of Chronicles was written. Chronicles, more so than Samuel--Kings, is characterized in terms of ``memory'' because it is clear that the Chronicler%
    \footnote{My use of the term ``Chronicler'' is meant only to reference the author(s) of the book of Chronicles. Although the term is sometimes associated with a particular theory about the composition of Chronicles, Ezra and Nehemiah, I am not using it as such.}
used Samuel--Kings as a primary source and the differences between the sources and the end-product are demonstrable. In other words, because we know that Chronicles is secondary to Samuel--Kings and we can see where the Chronicler departed from Samuel--Kings, it is easy to characterize those changes as the result of changes in cultural memory. But it is important to remember that even Samuel--Kings is the product of mnemonic construction and the David presented there already functioned as a special site of memory for ancient Israel. In other words, despite the fact that Samuel--Kings functions as a foundational source \emph{for Chronicles}, it should not be treated as if it was the origin of all Davidic traditions.\autocite{frohlich_frohlich2019}

% David was important before Chronicles in the Ancient World 
Even setting aside the biblical material (e.g., Samuel--Kings, Psalms, et al.), it is demonstrably the case that the Davidic \emph{dynasty}---whatever one might think about David as an historical figure---had symbolic meaning in the ancient world which extended beyond the borders of Israel. For example, we know from the Old Aramaic inscription from Tel Dan that the term \aram{בת דוד} ``house of David'' was used as a dynastic name for the monarchy of the kingdom of Judah in the \bce{late ninth or early eighth centuries}.%
    \footnote{The \emph{editio princeps} were published in two articles: the first find as \cite{biran-naveh_iej1993}, and the subsequent fragments as \cite{biran-naveh_iej1995}.}
Likewise, it has been suggested that the Mesha Stele, too, refers to the ``house of David,'' although this reading is not secure.%
    \footnote{The reading \aram{בת דוד} was proposed by Lemaire, but his reading is not universally accepted. See \cite{lemaire_sel1994} and \cite{lemaire_bar1994}. The Mesha inscription is typically dated to the \bce{mid-ninth century} and thus would be slightly earlier than the reference in the Tel Dan inscription, if Lemaire is correct.}
Although such references have traditionally been used to bolster claims of an historical David, for our purposes it suffices to say that around the turn of the \bce{eight century}, ``David'' existed as a meaningful eponymous symbol and site of memory with respect to the monarchy of Judah. Thus, when we turn to the biblical portrayals of the figure David (which, by most accounts were products of later periods of Israelite history than Tel Dan and Mesha), it is important to keep in mind that those portrayals are participating in established discourses about David. This is all the more important when we consider the book of Chronicles which represents some of the latest strata of memory preserved in the Hebrew Bible. Thus when we discuss the figure of David as a site of memory which the book of Chronicles engages with extensively, I want to emphasize that the processes of constructing the remembered figure of David did not begin with the Chronicler just as it did it end with the Chronicler.\autocite{frohlich_frohlich2019}

Although the particular relationship between the book of Chronicles and the books of Samuel and Kings is a matter of scholarly debate, it is generally agreed that Samuel--Kings forms the basis for much of the Chronicler's depiction of Israel's history.%
    \footnote{The observation was made as early as de Wette in the early nineteenth century in his \cite*{dewette1806}. More recently, see especially the work of McKenzie
        \cite*{mckenzie1985};
        \cite{mckenzie_graham-mckenzie1999};
        \cite[66--71]{knoppers2003}; and 
        \cite[30--42]{klein2006} as well as that of 
        \cite[74--74]{carr2011}. Notable exceptions, however, do exist. See especially the work of 
        \cite{auld1994}; 
        \cite{auld_graham-mckenzie1999} and 
        \cite{person2010}.}
A great deal of work has been done analyzing the particular literary relationship between Samuel--Kings and Chronicles and the textual processes involved---e.g., what version(s) of Samuel--Kings the Chronicler may have used, etc.---but thinking in terms of social memory requires us to consider the relationship between the texts in \emph{social} terms. In other words, not just to ask \emph{what} the received traditions about David said, but to consider the \emph{role} and \emph{status} of those traditions and to consider why they were (or were not) significant within a particular social context.

% Social context matters; what mattered then might not matter now.
Take, for example, the so-called History of David's Rise (HDR) narrative  and the dramatic family disputes that preoccupied the latter years of David's reign and those of his son Solomon (the so-called ``Succession Narrative'' [SN]) which form a core set of narratives for Samuel--Kings, but whose intrigue are essentially absent from the book fo Chronicles.%
    \footnote{Although the compositional and redactional history of the Deuteronomistic History is hotly debated---with wildly divergent scholarly opinions---I will take as my point of departure the centrist view of McCarter, Halpern, and specifically Knapp which view the HDR and SN (collectively, the ``Court Narrative'' [CN] or ``Traditions of David's Rise and Reign'' [TDRR] \emph{per} Knapp) as royal apologia. I follow Knapp in his view that these traditions do not represent ``the residue of a single apologetic composition'' (161), but rather a diverse set of traditions. However, because the sources cannot meaningfully be parsed, I will also follow him in ``[dealing] with the early narrative traditions in their entirety'' (161). See 
        \cite{knapp2015};
        \cite{mccarter_interpretation1981};
        \cite{mccarter_jbl1980};
        \cite{mccarter1980};
        \cite{halpern2001}.}
It is widely held that that the HDR and SN should be understood as forms of ancient royal apologia---an effort by the author(s) to legitimize David's actions which might otherwise have been construed as a usurpation of the divinely elected king, Saul. Knapp, for example observes that ``[i]n some ways, [the Traditions of David's Rise and Reign] is the paradigmatic ancient Near Eastern apology.''%
    \autocite[218]{knapp2015}
He elaborates:

\begin{quote}
    The apologist employs nearly every apologetic motif in his effort to legitimize David, including passivity, transcendent non-retaliation, the unworthy predecessor, military prowess, and the entire triad of establishing legitimacy.%
    \autocite{knapp2015}
\end{quote}

Clearly the apologist sought to make a forceful and potent argument in favor of David's legitimacy. The apologist operated within his social context---using literary devices and forms which were meaningful in his society---and engaged in discourses about David's legitimacy in an attempt to define David's rise and reign in a particular (positive) way. Of course, we know that for ancient Israel David \emph{did} become known as the legitimate king of Israel and Judah \emph{par excellence}---a figure against whom subsequent kings would be measured. In this way, the construction of the apologist's David was ultimately successful.%
    \footnote{It is worth pointing out that such an apologia likely arose in response to accusations of usurpation. Thus, we can imagine that the HDR is representative of the ``last word'' on the matter, which was an attempt to suppress alternative voices in the matter which questioned the legitimacy of David's rule, the means by which he gained the throne, and the manner of his succession. These discourses were not entirely suppressed from the Hebrew Bible, as evidenced by the figure Shimei and his condemnation of David as a usurper, ``Come out, come out, Oh man of blood!, Oh worthless man! Yahweh has repaid you all the blood of the house of Saul, in whose place you reign'' (2~Sam 16:7b).}

For all the potency of these stories, one may wonder why they did not make it into the Chronicler's history. That is to say, why remove such persuasive, and effective material? The answer, I think, is quite simple: the Chronicler was operating within a social milieu which not only accepted the legitimacy of David and his heirs, but celebrated them as foundational figures. In other words in the symbolic world of the Chronicler, David was significant \emph{because} he was king and---his legitimacy was assumed and celebrated. The discourses that HDR participated in had long been resolved and the Davidic dynasty was thoroughly legitimate in the mind of the Chronicler. As such, it was sufficient for the Chronicler to simply recount the death of Saul---which David had no part in---and the subsequent anointing of David. No mess; a thoroughly unremarkable transfer of power. Similarly, the Chronicler makes no mention of the difficult power struggles that occurred near the end of David's life between him and his sons. Instead, it sufficed for the Chronicler to state: 
    \begin{hebrewtext}
        \versenum{1 Chr 23:1}
        וְדָוִיד זָקֵן וְשָׂבַע יָמִים וַיַּמְלֵךְ אֶת־שְׁלֹמֹה בְנוֹ עַל־יִשְׂרָאֵל׃
    \end{hebrewtext}
    \begin{translation}
        When David was old and full of days, he made Solomon, his son, king over Israel
    \end{translation}
\noindent
The struggle between Solomon and Adonijah following David's death is likewise omitted. Instead, opening verse of 2~Chronicles reads simply:
    \begin{hebrewtext}
        \versenum{2 Chr 1:1}
        וַיִּתְחַזֵּק שְׁלֹמֹה בֶן־דָּוִיד עַל־מַלְכוּתוֹ וַיהוָה אֱלֹהָיו עִמּוֹ וַיְגַדְּלֵהוּ לְמָעְלָה׃
    \end{hebrewtext}
    \begin{translation}
        Solomon, the son of David, established himself in his kingdom, and Yahweh his God was with him and made him exceedingly great.   
    \end{translation}
\noindent
It seems, therefore, that the DH was so successful in its apologetic that it precluded the need for continued apologia. The Chronicler had no need to ``legitimize'' the \emph{fact of} the Davidic dynasty, but instead would focus his attention on defining the \emph{significance of} that dynasty for his own readers in a dramatically different social setting. 

Thus the process of ``remembering'' David in Chronicles can be viewed from two different angles which map onto the dual valences of the term ``remember'': to ``recall'' and to ``commemorate.'' On the one hand, the Chronicler ``recalls'' stories about David which are adapted to the frameworks of the Chronicler's social situation. The Chronicler is a product of his time and society and as such inherited sets of traditions about David, the past, and the world more broadly, which color how he understands the history of Israel and David in particular. On the other hand, the composition of the book of Chronicles is itself an act of commemoration which (as we've noted) is a conscious, constructive process. It represents the process of memory encoding and the construction of cultural memory from which future rememberers would draw. As a work literature, it also bears the idiosyncrasies of its author(s), however constrained by their social milieu they may have been. In fact, determining which of these processes best accounts for any particular ``innovation'' of Chronicles is quite difficult. Was the Chronicler consciously ``reshaping'' the memory of David? Or was the Chronicler more passively reproducing a composite picture of David that he inherited from his culture? Traditional approaches to the book of Chronicles have tended to attribute a great deal of agency to the Chronicler as an innovator of tradition. But thinking in terms of cultural memory pushes us to consider a fuller picture of how cultural memory is created and calls into question whether every theological or ideological augmentation of the Chronicler should be attributed to his novel understanding of the Israelite past. Such an approach takes into account that textual ``sources'' are not merely copied and ``altered,'' but are read, internalized, believed, understood, and reasoned about, which is to say, \emph{remembered}.

\subsection{The David of Chronicles}
How then was David remembered in Chronicles? This question carries with it the assumption that the author of Chronicles was not simply copying-and-changing Samuel--Kings (or other traditions), but rather was a product of a \emph{remembering community} and participated in memory discourses at various sites within the cultural memory of \secondtemple Judaism. Answering this question requires that we not only consider what sources the Chronicler may have used and how he altered those sources, but also to consider the social frameworks which shaped how those sources were received by the Chronicler and how they affected how the Chronicler presented (or commemorated) his work.


[[ TRANSITION ]]
\subsubsection{David the Rightful King}
    % Rightful King
    % Temple-builder
    % David the musician
    % Pious prayer-maker
    % "After God's own Heart"

\subsection{Historicizing the Chronicler's Memory of David}
    % What informs this portrait?
    % Biblical interpretation? perhaps?
    % Biblical Memory? Yes.
        % Talk about Psalms
        % Where does this tradition come from?