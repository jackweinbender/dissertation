% !TEX root = dissertation.tex

%%%%%%%%%%%%%%%%%%%%%%%%%%%%%%%
%% DAVID AS A SITE OF MEMORY %%
%%%%%%%%%%%%%%%%%%%%%%%%%%%%%%%
\section{King David as a Site of Memory}

% David was important before Chronicles in the Bible
It is important to note that although the book of \chronicles is a work of cultural memory, it is unquestionably the case that the figure David was a prominent site of memory for ancient Israel long before the book of \chronicles was written. \chronicles, more so than Samuel--Kings, is characterized in terms of ``memory'' because it is clear that the \chronicler%
    \footnote{My use of the term ``Chronicler'' is meant only to reference the author(s) of the book of \chronicles. Although the term is sometimes associated with a particular theory about the composition of \chronicles, Ezra and Nehemiah, I am not using it as such.}
used Samuel--Kings as a primary source and the differences between the sources and the end-product are demonstrable. In other words, because we know that \chronicles is secondary to Samuel--Kings and we can see where the \chronicler departed from Samuel--Kings, it is easy to characterize those changes as the result of changes in cultural memory. But it is important to remember that even Samuel--Kings is the product of mnemonic construction and the David presented there already functioned as a special site of memory for ancient Israel. In other words, despite the fact that Samuel--Kings functions as a foundational source \emph{for \chronicles}, it should not be treated as if it was the origin of all Davidic traditions.\autocite{frohlich_frohlich2019}

% David was important before Chronicles in the Ancient World 
Even setting aside the biblical material (e.g., Samuel--Kings, Psalms, et al.), it is demonstrably the case that the Davidic \emph{dynasty}---whatever one might think about David as an historical figure---had symbolic meaning in the ancient world which extended beyond the borders of Israel. For example, we know from the Old Aramaic inscription from Tel Dan that the term \aram{בת דוד} ``house of David'' was used as a dynastic name for the monarchy of the kingdom of Judah in the \bce{late ninth or early eighth centuries}.%
    \footnote{The \emph{editio princeps} were published in two articles: the first find as \cite{biran-naveh_iej1993}, and the subsequent fragments as \cite{biran-naveh_iej1995}.}
Likewise, it has been suggested that the Mesha Stele, too, refers to the ``house of David,'' although this reading is not secure.%
    \footnote{The reading \aram{בת דוד} was proposed by Lemaire, but his reading is not universally accepted. See \cite{lemaire_sel1994} and \cite{lemaire_bar1994}. The Mesha inscription is typically dated to the \bce{mid-ninth century} and thus would be slightly earlier than the reference in the Tel Dan inscription, if Lemaire is correct.}
Although such references have traditionally been used to bolster claims of an historical David, for our purposes it suffices to say that around the turn of the \bce{eight century}, ``David'' existed as a meaningful eponymous symbol and site of memory with respect to the monarchy of Judah. Thus, when we turn to the biblical portrayals of the figure David (which, by most accounts were products of later periods of Israelite history than Tel Dan and Mesha), it is important to keep in mind that those portrayals are participating in established discourses about David. This is all the more important when we consider the book of \chronicles which represents some of the latest strata of memory preserved in the Hebrew Bible. Thus when we discuss the figure of David as a site of memory which the book of \chronicles engages with extensively, I want to emphasize that the processes of constructing the remembered figure of David did not begin with the \chronicler just as it did it end with the \chronicler.\autocite{frohlich_frohlich2019}

Although the particular relationship between the book of \chronicles and the books of Samuel and Kings is a matter of scholarly debate, it is generally agreed that Samuel--Kings forms the basis for much of the \chronicler's depiction of Israel's history.%
    \footnote{The observation was made as early as de Wette in the early nineteenth century in his \cite*{dewette1806}. More recently, see especially the work of McKenzie
        \cite*{mckenzie1985};
        \cite{mckenzie_graham-mckenzie1999};
        \cite[66--71]{knoppers2003}; and 
        \cite[30--42]{klein2006} as well as that of 
        \cite[74--74]{carr2011}. Notable exceptions, however, do exist. See especially the work of 
        \cite{auld1994}; 
        \cite{auld_graham-mckenzie1999} and 
        \cite{person2010}.}
A great deal of work has been done analyzing the particular literary relationship between Samuel--Kings and \chronicles and the textual processes involved---e.g., what version(s) of Samuel--Kings the \chronicler may have used, etc.---but thinking in terms of social memory requires us to consider the relationship between the texts in \emph{social} terms. In other words, not just to ask \emph{what} the received traditions about David said, but to consider the \emph{role} and \emph{status} of those traditions and to consider why they were (or were not) significant within a particular social context.

Thus the process of ``remembering'' David in \chronicles can be viewed from two different angles which map onto the dual valences of the term ``remember'': to ``recall'' and to ``commemorate.'' On the one hand, the \chronicler ``recalls'' stories about David which are adapted to the frameworks of the \chronicler's social situation. the \chronicler is a product of his time and society and as such inherited sets of traditions about David, the past, and the world more broadly, which color how he understands the history of Israel and David in particular. On the other hand, the composition of the book of \chronicles is itself an act of commemoration which (as we've noted) is a conscious, constructive process. It represents the process of memory encoding and the construction of cultural memory from which future rememberers would draw. As a work literature, it also bears the idiosyncrasies of its author(s), however constrained by their social milieu they may have been. In fact, determining which of these processes best accounts for any particular ``innovation'' of \chronicles is quite difficult. Was the \chronicler consciously ``reshaping'' the memory of David? Or was the \chronicler more passively reproducing a composite picture of David that he inherited from his culture? Traditional approaches to the book of \chronicles have tended to attribute a great deal of agency to the \chronicler as an innovator of tradition. But thinking in terms of cultural memory pushes us to consider a fuller picture of how cultural memory is created and calls into question whether every theological or ideological augmentation of the \chronicler should be attributed to his novel understanding of the Israelite past. Such an approach takes into account that textual ``sources'' are not merely copied and ``altered,'' but are read, internalized, believed, understood, and reasoned about, which is to say, \emph{remembered}.

\subsection{The David of \chronicles}
How then was David remembered in \chronicles? This question carries with it the assumption that the author of \chronicles was not simply copying-and-changing Samuel--Kings (or other traditions), but rather was a product of a \emph{remembering community} and participated in memory discourses at various sites within the cultural memory of \secondtemple Judaism. Answering this question requires that we not only consider what sources the \chronicler may have used and how he altered those sources, but also to consider the social frameworks which shaped how those sources were received by the \chronicler and how they affected how the \chronicler presented (or commemorated) his work.

Although the David of \chronicles largely resembles that of the DH (he is recognizably the same figure), his function within the narrative of the book of \chronicles is different than that of the DH and that difference can be seen in how the \chronicler portrays and uses him. In both works David is beloved, but he is noticeably less-flawed in the book fo \chronicles. This is not to say that  David is treated as entirely faultless in the book of \chronicles, but I think it is fair to say that the overall portrait presented by the \chronicler is more willing to overlook (and literally to omit) some of David's more egregious acts, and to highlight his role as a model King. This positive portrayal of David in \chronicles is well documented and oft-repeated, so it will suffice for me to focus on two of the most significant features of the \chronicler's portrayal of David, specifically, his portrayal as a divinely elected king, and his role in the establishment of the Israelite cult in Jerusalem.%
    \footnote{See \cite{jarick_frohlich2019}; \cite[347--383]{japhet2009} \cite{knoppers_biblica1995}; \cite[47--48]{japhet1993}; \cite[44--48]{klein2006}; \cite[80--85]{knoppers2003}.}

\subsubsection{David the Divinely Elected King}
First, as I have just alluded to, in the book of \chronicles, David is portrayed as the quintessential, rightful Israelite ruler, elected by \yahweh (1~Chr 10:14) and anointed by the elders of Israel to lead the people (1 Chr 11:1--3). By comparison to the account in Samuel--Kings, the process by which David becomes the ruler of Israel is somewhat less contentious. The apologetic tone of the HDR narratives is nowhere to be found. The rationale for Saul's demise is, like in the DH, predicated on his supposed infidelity to \yahweh, with special reference to his consultation with a medium (although, the story is not told in \chronicles), however, the election of David as Saul's ``successor,'' as described by the \chronicler, does not include Saul aside from a passing reference to his death and infidelities. David himself offers his version of events in 1 Chr 28:4:
\begin{hebrewtext}
    \versenum{1 Chr 28:4}
    וַיִּבְחַר יְהוָה אֱלֹהֵי יִשְׂרָאֵל בִּי מִכֹּל בֵּית־אָבִי לִהְיוֹת לְמֶלֶךְ עַל־יִשְׂרָאֵל לְעוֹלָם כִּי בִיהוּדָה בָּחַר לְנָגִיד וּבְבֵית יְהוּדָה בֵּית אָבִי וּבִבְנֵי אָבִי בִּי רָצָה לְהַמְלִיךְ עַל־כָּל־יִשְׂרָאֵל׃ 
\end{hebrewtext}
\begin{translation}
    \versenum{1 Chr 28:4}
    \yahweh, the God of Israel chose me from among my father's whole house to be king over Israel forever. He chose Judah to be a leader and (from) the house of Judah, the house of my father and (from) the house of my father, he took delight in me to make (me) king over all Israel.
\end{translation}
Conspicuously absent from the \chronicler's narrative and David's summary, are the major conflicts with Saul during David's rise to power. In fact, if one did not know better, simply removing all references to Saul in \chronicles would not meaningfully change how David's election is described.%
    \footnote{This fact raises the question of why the \chronicler \emph{did not} simply omit Saul. I suspect that, although not favored Saul was a useful foil narratively and was a well-enough known figure that omitting him entirely simply did not make sense. Saul was, doubtless, a major figure in the traditions of early Israel.}

 Similarly, the tumult within David's court at the end of his life and the succession of Solomon are omitted by the \chronicler, where 1~Kings begins with a feeble, impotent David and his messy succession by Solomon, 1~Chr 23:1 is content simply to report that:
\begin{hebrewtext}
    \versenum{1 Chr 23:1}
    וְדָוִיד זָקֵן וְשָׂבַע יָמִים וַיַּמְלֵךְ אֶת־שְׁלֹמֹה בְנוֹ עַל־יִשְׂרָאֵל׃
\end{hebrewtext}
\begin{translation}
    When David was old and full of days, he made Solomon, his son, king over Israel.
\end{translation}
\noindent
It went \emph{so} well, in fact, that David saw fit to do it a second time, according to 1 Chr 29:22b--23:
\begin{hebrewtext}
    \versenum{1 Chr 29:22b}
    וַיַּמְלִיכוּ שֵׁנִית לִשְׁלֹמֹה בֶן־דָּוִיד וַיִּמְשְׁחוּ לַיהוָה לְנָגִיד וּלְצָדוֹק לְכֹהֵן׃ 
    \versenum{23}
    וַיֵּשֶׁב שְׁלֹמֹה עַל־כִּסֵּא יְהוָה לְמֶלֶךְ תַּחַת־דָּוִיד אָבִיו וַיַּצְלַח וַיִּשְׁמְעוּ אֵלָיו כָּל־יִשְׂרָאֵל׃
\end{hebrewtext}
\begin{translation}
    \versenum{1 Chr 29:22b}
    Then they made Solomon, son of David, king a second time and they anointed him by \yahweh as a prince as well as Zadok as a priest.
    \versenum{23}
    And Solomon sat on the throne of \yahweh as king in place of David, his father. And he prospered and all Israel obeyed him.
\end{translation}
\noindent
These matter-of-fact descriptions contrast sharply with the events depicted in 1 Kings: Adonijah's self-exaltation (1~Kgs 1:5--53), David's deathbed speech to Solomon (1~Kgs 2:1--9), Solomon's subsequent conflict with Adonijah over Abishag (1~Kgs 2:13--25), with Joab (1~Kgs 2:28--35), and with Shimei (1~Kgs 2:36--46); all of which culminates with the ominous pronouncement of 1~Kgs 2:46b:
\begin{hebrewtext}
    וְהַמַּמְלָכָה נָכוֹנָה בְּיַד־שְׁלֹמֹה׃
\end{hebrewtext}
\begin{translation}
    So the kingdom was established in the hand of Solomon.
\end{translation}
The contrast between the violent establishment of the kingdom ``in the hand of Solomon'' and the popular assent of the people to both the reigns of David and Solomon in \chronicles could not be more clear. On the one hand, the accounts of 1 Kings offer narratives which provide \emph{rationale} for the events that take place---everything that David and Solomon do is framed as a sensible response to wrongdoing. The  descriptions of \chronicles, on the other hand, are not at all interested in providing such rationales, but rather \emph{assume} the premise of 1~Kings. Instead, \chronicles offers plain, black-and-white, narratives which---by virtue of their declarative rhetoric---help to reinforce the idea that David and his successors were not only elect by \yahweh, but were ``good'' kings whose reigns were not contested, but were supported by the population at large.

\subsubsection{David the Temple-builder (Almost)}
While the portrayal of David as the unquestioned founder of the Israelite monarchy in \chronicles is accomplished primarily through omitting details of David's faults, the book of \chronicles makes its most significant \emph{positive} contribution to its picture of David by crediting him as the founder of the Jerusalem Temple. David did not build the Temple in Jerusalem, of course, but the picture that the \chronicler paints of how the Jerusalem Temple came about leaves little doubt about whose idea it \emph{really} was, namely, David's. In the mind of the \chronicler, although Solomon may have been the one to \emph{build} the Temple, David wrote, directed, funded, and produced the project.

In both 2 Sam 7 and 1 Chr 17, David expresses a desire to build a temple for \yahweh and in both cases is rebuffed by \yahweh through the prophet Nathan. Instead, Nathan tells David that \yahweh would establish David's line through his son, Solomon. Although neither account gives a reason for \yahweh's preference toward Solomon, later in \chronicles, David states that the reason \yahweh passed over him was that David was a man of war, while Solomon would be a man of peace:

\begin{hebrewtext}
    \versenum{1 Chr 22:7}
    וַיֹּאמֶר דָּוִיד לִשְׁלֹמֹה בְּנוֹ [בְּנִי] אֲנִי הָיָה עִם־לְבָבִי לִבְנוֹת בַּיִת לְשֵׁם יְהוָה אֱלֹהָי׃ 
    \versenum{8}
    וַיְהִי עָלַי דְּבַר־יְהוָה לֵאמֹר דָּם לָרֹב שָׁפַכְתָּ וּמִלְחָמוֹת גְּדֹלוֹת עָשִׂיתָ לֹא־תִבְנֶה בַיִת לִשְׁמִי כִּי דָּמִים רַבִּים שָׁפַכְתָּ אַרְצָה לְפָנָי׃
    \versenum{9}
    הִנֵּה־בֵן נוֹלָד לָךְ הוּא יִהְיֶה אִישׁ מְנוּחָה וַהֲנִחוֹתִי לוֹ מִכָּל־אוֹיְבָיו מִסָּבִיב כִּי שְׁלֹמֹה יִהְיֶה שְׁמוֹ וְשָׁלוֹם וָשֶׁקֶט אֶתֵּן עַל־יִשְׂרָאֵל בְּיָמָיו׃
\end{hebrewtext}
\begin{translation}
    \versenum{1 Chr 22:7}
    And David said to Solomon, ``My son, my heart desired to build a temple for the name of \yahweh, my God
    \versenum{8}
    but the word of \yahweh came to me saying, `You have spilled much blood and fought in great battles. You shall not build a temple for my name because you have spilled so much blood on the earth before me.
    \versenum{9}
    Rather, a son will be born to you. He will be a man of rest. And I will give him rest from all his enemies who surround him. Thus, Solomon will be his name and I will give peace and quiet to Israel during his days.'''
\end{translation}
\noindent
Here, the point is made particularly explicit through the word-play of ``Solomon'' (Heb. \hebrew{שְׁלֹמֹה}) and ``peace'' (Heb. \hebrew{שָׁלוֹם}) in v.~9.%
    \footnote{The same logic is echoed in 1~Chr 28:3: \hebrew{וְהָאֱלֹהִים אָמַר לִי לֹא־תִבְנֶה בַיִת לִשְׁמִי כִּי אִישׁ מִלְחָמוֹת אַתָּה וְדָמִים שָׁפָכְתָּ׃} ``But God said to me, `You shall not built a house for my name because you are a man of war and had spilled blood.'''}

Despite David's assertion that Solomon be the one who builds the temple, the \chronicler credits David with making all the preparations and providing the bulk of the necessary building supplies for its construction. 
While Solomon would provide the labor, not only was the \emph{idea} of building the Temple David's, but he financed the operation \hebrew{בְעָנְיִי} ``with great pains'' (lit. ``in my oppression''; 1~Chr 22:14). This may be contrasted with Samuel--Kings which does not contain any of this material. 

\subsection{Historicizing the \chronicler's Memory of David}

Historicizing the \chronicler's memory of David's roles as King and cult-founder asks us to account for the similarities and differences between the portrayal of David in Samuel--Kings and \chronicles based on an historical understanding of the social frameworks from which each text emerged. In other words it asks us to utilize what we know historically about the societies which produced these texts to help to explain the similarities and differences between them using the language and theoretical models of social and cultural memory theory.

As I alluded to above, the portrayal of David as the unquestionably elect ruler of Israel and his succession by Solomon in \chronicles is a conspicuously tidy treatment of the very messy account of the so-called History of David's Rise (HDR) narrative and the dramatic family disputes that preoccupied the latter years of David's reign and those of his son Solomon (the so-called ``Succession Narrative'' [SN]). While these stories form a core set of narratives for Samuel--Kings, they are almost completely absent from the book fo \chronicles.%
    \footnote{Although the compositional and redactional history of the Deuteronomistic History is hotly debated---with wildly divergent scholarly opinions---I will take as my point of departure the centrist view of McCarter, Halpern, and specifically Knapp which view the HDR and SN (collectively, the ``Court Narrative'' [CN] or ``Traditions of David's Rise and Reign'' [TDRR] \emph{per} Knapp) as royal apologia. I follow Knapp in his view that these traditions do not represent ``the residue of a single apologetic composition'' (161), but rather a diverse set of traditions. However, because the sources cannot meaningfully be parsed, I will also follow him in ``[dealing] with the early narrative traditions in their entirety'' (161). See 
        \cite{knapp2015};
        \cite{mccarter_interpretation1981};
        \cite{mccarter_jbl1980};
        \cite{mccarter1980};
        \cite{halpern2001}.}

It is widely held that that the HDR and SN should be understood as forms of ancient royal apologia---an effort by the author(s) to legitimize David's actions which might otherwise have been construed as a usurpation of the divinely elected king, Saul. Andrew Knapp, for example, observes that ``[i]n some ways, [the Traditions of David's Rise and Reign] is the paradigmatic ancient Near Eastern apology.''%
    \autocite[218]{knapp2015}
He elaborates:

\begin{quote}
    The apologist employs nearly every apologetic motif in his effort to legitimize David, including passivity, transcendent non-retaliation, the unworthy predecessor, military prowess, and the entire triad of establishing legitimacy.%
    \autocite{knapp2015}
\end{quote}

Clearly the apologist sought to make a forceful and potent argument in favor of David's legitimacy. The apologist operated within his social context---using literary devices and forms which were meaningful in his society---and engaged in discourses about David's legitimacy in an attempt to define David's rise and reign in a particular (positive) way. As such, it has been argued that this apologetic form suggests that the HDR narratives functioned as a \emph{contemporary} form of apologia, implying that these narratives originated at-or-around the time of the presumed historical figure.%
    \footnote{See especially \cite{mccarter_interpretation1981}; \cite{mccarter_jbl1980}; and to a lesser degree \cite[75--76]{halpern2001}. Some clarification is in order here. McCarter et al. are generally talking about where these stories \emph{originated}. They are engaging primarily with minimalist scholars who discount reality of the historical figure of David.}
By this reasoning, such an apologia would have arisen in response to accusations of usurpation. Thus, we would imagine that the HDR was representative of the ``last word'' on the matter or an attempt to suppress alternative voices that questioned the legitimacy of David's rule, the means by which he gained the throne, and the manner of his succession. These discourses were not entirely suppressed from the Hebrew Bible, as evidenced by the figure Shimei and his condemnation of David as a usurper in 2~Sam 16:7b-8:
\begin{hebrewtext}
    \versenum{2 Sam 16:7b}
    וְכֹה־אָמַר שִׁמְעִי בְּקַלְלוֹ צֵא צֵא אִישׁ הַדָּמִים וְאִישׁ הַבְּלִיָּעַל׃ 
    \versenum{8}
    הֵשִׁיב עָלֶיךָ יְהוָה כֹּל דְּמֵי בֵית־שָׁאוּל אֲשֶׁר מָלַכְתָּ תַּחְתָּו [תַּחְתָּיו] וַיִּתֵּן יְהוָה אֶת־הַמְּלוּכָה בְּיַד אַבְשָׁלוֹם בְּנֶךָ וְהִנְּךָ בְּרָעָתֶךָ כִּי אִישׁ דָּמִים אָתָּה׃
\end{hebrewtext}
\begin{translation}
    \versenum{2 Sam 16:7b}
    Thus Shimei spoke cursing him, ``Go out! Go out! Oh man of blood; Oh worthless man! \yahweh has repaid you all the blood of the house of Saul, in whose place you reign. May \yahweh give your kingdom into the hand of Absalom, your son. Look at your evil! Because you are a man of blood.''
\end{translation}
\noindent
Although I am not entirely convinced by this line of reasoning, from the perspective of social memory it is safe to say that at the time of the narrative's composition, the question of whether David should be remembered as a the leader of a victorious \emph{coup d'état} over Saul, or a reluctant leader divinely chosen by \yahweh was a matter of debate. The complex redactional history of the Deuteronomistic History makes saying anything more specific than this difficult and I am open to the possibility that there could have been other social contexts in which such apologia would be potent, either as the original context of their composition or as a new context for an old set of stories.%
    \footnote{For example, Diana Edelman has suggested that a Saulide--Davidic rivalry could have resurfaced during the early Persian period. See \cite{edelman_dearman-graham2002}. Or perhaps the Saul/David struggle could hint at a Benjaminite/Judahite conflict even after the the fall of Israel. This is all idle speculation, of course, but I want to allow for the fact that \emph{other} social situations could make these apologetic discourses potent.}
Whatever the case, we know that for ancient Israel David \emph{did} become known as the legitimate king of Israel and Judah \emph{par excellence} and a figure against whom subsequent kings would be measured. In this way, the construction of the apologist's David was ultimately successful.%

For all the potency of these stories, one may wonder why they were not included in the \chronicler's history. That is to say, why omit such persuasive, and effective material? The answer, I think, is quite simple: the \chronicler was operating within a social milieu which not only accepted the legitimacy of David and his heirs, but celebrated them as foundational figures. In other words in the symbolic world of the \chronicler, David was significant \emph{because} he was king and---his legitimacy was assumed and celebrated. In other words, for the \chronicler, remembering David into his social context found particular parts of the received tradition more useful for the set of discourses that he was participating in. The discourses that HDR participated in had been resolved and the Davidic dynasty was thoroughly legitimate in the mind of the \chronicler. As such, it was sufficient for the \chronicler to simply recount the death of Saul---which David had no part in---and the subsequent anointing of David. From the perspective of the \chronicler it was a thoroughly unremarkable transfer of power. Similarly, the \chronicler makes no mention of the difficult power struggles that occurred near the end of David's life between him and his sons. Instead, it sufficed for the \chronicler to state: 
    \begin{hebrewtext}
        \versenum{1 Chr 23:1}
        וְדָוִיד זָקֵן וְשָׂבַע יָמִים וַיַּמְלֵךְ אֶת־שְׁלֹמֹה בְנוֹ עַל־יִשְׂרָאֵל׃
    \end{hebrewtext}
    \begin{translation}
        When David was old and full of days, he made Solomon, his son, king over Israel
    \end{translation}
\noindent
The struggle between Solomon and Adonijah following David's death is likewise omitted. Instead, opening verse of 2~\chronicles reads simply:
    \begin{hebrewtext}
        \versenum{2 Chr 1:1}
        וַיִּתְחַזֵּק שְׁלֹמֹה בֶן־דָּוִיד עַל־מַלְכוּתוֹ וַיהוָה אֱלֹהָיו עִמּוֹ וַיְגַדְּלֵהוּ לְמָעְלָה׃
    \end{hebrewtext}
    \begin{translation}
        Solomon, the son of David, established himself in his kingdom, and Yahweh his God was with him and made him exceedingly great.   
    \end{translation}
\noindent
It seems, therefore, that the DH was so successful in its apologetic that the memory constructed by its rhetoric precluded the need for continued apologia in the work of the \chronicler. The \chronicler had no need to ``legitimize'' the \emph{fact of} the Davidic dynasty, but instead would focus his attention on defining the \emph{significance of} that dynasty for his own readers in a dramatically different social setting. 

Instead of these questions of legitimacy, what seems more important to the \chronicler are questions revolving around David's role in cultic activity before Solomon's temple. 

In particular we can see how the rationale for explaining why David \emph{did not} build a temple for \yahweh may have developed through the influence of other traditions. In 1~Kings 5:15 Solomon explains that the reason his father, David, was unable to build the Temple was due to the persistence of David's many enemies:

\begin{hebrewtext}
    אַתָּה יָדַעְתָּ אֶת־דָּוִד אָבִי כִּי לֹא יָכֹל לִבְנוֹת בַּיִת לְשֵׁם יְהוָה אֱלֹהָיו מִפְּנֵי הַמִּלְחָמָה אֲשֶׁר סְבָבֻהוּ עַד תֵּת־יְהוָה אֹתָם תַּחַת כַּפּוֹת רַגְלָו [רַגְלָי׃] 
\end{hebrewtext}
\begin{translation}
    You knew David, my father; that he was not able to build a house for the name of \yahweh, his God, on account of the war which surrounded him until \yahweh put them beneath the soles of his feet.
\end{translation}
\noindent
It is important to note that the rationale here is not that David divinely prohibited from building the temple, but that the presence of his enemies \emph{prevented} him from building the temple. In fact, this statement is inconsistent with the description of David in 2~Sam~7:1, which explicitly states that it was after \yahweh had given David rest from his enemies that David first considered building a temple for the deity:
    \begin{hebrewtext}
        \versenum{2 Sam 7:1}
        וַיְהִי כִּי־יָשַׁב הַמֶּלֶךְ בְּבֵיתוֹ וַיהוָה הֵנִיחַ־לוֹ מִסָּבִיב מִכָּל־אֹיְבָיו׃
        \versenum{2}
        וַיֹּאמֶר הַמֶּלֶךְ אֶל־נָתָן הַנָּבִיא רְאֵה נָא אָנֹכִי יוֹשֵׁב בְּבֵית אֲרָזִים וַאֲרוֹן הָאֱלֹהִים יֹשֵׁב בְּתוֹךְ הַיְרִיעָה׃
    \end{hebrewtext}
    \begin{translation}
        \versenum{2 Sam 7:1}
        It came about that when the king was sitting in his house---\yahweh having given him rest all around from all his enemies---
        \versenum{2}
        the king said to Nathan the prophet, ``Look! I am sitting in a house of cedar but the ark of God is sitting in the midst of curtains!''
    \end{translation}
\noindent
Rather conspicuously, however, the parallel account in 1~Chr~17:1 omits that David had been given rest:
\begin{hebrewtext}
    \versenum{1 Chr 17:1}
    וַיְהִי כַּאֲשֶׁר יָשַׁב דָּוִיד בְּבֵיתוֹ וַיֹּאמֶר דָּוִיד אֶל־נָתָן הַנָּבִיא הִנֵּה אָנֹכִי יוֹשֵׁב בְּבֵית הָאֲרָזִים וַאֲרוֹן בְּרִית־יְהוָה תַּחַת יְרִיעוֹת׃
\end{hebrewtext}
\begin{translation}
    \versenum{1 Chr 17:1}
    Now, when David was sitting in his house, David spoke to Nathan the prophet, ``I am sitting in a house of cedar but the ark of the covenant of \yahweh is under curtains!'' 
\end{translation}
One obvious way to explain this difference is to attribute the omission to the \chronicler's desire for narrative consistency and to assert that it was not until the reign of Solomon that ``peace and quiet'' would be achieved in Israel. Indeed, this ultimately is the position of the \chronicler, which he makes explicit in 1 Chr 22:7 (above). 

While Japhet and others finds this omission consistent with the \chronicler's broader methodology and ideological project,%
    \autocite[328]{japhet1993}.
there is some debate about whether the reference to \yahweh giving rest to David was original to 2 Samuel or whether it was a late Deuteronomistic addition.%
    \footnote{McCarter states confidently that this is an addition to the MT, despite the fact that all known witnesses include the phrase. See \cite[191]{mccarter1984}.}
As a result, there is also some question whether it was a part of the \vorlage of the \chronicler at all and therefore whether the minus in 1 Chr 17 should be attributed to the \chronicler. McKenzie in particular goes so far as to say that this was a late Deuteronomistic addition to 2 Samuel and argues that the phrase simply was not a part of the \vorlage from which the \chronicler drew.%
    \footnote{\cite[63]{mckenzie1985}. Knoppers does not make a strong recommendation either way, but makes it a point to include haplography as a possible explanation of the omission. See \cite[666]{knoppers2007}.}
Even allowing for the possibility that this aside was not a part of the \chronicler's \vorlage, however, there remain at least two related questions to be answered: 1) What prompted the supposed insertion into 2 Sam 7, and 2) how did David's \emph{preoccupation} with his enemies turn into a divine \emph{disqualifier}.

To answer the first question, numerous scholars have observed the clear connection between this reference to finding ``rest'' with Deuteronomy 12:10--11, which establishes a timeline for the construction of a permanent cultic site in \yahweh's chosen locale:

\begin{hebrewtext}
    \versenum{Deut 12:10}
    וַעֲבַרְתֶּם אֶת־הַיַּרְדֵּן וִישַׁבְתֶּם בָּאָרֶץ אֲשֶׁר־יְהוָה אֱלֹהֵיכֶם מַנְחִיל אֶתְכֶם וְהֵנִיחַ לָכֶם מִכָּל־אֹיְבֵיכֶם מִסָּבִיב וִישַׁבְתֶּם־בֶּטַח׃ 
    \versenum{11}
    וְהָיָה הַמָּקוֹם אֲשֶׁר־יִבְחַר יְהוָה אֱלֹהֵיכֶם בּוֹ לְשַׁכֵּן שְׁמוֹ שָׁם שָׁמָּה תָבִיאוּ אֵת כָּל־אֲשֶׁר אָנֹכִי מְצַוֶּה אֶתְכֶם עוֹלֹתֵיכֶם וְזִבְחֵיכֶם מַעְשְׂרֹתֵיכֶם וּתְרֻמַת יֶדְכֶם וְכֹל מִבְחַר נִדְרֵיכֶם אֲשֶׁר תִּדְּרוּ לַיהוָה׃
\end{hebrewtext}
\begin{translation}
    \versenum{Deut 12:10}
    And you will cross over the Jordan and settle in the land that \yahweh your God is giving to you. And he will give you rest from all your enemies around (you) and you will live safely.
    \versenum{11}
    Then the place at which \yahweh your God will establish his name will be (the place) that you will bring everything that I command you---your burnt offerings and your sacrifices, tithes, the contributions of your hand, and all your finest votive offerings that you might vow to \yahweh.
\end{translation}
\noindent
According to this passage, it is only after the Israelites completely conquer the land and find ``rest'' will the central cultic site be established.  As a matter of inner-biblical interpretation, it makes sense that some late redactor of 2 Sam 7 might note that David sought to build the temple only after ``rest'' had been established and simply did not take into account the rationale given by Solomon in 1~Kings 5:15.

Such editorial or redactional changes may be subsumed under the rubric of memory insofar as such changes come about in order to better align one mnemonic node within a the broader framework of memory. In other words, we can account for this textual change by positing that the redactor's understanding of \emph{when} the temple could be built was informed by the tradition of Deut 12:10--11 (or one like it). Thus, when the redactor read about David's attempt to build a temple, he interpreted David's actions based on this other knowledge. Although it would be easy enough to circumvent the issue by noting that David is \emph{rebuffed} by \yahweh and that the temple is ultimately built by Solomon, doing so leaves David somewhat vulnerable to critique. If the redactor thought David to have access to the ``Torah'' one must suppose that David either did not know Deut 12, did not care about Deut 12, or that (as the redactor concluded) ``rest'' \emph{had in fact} come about in Israel.%
    \footnote{As a way to rationalize the apparent contradiction with the fact that David engages in battle in the very next chapter, one might speculate, for example, that David only ``thought'' that he had vanquished all his enemies, or that there was ``rest,'' but that it was short-lived.}


TODO: Something about how all of this is historicizing. I need to be explicit about this.
% Deut 12:8 says that yahweh would give rest before building the temple