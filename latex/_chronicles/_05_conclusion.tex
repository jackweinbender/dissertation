% !TEX root = dissertation.tex

\section{Conclusion}

Although traditional approached to the book of \chronicles have tended to focus on the ideological and theological agenda of the the \chronicler and the changes that he made over-and-against his \vorlage as innovations, in this chapter I have shown how a memory approach both problematizes this characterization and offers a more robust understanding of the kinds of processes which affect how societies relate to their remembered past and how those systems of remembrance can change over time. I have argued that the concept of ``sites'' of memory offer a robust way to think about the way that specific ideas become locations for engaging with social discourses and that these sites operate within complex systems of symbolic meaning. Although such sites are not bound to reflect ``history'' in the modern sense, but the processes by which these sites of memory interact and evolve can be observed and reasoned about \emph{historically}. The kinds of changes that social and cultural memory exhibits diachronically can be tied to historical social and cultural changes.

The book of \chronicles exhibits numerous changes over-and-against its putative \vorlage and those changes, I have argued, can be attributed not only to the genius of the \chronicler, but more fundamentally to the social context in which the \chronicler lived. The individual and idiosyncratic reinterpretations that the \chronicler provides are inextricably linked to the to his social frameworks. Characterizing this reimagining simply as an effort by one individual to ``smooth over'' problematic aspects of Samuel--Kings or as an effort to explain difficult sections of his \vorlage sells-short the social processes that contributed to the much larger reconfiguration of Israel's remembered past that the book of \chronicles represents.