% !TEX root = dissertation.tex

\section{Magnetism and Convergence of Mnemonic Sites}

The questions that the book of \chronicles seeks to answer and the assumptions which it carries are different than that of Samuel--Kings and affect the way that the \chronicler not only read and interpreted his sources, but also the way that he situated various sites of memory with respect to one another. Thus David's role in the construction of the temple is not isolated to the question of why he could not build it, but extends to the way that David, as site of memory, relates to the temple \emph{as a site of memory}.

\subsection{The Temple as a Site of Memory}
The temple in Jerusalem was already an important site of memory for ancient Israel long before the book of \chronicles was written. Already in the book of Deuteronomy the mythology surrounding the divine selection of Jerusalem and the uniquely ordained site of the Solomonic temple had been well-established. This development is easily seen by contrasting the ways that the Covenant Code of \exod 20 in which \yahweh seems to command (or, at the very least not \emph{prohibit}) the Israelites to establish cult sites \hebrew{בְּכָל־הַמָּקוֹם אֲשֶׁר אַזְכִּיר אֶת־שְׁמִי} ``in every place that I commemorate my name'' (\exod 20:24) with that of Deuteronomy, in which \yahweh commands the Israelites to destroy all cult sites within the land and furthermore that: 
\begin{hebrewtext}
    \versenum{\deut 12:5}
    כִּי אִם־אֶל־הַמָּקוֹם אֲשֶׁר־יִבְחַר יְהוָה אֱלֹהֵיכֶם מִכָּל־שִׁבְטֵיכֶם לָשׂוּם אֶת־\\שְׁמוֹ שָׁם לְשִׁכְנוֹ תִדְרְשׁוּ וּבָאתָ שָׁמָּה׃
    \versenum{6}
    וַהֲבֵאתֶם שָׁמָּה עֹלֹתֵיכֶם וְזִבְחֵיכֶם וְאֵת מַעְשְׂרֹתֵיכֶם וְאֵת תְּרוּמַת יֶדְכֶם וְנִדְרֵיכֶם וְנִדְבֹתֵיכֶם וּבְכֹרֹת בְּקַרְכֶם וְצֹאנְכֶם׃
\end{hebrewtext}
\begin{translation}
    \versenum{\deut 12:5}
    But you shall seek the place that \yahweh your God will choose from among all your tribes as his dwelling to put his name there. You shall go there
    \versenum{6}
    and you will bring your burnt offerings there as well as your sacrifices, your tithes and the offerings of your hands, your votive gifts, your freewill offerings, and the firstborn of your cattle and flocks. 
\end{translation}
\noindent
It is hard not to speculate that the textual variants in \exod 20:10 are due to the implication that \yahweh could be commemorate his name in multiple places, compared to its counterpart in \deut 12:5. This discomfort is illustrated in  \sampent's omission of \hebrew{כל} with the result that \hebrew{מָקוֹם} is conceptually singular (in \emph{the} place), while \lxx, Syriac, and the Targums all support the reading ``in every place.''%
    \footnote{In the case of the \sampent, the editor may have had in mind ``Samaria'' rather than ``Jerusalem,'' but the impulse is the same.}
Such a reading implies that the author had in mind an \emph{itinerant} cult site. The Niqqudim make it a point to separate the ideas, emphasizing that the clause \hebrew{בְּכָל־הַמָּקוֹם} ``in every place'' modifies the following clause \hebrew{אָבוֹא אֵלֶיךָ וּבֵרַכְתִּיךָ} ``I will come to you and bless you'' and not completing the action of the preceding \hebrew{מִזְבַּח אֲדָמָה תַּעֲשֶׂה־לִּי וְזָבַחְתָּ עָלָיו} ``you will make an earthen altar for me and make sacrifices upon it.'' Indeed, the first person form \hebrew{אַזְכִּיר} favors the former reading. Even so, \exod 20 seems to presuppose that \yahweh could or would cause his name to be commemorated in more than one place. On the other hand, the book of Deuteronomy states clearly that the the Israelite were only to bring their offerings to the \emph{the} place that \yahweh would choose from among the tribes. The historical reality of Israelite shrines and cult sites outside of Jerusalem during the monarchic period such as those from Dan, Arad and others is well documented.%
    \footnote{For a concise overview of the archaeological evidence, see \cite[319--352]{king-stager2001}.}
While these sites were condemned as idolatrous by the deuteronomistic editor(s), there is no evidence to suggest that contemporaries of the \bce{seventh century} (or earlier) saw them as such.

The increased importance of the Jerusalem Temple brought about by the cult centralization efforts of Hezekiah and Josiah after the destruction of the Northern Kingdom similarly consolidated the religious memory of ancient Israel around Jerusalem and temple of Solomon. Insofar as the real religious practices of Israel (putatively) became increasingly focused on the city of Jerusalem and Solomon's temple leading up to its destruction at the beginning of the \bce{sixth century}, ancient Israel's memory about other ``marginal'' religious practices was quite literally demolished through the intentional destruction of \translit{bāmôt} and other sacred sites through the religious reforms of Hezekiah and Josiah. It was the socio-political \emph{reality} of the Jerusalem Temple's significance at the end of the \bce{sixth century}---brought about by intentional religious reforms---which informed the deuteronomistic editor's memory of earlier \yahwistic cult practices and which would form the basis for the \chronicler's perception of religious practice during the early monarchic period. 

Regardless of how centrally significant the Jerusalem temple actually had been during the early monarchic period, or even how successful the practical aspects of Hezekiah and Josiah's reforms had been, between the end of the \bce{seventh century} and the time of the \chronicler, the memory of the temple had continued to accrue significance. In particular the trauma of the temple's destruction at the beginning of the \bce{sixth century} commemorated the dissolution of a particular form of ... TODO:

\subsection{Magnetism}
Although both David and the Temple maybe thought of as discrete sites of memory, it is important to remember that they participate in a \emph{network} of symbolic social meaning. Thus, ``discrete'' here does not mean ``isolated.'' Moreover, not all sites of memory carry the same weight of significance within a particular symbolic system. In other words, not all sites of memory are created equal; David is a much more prominent and potent node within the social memory of ancient Israel than was Shimei, his critic. Though they participate within the same discursive space---even in the Bible---David is a more significant symbol. Likewise the Temple's symbolic significance far outweighs that of the \translit{bāmôt}, despite the fact that---functionally---their social function was similar.

But, what do we mean by ``significance''? One way to think about a symbol's significance within a social space is by considering not the ``size'' of the node (whatever that might mean), but by how ``connected'' the node is within the social network. More highly-connected sites of memory---those which for one reason or another have been connected to many other such sites within the social memory---may be viewed as more ``significant,'' while sites with fewer connections are comparatively less significant with respect to social and cultural memory.%
    \footnote{Of course, when I say that a king is more ``significant'' than, say, a peasant, I am making an assessment of the social impact of the individual on the society broadly and not making a judgment of the intrinsic value or importance of the individual. Moreover, I am not saying that such significance ought to guide the historian. This is merely meant as a description of this particular social phenomenon.}


    % MAGNETISM OF SPECIFIC SITES: 
        % Give basic examples for:
            % David (Goliath?)
            % Temple (Creation?)
    % MAGNETISM BETWEEN SITES
        % Look at the Araunah/Census story of Chr.
            % Cf. with Sam--Kings
            % David and Temple converge (within certain boundaries) 
            % How might external factors affect how this was "remembered"
                    % Was Araunah's threshing floor known?
                    % What about the imagined role of the king?
                        % David as a cult leader
                        % Why not Solomon?
        % Mt. Moriah is an example of further `magnetism' to another potent site of memory (the Aqedah)
