% !TEX root = dissertation.tex
\section{Reimagining Foundations: David's Census and Ornan's Threshing Floor}

As a final example of the mnemonic processes at work in the book of \chronicles, I will discuss the account of David's Census and the Threshing Floor of Araunah/Ornan found in 1 Chr 21:1--22:1 as a product of cultural memory. When compared to the parallel account in 2 Sam 24:1--24, the story as told in \chronicles is not simply a modified or ``cleaned up'' version of the story, but offers a narrative which functions in a distinct fashion compared to that of 2 Sam 24. This recontextualized narrative not only operates on different sets of ideological and theological presuppositions, but re-situates the narrative within the social memory by making magnetic connections to the major themes and ideas of the \chronicler's society.

The story of David's census in 2 Sam 24:1--24 begins with \yahweh inciting David to take a census of his army because \yahweh was angry with him. He dispatches his general, Joab, to take the census. Joab is successful and, after nine months, reports back to David the results of the census. Upon hearing the results, ``David's heart was stricken because he had numbered the people'' (Heb. \hebrew{וַיַּךְ לֵב־דָּוִד אֹתוֹ אַחֲרֵי־כֵן סָפַר אֶת־הָעָם}). David asks for forgiveness from \yahweh, who gives him three options for punishment. After the punishment of pestilence is executed against several thousand Israelites, \yahweh relents before striking Jerusalem and instructs his angel (Heb. \hebrew{מַלְאַךְ})---who at that moment was at the threshing floor of a man named Araunah (a Jebusite)---to cease destroying the people. Later, the prophet Gad instructs David to build an altar to \yahweh at the site of Araunah's threshing floor. David purchases the threshing floor, offers sacrifices to \yahweh, and ``the plague was averted from upon Israel'' (\hebrew{וַתֵּעָצַר הַמַּגֵּפָה מֵעַל יִשְׂרָאֵל}).

With the exception of a few key details and additions, the version presented in 1 Chr 21:1--22:1 follows the version in 2 Samuel very closely.%
    \footnote{In smaller textual matters one must be careful to remember that the \vorlage of the \chronicler was not identical to the Masoretic text. Indeed the textual plurality of the \lxx, \q{4}{Sam}{a}, and references in Josephus's \ant caution against assuming every difference between the \chronicler's account and that of 2 Samuel (MT) are the result of some change made by the \chronicler. See \cite[761--762]{knoppers2007}.}
The description of the census itself is mostly the same as the MT of 1 Samuel, though differences (such as the specific numbers reported) do occur. The most significant change to the first part of the story, however, is the attribution of David's incitement to take the census to an entity referred to as ``(a) Satan'' (Heb. \hebrew{שָׂטָן}):

\begin{hebrewtext}
    \versenum{1 Chr 21:1}
    וַיַּעֲמֹד שָׂטָן עַל־יִשְׂרָאֵל וַיָּסֶת אֶת־דָּוִיד לִמְנוֹת אֶת־יִשְׂרָאֵל׃
\end{hebrewtext}
\begin{translation}
    \versenum{1 Chr 21:1}
    (A) Satan arose against Israel and incited David to number Israel.
\end{translation}
\noindent
Scholars remain divided over whether \hebrew{שָׂטָן} should be understood as a simple indefinite noun ``an adversary,''%
    \footnote{%
        \cite{stokes_jbl2009};
        \cite[114--117]{japhet2009};
        \cite[370--390]{japhet1993}.}
or whether the absence of the definite article indicates that by the time of the Chronicler, Satan referred to a malevolent spirit which prefigured the more developed, personified ``Satan'' found in the New Testament.%
    \autocite[4--5]{rollston_keith-stuckenbruck2016}
The most common usage of the term \hebrew{שָׂטָן} in the Hebrew Bible refers to human adversaries and accusers (Num 22:22, 32; 1~Sam 29:4, 2~Sam 19:23; 1~Kgs 5:18, 11:14, 23, 25; Ps 38:21, 71:13, 109:4, 6, 20, 29). However, the figure \hebrew{הַשָּׂטָן} (with definite article) in both the prologue to Job (Job 1--2) and Zech 3:1--2 appears as a celestial figure to whom Yahweh speaks directly.
    \footnote{This notion is more clear in Job, where \hebrew{הַשָּׂטָן} is described in the heavenly courts and is described as having supernatural powers over the health and prosperity of those on the Earth. On the other hand, the reference in Zechariah is somewhat ambiguous. Zech 3:1 reads: \hebrew{ וַיַּרְאֵנִי אֶת־יְהוֹשֻׁעַ הַכֹּהֵן הַגָּדוֹל עֹמֵד לִפְנֵי מַלְאַךְ יְהוָה וְהַשָּׂטָן עֹמֵד עַל־יְמִינוֹ לְשִׂטְנוֹ}, ``And he showed me Joshua, the high priest standing before the angel of Yahweh, and \translit{haśśāṭān} was standing on his right (side) to accuse him.'' The antecedent of ``his'' in ``his right(side)'' is unclear. If ``his'' refers to the \hebrew{מַלְאַךְ} Yahweh, then \hebrew{הַשָּׂטָן} likely refers to some kind of spiritual being. However, it is possible that ``his'' refers to Joshua, and that \hebrew{הַשָּׂטָן} should be understood as a human adversary.}
Proponents of reading \hebrew{שָׂטָן} as the personal name of a malevolent spirit argue that the absence of the definite article indicates that the idea of \emph{the} \hebrew{שָׂטָן} of Job and Zechariah had evolved into a fully personified Satan by the time of the Chronicler.%
    \footnote{%
        \cite[216--217]{braun1986};
        \cite[107]{coggins1976}. Rollston also finds this reading compelling, though, not without difficulties. See 
        \cite[4--5]{rollston_keith-stuckenbruck2016}.}
Additionally, while the \lxx hails from a later chronological horizon than Chronicles, it is worth noting that the translator used the indefinite substantive \greek{διάβολος} to translate \hebrew{שָׂטָן}---the same term used in Job and Zechariah (also, Ps 108:6) \emph{with} a definite article---which gives some indication that, in the mind of the translator, these passages likely referred to the same entity.% 
    \footnote{Elsewhere the \lxx renders the nominal forms of \hebrew{שָׂטָן} with the feminine \greek{διαβολή} or, in the case of 1~Kgs 11:14, simply in transliteration as \greek{σαταν}. It should be noted, however, that Esth 7:4 and 8:1 render the Hebrew √ṣrr as the masculine \greek{διάβολος} as well.} 

 Critics of this view, however, have pointed to the fact that in other cases in the Hebrew Bible, generic nouns that are treated as personal names or titles often \emph{do} retain the definite article.%
    \footnote{\cite[114--117]{japhet2009};
        \cite[370--390]{japhet1993}. Japhet, for example, notes that direct references to the Canaanite deity Baʿal are always accompanied by the definite article. In every instance, the name/title \hebrew{בַּעַל} is made grammatically definite whether by adding the definite article, pronominal suffixes, or being in construct with an explicitly definite noun. 
        \cite[115]{japhet2009} citing 
        \cite[§126d]{gkc}.}
In such a case, \hebrew{שָׂטָן} should simply be understood as an indefinite noun, ``an accuser'' and may be understood as a human antagonist of David.%
    \footnote{See 
        \cite{stokes_jbl2009};
        \cite[114--117]{japhet2009}; 
        \cite[370--390]{japhet1993}.} 
For our purposes, it is not essential that we know for certain how \hebrew{שָׂטָן} was intended to be understood by the \chronicler. What \emph{is} important, however, is that the author of \chronicles plainly understood the mechanisms at work differently than the author of 2~Sam 24. Chief among these differences is the fact that the \chronicler shifts the incitement of the census away from \yahweh and onto a third party. One of the more perplexing aspects of the 2 Sam 24 narrative is that \yahweh seems function as an antagonist to David. It is \yahweh's anger which prompts \yahweh to ``incite'' David to take the census, for which he is punished. At least to modern readers, the resulting narrative appears to be one of a sort of divine ``entrapment'' of David that makes \yahweh seem rather ``mercurial.''%
    \autocite[4]{rollston_keith-stuckenbruck2016}
That \yahweh would incite David to sin then punish him for it was understandably confusing for the \chronicler, and equally confusing is why \yahweh would punish Israel for taking the census to begin with. But within the narrative discourse, there is no hint of confusion. All the characters operate as if \yahweh's response is perfectly reasonable and seem to know the proper actions to take to avert the disaster. Joab, for example, seems reticent about David's request to take the census:
\begin{hebrewtext}
    \versenum{2 Sam 24:3}
    וַיֹּאמֶר יוֹאָב אֶל־הַמֶּלֶךְ וְיוֹסֵף יְהוָה אֱלֹהֶיךָ אֶל־הָעָם כָּהֵם וְכָהֵם מֵאָה פְעָמִים וְעֵינֵי אֲדֹנִי־הַמֶּלֶךְ רֹאוֹת וַאדֹנִי הַמֶּלֶךְ לָמָּה חָפֵץ בַּדָּבָר הַזֶּה׃
\end{hebrewtext}
\begin{translation}
    \versenum{2 Sam 24:3}
    And Joab said to the king, ``May \yahweh your God increase the people a hundred times while the eyes of my lord the king may see (them). But why does my lord the king desire this thing?''
\end{translation}
\noindent
He seems to know that taking the census could be risky. 

What these two insertions reveal is that for the \chronicler, the narrative logic of 2 Samuel did not work within his own set of social frameworks. Instead of reading the \chronicler's attribution of incitement to \hebrew{שָׂטָן} as an attempt at ``absolving'' or explaining away \yahweh's actions because they were offensive to the \chronicler, perhaps it is better to think about the \chronicler attempting to fit the story of 2 Sam 24 into a different theological framework and into a different system of narrative logic. From this perspective, it was not that the \chronicler was scandalized by \yahweh's incitement of David but rather a reflection on the fact those actions lacked a sort of ``theological verisimilitude'' (my term) within the worldview of the \chronicler.%
    \footnote{I suspect that most religious laypeople for whom 2 Sam 24 is scripture, too, would find \yahweh's portrayal in this text out-of-character with the way that God is portrayed elsewhere in their Bibles (not least, in \chronicles!).}
Likewise where the narrative in 2 Sam 24 assumes that the reader understands why Joab would question David about taking the census, the \chronicler supplies the details for Joab's reservations:
\begin{hebrewtext}
    \versenum{1 Chr 21:3}
    וַיֹּאמֶר יוֹאָב יוֹסֵף יְהוָה עַל־עַמּוֹ כָּהֵם מֵאָה פְעָמִים הֲלֹא אֲדֹנִי הַמֶּלֶךְ כֻּלָּם לַאדֹנִי לַעֲבָדִים לָמָּה יְבַקֵּשׁ זֹאת אֲדֹנִי לָמָּה יִהְיֶה לְאַשְׁמָה לְיִשְׂרָאֵל׃
\end{hebrewtext}
\begin{translation}
    \versenum{1 Chr 21:3}
    And Joab said, ``May \yahweh increase his people a hundred times. Are they not all, my lord the king, servants of my lord? Why does my lord seek this? Why shall he bring guilt on Israel?''
\end{translation}
\noindent
Here the \chronicler makes explicit what is implicit in 2 Sam 24---that census taking carries a risk. Although a number of scholars have speculated about what the rationale had been for the original author(s) of 2 Sam 24, the \chronicler infers a plausible rationale from his own theological and ideological frameworks.%
    \footnote{See, especially the discussion in \cite[512--514]{mccarter1984}.}
The same basic tendency to clarify the narrative logic of several apparently inconsistent portions of 2 Sam 24 can be seen throughout the narrative.%   
    \footnote{For example, the angel of \yahweh appears suddenly in the Samuel narrative and seems to be carrying out some violence against the land. The \chronicler introduces the angel as explicitly sent by \yahweh. More than likely there is some kind of textual corruption in the Samuel account, so it is not certain that every additional detail that the \chronicler supplies is original to the \chronicler.}

Perhaps the most significant addition to the \chronicler's account, however, comes at the very end of the story with the identification of Araunah/Ornan's threshing floor with the future site of the Jerusalem temple. In both accounts David is instructed by the prophet Gad to build an altar to \yahweh and offer sacrifices at the threshing floor of Araunah/Ornan.%
    \footnote{On the threshing floor as  sacred space and location for temple-construction, see 
        \cite[125--144]{waters2015}.}
The sacrifices that David makes in the 2~Samuel serve as a means to placate \yahweh and avert continued punishment. The \chronicler, clearly, was perplexed by this practice. As Japhet notes, implicit in the \chronicler's desire to explain this practice is the presupposition that the cult of \yahweh was centralized into a single location. Although the temple (the permanent, legitimate site for worship) had yet to be build, the tabernacle provided a singular (if itinerant) site of legitimate worship.%
    \footnote{\cite[389]{japhet1993}; \cite[760--761]{knoppers2007}}
To addresses the problem of how/why David made sacrifices outside the singular cult site, the tabernacle, the \chronicler rationalizes that because the tabernacle was in Gibeon, it was therefore inaccessible to him in a timely fashion:
\begin{hebrewtext}
    \versenum{1 Chr 21:29}
    וּמִשְׁכַּן יְהוָה אֲשֶׁר־עָשָׂה מֹשֶׁה בַמִּדְבָּר וּמִזְבַּח הָעוֹלָה בָּעֵת הַהִיא בַּבָּמָה בְּגִבְעוֹן׃ 
    \versenum{30}
    וְלֹא־יָכֹל דָּוִיד לָלֶכֶת לְפָנָיו לִדְרֹשׁ אֱלֹהִים כִּי נִבְעַת מִפְּנֵי חֶרֶב מַלְאַךְ יְהוָה׃
\end{hebrewtext}
\begin{translation}
    \versenum{1 Chr 21:29}
    Now, the tabernacle of \yahweh which was built by Moses in the wilderness and the altar of burnt offering at that time was at the high place at Gibeon
    \versenum{30}
    and David was not able to go before it to inquire of God because he was terrified of the sword of the angel of \yahweh.
\end{translation}
\noindent
From a memory perspective, the issue is not necessarily that it was \emph{offensive} for David to make sacrifices outside of the central cult site, but a difference set of presuppositions about how proper worship \emph{should} work. The \chronicler does not change the fact that David offered sacrifices outside the tabernacle, but provides a rationale for why it was expedient for David to bend the rules. The \chronicler gives David the benefit of the doubt and explains his actions based on the \chronicler's system of theological rationale.

But the \chronicler goes further than simply rationalizing David's actions and completely recontextualizes this narrative within the framework of Israel's history:
\begin{hebrewtext}
    \versenum{1 Chr 22:1}
    וַיֹּאמֶר דָּוִיד זֶה הוּא בֵּית יְהוָה הָאֱלֹהִים וְזֶה־מִּזְבֵּחַ לְעֹלָה לְיִשְׂרָאֵל׃ 
\end{hebrewtext}
\begin{translation}
    \versenum{1 Chr 22:1}
    And David said, ``This is it; the house of \yahweh. This is [the] altar for burnt offerings for Israel. ''
\end{translation}
\noindent
Thus the \chronicler presents Gad's command to build an altar at the threshing floor of Araunah/Ornan as tantamount to commanding that David establish the site of the new temple.  

The reimagining of David's altar at Araunah/Ornan's threshing floor as the foundation of the temple in Jerusalem reflects the magnetic tendency between major sites of memory. Note that the purpose of this pericope in \chronicles is no longer simply a (weird) story about how David avoided disaster by \yahweh's mercy, but about how \yahweh indicated to David where the temple would be constructed. The common translation of \hebrew{זֶה הוּא בֵּית יְהוָה} as ``Here shall be the house of the \textsc{Lord} God'' (NRSV), however, misses the force of David's declaration. David does not say that the temple \emph{will be} ``here,'' but performatively declares that the location of ``this altar'' \emph{is now the temple} and that \emph{this altar is THE altar}. By doing this, the \chronicler has explicitly credited the figure of David to the establishment of the Solomonic temple. It is not a coincidence, therefore, that the rest of 1 Chr 22 describes the preparations that David makes for the construction of the temple, David's admonition to Solomon to build the temple, and a command to the leaders of Israel to support Solomon in this endeavor. Thus the narrative of David's census has becomes the means by which the connection between the two major sites of memory in the book of \chronicles---David and the temple---becomes explicit and within the social memory of the \chronicler, the two sites of memory have become more deeply entangled with one another.

The magnetic process was not limited to the mnemonic sites of David and the temple. Later in \chronicles when Solomon is beginning to build the temple, the \chronicler makes an explicit the connection between the site of the new temple and ``Mt. Moriah'':
\begin{hebrewtext}
    \versenum{2 Chr 3:1}
    וַיָּחֶל שְׁלֹמֹה לִבְנוֹת אֶת־בֵּית־יְהוָה בִּירוּשָׁלִַם בְּהַר הַמּוֹרִיָּה אֲשֶׁר נִרְאָה לְדָוִיד אָבִיהוּ אֲשֶׁר הֵכִין בִּמְקוֹם דָּוִיד בְּגֹרֶן אָרְנָן הַיְבוּסִי׃
\end{hebrewtext}
\begin{translation}
    \versenum{2 Chr 3:1}
    Solomon began to build the temple of \yahweh in Jerusalem on Mt. Moriah where [\yahweh]%
        \footnote{Following \lxx: 
        \greek{οὗ ὤφθη κύριος τῷ Δαυιδ πατρὶ αὐτοῦ}.
        The Hebrew does not make sense as written.}
    appeared to David his father;
    at the place where David had established; 
    at the threshing floor of Ornan the Jebusite.
\end{translation}
\noindent
Although this is the only reference to the \emph{mountain} of Moriah in the Hebrew Bible, the \emph{land} of Moriah is only mentioned one other time in the Hebrew Bible as the setting of the Aqedah (Gen 22). In the well-known story of the near-sacrifice of Isaac Abraham was instructed by \yahweh to bring his only son, Isaac,  to the land of Moriah and to sacrifice him (on a mountain!). This reference to Moriah appears to be another example of the magnetic processed between the core sites of memory within a community's social memory. In the same way that David's sacrificial acts at the threshing floor of Ornan---in the memory of the Chronicler---connected the idea of David as temple-founder with the Jerusalem temple, so too the near-sacrifice of Isaac, by its geographic association with ``Moriah'' converges on the site of the Jerusalem temple and the near-sacrifice of Isaac becomes a prototype for the sacrificial cult.%
    \footnote{In fact, \vermes makes this point explicit and traces the tradition into early Christianity. See \cite[204--211]{vermes1961}. This connection has also been fruitfully analyzed by \cite{kalimi_htr1990}; \cite[190--191]{kalimi_jnes2009} and \cite{amit_brenner-polak2009}.}

Thus the story of David's census was not only reimagined within a different theological or ideological system than the parallel account in 2 Sam 24, but it was resituated within the social memory of the \chronicler as the crossroads of three major metanarrative arcs within remembered past of ancient Israel---the Davidic Monarchy, the Jerusalem temple, and the Abrahamic covenant.