
Within the story of David's census and the threshing floor of Oruna in 1 Chr 21, several mnemonic sites can be understood as exhibiting magnetic qualities. First, although the identity of \emph{śāṭān} is not clear in the text, later treatments of the story, including in the LXX, treat the character as a malevolent supernatural being. The potency---at least in the interpretation of later periods---of Satan was able to don this particular mischief and take credit for the incitement of David to sin. One might imagine, given \emph{haśśāṭān}'s subordinate position \visavis Yahweh in the prologue to Job, that the two passages were not seen to be in direct conflict with one another, whatever the original intent of the Chronicler might have been. Shifting the incitement of David away from Yahweh can be explained by the ostensible discomfort (or confusion) created by the story in 2 Sam 24, but, supposing \emph{śāṭān} was imagined by the Chronicler to be a supernatural being, we can account for the Chronicler's choice as an example of magnetism on the part of Satan. 

Second, the identification of Ornan's threshing floor with the future site of the Solomonic Temple creates an explicit connection between two very potent mnemonic sites: David and the Temple. The discourse surrounding the fact that it was Solomon and not David who constructed the Temple of Yahweh predates even the DtrH, as indicated by 2 Sam 7, but the degree to which David involves himself in making preparations for the construction of the temple in 1 Chr 22:2--29:22 indicates that DtrH did not get the last word on the matter. While 2 Sam 7 provides an apologia for why it was Solomon, and not David, who built the temple, Chronicles takes it a step further by attributing the planning and preparation of the temple's construction to David, leaving only the most nominal tasks for the temple's ``builder,'' Solomon. The inability of David's mnemonic gravitas to fully absorb the temple's construction illustrates the immutability of certain mnemonic sites, particularly those such as buildings and geographic features, despite the relative significance of David for the continued maintenance of Judean identity continued into the \secondtemple Period \visavis Solomon. 

Finally, after identifying the site of the temple of Solomon with Ornan's threshing floor, and drawing the mnemonic sites of the temple and king David even closer together, the Chronicler includes one additional piece of information to the reader in the description of the temple's construction in 2 Chr 3:1: 

\begin{quote} \emph{wayyāḥel šəlōmōh liḇnôṯ ʾeṯ-bêṯ-yhwh bı̂rûšālaim bəhar hammôrı̂ā ʾăšer nirʾāh ləḏāwı̂ḏ ʾāḇı̂hû ʾăšer hēḵı̂n bimqôm dāwı̂ḏ bəḡōren ʾornān hayəḇûsı̂} 

Solomon began to build the temple of Yahweh in Jerusalem \emph{on the mountain of Moriah} where he appeared to David, his father, the place which David designated, on the threshing floor of Ornan, the Jebusite. \end{quote} 

Although this is the only reference to the \emph{mountain} of Moriah in the Hebrew Bible, the \emph{land} of Moriah is mentioned only in Gen 22, the Aqeda, as the land to which Abraham was to bring Isaac for sacrifice (on a mountain!).\autocite[358--359]{kalimi_htr1990} The reference to Moriah appears to be another example of the magnetic quality of core events in a community's identity. In the same way that David's sacrificial acts at the threshing floor of Ornan---in the memory of the Chronicler---prefigured and made acceptable the offerings made by David there, so too the near-sacrifice of Isaac, by its geographic association with the foundation of the temple in Jerusalem, becomes a prototype for the sacrificial cult.\autocite[In fact, \vermes makes this point explicit and traces the tradition into early Christianity. See][204--211]{vermes1961} 