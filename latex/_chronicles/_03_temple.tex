% !TEX root = dissertation.tex

\section{The Jerusalem Temple as a Site of Memory}
Another potent site of memory in Chronicles is the temple in Jerusalem. As with David, the memory of the Temple in Chronicles is not entirely novel. Already in the book of Deuteronomy the mythology surrounding the divine selection of Jerusalem and the uniquely ordained site of the Solomonic temple had been well-established. This development is easily seen by contrasting the ways that the Covenant Code of \exod 20 in which \yahweh seems to command (or, at the very least not \emph{prohibit}) the Israelites to establish cult sites \hebrew{בְּכָל־הַמָּקוֹם אֲשֶׁר אַזְכִּיר אֶת־שְׁמִי} ``in every place that I commemorate my name'' (\exod 20:24)%
    % FIXME: Fix this cluster, It's nonsense
    \footnote{It is hard not to speculate that a number of the textual variants in this verse are due to the implication that \yahweh could be commemorate his name in multiple places, compared to its counterpart in \deut 12:5. This discomfort is illustrated in  \sampent's omission of \hebrew{כל} with a result that \hebrew{מָקוֹם} is conceptually singular (in \emph{the} place), while \lxx, Syriac, and the Targums all support the reading ``in every place.'' The Niqqudim make it a point to separate the ideas, emphasizing that the clause \hebrew{בְּכָל־הַמָּקוֹם} ``in every place'' modifies only the following clause \hebrew{אָבוֹא אֵלֶיךָ וּבֵרַכְתִּיךָ} ``I will come to you and bless you'' and not completing the action of the preceding \hebrew{מִזְבַּח אֲדָמָה תַּעֲשֶׂה־לִּי וְזָבַחְתָּ עָלָיו} ``you will make an earthen altar for me and make sacrifices upon it.'' Indeed, the first person form \hebrew{אַזְכִּיר} favors the former reading. The Syriac does, however, offer a variant suggesting a possible second person \emph{Vorlage} \hebrew{תַּזְכִּיר*}, which I find intriguing, ``in every place that \emph{you} commemorate my name, I will come....'' Another explanation for the grammar is to read the imperfect form \hebrew{אָבוֹא} as a volitive ``[in order that] I might come to you and bless you,'' though one would expect a \hebrew{ו}. I admit that all of these options are tenuous, and it may be that the grammar is unremarkable. Even so, \exod 20 seems to presuppose that \yahweh could or would cause his name to be commemorated in more than one place. And, at least for the author of Deut 12, this seemed ambiguous enough that he felt the need to forcefully clarify his position.}
with that of Deuteronomy, in which \yahweh commands the Israelites to destroy all cult sites within the land and furthermore that: 

\begin{hebrewtext}
    \versenum{\deut 12:5}
    כִּי אִם־אֶל־הַמָּקוֹם אֲשֶׁר־יִבְחַר יְהוָה אֱלֹהֵיכֶם מִכָּל־שִׁבְטֵיכֶם לָשׂוּם אֶת־\\שְׁמוֹ שָׁם לְשִׁכְנוֹ תִדְרְשׁוּ וּבָאתָ שָׁמָּה׃
    \versenum{6}
    וַהֲבֵאתֶם שָׁמָּה עֹלֹתֵיכֶם וְזִבְחֵיכֶם וְאֵת מַעְשְׂרֹתֵיכֶם וְאֵת תְּרוּמַת יֶדְכֶם וְנִדְרֵיכֶם וְנִדְבֹתֵיכֶם וּבְכֹרֹת בְּקַרְכֶם וְצֹאנְכֶם׃
\end{hebrewtext}
\begin{translation}
    \versenum{\deut 12:5}
    But you shall seek the place that \yahweh your God will choose from among all your tribes as his dwelling to put his name there. You shall go there
    \versenum{6}
    and you will bring your burnt offerings there as well as your sacrifices, your tithes and the offerings of your hands, your votive gifts, your freewill offerings, and the firstborn of your cattle and flocks. 
\end{translation}

% The Temple
    % How is the temple portrayed in Chr?
        % Levites ?
    % Is there anything unique about this?
    % What about the Physicality of the temple makes it unique as a site of memory?