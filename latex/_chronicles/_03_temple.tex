% !TEX root = dissertation.tex

\section{The Temple as a Site of Memory}
The temple in Jerusalem was already an important site of memory for ancient Israel long before the book of \chronicles was written. Already in the book of Deuteronomy the mythology surrounding the divine selection of Jerusalem and the uniquely ordained site of the Solomonic temple had been well-established. This development is easily seen by contrasting the ways that the Covenant Code of \exod 20 in which \yahweh seems to command (or, at the very least not \emph{prohibit}) the Israelites to establish cult sites \hebrew{בְּכָל־הַמָּקוֹם אֲשֶׁר אַזְכִּיר אֶת־שְׁמִי} ``in every place that I commemorate my name'' (\exod 20:24) with that of Deuteronomy, in which \yahweh commands the Israelites to destroy all cult sites within the land and furthermore that: 
\begin{hebrewtext}
    \versenum{\deut 12:5}
    כִּי אִם־אֶל־הַמָּקוֹם אֲשֶׁר־יִבְחַר יְהוָה אֱלֹהֵיכֶם מִכָּל־שִׁבְטֵיכֶם לָשׂוּם אֶת־\\שְׁמוֹ שָׁם לְשִׁכְנוֹ תִדְרְשׁוּ וּבָאתָ שָׁמָּה׃
    \versenum{6}
    וַהֲבֵאתֶם שָׁמָּה עֹלֹתֵיכֶם וְזִבְחֵיכֶם וְאֵת מַעְשְׂרֹתֵיכֶם וְאֵת תְּרוּמַת יֶדְכֶם וְנִדְרֵיכֶם וְנִדְבֹתֵיכֶם וּבְכֹרֹת בְּקַרְכֶם וְצֹאנְכֶם׃
\end{hebrewtext}
\begin{translation}
    \versenum{\deut 12:5}
    But you shall seek the place that \yahweh your God will choose from among all your tribes as his dwelling to put his name there. You shall go there
    \versenum{6}
    and you will bring your burnt offerings there as well as your sacrifices, your tithes and the offerings of your hands, your votive gifts, your freewill offerings, and the firstborn of your cattle and flocks. 
\end{translation}
\noindent
It is hard not to speculate that the textual variants in \exod 20:10 are due to the implication that \yahweh could be commemorate his name in multiple places, compared to its counterpart in \deut 12:5. This discomfort is illustrated in  \sampent's omission of \hebrew{כל} with the result that \hebrew{מָקוֹם} is conceptually singular (in \emph{the} place), while \lxx, Syriac, and the Targums all support the reading ``in every place.''%
    \footnote{In the case of the \sampent, the editor may have had in mind ``Samaria'' rather than ``Jerusalem,'' but the impulse is the same.}
Such a reading implies that the author had in mind an \emph{itinerant} cult site. The Niqqudim make it a point to separate the ideas, emphasizing that the clause \hebrew{בְּכָל־הַמָּקוֹם} ``in every place'' modifies the following clause \hebrew{אָבוֹא אֵלֶיךָ וּבֵרַכְתִּיךָ} ``I will come to you and bless you'' and not completing the action of the preceding \hebrew{מִזְבַּח אֲדָמָה תַּעֲשֶׂה־לִּי וְזָבַחְתָּ עָלָיו} ``you will make an earthen altar for me and make sacrifices upon it.'' Indeed, the first person form \hebrew{אַזְכִּיר} favors the former reading. Even so, \exod 20 seems to presuppose that \yahweh could or would cause his name to be commemorated in more than one place. On the other hand, the book of Deuteronomy states clearly that the the Israelite were only to bring their offerings to the \emph{the} place that \yahweh would choose from among the tribes. The historical reality of Israelite shrines and cult sites outside of Jerusalem during the monarchic period such as those from Dan, Arad, Beer-Sheeba, and others is well documented.%
    \footnote{For a concise overview of the archaeological evidence, see \cite[319--352]{king-stager2001}. See also \cite{edelman_barton-stavrakopoulou2010} and \cite[160--181]{smith2002}.}
While these sites were condemned as idolatrous by the deuteronomistic editor(s), there is no evidence to suggest that contemporaries of the \bce{seventh century} (or earlier) saw them as such.

The increased importance of the Jerusalem temple brought about by the cult centralization efforts of Hezekiah and Josiah after the destruction of the Northern Kingdom similarly consolidated the religious memory of ancient Israel around Jerusalem and temple of Solomon. Insofar as the real religious practices of Israel (putatively) became increasingly focused on the city of Jerusalem and Solomon's temple leading up to its destruction at the beginning of the \bce{sixth century}, ancient Israel's memory about other ``marginal'' religious practices was quite literally demolished through the intentional destruction of \translit{bāmôt} and other sacred sites through the religious reforms of Hezekiah and Josiah.%
    \footnote{\cite[182--199]{smith2002}; \cite[191--209]{romer2015}.}

Thus it was the socio-political \emph{reality} of the Jerusalem temple's significance at the end of the \bce{sixth century}---brought about by the intentional religious reforms of Josiah---which informed the deuteronomistic editor's memory of earlier \yahwistic cult practices and which would form the basis for the \chronicler's perception of religious practice during the early monarchic period. Regardless of how centrally significant the Jerusalem temple actually had (or had not) been during the early monarchic period, or how successful the practical aspects of Hezekiah and Josiah's reforms had been, between the end of the \bce{seventh century} and the time of the \chronicler the memory of the temple had accrued meaning as a site of memory for the Golah community of Persian Yehud.%
    \footnote{I restrict my discussion here to the memory of Persian \emph{Yehud}, meaning the Golah community. The Samaritans, \translit{ʕam hāʔāreṣ}, and the Jewish garrison at Elephantine, presumably, had their own systems of memory which, while historically related, would have been distinct in this period.}
In other words, what is important for our purposes is not what the historical function of the temple had been during the monarchic period, but the function the Jerusalem temple played in the memory of the \chronicler and what kinds of social factors contributed to that function.

Consider, for example, the foundational role that the temple played in the reestablishment of the Golah community, as presented in the closing verses of 2~Chronicles (2~Chr 36:22--23|| Ezra 1:1--4). According to these texts, the first task of the returnees was to construct the \emph{temple}. The construction project (as presented here) comes as a result of Cyrus' desire to build a temple for \yahweh. This emphasis on the importance of the temple's reconstruction aligns with sentiments from the other accounts of the temple's construction in Haggai and Zechariah. The account in Haggai 1:1--4, in particular, evokes the same rhetorical question asked by David in 2~Sam~7 and 1~Chr~17 and asserts that the struggles that the Golah community was facing were tied to the fact that they had yet to reconstruct the temple:
\begin{hebrewtext}
    \versenum{Haggai 1:1}
    ‏בִּשְׁנַת שְׁתַּיִם לְדָרְיָוֶשׁ הַמֶּלֶךְ בַּחֹדֶשׁ הַשִּׁשִּׁי בְּיוֹם אֶחָד לַחֹדֶשׁ הָיָה דְבַר־יְהוָה בְּיַד־חַגַּי הַנָּבִיא אֶל־זְרֻבָּבֶל בֶּן־שְׁאַלְתִּיאֵל פַּחַת יְהוּדָה וְאֶל־יְהוֹשֻׁעַ בֶּן־יְהוֹצָדָק הַכֹּהֵן הַגָּדוֹל לֵאמֹר׃ 
    \versenum{2}
    כֹּה אָמַר יְהוָה צְבָאוֹת לֵאמֹר הָעָם הַזֶּה אָמְרוּ לֹא עֶת־בֹּא עֶת־בֵּית יְהוָה לְהִבָּנוֹת׃  
    \versenum{3}
    וַיְהִי דְּבַר־יְהוָה בְּיַד־חַגַּי הַנָּבִיא לֵאמֹר׃ 
    \versenum{4}
    הַעֵת לָכֶם אַתֶּם לָשֶׁבֶת בְּבָתֵּיכֶם סְפוּנִים וְהַבַּיִת הַזֶּה חָרֵב׃
\end{hebrewtext}
\begin{translation}
    \versenum{Haggai 1:1}
    In the second year of Darius the King, in the sixth month, on the first day of the month, the word of \yahweh came by the hand of Haggai the prophet to Zerubbabel, son of Shealtiel, the governor of Judah and to Joshua, son of Jehozadak, the high priest, saying: 
    \versenum{2}
    ``Thus says \yahweh of Hosts: `This people says ``The time has not come (yet) to build a temple for \yahweh'''''
    \versenum{3}
    And the word of \yahweh came by the hand of Haggai the prophet, saying:
    \versenum{4}
    ``Is it time for you to live in your (own) paneled houses while this temple is in ruins?''
\end{translation}
\noindent
In all of these cases, the construction of the temple is of central concern to the authors of the biblical text. In the case of 2 Chronicles and Ezra, as the impetus for the returnees to go back to the land, and for Haggai and Zechariah as a way to complete the establishment of the Golah community. These text show considerable diversity, however, in a number of particulars. 

It is conspicuous to me, for example, that although all the accounts operate within the Persian administrative context, the effort to reconstruct the temple in \chronicles and Ezra is instigated by Cyrus as a part of his benevolent edict. Because temple-construction was thought to be one of the central responsibilities of kings in the ancient world, it is understandable that it is \emph{Cyrus} who gives the command to rebuild the temple in Jerusalem.%
    \footnote{On temple construction as a royal activity, see
        \cite{kapelrud_orientalia1963};
        \cite{petersen_cbq1974};
        \cite{laato_zaw1994}.

        It is also worth noting that the Persian Empire \emph{did} in fact commission the reconstruction of religious and cultural apparatuses, as evidenced by the Egyptian Udjahorresnet. See 
            \cite{lloyd_jea1982}. On the relationship of Udjahorresnet to Ezra and Nehemiah, see 
            \cite{blenkinsopp_jbl1987}.
        Whether or not this was a part of a broader practice of so-called imperial authorization of local customs remains a matter of debate. See especially
            \cite{frei_frei1984};
            \cite{frei_watts2001}.}
These accounts lack any hint of nationalistic aspirations for the return---the returnees are Persian subjects working at the behest of the benevolent and pious Persian king (2~Chr 36:22--23|| Ezra 1:1--4). Note further Zerubbabel's response in Ezra 4:3 to the \translit{ʕam hāʔāreṣ} who wished to assist in the temple's reconstruction:
\begin{hebrewtext}
    \versenum{Ezra 4:3}
    וַיֹּאמֶר לָהֶם זְרֻבָּבֶל וְיֵשׁוּעַ וּשְׁאָר רָאשֵׁי הָאָבוֹת לְיִשְׂרָאֵל לֹא־לָכֶם וָלָנוּ לִבְנוֹת בַּיִת לֵאלֹהֵינוּ כִּי אֲנַחְנוּ יַחַד נִבְנֶה לַיהוָה אֱלֹהֵי יִשְׂרָאֵל כַּאֲשֶׁר צִוָּנוּ הַמֶּלֶךְ כּוֹרֶשׁ מֶלֶךְ־פָּרָס
\end{hebrewtext}
\begin{translation}
    \versenum{Ezra 4:3}
    And Zerubbabel and Joshua and the remaining heads of the families of Israel said to them, ``It is not your place, but ours, to build a temple for our God. But we alone will build (it) for \yahweh, the God of Israel, \emph{as Cyrus the king of Persia commanded us}.'' (Emphasis added)
\end{translation}
\noindent
On the other hand, the accounts of Haggai and Zechariah, although not overtly nationalistic or anti-imperial, focus on the figure of Zerubbabel as a semi-royal, Davidic figure charged with the rebuilding of the temple (along with the high priest, Joshua) \emph{by \yahweh}. In other words, this royal responsibility was taken on by Zerubbabel and Joshua (the high priest) and \emph{not} by the Persian king, which has lead some scholars to suggest that Zerubbabel was viewed as a royal-messianic figure. Moreover, both Haggai and Zechariah betray certain images that point toward some kind of semi-royal or messianic hope associated with him, most notably in Haggai's reference to \yahweh making Zerubbabel ``like a signet ring'' (Hag~2:23). Although the text does not make it explicit that Zerubbabel was viewed as a semi-royal figure, it is difficult to read Haggai's use of the term ``signet ring'' as anything but an allusion to Jer 22:24 which describes ``Coniah'' as a signet ring on \yahweh's hand which would be removed and cast into exile:
\begin{hebrewtext}
    \versenum{Jer 22:24}
    חַי־אָנִי נְאֻם־יְהוָה כִּי אִם־יִהְיֶה כָּנְיָהוּ בֶן־יְהוֹיָקִים מֶלֶךְ יְהוּדָה חוֹתָם עַל־יַד יְמִינִי כִּי מִשָּׁם אֶתְּקֶנְךָּ׃ 
    \versenum{25}
    וּנְתַתִּיךָ בְּיַד מְבַקְשֵׁי נַפְשֶׁךָ וּבְיַד אֲשֶׁר־אַתָּה יָגוֹר מִפְּנֵיהֶם וּבְיַד נְבוּכַדְרֶאצַּר מֶלֶךְ־בָּבֶל וּבְיַד הַכַּשְׂדִּים׃
\end{hebrewtext}
\begin{translation}
    \versenum{Jer 22:24}
    As I live---an utterance of \yahweh---even if Coniah, son of Jehoiakim, king of Judah, were a signet ring upon my right hand, even from there I would tear you off
    \versenum{25}
    and I would give you into the the hand of those who seek your life and into the hand of those you dread and into the hand of Nebuchadnezzar, the king of Babylon and into the hand of the Chaldeans.
\end{translation}
\noindent
In Haggai, this negative image of \yahweh removing and discarding the ``signet ring'' is used positively, ostensibly, to mark Zerubbabel as \yahweh's agent on earth and, probably, as a royal messianic figure.%
    \footnote{See
        \cite[71--103]{blenkinsopp2013};
        \cite[2:281--284]{vonrad1962};
        \cite[187]{redditt_interpretation2007}.}
The fact that Zerubbabel is referred to as the ``servant'' (Heb. \hebrew{עֶבֶד}) evokes the way that David was characterized as \yahweh's servant and supports this general conclusion.

All of this is to say that during the \secondtemple period the temple itself was overloaded with significance. The historical reality of Zerubbabel's failure to come into his kingship (if  indeed this was what Haggai and Zechariah allude to), meant that it was the temple and not the kingship that provided continuity between the present and the remembered past. The significance of the Solomonic temple in the memory of \secondtemple Judaism, is a reflection of the significance of the second temple in the lived experience of the Golah community. This is not to suggest that the Solomonic temple would not have been significant had the second temple not been built, only that the \emph{presence} of the second temple augmented and affected the memory of the former temple in the memory of \secondtemple Judaism.
