% !TEX root = dissertation.tex
\section{Introduction}

Scholars of the Hebrew Bible have long observed that the book of \chronicles is a derivative work of Samuel--Kings reflecting the concerns and religious sensibilities of the late Persian or Early Hellenistic Periods. For example, in his classic work on the history of ancient Israel, Julius Wellhausen begins his treatment of the history of traditions within Judaism with a lengthy discussion of the book of \chronicles. In his treatment of the book, Wellhausen argues that the history of Israel as portrayed in \chronicles differs from that portrayed in Samuel--Kings due to the chronological distance of the works and the intervention of the Priestly Code into the theology of \secondtemple Judaism.%
    \footnote{%
        \cite[171--172]{wellhausen1957}. See also
        \cite{wright_ulrich-wright1992}.}
\chronicles, according to Wellhausen, provides a clear example of the way that ancient Israel's traditions evolved over time. Just as the legal material of the Hexateuch developed over the centuries, so too the traditions of the historical books were subject to the changing theologies of later centuries. While the particulars of the relationship of Samuel--Kings to \chronicles and the nuances of priestly influences on the Hebrew Bible remain subject to scholarly debate, the broad consensus remains that 1) \chronicles was written sometime in the late Persian or early Hellenistic periods, 2) it heavily utilized Samuel--Kings as a literary source, and 3) it bears an ideological \emph{Tendenz} influenced by (though not identical to) the final layers of the Pentateuch.%
    \footnote{For a thorough and reasonably recent summary of the \emph{status quaestionis}, see 
        \cite[72--89]{knoppers2003}. See also
        \cite{japhet1993};
        \cite{japhet2009};
        \cite{braun1986} and
        \cite{coggins1976}.}

 The scholarly discourse surrounding the formation of the Hebrew Bible has increasingly turned to memory studies as a robust framework for describing the processes by which the biblical traditions were produced and transmitted.%
    \footnote{%
        \cite{wright2014};
        \cite{blenkinsopp2013};
        \cite{rogerson2010};
        \cite{davies2008};
        \cite{hendel2005};
        \cite{smith_cbq2002}.}
Wellhausen chose to begin his treatment of the history of traditions with \chronicles because of the relative security with which scholars are able to date Samuel--Kings and the major Pentateuchal strata \visavis \chronicles. For the very same reason, \chronicles likewise has played an important role in early applications of memory theory within biblical studies.%
    \footnote{%
        \cite{benzvi_st2017};
        \cite{benzvi-a_evans-williams2013};
        \cite{benzvi-b_evans-williams2013}.}
While \chronicles is not the \emph{latest} book in the Hebrew Bible, it is uniquely situated at the end of the traditioning process preserved in the Hebrew Bible. In some ways, therefore, \chronicles gets the last word on a certain set of traditions surrounding the monarchic period, in particular those of David, Solomon, and the kings of Judah. 

Although it is broadly agreed upon that \chronicles exhibits a hypertextual relationship to Samuel--Kings, treating \chronicles as an exemplar of Rewritten Bible (\rwb) is somewhat less common.%
    \footnote{Though, not particularly \emph{un}common. See 
        \cite{campbell_zsengeller2014};
        \cite{zahn_lim-collins2010};
        \cite{bernstein_textus2005}.
        Alexander considers \chronicles to be a ``prototype'' of \rwb, see, 
        \cite[100]{alexander_carson-williamson1988}.
        I have adopted the terminology of hyper-/hypotext from Genette. In this case, to say that \chronicles is a ``hypertext'' of Samuel--Kings is to say that it is derivative, but not a commentary on Samuel--Kings. See
        \cite[5]{genette1997}.}
The ambivalence of scholars towards treating \chronicles as \rwb is rooted, unsurprisingly, in the confusion surrounding the definition of the term. Knoppers, for example, takes special care to treat the question of whether \chronicles should be understood as \rwb in the introduction of his commentary and notes, from the very beginning, that he will answer the question based on what he understands to be the essential elements of \rwb.\autocite[129--134]{knoppers2003} He writes: 

 \begin{quote}
    They [\rwb] select from, interpret, comment on, and expand portions of a particular biblical book (or group of books), addressing obscurities, contradictions, and other perceived problems with the source text. Rewritten Bible texts normally emulate the form of the source text and follow it sequentially. The major intention of such works seems to be to provide a coherent interpretive reading of the biblical text.\autocite[130]{knoppers2003}
\end{quote} 

Knoppers observes that, while \chronicles exhibits most of the specific literary moves which \rwb is known for (expanding, harmonizing, and augmenting its \emph{Vorlage}), the presence of material which is entirely unique to \chronicles cannot be attributed to a purely exegetical or explanatory impulse. In addition to the narrative additions within \chronicles, the genealogies of 1 Chr 1--9 have no corollary in Samuel--Kings, and can hardly be considered a rewritten form of the Pentateuch.\autocite[132]{knoppers2003} Thus, while Knoppers affirms that certain portions of \chronicles ``may be profitably compared with a number of rewritten Bible texts,''\autocite[131]{knoppers2003} ultimately he concludes that \chronicles ``is more than a paraphrase or literary elaboration of the primary history''\autocite[134]{knoppers2003} and thus \rwb cannot can account for \chronicles as a whole, instead suggesting that it should be treated as its own work.\autocite[131--134]{knoppers2003} 

While Knoppers's assessment is fair given the definition that he supplies for \rwb, in the preceding chapters of this dissertation, I have argued that similar extra-exegetical qualities exist within the \rwb corpus which push us to consider the function of \rwb as more than a method for explaining sacred texts. The rationale for why \rwb cannot account for the complexity of the whole book of \chronicles, according to Knoppers, is the same basic argument that I have made for why an \emph{exegetically} focused definition of \rwb is insufficient to account for the complexity of even the literature that scholars \emph{traditionally} consider to be \rwb. In other words, Knoppers's argument for why \chronicles should not be considered \rwb is the same essential argument that I am making for \ga and \jub. The same case could just as easily be applied to Josephus's \ant, the \templescroll, and others. The difference between my thesis and Knoppers, however, is that where he sees disjunction between \rwb and \chronicles, I am arguing that all of these texts represent the same fundamental social and cultural processes of memory and that memory theory offers a degree of abstraction for talking about these processes which highlights their similarities.

