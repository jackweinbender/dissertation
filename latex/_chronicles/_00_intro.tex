% !TEX root = dissertation.tex
Scholars of the Hebrew Bible have long observed that the book of \chronicles is a derivative work of Samuel--Kings reflecting the concerns and religious sensibilities of the late Persian or Early Hellenistic Periods. For example, in his classic work on the history of ancient Israel, Julius Wellhausen begins his treatment of the history of traditions within Judaism with a lengthy discussion of the book of \chronicles. In his treatment of the book, Wellhausen argues that the history of Israel as portrayed in \chronicles differs from that portrayed in Samuel--Kings due to the chronological distance of the works and the intervention of the Priestly Code into the theology of \secondtemple Judaism.%
    \footnote{%
        \cite[171--172]{wellhausen1957}. See also
        \cite{wright_ulrich-wright1992}.}
\chronicles, according to Wellhausen, provides a clear example of the way that ancient Israel's traditions evolved over time. Just as the legal material of the Hexateuch developed over the centuries, so too the traditions of the historical books were subject to the changing theologies of later centuries. While the particulars of the relationship of Samuel--Kings to \chronicles and the nuances of priestly influences on the Hebrew Bible remain subject to scholarly debate, the broad consensus remains that 1) \chronicles was written sometime in the late Persian or early Hellenistic periods, 2) it heavily utilized Samuel--Kings as a literary source, and 3) it bears an ideological \emph{Tendenz} influenced by (though not identical to) the final layers of the Pentateuch.%
    \footnote{For a thorough and reasonably recent summary of the \emph{status quaestionis}, see 
        \cite[72--89]{knoppers2003}. See also
        \cite{japhet1993};
        \cite{japhet2009};
        \cite{braun1986} and
        \cite{coggins1976}.}
Wellhausen chose to begin his treatment of the history of traditions with \chronicles because of the relative security with which scholars are able to date Samuel--Kings and the major Pentateuchal strata \visavis \chronicles. For the very same reason, \chronicles likewise has played an important role in early applications of memory theory within biblical studies.%
    \footnote{%
        \cite{benzvi_st2017};
        \cite{benzvi-a_evans-williams2013};
        \cite{benzvi-b_evans-williams2013}.}
While \chronicles is not the \emph{latest} book in the Hebrew Bible, it is uniquely situated at the end of the traditioning process preserved in the Hebrew Bible. In some ways, therefore, \chronicles gets the last word on a certain set of traditions surrounding the monarchic period, in particular those of David, Solomon, and the kings of Judah. 

Although the particular relationship between the book of \chronicles and the books of Samuel--Kings is a matter of scholarly debate, it is generally agreed that Samuel--Kings forms the basis for much of the \chronicler's depiction of Israel's history.%
    \footnote{The observation was made as early as de Wette in the early nineteenth century in his \cite*{dewette1806}. More recently, see especially the work of McKenzie
        \cite*{mckenzie1985};
        \cite{mckenzie_graham-mckenzie1999};
        \cite[66--71]{knoppers2003}; and 
        \cite[30--42]{klein2006} as well as that of 
        \cite[74--74]{carr2011}. Notable exceptions, however, do exist. See especially the work of 
        \cite{auld1994}; 
        \cite{auld_graham-mckenzie1999} and 
        \cite{person2010}.}
A great deal of work has been done analyzing the particular literary relationship between Samuel--Kings and \chronicles and the textual processes involved---e.g., what version(s) of Samuel--Kings the \chronicler may have used, etc. And although it is not universally accepted as an exemplar of \rwb, by the definitions that I have adopted, there is little (if any) reason to treat it as qualitatively different than \ga or \jub.%
    \footnote{A number of scholars include it in their lists of \rwb texts. See 
        \cite{campbell_zsengeller2014};
        \cite{zahn_lim-collins2010};
        \cite{bernstein_textus2005}.
        Alexander considers \chronicles to be a ``prototype'' of \rwb, see, 
        \cite[100]{alexander_carson-williamson1988}.}

Thinking in terms of social memory requires us to consider the relationship between the texts in \emph{social} terms. In other words, not just to ask \emph{what} received traditions the \chronicler used, but to consider the \emph{role} and \emph{status} of those traditions and to consider why they were (or were not) significant within a particular social context. Thus the process of ``remembering'' in \chronicles can be viewed from two different angles which map onto the dual valences of the term ``remember'': to ``recall'' and to ``commemorate.'' On the one hand, the \chronicler ``recalls'' stories which are adapted to the frameworks of the \chronicler's social situation. The \chronicler is a product of his time and society and as such inherited sets of traditions about Israel's remembered past and the world more broadly which color how he understands that history. On the other hand, the composition of the book of \chronicles is itself an act of commemoration which (as we have noted) is a conscious, constructive process. It represents the process of memory encoding and the construction of cultural memory from which future rememberers would draw. As a work literature, it also bears the idiosyncrasies of its author(s), however constrained by their social milieu they may have been. In fact, determining which of these processes best accounts for any particular ``innovation'' of \chronicles is quite difficult. Was the \chronicler consciously ``reshaping'' the memory of Israel's past? Or was the \chronicler more passively reproducing the memory that he inherited from his culture? Or perhaps both? Traditional approaches to the book of \chronicles have tended to attribute a great deal of agency to the \chronicler as an innovator of tradition. But thinking in terms of cultural memory pushes us to consider a fuller picture of how cultural memory is created and calls into question whether every theological or ideological augmentation of the \chronicler should be attributed to his idiosyncratic understanding of the Israelite past. Such an approach takes into account that textual ``sources'' are not merely copied and ``altered,'' but are read, internalized, believed, understood, and reasoned about, which is to say, \emph{remembered}.

% TODO: In this chapter...