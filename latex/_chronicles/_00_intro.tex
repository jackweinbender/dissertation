% !TEX root = dissertation.tex
Scholars of the Hebrew Bible have long held that the book of \chronicles is a derivative work of Samuel--Kings reflecting the concerns and religious sensibilities of the late Persian or early Hellenistic Periods.%
    \footnote{The observation that \chronicles drew from Samuel--Kings was made as early as de Wette in the early nineteenth century in his \cite*{dewette1806}. See also the work of McKenzie
        \cite*{mckenzie1985};
        \cite{mckenzie_graham-mckenzie1999};
        \cite[66--71]{knoppers2003}; and
        \cite[30--42]{klein2006} as well as that of
        \cite[74--74]{carr2011}. Notable exceptions do exist. See the work of
        \cite{auld1994};
        \cite{auld_graham-mckenzie1999} and
        \cite{person2010}.}
In his classic work on the history of ancient Israel, Julius Wellhausen began his treatment of the history of traditions within Judaism with a lengthy discussion of the book of \chronicles and the ways that it differed from Samuel--Kings. He argued that the portrayal of ancient Israel's history in \chronicles differed from Samuel--Kings due to the chronological distance between the works and the theological intervention of the Priestly Code into \secondtemple Judaism.%
    \footnote{%
        \cite[171--172]{wellhausen1957}. See also
        \cite{wright_ulrich-wright1992}.}
According to Wellhausen, \chronicles provided a clear example of how ancient Israel's traditions evolved over time. In the same way that the legal material of the Hexateuch developed over the centuries, the traditions of the historical books reflected the changing theologies of later centuries. While the relationship of Samuel--Kings to \chronicles and the nuances of priestly influences on the Hebrew Bible remain subjects of scholarly debate, the broad consensus remains that 1) \chronicles emerged sometime in the late Persian or early Hellenistic periods, 2) it used Samuel--Kings as a literary source, and 3) it exhibits an ideological \emph{Tendenz} influenced by---though not identical to---the final layers of the Pentateuch.%
    \footnote{For a thorough and reasonably recent summary of the \emph{status quaestionis}, see
        \cite[72--89]{knoppers2003}. See also
        \cite{japhet1993};
        \cite{japhet2009};
        \cite{braun1986} and
        \cite{coggins1976}.}
In this way, \chronicles offers the last word on a certain set of traditions surrounding the monarchic period---in particular those of David, Solomon, and the kings of Judah.

From my perspective, the relationship of \chronicles to Samuel--Kings exhibits all the requisite features for inclusion into the category of \rwb by nearly every system of classification that I discussed in \autoref{chap:rwb} (generic or otherwise). Indeed, for scholars who focus on the topic of \rwb, \chronicles often factors prominently in their discussions.%
    \footnote{See
        \cite{campbell_zsengeller2014};
        \cite{zahn_lim-collins2010};
        \cite{bernstein_textus2005}.
        Alexander considers \chronicles to be a ``prototype'' of \rwb, see,
        \cite[100]{alexander_carson-williamson1988}.}
In spite of this fact, scholars who work primarily on \chronicles seem rarely to give more than a passing nod to the scholarly literature on \rwb. Sara Japhet, for example, does not address the topic at all in her 1993 commentary.%
    \autocite{japhet1993}
Klein's recent commentary on \chronicles, published in 2006, only mentions the topic of \rwb in a footnote, stating:
\begin{quote}
    Perhaps \chronicles could also be compared with the genre called ``rewritten Bible,'' known from Qumran and in the works of Josephus. Such works retell some portion of the Bible while interpreting it through paraphrase, elaboration, allusion to other texts, expansion, conflation, rearrangement, and other techniques. In this case, of course, the ``rewritten Bible'' also became part of the Bible itself.%
    \autocite[17 n.157]{klein2006}
\end{quote}
\noindent
Setting aside Klein's rather anemic description of \rwb, it is striking---if not terribly surprising---to me that the comparison is not taken more seriously. Instead, Klein characterizes \rwb as a genre ``known from Qumran and ... Josephus'' and not as a phenomenon of Jewish literary production during the \secondtemple period in which \chronicles may also have been participating. In his slightly earlier commentary, published in 2003, Knoppers takes up the issue in a special section of his introduction and offers his thoughts on whether \chronicles is \rwb.%
    \autocite[129--134]{knoppers2003}
His conclusions, however, are not much more satisfying. According to Knoppers, \rwb is an interesting angle from which to approach \chronicles, but ultimately, ``\chronicles needs to be understood as its own work.''%
    \autocite[134]{knoppers2003}

The bracketing of \chronicles as \rwb is, I think, largely a function of the ways that conversations within scholarly sub-disciplines can be insular and resistant to traversing disciplinary boundaries. In this case, the book of \chronicles falls squarely within the field of ``biblical studies'' and ``Hebrew Bible'' while most of the scholarly work surrounding \rwb falls within the related, but distinct, fields of Early Judaism, Qumran Studies, and Second Temple Studies. Bridging such disciplinary gaps, even when the sub-disciplines are adjacent, can be difficult. Thus, although some commentators on \chronicles note the family resemblance, there is no commentary-length treatment of \chronicles which utilizes \rwb as the primary literary framework for reading \chronicles.

What we lose in scholarly consensus about the book of \chronicles' characterization as \rwb, we gain in its treatment as an example of cultural and social memory. Wellhausen chose to begin his treatment of the history of traditions with \chronicles because of the security with which scholars date the work relative to Samuel--Kings and the latest Pentateuchal strata. The relative chronology allowed Wellhausen to reason about how those changes may have emerged from an historical perspective and it is for this same reason that \chronicles has played an important role in the emerging applications of memory theory within the field of biblical studies.%
\footnote{%
    See especially the work of Ehud Ben Zvi,
    \cite*{benzvi_st2017};
    \cite*{benzvi-a_evans-williams2013};
    \cite*{benzvi-b_evans-williams2013}.
    See also \cite{wilson2017};
    \cite[26--30]{rogerson2010};
    \cite[104--114]{blenkinsopp2013};
    \cite[148-166]{wright2014};
    \cite{jarick_frohlich2019}.
    }
The relatively mature discourse surrounding the discussion of \chronicles as memory makes it a reasonable place to start for our purposes as well and it is for this reason that I have chosen to start my case studies with the book of \chronicles.

Thinking in terms of social memory requires us to consider the relationship between the texts in \emph{social} terms. In other words, not just to ask \emph{what} received traditions the \chronicler used, but to consider the \emph{role} and \emph{status} of those traditions and to consider why they were (or were not) significant within a particular social context. Thus the process of ``remembering'' in \chronicles can be viewed from two different angles which map onto the dual valences of the term ``remember'': to ``recall'' and to ``commemorate.'' On the one hand, the \chronicler ``recalls'' stories which are adapted to the frameworks of the \chronicler's social situation. The \chronicler is a product of his time and society and as such inherited sets of traditions about Israel's remembered past and the world more broadly which color how he understands that history. The \chronicler participates in discourses at various sites of memory and makes his own contributions to those sites. Thus the composition of the book of \chronicles is itself an act of commemoration which represents the process of memory encoding and the construction of cultural memory from which future rememberers would draw.

As a work literature, it also bears the idiosyncrasies of its author(s), however constrained by their social milieu they may have been. In fact, determining which of these processes best accounts for any particular ``innovation'' of \chronicles is quite difficult. Was the \chronicler consciously ``reshaping'' the memory of Israel's past? Or was the \chronicler more passively reproducing the memory that he inherited from his culture? Or perhaps both? Traditional approaches to the book of \chronicles have tended to attribute a great deal of agency to the \chronicler as an innovator of tradition. But thinking in terms of cultural memory pushes us to consider a fuller picture of how cultural memory is created and calls into question whether every theological or ideological augmentation of the \chronicler should be attributed to his idiosyncratic understanding of the Israelite past. Such an approach takes into account that textual ``sources'' are not merely copied and ``altered,'' but are read, internalized, believed, understood, and reasoned about, which is to say, \emph{remembered}.

In this chapter I will argue that the book of \chronicles---as both a work of \rwb and as an exemplar of social and cultural memory---reflects a distinct system of mnemonic discourses and a reconfiguration of Israel's remembered past over-and-against Samuel--Kings. This reconfiguration, I will argue, came about through specific mnemonic processes.  To demonstrate these processes, I will first discuss the ways that \chronicles engages with major sites of memory, adapting them to his own system of social frameworks. Second, I will discuss the phenomenon of ``magnetism'' between sites of memory and argue that magnetic tenancies between sites of memory may better account for some of the \chronicler's so-called ``harmonizing'' tendencies. Finally, I will discuss the account of David's Census and the Threshing Floor of Araunah/Ornan found in 1 Chr 21:1--22:1 as a product of cultural memory and illustrate the ways that it engages with major sites of memory draws together the major metanarratives of Israel's remembered past.
