% !TEX root = dissertation.tex

\section{Magnetism and Mnemonic Networks}

The questions that the book of \chronicles seeks to answer and the assumptions which it carries are different than that of Samuel--Kings or even Haggai and Zechariah and affect the way that the \chronicler not only read and interpreted his sources, but also the way that he situated various sites of memory with respect to one another. Thus David's role in the construction of the temple is not isolated to the question of why he could not build it, but extends to the way that David, as site of memory, relates to the temple \emph{as a site of memory}.

Although both David and the temple may be thought of as discrete sites of memory, it is important to remember that they participate in a \emph{network} of symbolic social meaning. Thus, ``discrete'' here does not mean ``isolated.'' For example, as I have already demonstrated, the figure of Zerubbabel is connected both to the construction of the second temple as well as to the figure of David, who himself is related to the construction of the first temple, albeit not as its builder. Moreover, not all sites of memory carry the same weight of significance within a particular symbolic system. In other words, not all sites of memory are created equal; David is a much more prominent and potent node within the social memory of ancient Israel than was Shimei, his critic. Though they participate within the same discursive space---even in the Bible---David is a more significant symbol. Likewise the temple's symbolic significance far outweighs that of the \translit{bāmôt}, despite the fact that---functionally---their social function was similar.

But, what do we mean by ``significance?'' I would like to suggest that a particularly useful model for thinking about mnemonic significance is to consider social memory as a complex \emph{network} of meaning---as a graph with nodes and edges.%
    \footnote{Graphs and graph theory are a part of discrete mathematics and have a long and distinguished history going back to Euler. Although more recent applications of graph theory within sociology have focused on, for example, social networks on the internet, so-called social network analysis has been in use within sociology back to the early 19th century. See \cite[10--16]{linton2004}. Scale-free networks, in particular, are of interest to us. See \cite{barabasi_science2009}.}
In such a system each node represents a site of memory and the size or weight of that node is determined by the kinds of details, ideas, and themes that are remembered about that mnemonic site. In the case of the temple, one might argue that the detail with which the it is described in the Pentateuch as well as the themes of atonement and covenant provide considerable ``weight'' to the node within Israel's memory. But what make networks interesting, of course, are the connections that nodes make to one another. Those nodes which are more highly ``connected'' (those that have more edges linking them to other sites of memory) are more entangled with the entire symbolic system. Entangled nodes are more stable within the network and are less likely to be forgotten by virtue of the fact that they are defined with reference to more sites of memory. Severing one or two connections will not completely isolate the node from the rest of the graph. The inverse, then, is also true. Those sites which are less clearly situated within the graph are more susceptible to being forgotten.

One way to think about a symbol's significance within a social space, therefore, is to consider the size of the node and how ``connected'' the node is within the social network. Larger, more highly-connected sites of memory---those which for one reason or another have been connected to many other such sites within the social memory---may be viewed as more ``significant,'' while smaller sites with fewer connections are comparatively less significant with respect to social and cultural memory.%
    \footnote{Of course, when I say that a king is more ``significant'' than, say, a peasant, I am making an assessment of the social impact of the individual on the society broadly and not making a judgment of the intrinsic value or importance of the individual. Moreover, I am not saying that such significance ought to guide the historian. This is merely meant as a description of this particular social phenomenon.}
Larger, more highly connected nodes of meaning are not only more difficult to forget, but have the tendency to extend their connectedness to other highly significant nodes and furthermore to absorb lesser nodes and integrate their meaning.%
    \footnote{This is a property of scale-free networks more generally. When new new nodes make connections within the graph, they are not connected at random, but tend to connect to more highly connected nodes (so-called ``preferential attachment''). See \cite[412]{barabasi_science2009}.}
This process by which sites of memory attract one another is what Ehud Ben Zvi refers to as ``magnetism'' in memory. He writes:
\begin{quote}
    Of course, not all sites of memory draw the same attention in a group. the most prominent sites of memory are ``magnets'' for core meaning, ideas, and concepts and tend to evoke and deeply intertwine several of the groups's main metanarratives. Conversely , the more a site of memory can embody, intertwine and communicate several of these metanarratives, the more central the site of memory will become for the group.\autocite[73]{benzvi_st2017}
\end{quote}
The processes by which these major ``hubs'' of meaning continue to attract more and more connections and creates a kind of feedback loop wherein the ``rich get richer'' and the more significant sites of memory get more significant.\autocite[412]{barabasi_science2009}

One likely example of magnetism observable within the biblical text is the story of David and Goliath. Although the story of David and Goliath is very well known, is surprising to find that in 2 Sam 21:19b, someone else is credited with slaying Goliath:
\begin{hebrewtext}
    \versenum{2 Sam 21:19}
    וַתְּהִי־עוֹד הַמִּלְחָמָה בְּגוֹב עִם־פְּלִשְׁתִּים וַיַּךְ אֶלְחָנָן בֶּן־יַעְרֵי אֹרְגִים בֵּית הַלַּחְמִי אֵת גָּלְיָת הַגִּתִּי וְעֵץ חֲנִיתוֹ כִּמְנוֹר אֹרְגִים׃
\end{hebrewtext}
\begin{translation}
    \versenum{2 Sam 21:19}
    Then another battle came about in Gob with the Philistines and \emph{Elhanan son of Jaare-oregim the Bethlehemite struck down Goliath} the Gittite the shaft of whose spear was like a weaver's beam.
\end{translation}
\noindent
Of course, Elhanan is not remembered as the one who killed Goliath. The book of \chronicles, notably, explains this discrepancy by emending the text to say that Elhanan killed \emph{Lahmi} the \emph{brother} of Goliath.
\begin{hebrewtext}
    \versenum{1 Chr 20:5b}
    וַתְּהִי־עוֹד מִלְחָמָה אֶת־פְּלִשְׁתִּים וַיַּךְ אֶלְחָנָן בֶּן־יָעוּר [יָעִיר] אֶת־לַחְמִי אֲחִי גָּלְיָת הַגִּתִּי וְעֵץ חֲנִיתוֹ כִּמְנוֹר אֹרְגִים׃ 
\end{hebrewtext}
\begin{translation}
    \versenum{1 Chr 20:5b}
    Then another battle came about with the Philistines and Elhanan, son of Jair struck down Lahmi, the brother of Goliath, the Gittite, the shaft of whose spear was like a weaver's beam.
\end{translation}
\noindent
Thinking in terms of magnetism, we may suppose that the more significant and highly-connected mnemonic node (David) absorbed the comparatively poorly-connected node of Elhanan, who was relegated to the footnotes of Israelite history. 

In this particular case, it is also important to note that the extended narrative of David and Goliath from 1 Sam 17 likewise bears the signs of magnetism. In fact, the name ``Goliath'' only occurs twice in the extended narrative, in vv. 4 and 23; in every other instance throughout the narrative, the man is referred to simply as ``the Philistine.'' This fact has caused some scholars to question whether the identification of the ``the Philistine'' with Goliath was, like David, a secondary addition. Thus, it is supposed that both major characters in the story originally may have been anonymous, and only later were these figures identified with David and Goliath.%
    \footnote{For a fuller account of the textual issues surrounding the main narrative about David and Goliath, see \cite[280--309]{mccarter1980} and \cite[69--77]{mckenzie2000}.}
It is not difficult to speculate about why this convergence of character might have happened. David was remembered as a warrior (1~Sam 18:7; 29:5) who fought with and against Philistines (with: 1 Sam 27:1--28:2; against: 2~Sam5:17--25). Narratively, his character arc---even without the Goliath story---is one of humble beginnings and a meteoric rise by God's favor. All of these connections and narrative patterns lend themselves to attachment to the story of the humble Israelite boy (Elhanan?) who successfully slew the Philistine ``the shaft of whose spear was as like a weaver's beam.'' The identification of Goliath with ``the Philistine,'' too, is quite easy. Although the Philistine's stature is exaggerated in the Masoretic Text,%
    \footnote{The Masoretic text lists Goliath's height to be ``six cubits and a span'' (Heb. \hebrew{שֵׁשׁ אַמּוֹת וָזָרֶת}), or about three meters. This number is contested by both Josephus (\ant 6.171) at least two major Greek versions (Codex Vaticanus and the Lucianic texts), as well as \q{4}{Sam}{a}, which all read ``\emph{four} cubits and a span,'' or about two meters. Still quite large, but hardly a ``giant.'' See \cite[286]{mccarter1980}.}
presumably a man who could weird a spear whose size was remarkable would also be a large man.%
    \footnote{The reference to Goliath's spear shaft being the size of a weaver's beam in 1 Samuel 17:7 is commonly read as an insertion \emph{based on} 2~Sam 21:19. As such, it is a further example of how the more central mnemonic site is able incorporate lesser sites. It does not, however, help to explain the connection \emph{initially}.}
Moreover, since Gath was a Philistine city and, so far as anyone can tell ``Goliath'' is an authentically Philistine name,\autocite[291]{mccarter1980} it is not so difficult to identify the tall ``Philistine'' with Goliath.

The idea that the David ``absorbed'' the deeds of Elhanan into his own legacy is not new. Indeed, McCarter notes this in his commentary, writing: ``Deeds of obscure heroes tend to attach themselves to famous heroes.''\autocite[450]{mccarter1984} Although I am in agreement with McCarter, it is noteworthy that he makes this statement without providing any supporting rationale for \emph{why} this may have been the case or \emph{how} it came about. The ability to articulate social processes such as magnetism is one of the major contributions that memory studies offer to these kinds of discussions.

Thus when we consider the relationship between David and the temple in the book of \chronicles, it should come as no surprise that these two large, highly connected nodes within the cultural memory of Persian Yehud have continued to entangle themselves with the major metanarratives and ideas of the \chronicler's society. Indeed, we have already discussed at length the way that David's role in the construction of the temple expanded in the memory of the \chronicler \visavis Samuel--Kings. This expansion can be understood as a process of mnemonic magnetism whereby the most typologically significant and highly-connected node of memory about the remembered political kingdom of Israel (David) converges with the highly-connected and typologically significant center of Judahite identity during the \secondtemple period (the temple). 
