% !TEX root = dissertation.tex

\section{Memory, History, and the ``Actual Past''}

\halbwachs's did not see any reason to assume that the remembered past had any meaningful connection to the ``actual past.'' Because memory is always constructed in the present for use in the present, the ``actual past'' does not carry any meaningful influence on this (re)construction. It was in his \citetitle{halbwachs1941} that \halbwachs makes this case most forcefully, and I think he does so quite convincingly. \halbwachs's understanding of memory as a phenomenon of the present has thus earned him the label of ``presentist'' or ``constructivist'' over and against a number of more recent theorists who wish to attribute some normative force to the past.\autocite[27--30]{coser_halbwachs1992}  

\hypertarget{the-presentist-perspective}{%
\subsection{The Presentist Perspective}\label{the-presentist-perspective}}  

This presentist mantle has been donned by a number of more recent scholars, perhaps most notably by the German scholars Jan and Aleida Assmann.\autocites[See esp.][]{assmann_nikulin2015}{assmann2011}[and][]{a_assmann2011} Where \halbwachs distinguished between autobiographical and historical memory, Jan Assmann describes what he calls communicative and cultural memory (German: \emph{kommunikatives} and \emph{kulturelles} \emph{Gedächtnis}, respectively).\autocites[36]{assmann2011}[For a concise terminological crash-course, see][182--183]{hubenthal_carstens-hasselbalch2012} Rather than focus on the relationship of the rememberer to the experience (viz. whether the memory is ``autobiographical''), this terminology essentially distinguishes between synchronic and diachronic processes of memory. On the one hand, communicative memory represents a synchronic, or ``horizontal'' memory shared by a society at a particular chronological horizon based on direct communication between individuals. According to Assmann, this memory has a temporal horizon of 80--100 years---limited by spatial (where people are) and chronological (how long people live) factors. He writes:  

\begin{quote} A typical instance would be generational memory that accrues within the group, originating and disappearing with time or, to be more precise, with its carriers. Once those who embodied it have died, it gives way to a new memory.\autocite[36]{assmann2011} \end{quote}  

On the other hand, at the end of this crucial period, as particular memories begin to drop from current discourse and lose relevance; as those individuals with direct connections to the events, people, and places which the memories involve die off, the remembering community will either forget or transform the memory for long-term transmission in the form of \emph{cultural memory}. The canonization of memory at points during this period is a conscious, \emph{constructive} activity by a remembering group. \autocite[45]{assmann2011}  

Where \halbwachs's terminology took as its point of departure the psychological perspectives of Freud\autocite{terdiman_radstone-schwarz2011} and Bergson\autocite{ansellpearson_radstone-schwarz2011}, Assmann's taxonomy is rooted in ethnological research on oral traditions, specifically that of Jan Vansina and his notion of a ``floating gap'' between the present and the distant past.\autocite{vansina1985} Vansina observes that in oral cultures often there is an abundance of common knowledge about current goings-on and a similar abundance of shared knowledge about the distant past (esp.~with regard to origin stories and the like), but there often exists a gap for the not-so-distant past. The proportion of collective knowledge, therefore, is unevenly distributed between two chronological poles of memory, although members of the society in question may not perceive it as such in their own reconstructions of the past.\autocites[23--24]{vansina1985}[As Assmann, observes, ``In the cultural memory of a group, both levels of the past merge seamlessly into one another.''][35]{assmann2011} In other words, from the perspective of the remembering society, often there exists a continuity between the distant past (often legendary or mythic) and the near-past (a few generations, at most) where in reality a good deal of the not-so-recent past has fallen from memory. Memory of the near-past---those things which, while not necessarily ``autobiographical'' to every rememberer, nevertheless are reinforced by those with autobiographical memory---is categorized as ``communicative'' because it is memory that it generated and spread in the present by those with direct access to the events in question. Those memories which are deemed significant enough to not be forgotten---those which will make up cultural memory---undergo a process by which they are transformed from ``factual into remembered history,'' and may take the form of myth or legend.\autocite[37--38]{assmann2011} Thus, according to Assmann, myth and legend cannot be distinguished from ``history'' as a part of cultural memory. The significance of an event is not tied to whether or not it is ``factual,'' but by its ``truth'' seen through its continued relevance to the remembering community in the present.\autocite[Paul Veyne offers a particularly stimulating discussion of the perception of the past and its relationship to myth. He concludes his book with the insightful quote, ``The theme of this book was very simple. Merely by reading the title, anyone with the slightest historical background would immediately have answered, `but of course they believed in their myths!' We have simply wanted also to make clear that what is true of `them' is also true of ourselves and to bring out the implications of this primary truth.''][128--129]{veyne1988} Assmann writes:  

\begin{quote} Myth is foundational history that is narrated in order to illuminate the present from the standpoint of its origins. The Exodus, for instance, regardless of any historical accuracy, is the myth behind the foundation of Israel; thus it is celebrated at Pesach and thus it is part of the cultural memory of the Israelites. Through memory, history becomes myth. This does not make it unreal---on the contrary, this is what makes it real, in the sense that it becomes a lasting, normative, and formative power.\autocite[38]{assmann2011} \end{quote}  

Assmann's understanding of the relationship of the actual past to a society's cultural memory, therefore is not concerned with the discussion of the historicity of cultural memory. Although Assmann does not dismiss cultural memory as a source for historical inquiry, like \halbwachs, his primary interest is in exploring the constructive, presentist aspects of memory.  

\hypertarget{the-continuity-perspective}{%
\subsection{The Continuity Perspective}\label{the-continuity-perspective}}  

Critics of \halbwachs's presentist posture (and the similar approaches of Jan and Aleida Assmann) agree that memory is malleable but argue that there are constraints placed upon memory which mitigate unbounded fictionalization of the remembered past. This so-called ``continuity'' (or ``essentialist'') perspective---primarily associated with the American sociologist Barry Schwartz---insists that the ``actual'' past carries some normative force in the shaping of collective memory in addition to the ``received'' past.\autocites[Schwartz has made numerous contributions to the field of memory studies. See esp.][]{schwartz_sf1982}{schwartz_asr1991}[and][]{schwartz2000}[Note also the SBL volume specifically interacting with his work:][]{thatcher2014} Critiquing \halbwachs, Schwartz writes:  

\begin{quote} Unfortunately, this [\halbwachs's presentist] perspective has problems of its own. It promotes the idea that our conception of the past is entirely at the mercy of current conditions, that there is no objectivity in events, nothing in history which transcends the peculiarities of the present.\autocite[376]{schwartz_sf1982} \end{quote}  

At the heart of the so-called ``continuity'' approach is the conviction that while memories are always conditioned by the present, there is a limit to the amount of distortion acceptable to the remembering community. As Michael Schudson puts it, ``The past is in some respects and under some conditions, highly resistant to efforts to make it over.''\autocite[107]{schudson_communication1989}  

In fact, I think the conceptual distance between the presentist and continuity perspectives is not as great as it may initially appear. Neither \halbwachs nor Assmann assert that there \emph{cannot} be any historical value to cultural/collective memory, nor that such memory cannot be used for historiographical purposes. For example, in \citetitle{halbwachs1941}, \halbwachs takes seriously that the figure of Jesus \emph{did} exist as an historical person while making it clear that he does not accept the Gospels at face value as historically accurate (he explicitly compares his basic approach toward the historicity of the Gospels as similar to that of Ernst Renan).\cite[TODO: page]{coser_halbwachs1992} Throughout the work, \halbwachs does talk about the ``actual'' past and allows for the possibility that the Gospels do refer, at least partially, to real events. In other words, he does not make the argument that the Gospels were fabricated of whole-cloth and instead takes the position that the ``actual past'' is irretrievable and unknowable and that historical memory has no obligation to align with the actual past as such. On the other hand, Schwartz and the continuity perspective never argue that memory is \emph{accurate}, but instead that memory ought not be treated as \emph{entirely} arbitrary. In other words, the two perspectives both agree on the central premise that memory is shaped by society in the present, but they each approach the question of memory's connection to the actual past from opposite ends of the epistemological spectrum.  

This difference in perspective, I think, is attributable to the respective fields that Assmann and Schwartz deal with in their own research. As an Egyptologist dealing with literatures from the ancient Near East, Assmann necessarily is reliant on scant documentary evidence that may or may not have any supporting evidence whatsoever. The same can be said of other ancient fields such as biblical studies, Assyriology and Classics. Under these circumstances, the historian \emph{must} approach her sources with an appreciation for the intellectual gap that exists between the historian her source, particularly when not corroborated by an independent alternate source. On the other hand, Schwartz, as an Americanist, is able to marshal a plethora of contemporary sources for reconstructing the collective memory of the antebellum United States. What each scholar is able to assume about his sources speaks to their instincts toward the reliability of those sources. Furthermore, Schwartz deals with a comparatively disenchanted society whose historical consciousness more closely resembles our own, while Assmann deals with societies for whom myth and legend are not distinguished from history. Their historical methodologies may be the same, but the \emph{kinds} of sources that each field deals with creates a different set of scholarly instincts for dealing with the idea of memory and its relation to the actual past.  

Because this dissertation deals with the way that early Judaism interacted with its own received collective memory (rather than how it created those memories to begin with), I will tend to interact with the topic of collective memory from the perspective of \halbwachs and Assmann. This is not to say that I am entirely unsympathetic to Schwartz's critique of a purely presentist approach, only that the particulars of this project preclude the need to discuss the relationship between memory and the ``actual'' past.  