% !TEX root = dissertation.tex

\section{Memory and \RwB}

Having laid the theoretical foundation of modern memory studies, we may now turn our attention to the particulars of this study, namely, addressing the way that social memory studies can meaningfully augment the scholarly discussion surrounding \rwb.\autocite[See also][]{brooke_zsengeller2014}  

At this point it should be fairly obvious how the Hebrew Bible may be convincingly framed as both the product and progenitor of collective and cultural memory during the late Persian and Early Hellenistic periods. In Halbwachs's terms, the biblical text represents the common, collective memory of \secondtemple Judaism which formed the basis for Jewish collective identity as a people of the land which \yahweh promised to Abraham and into which \yahweh lead the people of Israel, ``with a might hand and an outstretched arm'' after their miraculous escape from the land of Egypt and subsequent desert wanderings. Bracketing any discussions of the historicity of these biblical narratives, by the late \secondtemple period they would have been perceived as the true and central foundation narratives to any number of Jewish groups both in and out of Persian Yehud and Roman Palestine. In Assmann's terms, the biblical texts---and in particular the stories of the patriarchs, Exodus, and Conquest narratives---carried ``a lasting, normative, and formative power,''\autocite[38]{assmann2011} which can be observed concretely by their preservation both in antiquity (e.g., at \qumran as well in translation) and into the modern era.  

The process of textual interpretation, therefore, is itself a mnemonic process. Just as memories are recalled into and shaped by a set of social frameworks which may be alien to their original context, so too the interpretation of texts and traditions is shaped by the social frameworks of the interpreter. Because any single text or narrative represents only a sliver of the thick nexus of ideas that is collective memory, not only is a text always read into new social circumstances, but it is always read into a new literary context and discursive arena. No two readings of a given text will every be the same. Each reading is affected by the collective memory of the reader(s) which is constantly adapting and in flux as new memories are added and others are forgotten.  

\RwB, therefore, can be understood as a set of snapshots revealing the ways that the collective memory of \secondtemple Judaism was being used within Jewish communities (or, at the very least within some scribal circles) to shape remembering communities' identities. This shaping, however, was not a passive process, but elicited creative, constructive participation to not only ``read'' the past, but to rewrite it as well. These texts themselves would have contributed to the collective memories of their respective groups. The disparate ways that \rwb texts were passed on or jettisoned from various religious groups in antiquity illustrate the ways that new memories can be added to a group's cultural memory and be adopted as a part of its historical self-understanding. The three texts which I will treat in this dissertation each meet a different outcome. Chronicles---which I have framed (loosely) as a rewriting of Samuel--Kings---was adopted by both Jews and Christians in antiquity as a part of their cultural memory and became a part of both traditions' canon of scripture. The \ga, on the other hand, seems to have not survived within Judaism beyond the first century CE (although, it may have impacted some later traditions). Finally, \jub was not retained in Jewish circles, but \emph{was} passed on within certain segments of early Christianity and remains in liturgical use by the Ethiopian Orthodox Tewahedo Church.%
%
\footnote{\cite{baynes_mason-etal2012}; \cite{asale_bt2016}.}

Simply labeling these \rwb texts as examples of social or cultural memory, however, is rather uncontroversial. Such an assertion hardly requires a dissertation-length study and the task has already been sufficiently accomplished, to my mind, by Brooke.\autocite{brooke_zsengeller2014} Thus, this dissertation will attempt to go beyond simply labeling \rwb texts as exemplars of memory and instead attempt to offer a description of the processes by which \rwb texts functioned within the collective memory of \secondtemple Judaism(s). Many of these processes already exist within the scholarly discourse surrounding \rwb. For example, from the perspective of textual production, the topics of biblical interpretation, inner-biblical exegesis, and scribal culture are not new to the topic of biblical or \qumran studies. But each can provide valuable insights into the ways that groups of individuals understood and recalled their cultural memory and what in particular they found most valuable about their cultural memories. Approaching \rwb through the lens of social memory studies attempts to take a step back and address their \emph{function} as the means by which \secondtemple Judaism experienced its past in its present. Social memory studies, therefore, is not an alternative to more traditional modes of analysis, but a complement. 

Memory studies, therefore, provide a rich toolbox for thinking about and addressing the kinds of \emph{functional} questions which we raised in the first chapter. Discussing the ``purpose'' or ``function'' of a text is tantamount to discussing how a text can be both the product of and and participant in collective memory of its society. In other words, framing \rwb texts within the discourse of social and cultural memory means treating \rwb texts as more than creative ``exegetical'' works but also as cultural products which participate in cultural structures and discourses with concerns other than the explication of sacred texts.