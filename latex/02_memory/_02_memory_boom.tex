% !TEX root = dissertation.tex

\section{The Memory Boom}

\halbwachs's work, while not ignored, would not make its most significant impact until well after his death. It is frequently argued that the so-called ``memory boom,'' which began in the 1980's in the wake of the ``theory boom,'' picked up \halbwachs's terminology and central ideas in an attempt to deal with the perceived insufficiency of traditional historiography to deal with the sorts of major, traumatic events which characterized the mid 20th Century.%
%
\footnote{\cite[1--2]{galinsky_galinsky2016}. See also \cite[29--36]{olick_olick-etal2011}. One cannot help but speculate that---at least in the English-speaking world---the translation of \emph{The Collective Memory} in 1980 contributed to the popularity of \halbwachs's terminology.}
%
Works such as Yosef \yerushalmi's \citetitle{yerushalmi1989} and Pierre Nora's \emph{Les Lieux de mémoire} are typically cited as the foundational works of the modern memory boom.\autocites[112--113]{klein2011}{yerushalmi1989}[Nora's massive project has been abridged and translated into English as][]{nora1996}  

In \citetitle{yerushalmi1989}, \yerushalmi is quick to identify the tension between what traditional cultures and societies remember about their past and how the discipline of history treats the past. For remembering groups, what is preserved in the collective memory is what is useful for the edification of that group---whether through religious ritual, family stories, or some other combination of received traditions. Of course, prior to the enlightenment, this was the default mode of understanding the past for most people, and remains so for many social groups, including those within modern, Western societies. In particular, \yerushalmi addresses this tension for the Jewish historian---a vocation which, he notes, is a recent phenomenon.\autocite[81--103]{yerushalmi1989} Although, ancient Israel and Judah, clearly, were concerned with matters of the perceived past---much of the Hebrew Bible is preoccupied with narrating events from the perceived past---these codified traditions are preserved in a plurality of socio-religious groups for a complex set of purposes spanning cultural, social, and theological modes of discourse which are fundamentally at odds with the discipline of history.%
%
\footnote{I am aware of the problematic nature of this statement. Contemporary approaches to historiography are emphatically \emph{not} attempting to write ``objective history.'' Yet, referring only to ``Modernist'' historiography does not give due consideration to the fact that common discourse around the idea of ``history'' is largely influenced by Modernism. Even taking into account that contemporary historiography has moved beyond discussions of ``objectivity'' the methodological underpinning of historical discourse remains fundamentally different, if only by the existence of its own meta-discourse. As Daniel Pioske puts it, ``from the recounting of a culture's sanctioned memories is consequently the historian's determination to isolate and compare disparate testimonies about the past with other past traces that may corroborate or discredit their claims.'' \cite[12--13]{pioske_bibint2015}.}
%
Thus, the biblical command to ``remember,'' is not a command to keep tedious notes of historically accurate events, but a cultural and theological imperative to maintain the foundational narratives of the community. \yerushalmi writes:  

\begin{quote} There the fact that history has meaning does not mean that everything that happened in history is meaningful or worthy of recollection. Of Manasseh of Judah, a powerful king, who reigned for fifty-five years in Jerusalem, we hear only that ``he did what was evil in the sight of the Lord'' (II Kings 21:2).\autocite[10]{yerushalmi1989} \end{quote}  

In other words, what was remembered about Manasseh by the biblical tradants were those details which were useful for their socio-religious projects. The rules and methods of this process---remembering what is important and forgetting what is not---are generally not explicit or transparent.  

The discipline of history, on the other hand, generally attempts to uphold a certain set of explicit methodological and theoretical criteria which---while not exempt from distortion by the subjectivity of the historian---can be corroborated or contradicted by evidence and argumentation.\autocite[12--13]{pioske_bibint2015} While the historian participates in the collective memory of her own society, her reconstruction of the past attempts to approach the topic from the outside. The historian, too, (re)constructs the past, but the goals of the historian are, as \yerushalmi puts it, to recreate ``an ever more detailed past whose shapes and textures memory does not recognize.''\autocites[94]{yerushalmi1989}[See also][532]{verovsek_pgi2016} Even the most theory-conscious historian cannot help but struggle in avoiding older discourses about ``what really happened,'' particularly when stated over and against memory in the form of received tradition. All of this is not to say that modern history writing is in any meaningful sense ``objective,'' nor that the historian is able to remove herself from her own socio-political context. So, although memory and history both offer reconstructions of the past, it is important to affirm that their modes of doing so are radically different and for different purposes.%
%
\footnote{See esp. \cite[497]{ricoeur2004}. As Pioske observes, ``The epistemological tension observed by Ricoeur between memory and history is thus understood as the outcome of two processes that, though having the similar intent of re-presenting former phenomena, nevertheless pursue and mediate the past through quite disparate means.'' \cite[12]{pioske_bibint2015}.}

Thus the memory ``boom'' has, in some circles, been viewed as anti-historical and an attempt at ``resacrilization of the past'' to counter the disenchantment brought about by modern historical consciousness.\autocite[282]{winter2006} Kerwin Klein, for example, traces the origins of scholarly interest surrounding memory and lists five narratives that others have offered as explanations for the origins of memory discourse in society generally:  

\begin{quote} We have, then, several alternative narratives of the origins of our new memory discourse. The first, following Pierre Nora, holds that we are obsessed with memory because we have destroyed it with historical consciousness. A second holds that memory is a new category of experience that grew out of the modernist crisis of the self in the nineteenth century and then gradually evolved into our current usage. A third sketches a tale in which Hegelian historicism took up pre-modern forms of memory that we have since modified through structural vocabularies. A fourth implies that memory is a mode of discourse natural to people without history, and so its emergence is a salutary feature of decolonization. And a fifth claims that memory talk is a belated response to the wounds of modernity.\autocite[134]{klein_klein2011} \end{quote}  

Although Klein finds none of these ``fully satisfying,'' it is noteworthy that the general trend among these narratives corroborates the thesis that memory represents a ``reaction'' against history in some form.  

Whatever combination of these causes may have ultimately brought about the memory boom, the problem remains, according to Klein, that memory has come to dominate historical discourse as a ``therapeutic alternative'' to history in place of a rigorous scientific methodology.\autocite[137]{klein2011} As Winter puts it, ``It is a fix for those who cannot stand the harshness of critical thinking or historical analysis.''\autocite[283 (summarizing Klein)]{winter2006} Although I think Klein under-appreciates the value of the memory discourse as a meaningful mode of inquiry, I am in fact, quite sympathetic to his critique overall. As methodologies for querying the past, memory and history operate on different sets of hermeneutical and epistemological foundations, which is, I think, one of \yerushalmi's main points. 
%
% TODO: See Keith's long note on p. 23
%
However, what Klein does not address is the way that, for modern Westerners, history \emph{is} our collective memory (or at least, heavily influences our collective memory). This is what Nora means when he says that ``We speak so much of memory because there is so little of it left.''\autocite[7]{nora_representations1989} And for Klein, this is a good thing---historical consciousness is uniquely valuable as a scientific endeavor and jettisoning this critical posture toward the past is tantamount to abandoning the enlightenment.  

For modern historians studying the cultural memory of other modern people, it is easy to conflate the historical consciousness of the historian subject and that of the object. Such historical work relies on court documents, news articles, eyewitness accounts, and other documentary evidence that operates within an historical consciousness that closely resembles that of the historian. As a result, the historian can utilize her own historical intuitions when interacting with her sources. In \halbwachs's terms, the social frameworks (in this case the understanding of the way ``history'' is done) of the historian and their object of study are quite similar. For example, reading news reports from the mid-twentieth century does not require the historian to dramatically reorient her understanding of what ``news'' is. On the other hand, when studying ancient history, the intellectual distance between the source and the historian is, often, much more pronounced. Reading ``historical'' texts from antiquity often requires a kind of hermeneutical suspicion that is different from that used by scholars reading texts from the recent past.%
%
\footnote{For example competent readers of modern newspapers know to bring a different set of suspicions to ``news'' articles versus editorials. Similarly, historians can read personal correspondence with a different kind of suspicion than monumental inscriptions.}

In fact, biblical scholars in particular have been dealing with this problem since the enlightenment. The tension between memory and history is played out clearly in both modern Jewish and Christian circles \visavis historical-critical study of the Bible. Insofar as the Bible forms a major portion of both Jewish and Christian collective memory, historical-critical approaches to the biblical text continue to be met with fervent opposition in more conservative traditions. Parallels to what Klein describes within the discipline of history can be seen within biblical studies as well. Consider, for example, the way that Brevard Childs's canonical approach attempted to ``overcome the long-established tension between the canon and criticism.''\autocite[45]{childs1979} For Childs, writing an introduction to the Old Testament in the traditional manner (i.e., as an historical-critical introduction) was insufficient for use in churches or synagogues because it bypassed a fundamental aspect of the biblical text, the canon. Although he does not use the language of memory in his discussion of canon (though, it should be noted he made an important early contribution to the idea of memory in the biblical tradition which, I imagine, is not a coincidence\autocite{childs1962}), here we can see that the various canons of scripture in use by Christians and Jews throughout the world nevertheless function as a form of collective memory by constructing and filtering what should and should not be remembered by the community.  

The tension between history and memory is most problematic---as evidenced by Childs---when the historian participates within the collective memory of the community under investigation. This is why both Childs and \yerushalmi express their discomfort and dissatisfaction while attempting to operate with one foot in each world. This is the central critique of Klein: historians operate from the outside looking in (an etic approach), while practitioners of memory operate from within (an emic approach). Yet, this etic/emic distinction only makes sense when memory is placed on equal footing with history as a means of interrogating the past. From this perspective, I wholeheartedly agree with Klein that such an approach undercuts the epistemological foundations of modern historical inquiry. However, Klein does not address memory as the \emph{object} of historical study. I think this is what makes \yerushalmi's approach so intriguing. Although he acknowledges his precarious position as a Jewish historian, \yerushalmi discusses memory \emph{as an historian} and it is this approach which I think is the most fruitful avenue of memory research. Thus, this dissertation will treat memory as a phenomenon which can be studied historically rather than as a source of information about the past.  