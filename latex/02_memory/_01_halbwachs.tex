% !TeX root = ../dissertation.tex

\section{The Work of \Halbwachs: A Very Brief Overview}

Although the topic of memory has been of interest to philosophers and thinkers since the antiquity,\autocite{carruthers_radstone-schwarz2011} as Olick and Robbins note, modern social-scientific approaches (which concern this study) almost exclusively trace their genealogy to the early 20th century work of sociologist \Halbwachs.%
%
\footnote{\cites[106]{olick-robbins_ars1998}. It should be noted, however, that \halbwachs was not the first or only person to do work on memory or the impact of social structures on memory. See \cite[8--36]{olick_olick-etal2011}.}
%
Although \halbwachs's scholarly contributions were not limited to the topic of social memory (he also made contributions to statistics and probability theory, as well as sociological work on the topic of suicide and social morphology), the influence of his work in this area not only made a more lasting impact on the field of sociology than his other contributions, but it has also made a profound impact in a number of other fields such as history, anthropology, and biblical studies.\autocite[13--20]{coser_halbwachs1992}

\halbwachs published three primary works on the topic of memory, the first of which appeared in 1925 under the title \citetitle*{halbwachs1925}.\autocite{halbwachs1925} This monograph, along with the concluding chapter of his second monograph---dealing with the remembered geography of the Holy Land---was excerpted and translated by Lewis Coser in a single volume under the title \emph{On Collective Memory} in 1992.%
%
\footnote{Several of the most important chapters of \citetitle{halbwachs1925} were included in full. Likewise, the entirety of the conclusion of \citetitle{halbwachs1941} was included. \cites{halbwachs1992}; \cite{halbwachs1941}.}
%
His third and final contribution, entitled \citetitle*{halbwachs1980}, was published posthumously in 1950 and was translated into English in 1980 with an editorial introduction by Mary Douglas.\autocite{halbwachs1980} This work simultaneously represents some of \halbwachs's most developed ideas (responding to critics such as Charles Blondel) while evincing an incompleteness which posthumous publications often suffer.%
%
\footnote{%FIXME: Figure out why no Capital Ibid here.
\Cite{halbwachs1980}. As Coser observes, ``One may doubt that the author himself would have been willing to publish it in what seems to be an unfinished state. The book nevertheless contains many further developments of \halbwachs's thought in regard to such matters as the relation of space and time to collective memory as well as fruitful definitions and applications of the differences between individual, collective, and historical memory.'' \cite[2]{coser_halbwachs1992}.}

The central contribution of \halbwachs's work was the notion that human memory is intrinsically and inextricably tied to social frameworks.\autocite[37--38]{halbwachs1992} Humans are social beings and as such human activities, such as memory, are only usable within the context of a society. This focus on the \emph{social} dimensions of memory betrays the deep influence that Émile Durkheim's work had on \halbwachs.\autocite[8--9]{coser_halbwachs1992} Unlike Durkheim, however, \halbwachs's approach was tempered by his desire to identify the physical location of memory within to be within the \emph{individual}.%
%
\footnote{To clarify, \halbwachs was not at all interested in the biological processes or locating within the brain where memory is stored, only that ``collective memory'' \emph{is} stored biologically but, more importantly, is socially conditioned.}
%
Although the term ``collective memory'' evokes an ethereal or metaphysical idea, \halbwachs's use of the term was intended to ensure that any discussion of memory remained in the concrete. Collective memory is the sum total of those memories kept by \emph{individuals} withing a society. An individual's ability to retrieve and utilize a particular memory, however, is inextricably entangled with the individual's social context. According to \halbwachs memories require social frameworks to function.\autocite[38]{halbwachs1992}  

To illustrate this point, \halbwachs begins \citetitle{halbwachs1925} by attempting to prove the negative. \emph{Without} a social framework, he argues, memories are always incomplete. Because humans---for all intents and purposes---always exist within a society, it is the dream state that most closely approximates the complete isolation of memory from society.\autocite[41--42]{halbwachs1992} Therefore, the way that the human brain deals with memories while dreaming can illustrate the (dis)function of memories lacking a social framework. Thus, he observes that ``dreams are composed of fragments of memory too mutilated and mixed up with others to allow us to recognize them.''\autocite[41]{halbwachs1992} Because the mind lacks the ability to ``check'' itself against anything external while in a dream state, dreams do not contain ``true memories.''\autocite[41]{halbwachs1992} This assertion is set against the ``purely individual psychology'' of Bergson and Freud which viewed \emph{memory} as a location of social isolation.%
%
\footnote{See \cite{ansellpearson_radstone-schwarz2011} and \cite{terdiman_radstone-schwarz2011}.}
%
Regarding the incompleteness of the dream state, he writes:  

\begin{quote}
    Almost completely detached from the system of social representations, {[}the dream state's{]} images are nothing more than raw materials, capable of entering into all sorts of combinations. They establish only random relations among each other---relations based on the disordered play of corporal modification.\autocite[42]{halbwachs1992}
\end{quote}  

The ``system of social representations'' that \halbwachs refers to is not limited, however, to macro structures such as familial, religious, or class groups. Although these structures certainly \emph{do} make up an important stratum of social frameworks, \halbwachs envisions something much more fundamental which betrays his broadly structuralist perspective. \halbwachs uses the phrase ``social representations'' to refer to a system of shared ``signs'' that encompassed not only these macro structures, but every aspect of a group's social framework---a sort of ``cultural \emph{langue}.'' Although, \halbwachs does not use the language of semiotics, the analogy is helpful. Just as Saussurian semiotics argues that the concrete arbitrary sign is given meaning only by participating in the broader, shared \emph{langue}, so too memories (read: ``signs'') require a framework to convey meaning, as do the concrete, individual expressions of remembrance (read: \emph{parole}).%
%
\footnote{This is my terminology with reference to \halbwachs. Of course, it is borrowed from Saussure. See \cite*{saussure1916}. For a brief overview of these terms see \cite[93--94]{smith-riley2009}.}
%
\halbwachs writes:  

\begin{quote}
    It is in this sense that there exists a collective memory and social frameworks for memory; it is to the degree that our individual thought places itself in these frameworks and participates in this memory that it is capable of the act of recollection.\autocite[38]{halbwachs1992}
\end{quote}  

Memories, therefore, cannot be understood in isolation from their social frameworks and therefore should not be analyzed without consideration to the social context of the rememberer.  

Of course, people participate in a plurality of social contexts, often simultaneously, and the experiences that are later to be recalled, too, must be situated within these frameworks. In order to bring these autobiographical memories to mind, according to \halbwachs, an individual attempts to mentally situate herself within those same frameworks.\autocite[38]{halbwachs1992} For instance, I find it much easier to recall whether a particular university course I have taken occurred in the Fall or Spring semester, rather than which month or even year it occurred. The social framework that is the ``academic year'' remains a potent framework for my own memories; I imagine the ``year'' beginning in the Fall, and often refer to ``next semester.'' On the other hand, my wife---who had the good sense to stop her formal education after one degree---no longer thinks in terms of semesters. Yet, when remembering events during her time at university, the semester once again becomes a useful framework for memory. It is for this reason that recent memories are more easy to call to mind: because the social frameworks that produced the memory (the people, places, customs, etc.) remain in close proximity for the rememberer and the effort required to situate the memory within the social frameworks that produced it is minimal.\autocite[52]{halbwachs1992} This notion is a central part of \halbwachs's thesis and provides a point of departure for his more in-depth studies of collective memory in the family, religion, and social classes.  

\subsection{A Note on \halbwachs's Terminology}

There is a grand tradition of imprecise and overlapping terminologies within memory studies going back to \halbwachs himself. For example, on page 40 of \emph{On Collective Memory}, \halbwachs uses each of the terms ``collective memory,'' ``social memory,'' ``social frameworks of memory,'' and ``collective frameworks of memory'' and it is not entirely clear how \halbwachs is distinguishing between them. The way that he is able to use the terms almost interchangeably has led some in the current discussion to treat them as synonyms. As Anthony Le Donne observes, ``In fact, {[}`social' and `collective' memory{]} are currently used synonymously with such frequency that their nuances vary from author to author.''\autocite[42 n.8]{ledonne2009} Yet, Le Donne points out, \halbwachs actually uses these terms with slightly different nuances. On the one hand, \halbwachs uses the term ``social'' memory when he is describing the way social structures affect memory, while on the other hand ``collective'' memory tends to refer to the content of memories which are transmitted between individuals and common to those of a particular group.  

In other words, when \halbwachs uses the term ``social'' memory, he is referring to the social frameworks in which individual memory participates, i.e.~how society provides the framework that makes individual memory possible.\autocite[180]{hubenthal_carstens-hasselbalch2012} On the other hand, when he uses the term ``collective'' memory, he tends to refer to shared memory, ``the shared cultural past to which individuals contribute and upon which they call; but ultimately a past that transcends individual memory.''%
%
\footnote{\cite[360]{keith_ec2015}. See also \cite[180]{hubenthal_carstens-hasselbalch2012}.}
%
The two ideas work together and mutually influence one another. As Hübenthal puts it, ``The difference [between social and collective memory] lies in the perspective: \emph{social memory} is using the framework, \emph{collective memory} is establishing it.''\autocite[180.]{hubenthal_carstens-hasselbalch2012} Hübenthal's use of the active verb ``establish'' is intentional: for \halbwachs, collective memory is not a passive social accretion, but an actively constructed part of the group's common identity which \emph{speaks to the concerns and needs of the community in the present}. Social frameworks shape the way that people remember. The retrieval of memories is shaped by those same frameworks, and as those frameworks shift, so too do the memories that are recalled in those societies.%
%
\footnote{For a modern assessments on the malleability of human memory and the effects of social networks on the formation of collective memory, see \cite{coman-etal_pnas2016}; \cite{yamashiro-hirst_jarmc2014}; \cite{coman-etal_yang-etal2012}.}
%

In his later work, \halbwachs distinguishes between two kinds of memory which can be identified by the experiential-relation of the rememberer to the object of memory: autobiographical and historical memory.\autocite[52]{halbwachs1980} Autobiographical memory refers to the sort of memories which are the result of individual, subjective experience, while historical memory refers to those which fall outside the experience of the individual. Elsewhere \halbwachs refers to these as ``internal'' and ``external'' memory. Autobiographical memory is rooted in the individual, sensory experiences which provide a full, ``thick'' memory---to borrow from Geertz\footnote{\cite[3--30]{geertz1973}. See also \cite[189--192]{smith-riley2009}.}--- while historical memory offers only a thin, schematic overview and by definition is never ``experienced'' by the rememberer.  

Although \halbwachs distinguished between these two forms of memory, he nevertheless emphasized their interrelatedness. In particular, \halbwachs notes that autobiographical memory necessarily is dependent upon historical memory, insofar as our lives participate in ``general history.''\autocite[52]{halbwachs1980} For example, memories of a more indirect nature are able to shape autobiographical memory by shaping the social frameworks which produced them and the frameworks into which they are recalled. The quintessential example for Americans of my age would be the events of September 11, 2001. Although comparatively few people directly witnessed the events (I was asleep on the West Coast when the first plane crashed), the impact that those events had (and continue to have) on the orientation of American national memory is unquestionably a part of many people's lived experience, including my own and would therefore constitute a part of America's current ``collective memory.'' Although the incoming undergraduate class at the University of Texas at Austin, many of whom will have been born after 2001, have \emph{no} autobiographical memory of these events, it is, nonetheless, a part of the collective memory of their society at large. On the other hand the War of 1812 is not a part of any living person's autobiographical memory and its impact on the collective memory of most Americans is likely restricted to a few popular media references, or localized to specific geographical regions with a close connection to major events in the conflict (e.g., New Orleans).\footnote{Such as Jimmy Driftwood's \emph{The Battle of New   Orleans}, best known as performed by Johnny Horton which topped   \emph{Billboard} charts in the US, Canada, and Australia in 1959 and   was recently acknowledged to be one of the Top 100 Western songs of   all time. See,   https://en.wikipedia.org/wiki/The\_Battle\_of\_New\_Orleans.}  

The memories of historical events, likewise, are shaped by the social frameworks of the rememberer. The events of September 11, 2001 in the memory of most Americans are now further shaped by the socio-political discourses surrounding the United States' continued military presence in the Middle East and its controversial pretexts for engagement in the region, especially with the invasion of Iraq in 2003. Likewise, although no living person has an autobiographical memory of the American Civil War, the construction of certain confederate monuments on the campus of the University of Texas at Austin during the Jim Crow era, and their subsequent removal in August of 2017, illustrates how historical memory can be (consciously, in this case) reshaped and restructured as the remembering society changes.\footnote{See https://www.nytimes.com/2017/08/21/us/texas-austin-confederate-statues.html.}  

It is the way that these remembered events change over time that makes social memory studies so interesting for the historian. \halbwachs's own work in the area of history is best seen in \emph{The Legendary Topography of the Holy Land}, where he focuses on the ways that memories relate to particular geographic sites. Notably, \halbwachs is not interested in ``doing'' history. Rather, \halbwachs's study focuses on the way that the geographical sites in and around the Galilee and Jerusalem were imbued with significance based on their putative connection with significant events related to Jesus, the Apostles, and early Christian communities. 

\halbwachs makes a number of observations about the way that memories are formed and the ways that they interact. His first observation comes in contrasting the portrayal of Jesus within the Gospels with what must have been the lived experience of the Apostles.\autocite[193--198]{halbwachs1992} The involvement of the apostles in the day-to-day life of Jesus in some sense would have prohibited them from achieving the kind of ``necessary detachment'' to write something like the the Gospels. In other words (and to use \halbwachs's later terminology), the memory of Jesus as portrayed in the Gospels is almost necessarily informed by \emph{historical} rather than autobiographical memory.\autocite[194]{halbwachs1992} Indeed, \halbwachs rightly observes that the Gospels present Jesus and his ministry ``as if Jesus's whole life was but a preparation for his death, as if this was what he had announced in advance.''\autocite[198]{halbwachs1992} Although the religious significance of Jesus's death continues to be remembered as a central component of Christianity, surely Jesus's mother remembered the death of her son differently than the way the Church later commemorated it.%
%
\footnote{Regardless of whether \halbwachs's conception of Early Christianity would be considered sound today, the idea that the Gospels represent several collective remembrances of Jesus's life, ministry and death each bearing marks from their own \emph{Sitz im Leben} (to borrow from the form critics) seems relatively uncontroversial. A number of studies on the Jesus and early Christian memory have come about in the past several years. See \cite{ledonne2009}; \cite{rodriguez2010}. For an overview of the modern impact of \halbwachs (and memory studies more generally) on the field of Historical Jesus studies, see \cite{keith_ec2015} and \cite{keith_ec2015b}.}

\halbwachs, drawing on the Pauline epistles, observes that the earliest recollections of Jesus make no mention of the location of his death (Jerusalem) nor of his ministry (Galilee). He writes:  

\begin{quote}
    In the authentic epistles of Paul, we are told only that the son of God has come to earth, that he died for our sins, and that he was brought back to life again. There is no allusion to the circumstances of his life, except for the Lord's Supper, which, Paul says, appeared to him in a vision (and not through witnesses). There is no indication of locality, no question of Galilee, or of the preachings of Jesus on the shores of the lake of Gennesaret.\autocite[209]{halbwachs1992}\footnote{Notably, the only Jesus scholar with whom \halbwachs interacts is Ernest Renan, a figure whose work has survived mostly as a punching-bag for later scholars and as an example of overt anti-Semitism in biblical scholarship. See \cite[39]{heschel2008}}
\end{quote}  

\halbwachs's point is that within the narrative of the Gospels, the location of Jesus's death---by virtue of the social and political reality of the day---\emph{had} to occur in Jerusalem.\autocite[211]{halbwachs1992} Whether or not it actually did, or whether or not that information was explicitly handed down to the authors of the Gospels is irrelevant for the purposes of collective memory. Sacred places become sacred through the process of memory \emph{construction}, not simply through the transmission of autobiographical experience. They are spaces where significant ideas within the collective memory of a group can take concrete form. He writes, ``Sacred places thus commemorate not facts certified by contemporary witnesses but rather beliefs born perhaps not far from these places and strengthened by taking root in this environment.''\autocite[199]{halbwachs1992} Localizing historical memory, therefore, functions as a way to move abstract ideas into the real world and reinforce fundamental components of the group's collective memory.  

Perhaps more interesting is \halbwachs's treatment of the ability for memories to coalesce and split over time. \halbwachs makes the observation that, according to tradition (i.e., the collective memory of the Church), certain places in the Holy Land mark the location of \emph{several} significant events. From an historical perspective \halbwachs, obviously, doubts that these assertions are accurate---even assuming the events indeed occurred at all---but finds the clustering of these events to be more than just coincidence. For example, he writes:  

\begin{quote}
One is surprised to find on the shores of the lake Gennesaret, near the Seven Fountains, the place where apostles were chosen, the Sermon on the Mount, the appearance of Jesus on the waters after the Resurrection---all in the same place.\autocite[220]{halbwachs1992}
\end{quote}  

\halbwachs's assumption is that there was something about the location \emph{itself}, some ``earlier consecration,''\autocite[220]{halbwachs1992} which attracted these memories to particular locales. Extending this rationale further, we can appreciate the fact that for Christianity, the significance of Jerusalem is not limited to the significance of the city as the location of Jesus's death, but rather by the prior significance of the city for Judaism.%
%
\footnote{In addition, Jerusalem was the location of the leadership for the earliest Christian Church according to Gal 1:18 (and attested throughout Acts).}
%
Within the collective memory of Christian tradition, one might say that Jerusalem is not significant because it is the location of the Passion and resurrection of Jesus, but that the Passion and resurrection of Jesus happened in Jerusalem \emph{because Jerusalem was significant}. \halbwachs writes:  

\begin{quote}
The Christian collective memory could annex a part of Jewish collective memory only by appropriating part of the latter's local remembrance while at the same time transforming its entire perspective of historical space.\autocite[215]{halbwachs1992}%
%
\footnote{Because the earliest Christians were Jewish, it stands to reason that the collective memory of \emph{earliest} Christianity was rooted in broader Jewish memory. In later periods---especially during and after the so-called parting of the ways (however problematic this term has become)---\halbwachs is certainly correct. Regarding the current discussion on the Jewish--Christian schism see \cite[19--60]{burns2016}.}
\end{quote}  

One might object to this suggestion by noting that, supposing Jesus \emph{actually was} crucified in Jerusalem, one hardly needs to re-appropriate Jewish tradition or attribute this remembrance to some special process. Yet, it is worth pointing out in cases where the historical data are lacking (or, perhaps, where eyewitness accounts certainly did not exist), this same basic phenomenon occurred. For example, \halbwachs points to the birth narratives of the Gospels, in particular that of Matthew, where Jesus is described as being born as a descendant of David in the town of Bethlehem (Matt 1:20; 2:1). Although there is no reason to think that Jesus was \emph{actually} born in Bethlehem, \halbwachs rightly observes, ``the authors of the gospels seem entirely to have invented this poetic history which has occupied a considerable place in Christian History.''\autocite[214]{halbwachs1992} In fact, Jesus's entire portrayal in the Gospel of Matthew is an exercise in collective remembrance which is structured on the foundational narratives of the Hebrew Bible: the slaughter of innocents (Matt 2:16--17), and Jesus's portrayal as a lawgiver ``on the mount''(Matt 5:1--7:29), and framed as the fulfillment of Jewish prophecy (Matt 1:23; 2:6, 18 \emph{et passim}). 

The inverse of this phenomenon is also observable. According to \halbwachs while some events converge to particular locations, other events diverge among several sites. One expression of this process is the way that significant events are themselves divided providing the opportunity for each portion of the event to be separately localized. For example, \halbwachs notes how the memory of specific important events, such as the Passion, may be split and localized at a very fine level of detail:  

\begin{quote} Around Golgotha and the Holy Sepulcher, for example, we find the rock of anointing, the rock of the angel, the rock of the gardener, the place where Jesus was stripped, etc.\autocite[220]{halbwachs1992} \end{quote}  

The proliferation of these micro-sites of memory, according to \halbwachs, aide and reinforce the collective memory through repetition. Furthermore, the added detail serves in ``renewing and rejuvenating an ancient image.''\autocite[220]{halbwachs1992}  

%% TODO: Get ancient source citations
The same event may also be localized in multiple places. \halbwachs describes several traditional locations of the Cenacle (the ``Upper Room'' from the Gospels), including the Mount of Olives, Gethsemane, and the Grotto of Jesus's teaching. These traditions coexisted into the fourth century, yet, later, the site was moved to the Christian hill of Zion. Likewise, \halbwachs notes that there were two locations for Emmaus and two different mountains on which Jesus is said to have appeared in Galilee after his resurrection. While it runs counter to conventional modern conceptions of the past, that seemingly contradictory traditions are able to coexist within a society---or even within the memory of a single individual---is well documented.
% TODO: get citation.
\halbwachs points out that autobiographical memory, however, does not allow for this kind of fragmentation.
% TODO: get citation.
%
We all realize that the same event from our own past can not have happened in two locations simultaneously. Yet, \halbwachs points out that should that same person belong to two groups who disagree on a particular remembered event from history (one that the individual did not personally witness), individuals are generally able to hold such memories together (if in tension) without the need assert one or the other. The same is true of complex social entities such as religious groups who are themselves composed of smaller sects which may possess their own unique collective memory. \halbwachs writes:  

\begin{quote} A community must often accommodate itself to contradictions introduced by diverse groups so long as none of these groups prevails, or so long as the community itself does not find a new reason for decisively settling the issue. This is especially true when the community faces a controversy about its rites, which are an anchor for its component groups.\autocite[224]{halbwachs1992} \end{quote} 

%% TODO: Some kind of summary here?