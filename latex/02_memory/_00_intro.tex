% !TeX root = ../dissertation.tex

%% TODO: Intro connecting to rwb discussion

Over the past several decades, a dramatic increase in scholarly interest toward the topic of ``memory'' has swept throughout the social sciences and humanities. The precipitous rise in scholarly literature dealing with topics of memory coupled with its proliferation in popular media discourses has prompted some in the field to refer to a ``memory industry'' and to describe the ubiquity of memory discourses as a ``boom'' fast-approaching a bust.%
%
\footnote{\cite{rosenfeld_jmh2009}; \cite{winter2006}; \cite{berliner_aq2005}; \cite{confino_ahr1997}.}
%
Yet, as Olick et al.~make clear in their Introduction to \emph{The Collective Memory Reader}, there remain a significant number of scholars throughout the social sciences and humanities who continue to find memory to be a useful heuristic and a compelling theoretical basis for their various and sundry analytical applications.\autocite[3--6]{olick_olick-etal2011}  

Although the topic of memory has been of interest to philosophers and thinkers since the antiquity,\autocite{carruthers_radstone-schwarz2011} as Olick and Robbins note, modern social-scientific approaches (which concern this study) almost exclusively trace their genealogy to the early 20th century work of sociologist \Halbwachs.%
%
\footnote{\cites[106]{olick-robbins_ars1998}. It should be noted, however, that \halbwachs was not the first or only person to do work on memory or the impact of social structures on memory. See \cite[8--36]{olick_olick-etal2011}.}
%
Although \halbwachs's scholarly contributions were not limited to the topic of social memory (he also made contributions to statistics and probability theory, as well as sociological work on the topic of suicide and social morphology), the influence of his work in this area not only made a more lasting impact on the field of sociology than his other contributions, but it has also made a profound impact in a number of other fields such as history, anthropology, and biblical studies.\autocite[13--20]{coser_halbwachs1992} 

%% TODO: I feel like you need a paragraph that outlines the relevance of your overview for Rewritten Bible at this point