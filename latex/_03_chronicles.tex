% !TEX root = dissertation.tex

\chapter{Chronicles}
\label{chap:chronicles}

Scholars of the Hebrew Bible have long held that the book of \chronicles is a derivative work of Samuel--Kings reflecting the concerns and religious sensibilities of the late Persian or early Hellenistic Periods.%
    \footnote{The observation that \chronicles drew from Samuel--Kings was made as early as de Wette in the early nineteenth century in his \cite*{dewette1806}. See also the work of McKenzie
        \cite*{mckenzie1985};
        \cite{mckenzie_graham-mckenzie1999};
        \cite[66--71]{knoppers2003}; and
        \cite[30--42]{klein2006} as well as that of
        \cite[74--74]{carr2011}. Notable exceptions do exist. See the work of
        \cite{auld1994};
        \cite{auld_graham-mckenzie1999} and
        \cite{person2010}.}
In his classic work on the history of ancient Israel, Julius Wellhausen began his treatment of the history of traditions within Judaism with a lengthy discussion of the book of \chronicles and the ways that it differed from Samuel--Kings. He argued that the presentation of ancient Israel's history in \chronicles differed from Samuel--Kings due to the chronological distance between the works and the theological intervention of the Priestly Code into \secondtemple Judaism.%
    \footnote{%
        \cite[171--172]{wellhausen1957}. See also
        \cite{wright_ulrich-wright1992}.}
According to Wellhausen, \chronicles provided a clear example of how ancient Israel's traditions evolved over time. In the same way that the legal material of the Hexateuch developed over the centuries, the traditions of Samuel--Kings and \chronicles reflected the ways that  theologies change through later centuries. While the relationship of Samuel--Kings to \chronicles and the nuances of priestly influences on the Hebrew Bible remain subjects of scholarly debate, the broad consensus remains that 1) \chronicles emerged sometime in the late Persian or early Hellenistic periods, 2) it used Samuel--Kings as a literary source, and 3) it exhibits an ideological \emph{Tendenz} influenced by---though not identical to---the final layers of the Pentateuch.%
    \footnote{For a thorough and reasonably recent summary of the \emph{status quaestionis}, see
        \cite[72--89]{knoppers2003}. See also
        \cite{japhet1993};
        \cite{japhet2009};
        \cite{braun1986} and
        \cite{coggins1976}.}
In this way, \chronicles offers the last word on a certain set of traditions surrounding the monarchic period---in particular those of David, Solomon, and the kings of Judah.

From my perspective, the relationship of \chronicles to Samuel--Kings exhibits all the requisite features for inclusion into the category of \rwb by nearly every system of classification that I discussed in \autoref{chap:rwb} (generic or otherwise). Indeed, for scholars who focus on the topic of \rwb, \chronicles often factors prominently in their discussions.%
    \footnote{See
        \cite{campbell_zsengeller2014};
        \cite{zahn_lim-collins2010};
        \cite{bernstein_textus2005}.
        Alexander considers \chronicles to be a ``prototype'' of \rwb, see,
        \cite[100]{alexander_carson-williamson1988}.}
In spite of this fact, scholars who work primarily on \chronicles seem rarely to give more than a passing nod to the scholarly literature on \rwb. Sara Japhet, for example, does not address the topic at all in her 1993 commentary.%
    \autocite{japhet1993}
Klein's recent commentary on \chronicles, published in 2006, only mentions the topic of \rwb in a footnote, stating:
\begin{quote}
    Perhaps \chronicles could also be compared with the genre called ``rewritten Bible,'' known from Qumran and in the works of Josephus. Such works retell some portion of the Bible while interpreting it through paraphrase, elaboration, allusion to other texts, expansion, conflation, rearrangement, and other techniques. In this case, of course, the ``rewritten Bible'' also became part of the Bible itself.%
    \autocite[17 n.157]{klein2006}
\end{quote}
\noindent
Setting aside Klein's rather anemic description of \rwb, it is striking---if not terribly surprising---to me that the comparison is not taken more seriously. Instead, Klein characterizes \rwb as a genre ``known from Qumran and ... Josephus'' and not as a phenomenon of Jewish literary production during the \secondtemple period in which \chronicles may also have been participating. In his slightly earlier commentary, published in 2003, Knoppers takes up the issue in a special section of his introduction and offers his thoughts on whether \chronicles is \rwb.%
    \autocite[129--134]{knoppers2003}
His conclusions, however, are not much more satisfying. According to Knoppers, \rwb is an interesting angle from which to approach \chronicles, but ultimately, ``\chronicles needs to be understood as its own work.''%
    \autocite[134]{knoppers2003}

The bracketing of \chronicles as \rwb is, I think, largely a function of the ways that conversations within scholarly sub-disciplines can be insular and resistant to traversing disciplinary boundaries. In this case, the book of \chronicles falls squarely within the field of ``biblical studies'' and ``Hebrew Bible'' while most of the scholarly work surrounding \rwb falls within the related, but distinct, fields of Early Judaism, Qumran Studies, and \secondtemple Studies. Bridging such disciplinary gaps, even when the sub-disciplines are adjacent, can be difficult. Thus, although some commentators on \chronicles note the family resemblance, there is no commentary-length treatment of \chronicles which utilizes \rwb as the primary literary framework for reading \chronicles.

What we lose in scholarly consensus about the book of \chronicles' characterization as \rwb, we gain in its treatment as an example of cultural and social memory. Wellhausen chose to begin his treatment of the history of traditions with \chronicles because of the security with which scholars date the work relative to Samuel--Kings and the latest Pentateuchal strata. The relative chronology allowed Wellhausen to reason about how those changes may have emerged from an historical perspective and it is for this same reason that \chronicles has played an important role in the emerging applications of memory theory within the field of biblical studies.%
\footnote{%
    See especially the work of Ehud Ben Zvi,
    \cite*{benzvi_st2017};
    \cite*{benzvi-a_evans-williams2013};
    \cite*{benzvi-b_evans-williams2013}.
    See also \cite{wilson2017};
    \cite[26--30]{rogerson2010};
    \cite[104--114]{blenkinsopp2013};
    \cite[148-166]{wright2014};
    \cite{jarick_frohlich2019}.
    }
The relatively mature discourse surrounding the discussion of \chronicles as memory makes it a reasonable place to start for our purposes as well and it is for this reason that I have chosen to start my case studies with the book of \chronicles.

Thinking in terms of social memory requires us to consider the relationship between the texts in \emph{social} terms. In other words, not just to ask \emph{what} received traditions the \chronicler used, but to consider the \emph{role} and \emph{status} of those traditions and to consider why they were (or were not) significant within a particular social context.%
% FIXME: "The problem is that you don't really talk about social context in this chapter so you need to probably drop this angle."
Thus the process of ``remembering'' in \chronicles can be viewed from two different angles which map onto the dual valences of the term ``remember'': to ``recall'' and to ``commemorate.'' On the one hand, the \chronicler ``recalls'' stories which are adapted to the frameworks of the \chronicler's social situation. The \chronicler is a product of his time and society and as such inherited sets of traditions about Israel's remembered past and the world more broadly which color how he understands that history. The \chronicler participates in discourses at various sites of memory and makes his own contributions to those sites. Thus the composition of the book of \chronicles is itself an act of commemoration which represents the process of memory encoding and the construction of cultural memory from which future rememberers would draw.

As a work literature, it also bears the idiosyncrasies of its author(s), however constrained by their social milieu they may have been. In fact, determining which of these processes best accounts for any particular ``innovation'' of \chronicles is quite difficult. Was the \chronicler consciously ``reshaping'' the memory of Israel's past? Or was the \chronicler more passively reproducing the memory that he inherited from his culture? Or perhaps both? Traditional approaches to the book of \chronicles have tended to attribute a great deal of agency to the \chronicler as an innovator of tradition. But thinking in terms of cultural memory pushes us to consider a fuller picture of how cultural memory is created and calls into question whether every theological or ideological augmentation of the \chronicler should be attributed to his idiosyncratic understanding of the Israelite past. Such an approach takes into account that textual ``sources'' are not merely copied and ``altered,'' but are read, internalized, believed, understood, and reasoned about, which is to say, \emph{remembered}.

% TODO: Write a better thesis. Be sure to include how the Ornan section illustrate the main aim of your argument?
In this chapter I will argue that the book of \chronicles---as both a work of \rwb and as an exemplar of social and cultural memory---reflects a distinct system of mnemonic discourses and a reconfiguration of Israel's remembered past over-and-against Samuel--Kings. This reconfiguration, I will argue, came about through specific mnemonic processes.  To demonstrate these processes, I will first discuss the ways that \chronicles engages with major sites of memory, adapting them to his own system of social frameworks. Second, I will discuss the phenomenon of ``magnetism'' between sites of memory and argue that magnetic tenancies between sites of memory may better account for some of the \chronicler's so-called ``harmonizing'' tendencies. Finally, I will discuss the account of David's Census and the Threshing Floor of Araunah/Ornan found in 1 Chr 21:1--22:1 as a product of cultural memory and illustrate the ways that it engages with major sites of memory draws together the major metanarratives of Israel's remembered past.

%%%%%%%%%%%%%%%%%%%%%%%%%%%%%%%%%%%%%
%% MAGNETISM AND MNEMONIC NETWORKS %%
%%%%%%%%%%%%%%%%%%%%%%%%%%%%%%%%%%%%%
\section{Magnetism and Mnemonic Networks}

A particularly useful way to think about social memory is to model it as a complex \emph{network} of symbolic meaning---as a graph with nodes and edges.%
    \footnote{Graphs and graph theory are a part of discrete mathematics and have a long and distinguished history going back to Euler. Although more recent applications of graph theory within sociology have focused on, for example, social networks on the internet, so-called social network analysis has been in use within sociology back to the early 19th century. See \cite[10--16]{linton2004}. Scale-free networks, in particular, are of interest to us. See \cite{barabasi_science2009}.}
In such a system each node represents a ``site of memory'' and the size or weight of that node is determined by the kinds of details, ideas, and themes that are remembered about that mnemonic site. Sites of memory are sites of discourse which grow and accrue meaning through normal discursive processes within a society. In the case of the \temple, for example, one might argue that the description of the \temple in the Hebrew Bible and the role that it played in the religious life of ancient Israel as mechanism for ritual atonement and as a symbol of the covenant adds ``weight'' or ``gravitas'' to the site (or node) within Israel's cultural memory. 

What make networks interesting, however, are the connections that exist between nodes and the systems of relationships that those connections create. In networks of social memory, nodes that are more highly ``connected''---those that have more edges linking them to other sites of memory---are more entangled with the entire system of symbolic meaning. Entangled nodes are more stable within the network and are less likely to be forgotten by virtue of the fact that they are defined with reference to more sites of memory. Severing one or two connections will not completely isolate the node from the rest of the graph. The inverse is also true. Those sites which are less clearly situated within the graph are more susceptible to being forgotten.

To evaluate the ``significance'' of a particular site within the social memory, therefore, one must consider not just the size of a node, but how ``connected'' the node is within the social network. Larger, more highly-connected sites of memory---those which for one reason or another have been connected to many other such sites within the social memory---may be viewed as more ``significant,'' while smaller sites with fewer connections are comparatively less significant with respect to social and cultural memory.%
    \footnote{Of course, when I say that a king is more ``significant'' than, say, a peasant, I am making an assessment of the social impact of the individual on the society broadly and not making a judgment of the intrinsic value or importance of the individual. Moreover, I am not saying that such significance ought to guide the historian. This is merely meant as a description of this particular social phenomenon.}
Larger, more highly connected nodes of meaning are not only more difficult to forget, but have the tendency to extend their connectedness to other highly significant nodes and furthermore to absorb lesser nodes and integrate their meaning. The process by which sites of memory attract one another is what Ehud Ben Zvi refers to as ``magnetism'' in memory. He writes:
\begin{quote}
    Of course, not all sites of memory draw the same attention in a group. the most prominent sites of memory are ``magnets'' for core meaning, ideas, and concepts and tend to evoke and deeply intertwine several of the groups' main metanarratives. Conversely , the more a site of memory can embody, intertwine and communicate several of these metanarratives, the more central the site of memory will become for the group.\autocite[73]{benzvi_st2017}
\end{quote}
The processes by which these major ``hubs'' of meaning continue to attract more and more connections creates a kind of feedback loop wherein the ``rich get richer'' and the more significant sites of memory get more significant. In fact, the tendency of highly-connected nodes within a graph to attract other (especially new) nodes is a property of scale-free networks, which has become an important theoretical model for describing network processes and evolution in nature and society.%
    \footnote{See \cite[30--33]{caldarelli2007}. The applicability of scale-free networks cuts across disciplines. To name only a few: 
        (computer virus infections) \cite{satorras-alessandro_prl2001};
        (economics) \cite{garlaschelli-etal_pysisca-a2005};
        (game theory) \cite{santos-pacheo_prl2005};
        (cellular biology) \cite{albert_jcs2005};
        (epidemiology) \cite{may-lloyd_pre2001}.}
The principle feature of scale-free networks is the disparity between highly connected nodes (hubs) and sparsely connected nodes. The imbalance between these nodes is exacerbated by the fact that when new nodes are inserted into the graph, they are not randomly connected, but tend to connect to nodes that already exhibit high-levels of connectedness. This is a process known as ``preferential attachment.''%
    \footnote{%
        \cite{barabasi-albert_science1999};
        \cite{jeong-etal_epl2003}.}

One likely example of magnetism observable within the biblical text is the story of David and Goliath. Although the story of David and Goliath is very well known, is surprising to find that in 2 Sam 21:19b, someone else is credited with slaying Goliath:
\begin{hebrewtext}
    \versenum{2 Sam 21:19}
    וַתְּהִי־עוֹד הַמִּלְחָמָה בְּגוֹב עִם־פְּלִשְׁתִּים וַיַּךְ אֶלְחָנָן בֶּן־יַעְרֵי אֹרְגִים בֵּית הַלַּחְמִי אֵת גָּלְיָת הַגִּתִּי וְעֵץ חֲנִיתוֹ כִּמְנוֹר אֹרְגִים׃
\end{hebrewtext}
\begin{translation}
    \versenum{2 Sam 21:19}
    Then another battle came about in Gob with the Philistines and \emph{Elhanan son of Jaare-oregim the Bethlehemite struck down Goliath} the Gittite the shaft of whose spear was like a weaver's beam.
\end{translation}
\noindent
And yet it is David and not Elhanan who is ultimately remembered as the person who killed Goliath. How might this mnemonic realignment have happened? McCarter makes the observation that ``Deeds of obscure heroes tend to attach themselves to famous heroes''%
    \autocite[450]{mccarter1984}
but offers it without providing any supporting rationale for \emph{why} this may have been the case or \emph{how} it came about. I do not find this explanation particularly satisfying. Instead I would like to suggest that the identification of David as the person who killed Goliath can best be understood through the process of magnetism.

This suggestion is rooted in the presupposition that in the extended narrative of 1~Sam17--18, the Philistine whom David killed was not originally known as ``Goliath.'' In fact, the name ``Goliath'' only occurs twice in the extended narrative, in vv. 4 and 23; in every other instance throughout the narrative, the man is referred to simply as ``the Philistine.'' Moreover, the occurrence in v. 23 is found in a section of the narrative that is not preserved in the (presumably) earlier form of the narrative known from the \lxx. Because of these fact, the identification of the ``the Philistine'' with Goliath is widely regarded as a secondary addition.
    \footnote{For a fuller account of the textual issues surrounding the main narrative about David and Goliath, see \cite[280--309]{mccarter1980} and \cite[69--77]{mckenzie2000}.}
The fact that this narrative shows signs of extensive supplementation within the \mt coupled its complete absence from the book of \chronicles hints at the idea that the extended narrative of David and the Philistine was a late addition to the book of 1~Samuel.%

The identification of the young Israelite with David may also have been a secondary attribution. The narrative that immediately precedes the ``Goliath'' narrative describes David as ``a knowledgable musician, and a mighty man of valor and a man of war'' (1~Sam 16:18; Heb. \hebrew{יֹדֵעַ נַגֵּן וְגִבּוֹר חַיִל וְאִישׁ מִלְחָמָה}). This description hardly fits the young man in 1~Sam 17--18. Moreover, in the earlier narrative David becomes Saul's armor bearer, but in 1~Sam 17, David is described as a young man with no battle experience and Saul seems to not know who he was.%
    \autocite[70--71]{mckenzie2000}.
It is not impossible that the stories are simply out of order but this seems to me more like a doublet for David's association with Saul's court. Although the narrative in both the \mt and the \lxx identifies the young Israelite as David, I suspect that in the earliest forms of the narrative, both the young Israelite and the big Philistine were anonymous.

If indeed both the young Israelite and the Philistine were anonymous, we might imagine the story as an isolated site of memory. in such a case, it is not difficult to speculate about why the young man might have been associated with David. As a major site of Israelite cultural memory, ``new'' sites of memory which were not securely contextualized within the social memory of ancient Israel would have been drawn to magnetic figures like David by the principle of preferential attachment. Moreover, the details of the story work well for David. David was remembered as a warrior (1~Sam 18:7; 29:5) who fought with and against Philistines (with: 1 Sam 27:1--28:2; against: 2~Sam 5:17--25)and, narratively speaking, David's character arc---even without the Goliath story---is one of humble beginnings and a meteoric rise by God's favor. All of these these and narrative patterns lend also themselves to attachment to the story of the humble Israelite boy who successfully slew the Philistine.
    
It is also quite easy to imagine how ``the Philistine,'' was attached to the figure of Goliath. Although Goliath (leaving out the extended narrative material) would have constituted a comparatively minor site of memory, he is one of only a few named Philistines in the Hebrew Bible.%
    \footnote{As a matter of verisimilitude, so far as anyone can tell ``Goliath'' is an authentically Philistine name. See \autocite[291]{mccarter1980}.}
As such, the principle of preferential attachment would likely apply. The specific details of the Philistine's description contribute to this identification as well. For example, although the stature of the Philistine is exaggerated in the Masoretic Text,%
    \footnote{The Masoretic text lists Goliath's height to be ``six cubits and a span'' (Heb. \hebrew{שֵׁשׁ אַמּוֹת וָזָרֶת}), or about three meters. This number is contested by both Josephus (\ant 6.171) at least two major Greek versions (Codex Vaticanus and the Lucianic texts), as well as \q{4}{Sam}{a}, which all read ``\emph{four} cubits and a span,'' or about two meters. Still quite large, but hardly a ``giant.'' See \cite[286]{mccarter1980}.}
the effect of describing a warrior who could wield a girthy spear is to imply that he was large and fit.%
    \footnote{The reference to Goliath's spear shaft being the size of a weaver's beam in 1 Samuel 17:7 is commonly read as an insertion \emph{based on} 2~Sam 21:19. As such, it is a further example of how the more central mnemonic site is able incorporate lesser sites. It does not, however, help to explain the connection \emph{initially}.}
The Philistine of the narrative, therefore, could be attached to the mnemonic site of the big Philistine whom Elhanan slew. Once the connection had been made to David, however, the gravitas of David as site of memory effectively absorbed Elhanan's accomplishment. The work was effectively completed when the \chronicler tried to come to terms with the discrepancy:
\begin{hebrewtext}
    \versenum{1 Chr 20:5b}
    וַתְּהִי־עוֹד מִלְחָמָה אֶת־פְּלִשְׁתִּים וַיַּךְ אֶלְחָנָן בֶּן־יָעוּר [יָעִיר] אֶת־לַחְמִי אֲחִי גָּלְיָת הַגִּתִּי וְעֵץ חֲנִיתוֹ כִּמְנוֹר אֹרְגִים׃ 
\end{hebrewtext}
\begin{translation}
    \versenum{1 Chr 20:5b}
    Then another battle came about with the Philistines and Elhanan, son of Jair struck down Lahmi, the brother of Goliath, the Gittite, the shaft of whose spear was like a weaver's beam.
\end{translation}
\noindent
Thus, according to the book of \chronicles, Elhanan killed \emph{Lahmi} the \emph{brother} of Goliath. Although Elhanan was not completely removed from the memory of Israel by the \chronicler, his reputation was considerably diminished. Moreover, by removing intermediary nodes of memory (in this case, Elhanan), the effect is that these larger sites of memory have connected to one another. The process of mnemonic magnetism, therefore, is catalyzed by the introduction of new (or otherwise unconnected) nodes of memory within the system and has the effect not only of drawing smaller sites of memory to larger sites, but drawing together larger sites to one another.

\section{Major Sites of Memory in Chronicles}

To understand the magnetic processes at work in the book of \chronicles, it is important for us first to discuss the most significant sites of memory on their own terms. Sites of memory are sites of discourse that evolve and accrue meaning over time which, in turn, builds connections and contributes to the significance of the mnemonic site within the social memory. The significance of these sites, however, is determined by contemporary discourses at work in the remembering societies. In other words, significance is not an innate quality of a site of memory, but an expression of discursive relevance in a particular society. Although there are many such sites of memory within the book of \chronicles, I will focus on two of the most potent sites for the \chronicler: 1) the figure of King David, and 2) \solomonstemple.

%%%%%%%%%%%%%%%%%%%%%%%%%%%%%%%
%% DAVID AS A SITE OF MEMORY %%
%%%%%%%%%%%%%%%%%%%%%%%%%%%%%%%
\subsection{King David as a Site of Memory}

% David was important before Chronicles in the Bible
Although the book of \chronicles is a work of cultural memory, it is unquestionably the case that the figure David was a prominent site of memory for ancient Israel long before the book of \chronicles was written. More so than Samuel--Kings, \chronicles is characterized in terms of ``memory'' because it is clear that the \chronicler%
    \footnote{My use of the term ``Chronicler'' is meant only to reference the author(s) of the book of \chronicles. Although the term is sometimes associated with a particular theory about the composition of \chronicles, Ezra and Nehemiah, I am not using it as such.}
used Samuel--Kings as a primary source. The differences between Samuel--Kings and \chronicles are easy to see and are securely dated relative to one another. In other words, because we know that \chronicles is secondary to Samuel--Kings and we can see where the \chronicler departed from Samuel--Kings, it is easy to attribute those changes the to developments within Israel's cultural memory. But it is important to remember that even Samuel--Kings is the product of mnemonic construction and the David presented there already functioned as a special site of memory for ancient Israel. In other words, despite the fact that Samuel--Kings functions as a foundational source \emph{for \chronicles}, it should not be treated as if it was the origin of all Davidic traditions.\autocite{frohlich_frohlich2019}

% David was important before Chronicles in the Ancient World 
Even setting aside the biblical material (e.g., Samuel--Kings, Psalms, et al.), it is demonstrably the case that the Davidic \emph{dynasty}---whatever one might think about David as an historical figure---had symbolic meaning in the ancient world which extended beyond the borders of Israel. For example, we know from the Old Aramaic inscription from Tel Dan that the term \aram{בת דוד} ``house of David'' was used as a dynastic name for the monarchy of the kingdom of Judah in the \bce{late ninth or early eighth centuries}.%
    \footnote{The \emph{editio princeps} were published in two articles: the first find in \cite{biran-naveh_iej1993}, and the subsequent fragments in \cite{biran-naveh_iej1995}.}
Likewise, it has been suggested that the Mesha Stele, too, refers to the ``house of David,'' although this reading is contested.%
    \footnote{The reading \aram{בת דוד} on line 31 was proposed by Lemaire, but his reading is not universally accepted. The Mesha inscription is typically dated to the \bce{mid-ninth century} and thus would be slightly earlier than the reference in the Tel Dan inscription, if Lemaire is correct. See \cite{lemaire_sel1994} and \cite{lemaire_bar1994}. A number of scholars have contested Lemaire's reading, however, see especially the recent reconstruction by \cite{finkelstein-etal_ta2019}. See also \cite[164 n. 792]{parker2013}.}
Although such references have traditionally been used to bolster claims of an historical David, for our purposes it suffices to say that around the turn of the \bce{eight century}, ``David'' existed as a meaningful eponymous symbol and site of memory with respect to the monarchy of Judah. Thus, when we turn to the biblical memories of the figure David (which, by most accounts were products of later periods of Israelite history than Tel Dan and Mesha), it is important to keep in mind that those memories are participating in established discourses about David. This is all the more important when we consider the book of \chronicles which represents some of the latest strata of memory preserved in the Hebrew Bible. Thus when we discuss the figure of David as a site of memory which the book of \chronicles engages with extensively, I want to emphasize that the processes of constructing the remembered figure of David did not begin with the \chronicler just as it did it end with the \chronicler.\autocite{frohlich_frohlich2019}

% The David of Chronicles
The question of how David was remembered in \chronicles carries with it the assumption that the author of \chronicles was not arbitrarily copying-and-changing Samuel--Kings (or other traditions), but was a member of a \emph{remembering community} and participated in memory discourses at various sites within the cultural memory of \secondtemple Judaism. This question requires that we not only consider what sources the \chronicler may have used and how he altered those sources, but also to consider the social frameworks that shaped how those sources were received by the \chronicler and how the \chronicler's work reflects a distinct network of meaning over-and-against Samuel--Kings.

Although the David of \chronicles largely resembles that of Samuel--Kings (he is recognizably the same figure), his \emph{function} within the narrative of the book of \chronicles is different than that of Samuel--Kings and that difference can be seen in how the \chronicler uses him rhetorically. Typically, the differences between the \chronicler's picture of David and that of Samuel--Kings are characterized as differences of \emph{portrayal} and focuses on \chronicler's willingness to overlook (and literally to omit) some of David's more egregious shortcomings and to highlight his role as a model King. This positive portrayal of David in \chronicles is well documented and oft-repeated. I would like to suggest, however, that the \emph{portrayal} of David in \chronicles is merely the ``symptom'' of deeper, more fundamental shifts in how David was remembered during the \secondtemple period. To illustrate these shifts, I would like to focus on two points of discourse within the \chronicler's memory of David: 1) David as the divinely elect king and, and 2) David as disqualified from constructing the \temple in Jerusalem.%
    \footnote{See \cite{jarick_frohlich2019}; \cite[347--383]{japhet2009} \cite{knoppers_biblica1995}; \cite[47--48]{japhet1993}; \cite[44--48]{klein2006}; \cite[80--85]{knoppers2003}.}

\subsubsection{David the Divinely Elect King}

First, as I have just alluded to, in the book of \chronicles, David is remembered as the quintessential, rightful Israelite ruler, elected by \yahweh (1~Chr 10:14) and anointed by the elders of Israel to lead the people (1 Chr 11:1--3). By comparison to the account in Samuel--Kings, the process by which David becomes the ruler of Israel is somewhat less contentious. The apologetic tone of the so-called History of David's Rise (HDR) and Succession Narratives (SN) is nowhere to be found. The rationale for Saul's demise is, like in Samuel--Kings, predicated on his supposed infidelity to \yahweh, with special reference to his consultation with a medium (although, the story is not told in \chronicles), however, the election of David as Saul's ``successor,'' as described by the \chronicler, does not include Saul aside from a passing reference to his death and infidelities. David himself offers his version of events in 1 Chr 28:4:
\begin{hebrewtext}
    \versenum{1 Chr 28:4}
    וַיִּבְחַר יְהוָה אֱלֹהֵי יִשְׂרָאֵל בִּי מִכֹּל בֵּית־אָבִי לִהְיוֹת לְמֶלֶךְ עַל־יִשְׂרָאֵל לְעוֹלָם כִּי בִיהוּדָה בָּחַר לְנָגִיד וּבְבֵית יְהוּדָה בֵּית אָבִי וּבִבְנֵי אָבִי בִּי רָצָה לְהַמְלִיךְ עַל־כָּל־יִשְׂרָאֵל׃ 
\end{hebrewtext}
\begin{translation}
    \versenum{1 Chr 28:4}
    \yahweh, the God of Israel chose me from among my father's whole house to be king over Israel forever. He chose Judah to be a leader and (from) the house of Judah, the house of my father and (from) the house of my father, he took delight in me to make (me) king over all Israel.
\end{translation}
\noindent
Conspicuously absent from the \chronicler's narrative and David's summary, are the major conflicts with Saul during David's rise to power. In fact, if one did not know better, simply removing all references to Saul in \chronicles would not meaningfully change how David's election is described.%
    \footnote{This fact raises the question of why the \chronicler \emph{did not} simply omit Saul. I suspect that, although not favored Saul was a useful foil narratively and was a well-enough known figure that omitting him entirely simply did not make sense. Saul was, doubtless, a major figure in the traditions of early Israel.}

Similarly, the tumult within David's court at the end of his life and the succession of Solomon are omitted by the \chronicler, where 1~Kings begins with a feeble, impotent David and his messy succession by Solomon, 1~Chr 23:1 is content simply to report that:
\begin{hebrewtext}
    \versenum{1 Chr 23:1}
    וְדָוִיד זָקֵן וְשָׂבַע יָמִים וַיַּמְלֵךְ אֶת־שְׁלֹמֹה בְנוֹ עַל־יִשְׂרָאֵל׃
\end{hebrewtext}
\begin{translation}
    When David was old and full of days, he made Solomon, his son, king over Israel.
\end{translation}
\noindent
It went \emph{so} well, in fact, that David saw fit to do it a second time, according to 1 Chr 29:22b--23:
\begin{hebrewtext}
    \versenum{1 Chr 29:22b}
    וַיַּמְלִיכוּ שֵׁנִית לִשְׁלֹמֹה בֶן־דָּוִיד וַיִּמְשְׁחוּ לַיהוָה לְנָגִיד וּלְצָדוֹק לְכֹהֵן׃ 
    \versenum{23}
    וַיֵּשֶׁב שְׁלֹמֹה עַל־כִּסֵּא יְהוָה לְמֶלֶךְ תַּחַת־דָּוִיד אָבִיו וַיַּצְלַח וַיִּשְׁמְעוּ אֵלָיו כָּל־יִשְׂרָאֵל׃
\end{hebrewtext}
\begin{translation}
    \versenum{1 Chr 29:22b}
    Then they made Solomon, son of David, king a second time and they anointed him by \yahweh as a prince as well as Zadok as a priest.
    \versenum{23}
    And Solomon sat on the throne of \yahweh as king in place of David, his father. And he prospered and all Israel obeyed him.
\end{translation}
\noindent
These matter-of-fact descriptions contrast sharply with the events depicted in 1~Kings: Adonijah's self-exaltation (1~Kgs 1:5--53), David's deathbed speech to Solomon (1~Kgs 2:1--9), Solomon's subsequent conflict with Adonijah over Abishag (1~Kgs 2:13--25), with Joab (1~Kgs 2:28--35), and with Shimei (1~Kgs 2:36--46); all of which culminates with the ominous pronouncement of 1~Kgs 2:46b:
\begin{hebrewtext}
    וְהַמַּמְלָכָה נָכוֹנָה בְּיַד־שְׁלֹמֹה׃
\end{hebrewtext}
\begin{translation}
    So the kingdom was established in the hand of Solomon.
\end{translation}
\noindent
The contrast between the violent establishment of the kingdom ``in the hand of Solomon'' (1~Kgs 1-2) and the popular assent of the people to both the reigns of David and Solomon in \chronicles (David: 1~Chr 11:3; Solomon: 1 Chr 29:22b--23) could not be more clear. 

The rhetorical force of the History of David's Rise and the Succession Narratives in Samuel--Kings offers a \emph{rationale} for the events that take place---although not every action is kind and benevolent, everything that David and Solomon do is framed as a sensible response to wrongdoing. Because of this, it is widely held that that the HDR and SN should be understood as forms of ancient royal apologia---an effort by the author(s) to legitimize David's actions which might otherwise have been construed as a usurpation of the divinely elected king, Saul. Andrew Knapp, for example, observes that ``[i]n some ways, [the Traditions of David's Rise and Reign] is the paradigmatic ancient Near Eastern apology.''%
    \autocite[218]{knapp2015}
He elaborates:
\begin{quote}
    The apologist employs nearly every apologetic motif in his effort to legitimize David, including passivity, transcendent non-retaliation, the unworthy predecessor, military prowess, and the entire triad of establishing legitimacy.%
\autocite{knapp2015}
\end{quote}
\noindent
In other words, the authors of Samuel--Kings sought to make a forceful and potent argument in favor of David's legitimacy and used literary devices and forms which were meaningful in their society.%
    \footnote{Although the compositional and redactional history of the Deuteronomistic History is hotly debated---with wildly divergent scholarly opinions---I take as my point of departure the centrist view of McCarter, Halpern, and specifically Knapp which view the HDR and SN as royal apologia. I follow Knapp in his view that these traditions do not represent ``the residue of a single apologetic composition'' (161), but rather a diverse set of traditions. However, because the sources cannot meaningfully be parsed, I will also follow him in ``[dealing] with the early narrative traditions in their entirety'' (161). See 
        \cite{knapp2015};
        \cite{mccarter_interpretation1981};
        \cite{mccarter_jbl1980};
        \cite{mccarter1980};
        \cite{halpern2001}.}
They engaged in discourses about David's legitimacy in an attempt to define David's rise and reign as \emph{legitimate}. As such, it has been argued that the HDR narratives functioned as a \emph{contemporary} form of apologia, implying that these narratives originated at-or-around the time of the presumed historical figure of David.%
    \footnote{See especially \cite{mccarter_interpretation1981}; \cite{mccarter_jbl1980}; and to a lesser degree \cite[75--76]{halpern2001}. Some clarification is in order here. McCarter et al. are generally talking about where these stories \emph{originated}. They are engaging primarily with minimalist scholars who discount reality of the historical figure of David.}
These apologia, therefore, would have arisen in response to accusations of usurpation. Thus, we would imagine that the HDR was representative of the ``last word'' on the matter or an attempt to suppress alternative voices that questioned the legitimacy of David's rule, the means by which he gained the throne, and the manner of his succession. Although I am not entirely convinced by this line of reasoning, from the perspective of social memory it is safe to say that at the time of the narrative's composition, the question of whether David should be remembered as a the leader of a victorious \emph{coup d'état} over Saul, or a reluctant leader divinely chosen by \yahweh was a matter of debate. Indeed, these discourses were not entirely suppressed from the Hebrew Bible, as evidenced by the figure Shimei and his condemnation of David as a usurper in 2~Sam 16:7b-8:
\begin{hebrewtext}
    \versenum{2 Sam 16:7b}
    וְכֹה־אָמַר שִׁמְעִי בְּקַלְלוֹ צֵא צֵא אִישׁ הַדָּמִים וְאִישׁ הַבְּלִיָּעַל׃ 
    \versenum{8}
    הֵשִׁיב עָלֶיךָ יְהוָה כֹּל דְּמֵי בֵית־שָׁאוּל אֲשֶׁר מָלַכְתָּ תַּחְתָּו [תַּחְתָּיו] וַיִּתֵּן יְהוָה אֶת־הַמְּלוּכָה בְּיַד אַבְשָׁלוֹם בְּנֶךָ וְהִנְּךָ בְּרָעָתֶךָ כִּי אִישׁ דָּמִים אָתָּה׃
\end{hebrewtext}
\begin{translation}
    \versenum{2 Sam 16:7b}
    Thus Shimei spoke cursing him, ``Go out! Go out! Oh man of blood; Oh worthless man! \yahweh has repaid you all the blood of the house of Saul, in whose place you reign. May \yahweh give your kingdom into the hand of Absalom, your son. Look at your evil! Because you are a man of blood.''
\end{translation}
\noindent
The complex redactional history of the Deuteronomistic History makes saying anything more specific than this difficult, so I am open to the possibility that there may have been other social contexts in which such apologia would be potent, either as the original context of their composition or as a new context for an old set of stories. For example, Diana Edelman has suggested that a Saulide--Davidic rivalry could have resurfaced during the early Persian period.%
    \autocite{edelman_dearman-graham2002}
Or perhaps the Saul/David struggle could hint at a Benjaminite/Judahite conflict even after the the fall of Israel. This is all idle speculation, of course, but we must allow for the fact that these apologetic discourses could be potent in a number of social contexts. All the same, what is certain is that the memory of David in Samuel--Kings is preoccupied with discourses on the \emph{legitimacy} of David as the divinely elect King of Israel. 

The book of \chronicles, on the other hand, is not at all interested in engaging discourses about David's legitimacy. Instead, \chronicles offers plain, black-and-white, accounts of David's rise that take for granted the legitimacy of David's elect status. It seems odd then, in some regards, that the \chronicler would omit the vast majority of the drama and court intrigue found in Samuel--Kings. In other words, for all the potency of these stories, why would the \chronicler omit such persuasive, and effective material? The answer, I think, is quite simple: the \chronicler was operating within a social milieu that not only accepted the legitimacy of David and his heirs, but celebrated them as foundational figures. The discourses that the authors of Samuel--Kings participated in had been resolved and the Davidic dynasty was thoroughly legitimate in the mind of the \chronicler. As such, it was sufficient for the \chronicler to simply recount the death of Saul---which David had no part in---and the subsequent anointing of David. From the perspective of the \chronicler David's rise to power was a thoroughly unremarkable transfer of the kingship from the rejected Saul to the elect David. Similarly, the \chronicler makes no mention of the difficult power struggles that occurred near the end of David's life between him and his sons. The struggle between Solomon and Adonijah following David's death is likewise omitted. Instead, opening verse of 2~\chronicles reads simply:
    \begin{hebrewtext}
        \versenum{2 Chr 1:1}
        וַיִּתְחַזֵּק שְׁלֹמֹה בֶן־דָּוִיד עַל־מַלְכוּתוֹ וַיהוָה אֱלֹהָיו עִמּוֹ וַיְגַדְּלֵהוּ לְמָעְלָה׃
    \end{hebrewtext}
    \begin{translation}
        Solomon, the son of David, established himself in his kingdom, and \yahweh his God was with him and made him exceedingly great.   
    \end{translation}
\noindent
It seems, therefore, that Samuel--Kings was so successful in its apologetic that the memory constructed by its rhetoric precluded the need for continued apologia in the work of the \chronicler. The \chronicler had no need to ``legitimize'' the \emph{fact of} the Davidic dynasty, but instead would focus his attention on defining the \emph{significance of} that dynasty for his own readers in a dramatically different social setting. What is at issue between the David of \chronicles and the David of Samuel--Kings, therefore, is not that he is \emph{portrayed} differently, but that the memory of David \emph{functioned} differently within each system of symbolic meaning. In other words, the discursive function of David was different for the \chronicler than it was for the authors and redactors of Samuel--Kings. In the memory of the \chronicler, David was significant \emph{because} he was king and his legitimacy was assumed and celebrated. The \chronicler, therefore, used terse declarative rhetoric and offered black-and-white accounts of David's election, rule, and succession that helped to reinforce the idea that David and his successors were not only elect by \yahweh, but were monarchs whose elections were celebrated by the population at large.

\subsubsection{David the Disqualified \temple-builder}
A similar discursive shift between Samuel--Kings and \chronicles can be seen through each source's engagement with the fact that it was Solomon and not David who built the \temple. Both \chronicles and Samuel--Kings offer explanations for this fact, but they do so while operating on different sets of presuppositions. 

In both 2 Sam 7 and 1 Chr 17, David expresses a desire to build a \temple for \yahweh and in both cases is rebuffed by \yahweh through the prophet Nathan. Instead, Nathan tells David that \yahweh would establish David's line through his son, Solomon. Although neither account gives a reason for \yahweh's preference toward Solomon, later in \chronicles, David states that the reason \yahweh passed over him was that David was a man of war, while Solomon would be a man of peace:
\begin{hebrewtext}
    \versenum{1 Chr 22:7}
    וַיֹּאמֶר דָּוִיד לִשְׁלֹמֹה בְּנוֹ [בְּנִי] אֲנִי הָיָה עִם־לְבָבִי לִבְנוֹת בַּיִת לְשֵׁם יְהוָה אֱלֹהָי׃ 
    \versenum{8}
    וַיְהִי עָלַי דְּבַר־יְהוָה לֵאמֹר דָּם לָרֹב שָׁפַכְתָּ וּמִלְחָמוֹת גְּדֹלוֹת עָשִׂיתָ לֹא־תִבְנֶה בַיִת לִשְׁמִי כִּי דָּמִים רַבִּים שָׁפַכְתָּ אַרְצָה לְפָנָי׃
    \versenum{9}
    הִנֵּה־בֵן נוֹלָד לָךְ הוּא יִהְיֶה אִישׁ מְנוּחָה וַהֲנִחוֹתִי לוֹ מִכָּל־אוֹיְבָיו מִסָּבִיב כִּי שְׁלֹמֹה יִהְיֶה שְׁמוֹ וְשָׁלוֹם וָשֶׁקֶט אֶתֵּן עַל־יִשְׂרָאֵל בְּיָמָיו׃
\end{hebrewtext}
\begin{translation}
    \versenum{1 Chr 22:7}
    And David said to Solomon, ``My son, my heart desired to build a \temple for the name of \yahweh, my God
    \versenum{8}
    but the word of \yahweh came to me saying, `You have spilled much blood and fought in great battles. You shall not build a \temple for my name because you have spilled so much blood on the earth before me.
    \versenum{9}
    Rather, a son will be born to you. He will be a man of rest. And I will give him rest from all his enemies who surround him. Thus, Solomon will be his name and I will give peace and quiet to Israel during his days.'''
\end{translation}
\noindent
Because David had shed so much blood, he was \emph{disqualified} from building the \temple. It is a qualitative assessment of David. The same logic is echoed in 1~Chr 28:3: 
\begin{hebrewtext}
    \versenum{1~Chr 28:3}
    וְהָאֱלֹהִים אָמַר לִי לֹא־תִבְנֶה בַיִת לִשְׁמִי כִּי אִישׁ מִלְחָמוֹת אַתָּה וְדָמִים שָׁפָכְתָּ׃
\end{hebrewtext}
\begin{translation}
    \versenum{1~Chr 28:3}
    But God said to me, `'`You shall not built a house for my name because you are a man of war and had spilled blood.''
\end{translation}
\noindent
Solomon, on the other hand, would be a man of peace and this innate quality is foreshadowed through the word-play of ``Solomon'' (Heb. \hebrew{שְׁלֹמֹה}) and ``peace'' (Heb. \hebrew{שָׁלוֹם}) in 1~Chr 22:9.

On the other hand, Samuel--Kings does not contain any of this material and offers a slightly different rationale for why David did not build the \temple. In 1~Kgs 5:15 Solomon explains that the reason his father, David, was unable to build the \temple was due to the persistence of David's many enemies:
\begin{hebrewtext}
    אַתָּה יָדַעְתָּ אֶת־דָּוִד אָבִי כִּי לֹא יָכֹל לִבְנוֹת בַּיִת לְשֵׁם יְהוָה אֱלֹהָיו מִפְּנֵי הַמִּלְחָמָה אֲשֶׁר סְבָבֻהוּ עַד תֵּת־יְהוָה אֹתָם תַּחַת כַּפּוֹת רַגְלָו [רַגְלָי׃] 
\end{hebrewtext}
\begin{translation}
    You knew David, my father; that he was not able to build a house for the name of \yahweh, his God, on account of the war which surrounded him until \yahweh put them beneath the soles of his feet.
\end{translation}
\noindent
It is important to note that the rationale here is not that David divinely prohibited from building the \temple, but that the presence of his enemies \emph{prevented} him from building the \temple. In fact, this statement is inconsistent with the description of David in 2~Sam~7:1, which explicitly states that it was after \yahweh had given David rest from his enemies that David first considered building a \temple for the deity:
    \begin{hebrewtext}
        \versenum{2 Sam 7:1}
        וַיְהִי כִּי־יָשַׁב הַמֶּלֶךְ בְּבֵיתוֹ וַיהוָה הֵנִיחַ־לוֹ מִסָּבִיב מִכָּל־אֹיְבָיו׃
        \versenum{2}
        וַיֹּאמֶר הַמֶּלֶךְ אֶל־נָתָן הַנָּבִיא רְאֵה נָא אָנֹכִי יוֹשֵׁב בְּבֵית אֲרָזִים וַאֲרוֹן הָאֱלֹהִים יֹשֵׁב בְּתוֹךְ הַיְרִיעָה׃
    \end{hebrewtext}
    \begin{translation}
        \versenum{2 Sam 7:1}
        It came about that when the king was sitting in his house---\yahweh having given him rest all around from all his enemies---
        \versenum{2}
        the king said to Nathan the prophet, ``Look! I am sitting in a house of cedar but the ark of God is sitting in the midst of curtains!''
    \end{translation}
\noindent
Rather conspicuously, however, the parallel account in 1~Chr~17:1 omits that David had been given rest:
\begin{hebrewtext}
    \versenum{1 Chr 17:1}
    וַיְהִי כַּאֲשֶׁר יָשַׁב דָּוִיד בְּבֵיתוֹ וַיֹּאמֶר דָּוִיד אֶל־נָתָן הַנָּבִיא הִנֵּה אָנֹכִי יוֹשֵׁב בְּבֵית הָאֲרָזִים וַאֲרוֹן בְּרִית־יְהוָה תַּחַת יְרִיעוֹת׃
\end{hebrewtext}
\begin{translation}
    \versenum{1 Chr 17:1}
    Now, when David was sitting in his house, David spoke to Nathan the prophet, ``I am sitting in a house of cedar but the ark of the covenant of \yahweh is under curtains!'' 
\end{translation}
\noindent
One obvious way to explain this difference is to attribute the omission to the \chronicler's desire for narrative consistency and to assert that it was not until the reign of Solomon that ``peace and quiet'' would be achieved in Israel. Indeed, this ultimately is the position of the \chronicler, which he makes explicit in 1 Chr 22:7 (above). 

While Japhet and others finds this omission consistent with the \chronicler's broader methodology and ideological project,%
    \autocite[328]{japhet1993}
there is some debate about whether the reference to \yahweh giving rest to David was original to 2 Samuel or whether it was a late Deuteronomistic addition.%
    \footnote{McCarter states confidently that this is an addition to the MT, despite the fact that all known witnesses include the phrase. See \cite[191]{mccarter1984}.}
As a result, there is also some question whether it was a part of the \vorlage of the \chronicler at all and therefore whether the minus in 1 Chr 17 should be attributed to the \chronicler. McKenzie in particular goes so far as to say that this was a late Deuteronomistic addition to 2 Samuel and argues that the phrase simply was not a part of the \vorlage from which the \chronicler drew.%
    \footnote{\cite[63]{mckenzie1985}. Knoppers does not make a strong recommendation either way, but makes it a point to include haplography as a possible explanation of the omission. See \cite[666]{knoppers2007}.}
Even allowing for the possibility that this aside was not a part of the \chronicler's \vorlage, however, there remain at least two related questions to be answered: 1) What prompted the supposed insertion into 2 Sam 7, and 2) how did David's \emph{preoccupation} with his enemies turn into a divine \emph{disqualifier}, as described in 1~Chr 22 and 28.

To answer the first question, numerous scholars have observed the clear connection between this reference to finding ``rest'' with Deuteronomy 12:10--11, which establishes a timeline for the construction of a permanent cultic site in \yahweh's chosen locale:

\begin{hebrewtext}
    \versenum{Deut 12:10}
    וַעֲבַרְתֶּם אֶת־הַיַּרְדֵּן וִישַׁבְתֶּם בָּאָרֶץ אֲשֶׁר־יְהוָה אֱלֹהֵיכֶם מַנְחִיל אֶתְכֶם וְהֵנִיחַ לָכֶם מִכָּל־אֹיְבֵיכֶם מִסָּבִיב וִישַׁבְתֶּם־בֶּטַח׃ 
    \versenum{11}
    וְהָיָה הַמָּקוֹם אֲשֶׁר־יִבְחַר יְהוָה אֱלֹהֵיכֶם בּוֹ לְשַׁכֵּן שְׁמוֹ שָׁם שָׁמָּה תָבִיאוּ אֵת כָּל־אֲשֶׁר אָנֹכִי מְצַוֶּה אֶתְכֶם עוֹלֹתֵיכֶם וְזִבְחֵיכֶם מַעְשְׂרֹתֵיכֶם וּתְרֻמַת יֶדְכֶם וְכֹל מִבְחַר נִדְרֵיכֶם אֲשֶׁר תִּדְּרוּ לַיהוָה׃
\end{hebrewtext}
\begin{translation}
    \versenum{Deut 12:10}
    And you will cross over the Jordan and settle in the land that \yahweh your God is giving to you. And he will give you rest from all your enemies around (you) and you will live safely.
    \versenum{11}
    Then the place at which \yahweh your God will establish his name will be (the place) that you will bring everything that I command you---your burnt offerings and your sacrifices, tithes, the contributions of your hand, and all your finest votive offerings that you might vow to \yahweh.
\end{translation}
\noindent
According to this passage, it is only after the Israelites completely conquer the land and find ``rest'' will the central cultic site be established.  As a matter of inner-biblical interpretation, it makes sense that some late redactor of 2 Sam 7 might note that David sought to build the \temple only after ``rest'' had been established and simply did not take into account the rationale given by Solomon in 1~Kgs 5:15.

Such editorial or redactional changes may be subsumed under the rubric of memory insofar as such changes come about in order to align some idea (or mnemonic node) within a the broader framework of the editor's social memory. In other words, we can account for this textual change by positing that the redactor's understanding of \emph{when} the \temple could be built was informed by the tradition of Deut 12:10--11 (or one like it). Thus, when the redactor read about David's attempt to build a \temple, he interpreted David's actions based on this other knowledge. Although it would be easy enough to circumvent the issue by noting that David is \emph{rebuffed} by \yahweh and that the \temple is ultimately built by Solomon, doing so leaves David somewhat vulnerable to critique. If the redactor thought David to have access to the ``Torah'' one must suppose that David either did not know Deut 12, did not care about Deut 12, or (as the redactor concluded) that ``rest'' \emph{had in fact} come about in Israel. As a way to rationalize the apparent contradiction with the fact that David engages in battle in the very next chapter, one might imagine that the \chronicler speculated that David only \emph{thought} that he had vanquished all his enemies or that there was ``rest,'' but that it was short-lived. If instead we locate the change at the pen of the \chronicler, the argument may be, in effect, reversed. By prioritizing the idea that ``rest'' would not be established in Israel until the reign of Solomon, the \chronicler is able to categorically dismiss the notion that David had already accomplished the task in 1 Chr 17. In fact, the two ideas are not mutually exclusive. The inclusion of 2 Sam 7:1b may have been both a late editorial addition \emph{and} a part of the \chronicler's \vorlage. In either case we can see the received tradition being adapted and fitted into a related, but distinct system of knowledge---social memory. The redactional process is---like rewriting---a process of memory.

The second question---how David's \emph{preoccupation} with his enemies turned into a divine \emph{disqualifier}---is a more difficult question to address. The transition from the practical exigency of ``not having time to build the \temple'' to the ideological position that David was disqualified from building the \temple based on his bloody past cannot be attributed to the same kinds of simple redactional processes as above. Accounting for such a fundamental reimagining of circumstances under which the \temple would be built requires us to consider not simply \emph{what} was remembered but \emph{how} and \emph{why} it was remembered. 

Knoppers notes that there are essentially three explanations for the \chronicler's disqualification of David. First, it could be that the \chronicler viewed David as ritually unclean from his bloodshed. This position, supported by Rudolph,\autocite[151]{rudolph1955} presumes that warfare disqualifies David from participating in cultic activities. As Knoppers points out, however, there is no indication with in the Hebrew Bible that this was the case, and moreover, David \emph{does} participate in other forms of cultic activity.\autocite[772]{knoppers2007} Second, David's bloodshed could be understood as an ``ethical lapse,'' i.e., that David bore guilt because for some wrong act, such as murder.%
    \footnote{This position is advocated by \cite[53]{dirksen_jsot1996} and \cite{kelly_jsot1998}. See also \cite{braun_jbl1976}.}
Yet, within chronicles, David is never described as incurring blood-guilt. The accusations of being a ``man of blood'' are made by Shimei in 2 Sam 16:8. The great faults of David known from Samuel--Kings (such as the death of Uriah) are not present in \chronicles.%
    \footnote{The one possible exception is when David takes a census and many people die as a result (1~Chr 21:1--13; see below). However, as Knoppers notes, we would more likely expect some kind of ``\emph{national} consequence'' (as we see with the census) if this were the case. For the \chronicler, David's prohibition from building the \temple is ``personal in nature.'' See \cite[772--773]{knoppers2007}.}
Finally, a number of scholars have suggested that it was David's martial activities that made him unfit---the \temple is place of peace, while David is a man of war.%
    \footnote{See especially \cite[396--397]{japhet1993} and \cite{mckenzie_graham-etal1999}.}
This interpretation has the advantage of taking the \chronicler's explanation in the most plain sense possible. But, here again, there is insufficient textual support in the rest of the Hebrew Bible, which does not associate warfare with the shedding of innocent blood, and nowhere in \chronicles is David accused of such. At the very least, this interpretation aligns most closely with the explicit connection between David and his military campaigns (i.e., not a ceremonial or ethical, shortcoming), and indeed, it is the rationale given in antiquity by Josephus (\ant 7.337).%  
    \footnote{\cite[773]{knoppers2007}.}
Knoppers adopts a modified version of this last position and implies that it was the shear volume of blood that David spilled which ultimately disqualified him, drawing on Pentateuchal notions that ``blood belongs to God.''\autocite[774]{knoppers2007} From my perspective, none of these explanations are particularly satisfying. 

The assumption of all these approaches has been that the \chronicler---with his distinct theological \emph{Tendenz}---sought to rationalize the fact that Solomon, and not David, built the \temple---to explain why it was that David \emph{did not} build the \temple. As Japhet writes:
\begin{quote}
    The portrayal of David as \emph{the} greatest of Israel's kings and the object of future hopes, the establishment of the \temple as the centre of Israel's religious experience, and the inalienable bond between the house of David and the city of Jerusalem with its \temple---all these had become theological cornerstones. The irrefutable fact that the \temple was built by Solomon rather than David did not cease to challenge theological thinking and demand explanation.\autocite[396]{japhet1993}
\end{quote}
\noindent
I would like to suggest, however, that the question the \chronicler sought to answer was not why David \emph{did not} build the \temple, but why David \emph{could not} build the \temple. The difference is subtle, but important. To be sure the rhetoric of Samuel--Kings assumes that David \emph{could have} built the \temple, and attempts to provide reasons for why he did not. The narrative of Samuel--Kings argues that David \emph{wanted} to build the \temple but ultimately  was unable to do so because ``rest'' had not been established in Israel. The \chronicler, on the other hand, assumes that the David \emph{did not} and \emph{should not have} built the \temple and sought to explain the reason why. The answer is simple: David could not build the \temple because \yahweh had appointed Solomon to do it. 

Implicit in this shift is the assumption that what \emph{did} happen
\emph{should} or \emph{must have} happened. God must have planned for Solomon to build the \temple because Solomon \emph{did} build the \temple. This shift marks an ideological transformation which almost certainly would have affected the way that the \chronicler interpreted the relevant material in his \emph{Vorlagen}. What Dirksen calls the \chronicler's ``\emph{ad hoc}'' reinterpretation of 1~Kgs 5:15 in fact reflects an ideology which is built on the \chronicler's belief that Solomon was \emph{supposed} to build the \temple. Operating within such an ideology, Deut 12:10 becomes a \emph{prediction of future events} rather than a description of the conditions under which \yahweh would establish a place for his name. \yahweh's decision to choose Solomon to build the \temple in 1 Chr 17 (and, by extension 2 Sam 7) is merely an elaboration on this plan set in place in Deut 12:10. That Israel had to wait for ``rest'' before \yahweh would established a place for his name, therefore, was tantamount to waiting for the reign of Solomon. 

This reading is supported by the way that the book of \chronicles ultimately presents David's involvement in the \temple's construction. In particular, the argument made by the \chronicler asserts that David did everything within his power to prepare the way for his son to successfully complete construction. Within an intellectual framework wherein David knew that he was disqualified from building the \temple, David shows his piety through restraint and deference toward his son's chosen status. In other words, the \chronicler has turned the whole discourse on its head. The implication that David was \emph{ineffectual} in his efforts to build a \temple (or, establish peace so the \temple could be build) is reinterpreted as an act of piety. Not only did David have good reasons for not building the \temple, but he was right \emph{not} to build it. The truth that the \temple could not be built until the the reign of the divinely appointed monarch is so fundamental that it is even encoded within the name of the king who would rule the kingdom during the requisite time of ``rest'' and ``peace'' (Heb. \hebrew{שָׁלוֹם}), namely, Solomon (Heb. \hebrew{שְׁלֹמֹה}). Where the narrative of Samuel--Kings can be read as an apology for David, the work of the \chronicler is a teleological interpretation of the fact that it was Solomon who ultimately built the \temple in Jerusalem.

Both the (putative) redactional insertion in 2 Sam 7:2b and the reformulation of David as disqualified from building the \temple, are examples of how received traditions are adapted and retrofitted into a broader contemporary system of knowledge. These differences reflect deeper and more fundamental reconfigurations of social memory that reimagined the mnemonic function of David. But the differences between the ways that \chronicles and Samuel--Kings remember David should not primarily be explained as novel \emph{portrayals} of David by an idiosyncratic author, but instead should be analyzed as reflecting the discourses and frameworks of the society that produced each. Thus, the \emph{portrayal} of David in \chronicles is merely the ``symptom'' of deeper, more fundamental shifts in how David was remembered during the \secondtemple period. 

%%%%%%%%%%%%%%%%%%%%%%%%%%%%%%%%%%%%
%% THE TEMPLE AS A SITE OF MEMORY %%
%%%%%%%%%%%%%%%%%%%%%%%%%%%%%%%%%%%%
\subsection{The Temple as a Site of Memory}

As with the figure of David, the \jerusalemtemple was already an important site of memory for ancient Israel long before the book of \chronicles was written and the memory of the \firsttemple was certainly influenced by this received tradition.

Already in the book of Deuteronomy the mythology surrounding the divine selection of Jerusalem and the uniquely ordained site of \solomonstemple had been well-established. This development is easily seen by contrasting the ways that the Covenant Code of \exod 20 commands (or, at least does not \emph{prohibit}) the Israelites to establish cult sites \hebrew{בְּכָל־הַמָּקוֹם אֲשֶׁר אַזְכִּיר אֶת־שְׁמִי} ``in every place that I commemorate my name'' (\exod 20:24) with Deuteronomy commands the Israelites to destroy all cult sites within the land and bring all their offerings to \emph{one} specific location: 
\begin{hebrewtext}
    \versenum{\deut 12:5}
    כִּי אִם־אֶל־הַמָּקוֹם אֲשֶׁר־יִבְחַר יְהוָה אֱלֹהֵיכֶם מִכָּל־שִׁבְטֵיכֶם לָשׂוּם אֶת־\\שְׁמוֹ שָׁם לְשִׁכְנוֹ תִדְרְשׁוּ וּבָאתָ שָׁמָּה׃
    \versenum{6}
    וַהֲבֵאתֶם שָׁמָּה עֹלֹתֵיכֶם וְזִבְחֵיכֶם וְאֵת מַעְשְׂרֹתֵיכֶם וְאֵת תְּרוּמַת יֶדְכֶם וְנִדְרֵיכֶם וְנִדְבֹתֵיכֶם וּבְכֹרֹת בְּקַרְכֶם וְצֹאנְכֶם׃
\end{hebrewtext}
\begin{translation}
    \versenum{\deut 12:5}
    But you shall seek the place that \yahweh your God will choose from among all your tribes as his dwelling to put his name there. You shall go there
    \versenum{6}
    and you will bring your burnt offerings there as well as your sacrifices, your tithes and the offerings of your hands, your votive gifts, your freewill offerings, and the firstborn of your cattle and flocks. 
\end{translation}
\noindent
It is hard not to speculate that the textual variants in \exod 20:10 are due to the implication that \yahweh may commemorate his name in multiple places, compared to its counterpart in \deut 12:5. This discomfort is illustrated in  \sampent's omission of \hebrew{כל} with the result that \hebrew{מָקוֹם} is conceptually singular (in \emph{the} place), while \lxx, Syriac, and the Targums all support the reading ``in every place.''%
    \footnote{In the case of the \sampent, the editor may have had in mind ``Samaria'' rather than ``Jerusalem,'' but the impulse is the same.}
Such a reading implies that the author had in mind an \emph{itinerant} cult site. The masoretic punctuation makes an effort to separate these ideas, emphasizing that the clause \hebrew{בְּכָל־הַמָּקוֹם} ``in every place'' modifies the following clause \hebrew{אָבוֹא אֵלֶיךָ וּבֵרַכְתִּיךָ} ``I will come to you and bless you'' rather than completing the action of the preceding \hebrew{מִזְבַּח אֲדָמָה תַּעֲשֶׂה־לִּי וְזָבַחְתָּ עָלָיו} ``you will make an earthen altar for me and make sacrifices upon it.'' Indeed, the first person form \hebrew{אַזְכִּיר} favors the former reading. Even so, \exod 20 seems to presuppose that \yahweh could or would cause his name to be commemorated in more than one place. On the other hand, the book of Deuteronomy states clearly that the the Israelite were only to bring their offerings to the \emph{the} place that \yahweh would choose from among the tribes. The historical reality of Israelite shrines and cult sites outside of Jerusalem during the monarchic period such as those from Dan, Arad, Beer-Sheeba, and others is well documented.%
    \footnote{For a concise overview of the archaeological evidence, see \cite[319--352]{king-stager2001}. See also \cite{edelman_barton-stavrakopoulou2010} and \cite[160--181]{smith2002}.}
While these sites were condemned as idolatrous by the deuteronomistic editor(s), there is no evidence to suggest that contemporaries of the \bce{seventh century} (or earlier) saw them as such.

The increased importance of the \jerusalemtemple brought about by the cult centralization efforts of Hezekiah and Josiah after the destruction of the Northern Kingdom similarly consolidated the religious memory of ancient Israel around Jerusalem and \temple of Solomon. Insofar as the real religious practices of Israel (putatively) became increasingly focused on the city of Jerusalem and \solomonstemple leading up to its destruction at the beginning of the \bce{sixth century}, ancient Israel's memory about other ``marginal'' religious practices was quite literally demolished through the intentional destruction of \translit{bāmôt} and other sacred sites through the religious reforms of Hezekiah and Josiah.%
    \footnote{\cite[182--199]{smith2002}; \cite[191--209]{romer2015}.}

Thus it was the sociopolitical \emph{reality} of the \jerusalemtemple's significance at the end of the \bce{sixth century}---brought about by the intentional religious reforms of Josiah---which informed the deuteronomistic editor's memory of earlier \yahwistic cult practices and which would form the basis for the \chronicler's perception of religious practice during the early monarchic period. Regardless of how centrally significant the \jerusalemtemple actually had (or had not) been during the early monarchic period or how successful the practical aspects of Hezekiah and Josiah's reforms had been, between the end of the \bce{seventh century} and the time of the \chronicler, the memory of the \temple had accrued meaning as a site of memory for the Golah community of Persian Yehud.%
    \footnote{I restrict my discussion here to the memory of Persian \emph{Yehud}, meaning the Golah community. The Samaritans, \translit{ʕam hāʔāreṣ}, and the Jewish garrison at Elephantine, presumably, had their own systems of memory which, while historically related, would have been distinct in this period.}
In other words, what is important for our purposes is not what the historical function of the \temple had been during the monarchic period, but the function the \jerusalemtemple played in the memory of the \chronicler and what kinds of social factors contributed to that function.

Consider, for example, the foundational role that the \temple played in the reestablishment of the Golah community, as presented in the closing verses of 2~Chronicles (2~Chr 36:22--23|| Ezra 1:1--4). According to these texts, the first task of the returnees was to construct the \emph{temple}. The construction project (as presented here) comes as a result of Cyrus' desire to build a \temple for \yahweh. This emphasis on the importance of the \temple's reconstruction aligns with sentiments from the other accounts of the \temple's construction in Haggai and Zechariah (albeit with important differences as well). The account in Haggai 1:1--4, in particular, evokes the same rhetorical question asked by David in 2~Sam~7 and 1~Chr~17 and asserts that the struggles that the Golah community was facing were tied to the fact that they had yet to reconstruct the \temple:
\begin{hebrewtext}
    \versenum{Haggai 1:1}
    ‏בִּשְׁנַת שְׁתַּיִם לְדָרְיָוֶשׁ הַמֶּלֶךְ בַּחֹדֶשׁ הַשִּׁשִּׁי בְּיוֹם אֶחָד לַחֹדֶשׁ הָיָה דְבַר־יְהוָה בְּיַד־חַגַּי הַנָּבִיא אֶל־זְרֻבָּבֶל בֶּן־שְׁאַלְתִּיאֵל פַּחַת יְהוּדָה וְאֶל־יְהוֹשֻׁעַ בֶּן־יְהוֹצָדָק הַכֹּהֵן הַגָּדוֹל לֵאמֹר׃ 
    \versenum{2}
    כֹּה אָמַר יְהוָה צְבָאוֹת לֵאמֹר הָעָם הַזֶּה אָמְרוּ לֹא עֶת־בֹּא עֶת־בֵּית יְהוָה לְהִבָּנוֹת׃  
    \versenum{3}
    וַיְהִי דְּבַר־יְהוָה בְּיַד־חַגַּי הַנָּבִיא לֵאמֹר׃ 
    \versenum{4}
    הַעֵת לָכֶם אַתֶּם לָשֶׁבֶת בְּבָתֵּיכֶם סְפוּנִים וְהַבַּיִת הַזֶּה חָרֵב׃
\end{hebrewtext}
\begin{translation}
    \versenum{Haggai 1:1}
    In the second year of Darius the King, in the sixth month, on the first day of the month, the word of \yahweh came by the hand of Haggai the prophet to Zerubbabel, son of Shealtiel, the governor of Judah and to Joshua, son of Jehozadak, the high priest, saying: 
    \versenum{2}
    ``Thus says \yahweh of Hosts: `This people says ``The time has not come (yet) to build a \temple for \yahweh'''''
    \versenum{3}
    And the word of \yahweh came by the hand of Haggai the prophet, saying:
    \versenum{4}
    ``Is it time for you to live in your (own) paneled houses while this \temple is in ruins?''
\end{translation}
\noindent
In all of the accounts of the returnees, the construction of the \temple is of central concern. In the case of 2 Chronicles and Ezra, the construction of the \temple was the impetus for the returnees to go back to the land and formed their central mission, as commanded by Cyrus, and for Haggai and Zechariah as a central competent of the establishment of the Golah community. 

The \secondtemple was constructed with reference to the memory of the \firsttemple (however inadequately at first) and with the purpose of filling the same socio-religious functions as the (remembered) \firsttemple. Unlike the idea of the Davidic dynasty (which was largely an abstraction during the early \secondtemple period), the memory of the \firsttemple could be concretely tied to not only a geographic location, but to real, physical structure that existed during the time of the \chronicler. Although the two temples were objectively distinct, from the perspective of cultural memory they were typologically the connected through what Barry Schwartz refers to as ``framing'' memories through ``keying.''%
    \footnote{\cite{schwartz_asr1996}; \cite{thatcher_thatcher2014}.}
The fact that the construction of the \secondtemple is often characterized as the ``rebuilding'' of the temple should clue us to the fact that the two structures share an overlapping mnemonic significance. Thus the significance and centrality of the \secondtemple was based on the memory of \firsttemple as remembered by Samuel--Kings and that understanding of \emph{what} the temple was and \emph{how} it should operate was ``framed'' by the memory of the \firsttemple. In other words the Golah community adopted the symbolic significance of the \firsttemple (``keying'' the \secondtemple to the First) and used it to construct the symbolic meaning of the \secondtemple.

But this is only half of the story. The \emph{reality} of the \secondtemple also affected the the memory of the \firsttemple for the \chronicler. Because memory construction is concerned with the present, the present circumstances and real-life practices of the \secondtemple period dictated  the way that the \firsttemple continued to be remembered during the \secondtemple period. Because the \secondtemple was keyed to the \firsttemple, contemporary discourses about the \secondtemple and the natural evolution that such discourses bring about within social memory likewise affect the way that the \firsttemple would ultimately be remembered. in Schwartz' terms, the \firsttemple not only provided a model \emph{for} the \secondtemple, but became a model \emph{of} the \secondtemple.
    \autocite[18]{schwartz2000}

\subsection{David and the Temple}

% TODO: Sum this up and transition to the next bit

\section{Reimagining Foundations: David's Census and Ornan's Threshing Floor}

As a final example of the mnemonic processes at work in the book of \chronicles, I will discuss the account of David's Census and the Threshing Floor of Araunah/Ornan found in 1 Chr 21:1--22:1 as a product of cultural memory. When compared to the parallel account in 2 Sam 24:1--24, the story as told in \chronicles is not simply a modified or ``cleaned up'' version of the story, but offers a narrative which functions in a distinct fashion compared to that of 2 Sam 24. This recontextualized narrative not only operates on different sets of ideological and theological presuppositions, but re-situates the narrative within the social memory by making magnetic connections to the major themes and ideas of the \chronicler's society.

The story of David's census in 2 Sam 24:1--24 begins with \yahweh inciting David to take a census of his army because \yahweh was angry with him. He dispatches his general, Joab, to take the census. Joab is successful and, after nine months, reports back to David the results of the census. Upon hearing the results, ``David's heart was stricken because he had numbered the people'' (Heb. \hebrew{וַיַּךְ לֵב־דָּוִד אֹתוֹ אַחֲרֵי־כֵן סָפַר אֶת־הָעָם}). David asks for forgiveness from \yahweh, who gives him three options for punishment. After the punishment of pestilence is executed against several thousand Israelites, \yahweh relents before striking Jerusalem and instructs his angel (Heb. \hebrew{מַלְאַךְ})---who at that moment was at the threshing floor of a man named Araunah (a Jebusite)---to cease destroying the people. Later, the prophet Gad instructs David to build an altar to \yahweh at the site of Araunah's threshing floor. David purchases the threshing floor, offers sacrifices to \yahweh, and ``the plague was averted from upon Israel'' (\hebrew{וַתֵּעָצַר הַמַּגֵּפָה מֵעַל יִשְׂרָאֵל}).

With the exception of a few key details and additions, the version presented in 1 Chr 21:1--22:1 follows the version in 2 Samuel very closely.%
    \footnote{In smaller textual matters one must be careful to remember that the \vorlage of the \chronicler was not identical to the Masoretic text. Indeed the textual plurality of the \lxx, \q{4}{Sam}{a}, and references in Josephus's \ant caution against assuming every difference between the \chronicler's account and that of 2 Samuel (MT) are the result of some change made by the \chronicler. See \cite[761--762]{knoppers2007}.}
The description of the census itself is mostly the same as the MT of 1 Samuel, though differences (such as the specific numbers reported) do occur. The most significant change to the first part of the story, however, is the attribution of David's incitement to take the census to an entity referred to as ``(a) Satan'' (Heb. \hebrew{שָׂטָן}):

\begin{hebrewtext}
    \versenum{1 Chr 21:1}
    וַיַּעֲמֹד שָׂטָן עַל־יִשְׂרָאֵל וַיָּסֶת אֶת־דָּוִיד לִמְנוֹת אֶת־יִשְׂרָאֵל׃
\end{hebrewtext}
\begin{translation}
    \versenum{1 Chr 21:1}
    (A) Satan arose against Israel and incited David to number Israel.
\end{translation}
\noindent
Scholars remain divided over whether \hebrew{שָׂטָן} should be understood as a simple indefinite noun ``an adversary,''%
    \footnote{%
        \cite{stokes_jbl2009};
        \cite[114--117]{japhet2009};
        \cite[370--390]{japhet1993}.}
or whether the absence of the definite article indicates that by the time of the Chronicler, Satan referred to a malevolent spirit which prefigured the more developed, personified ``Satan'' found in the New Testament.%
    \autocite[4--5]{rollston_keith-stuckenbruck2016}
The most common usage of the term \hebrew{שָׂטָן} in the Hebrew Bible refers to human adversaries and accusers (Num 22:22, 32; 1~Sam 29:4, 2~Sam 19:23; 1~Kgs 5:18, 11:14, 23, 25; Ps 38:21, 71:13, 109:4, 6, 20, 29). However, the figure \hebrew{הַשָּׂטָן} (with definite article) in both the prologue to Job (Job 1--2) and Zech 3:1--2 appears as a celestial figure to whom Yahweh speaks directly.
    \footnote{This notion is more clear in Job, where \hebrew{הַשָּׂטָן} is described in the heavenly courts and is described as having supernatural powers over the health and prosperity of those on the Earth. On the other hand, the reference in Zechariah is somewhat ambiguous. Zech 3:1 reads: \hebrew{ וַיַּרְאֵנִי אֶת־יְהוֹשֻׁעַ הַכֹּהֵן הַגָּדוֹל עֹמֵד לִפְנֵי מַלְאַךְ יְהוָה וְהַשָּׂטָן עֹמֵד עַל־יְמִינוֹ לְשִׂטְנוֹ}, ``And he showed me Joshua, the high priest standing before the angel of Yahweh, and \translit{haśśāṭān} was standing on his right (side) to accuse him.'' The antecedent of ``his'' in ``his right(side)'' is unclear. If ``his'' refers to the \hebrew{מַלְאַךְ} Yahweh, then \hebrew{הַשָּׂטָן} likely refers to some kind of spiritual being. However, it is possible that ``his'' refers to Joshua, and that \hebrew{הַשָּׂטָן} should be understood as a human adversary.}
Proponents of reading \hebrew{שָׂטָן} as the personal name of a malevolent spirit argue that the absence of the definite article indicates that the idea of \emph{the} \hebrew{שָׂטָן} of Job and Zechariah had evolved into a fully personified Satan by the time of the Chronicler.%
    \footnote{%
        \cite[216--217]{braun1986};
        \cite[107]{coggins1976}. Rollston also finds this reading compelling, though, not without difficulties. See 
        \cite[4--5]{rollston_keith-stuckenbruck2016}.}
Additionally, while the \lxx hails from a later chronological horizon than Chronicles, it is worth noting that the translator used the indefinite substantive \greek{διάβολος} to translate \hebrew{שָׂטָן}---the same term used in Job and Zechariah (also, Ps 108:6) \emph{with} a definite article---which gives some indication that, in the mind of the translator, these passages likely referred to the same entity.% 
    \footnote{Elsewhere the \lxx renders the nominal forms of \hebrew{שָׂטָן} with the feminine \greek{διαβολή} or, in the case of 1~Kgs 11:14, simply in transliteration as \greek{σαταν}. It should be noted, however, that Esth 7:4 and 8:1 render the Hebrew √ṣrr as the masculine \greek{διάβολος} as well.} 

 Critics of this view, however, have pointed to the fact that in other cases in the Hebrew Bible, generic nouns that are treated as personal names or titles often \emph{do} retain the definite article.%
    \footnote{\cite[114--117]{japhet2009};
        \cite[370--390]{japhet1993}. Japhet, for example, notes that direct references to the Canaanite deity Baʿal are always accompanied by the definite article. In every instance, the name/title \hebrew{בַּעַל} is made grammatically definite whether by adding the definite article, pronominal suffixes, or being in construct with an explicitly definite noun. 
        \cite[115]{japhet2009} citing 
        \cite[§126d]{gkc}.}
In such a case, \hebrew{שָׂטָן} should simply be understood as an indefinite noun, ``an accuser'' and may be understood as a human antagonist of David.%
    \footnote{See 
        \cite{stokes_jbl2009};
        \cite[114--117]{japhet2009}; 
        \cite[370--390]{japhet1993}.} 
For our purposes, it is not essential that we know for certain how \hebrew{שָׂטָן} was intended to be understood by the \chronicler. What \emph{is} important, however, is that the author of \chronicles plainly understood the mechanisms at work differently than the author of 2~Sam 24. Chief among these differences is the fact that the \chronicler shifts the incitement of the census away from \yahweh and onto a third party. One of the more perplexing aspects of the 2 Sam 24 narrative is that \yahweh seems function as an antagonist to David. It is \yahweh's anger which prompts \yahweh to ``incite'' David to take the census, for which he is punished. At least to modern readers, the resulting narrative appears to be one of a sort of divine ``entrapment'' of David that makes \yahweh seem rather ``mercurial.''%
    \autocite[4]{rollston_keith-stuckenbruck2016}
That \yahweh would incite David to sin then punish him for it was understandably confusing for the \chronicler, and equally confusing is why \yahweh would punish Israel for taking the census to begin with. But within the narrative discourse, there is no hint of confusion. All the characters operate as if \yahweh's response is perfectly reasonable and seem to know the proper actions to take to avert the disaster. Joab, for example, seems reticent about David's request to take the census:
\begin{hebrewtext}
    \versenum{2 Sam 24:3}
    וַיֹּאמֶר יוֹאָב אֶל־הַמֶּלֶךְ וְיוֹסֵף יְהוָה אֱלֹהֶיךָ אֶל־הָעָם כָּהֵם וְכָהֵם מֵאָה פְעָמִים וְעֵינֵי אֲדֹנִי־הַמֶּלֶךְ רֹאוֹת וַאדֹנִי הַמֶּלֶךְ לָמָּה חָפֵץ בַּדָּבָר הַזֶּה׃
\end{hebrewtext}
\begin{translation}
    \versenum{2 Sam 24:3}
    And Joab said to the king, ``May \yahweh your God increase the people a hundred times while the eyes of my lord the king may see (them). But why does my lord the king desire this thing?''
\end{translation}
\noindent
He seems to know that taking the census could be risky. 

What these two insertions reveal is that for the \chronicler, the narrative logic of 2 Samuel did not work within his own set of social frameworks. Instead of reading the \chronicler's attribution of incitement to \hebrew{שָׂטָן} as an attempt at ``absolving'' or explaining away \yahweh's actions because they were offensive to the \chronicler, perhaps it is better to think about the \chronicler attempting to fit the story of 2 Sam 24 into a different theological framework and into a different system of narrative logic. From this perspective, it was not that the \chronicler was scandalized by \yahweh's incitement of David but rather a reflection on the fact those actions lacked a sort of ``theological verisimilitude'' (my term) within the worldview of the \chronicler.%
    \footnote{I suspect that most religious laypeople for whom 2 Sam 24 is scripture, too, would find \yahweh's portrayal in this text out-of-character with the way that God is portrayed elsewhere in their Bibles (not least, in \chronicles!).}
Likewise where the narrative in 2 Sam 24 assumes that the reader understands why Joab would question David about taking the census, the \chronicler supplies the details for Joab's reservations:
\begin{hebrewtext}
    \versenum{1 Chr 21:3}
    וַיֹּאמֶר יוֹאָב יוֹסֵף יְהוָה עַל־עַמּוֹ כָּהֵם מֵאָה פְעָמִים הֲלֹא אֲדֹנִי הַמֶּלֶךְ כֻּלָּם לַאדֹנִי לַעֲבָדִים לָמָּה יְבַקֵּשׁ זֹאת אֲדֹנִי לָמָּה יִהְיֶה לְאַשְׁמָה לְיִשְׂרָאֵל׃
\end{hebrewtext}
\begin{translation}
    \versenum{1 Chr 21:3}
    And Joab said, ``May \yahweh increase his people a hundred times. Are they not all, my lord the king, servants of my lord? Why does my lord seek this? Why shall he bring guilt on Israel?''
\end{translation}
\noindent
Here the \chronicler makes explicit what is implicit in 2 Sam 24---that census taking carries a risk. Although a number of scholars have speculated about what the rationale had been for the original author(s) of 2 Sam 24, the \chronicler infers a plausible rationale from his own theological and ideological frameworks.%
    \footnote{See, especially the discussion in \cite[512--514]{mccarter1984}.}
The same basic tendency to clarify the narrative logic of several apparently inconsistent portions of 2 Sam 24 can be seen throughout the narrative.%   
    \footnote{For example, the angel of \yahweh appears suddenly in the Samuel narrative and seems to be carrying out some violence against the land. The \chronicler introduces the angel as explicitly sent by \yahweh. More than likely there is some kind of textual corruption in the Samuel account, so it is not certain that every additional detail that the \chronicler supplies is original to the \chronicler.}

Perhaps the most significant addition to the \chronicler's account, however, comes at the very end of the story with the identification of Araunah/Ornan's threshing floor with the future site of the \jerusalemtemple. In both accounts David is instructed by the prophet Gad to build an altar to \yahweh and offer sacrifices at the threshing floor of Araunah/Ornan.%
    \footnote{On the threshing floor as  sacred space and location for \temple-construction, see 
        \cite[125--144]{waters2015}.}
The sacrifices that David makes in the 2~Samuel serve as a means to placate \yahweh and avert continued punishment. The \chronicler, clearly, was perplexed by this practice. As Japhet notes, implicit in the \chronicler's desire to explain this practice is the presupposition that the cult of \yahweh was centralized into a single location. Although the \temple (the permanent, legitimate site for worship) had yet to be build, the tabernacle provided a singular (if itinerant) site of legitimate worship.%
    \footnote{\cite[389]{japhet1993}; \cite[760--761]{knoppers2007}}
To addresses the problem of how/why David made sacrifices outside the singular cult site, the tabernacle, the \chronicler rationalizes that because the tabernacle was in Gibeon, it was therefore inaccessible to him in a timely fashion:
\begin{hebrewtext}
    \versenum{1 Chr 21:29}
    וּמִשְׁכַּן יְהוָה אֲשֶׁר־עָשָׂה מֹשֶׁה בַמִּדְבָּר וּמִזְבַּח הָעוֹלָה בָּעֵת הַהִיא בַּבָּמָה בְּגִבְעוֹן׃ 
    \versenum{30}
    וְלֹא־יָכֹל דָּוִיד לָלֶכֶת לְפָנָיו לִדְרֹשׁ אֱלֹהִים כִּי נִבְעַת מִפְּנֵי חֶרֶב מַלְאַךְ יְהוָה׃
\end{hebrewtext}
\begin{translation}
    \versenum{1 Chr 21:29}
    Now, the tabernacle of \yahweh which was built by Moses in the wilderness and the altar of burnt offering at that time was at the high place at Gibeon
    \versenum{30}
    and David was not able to go before it to inquire of God because he was terrified of the sword of the angel of \yahweh.
\end{translation}
\noindent
From a memory perspective, the issue is not necessarily that it was \emph{offensive} for David to make sacrifices outside of the central cult site, but a difference set of presuppositions about how proper worship \emph{should} work. The \chronicler does not change the fact that David offered sacrifices outside the tabernacle, but provides a rationale for why it was expedient for David to bend the rules. The \chronicler gives David the benefit of the doubt and explains his actions based on the \chronicler's system of theological rationale.

But the \chronicler goes further than simply rationalizing David's actions and completely recontextualizes this narrative within the framework of Israel's history:
\begin{hebrewtext}
    \versenum{1 Chr 22:1}
    וַיֹּאמֶר דָּוִיד זֶה הוּא בֵּית יְהוָה הָאֱלֹהִים וְזֶה־מִּזְבֵּחַ לְעֹלָה לְיִשְׂרָאֵל׃ 
\end{hebrewtext}
\begin{translation}
    \versenum{1 Chr 22:1}
    And David said, ``This is it; the house of \yahweh. This is [the] altar for burnt offerings for Israel. ''
\end{translation}
\noindent
Thus the \chronicler presents Gad's command to build an altar at the threshing floor of Araunah/Ornan as tantamount to commanding that David establish the site of the new \temple.  

The reimagining of David's altar at Araunah/Ornan's threshing floor as the foundation of the \temple in Jerusalem reflects the magnetic tendency between major sites of memory. Note that the purpose of this pericope in \chronicles is no longer simply a (weird) story about how David avoided disaster by \yahweh's mercy, but about how \yahweh indicated to David where the \temple would be constructed. The common translation of \hebrew{זֶה הוּא בֵּית יְהוָה} as ``Here shall be the house of the \textsc{Lord} God'' (NRSV), however, misses the force of David's declaration. David does not say that the \temple \emph{will be} ``here,'' but performatively declares that the location of ``this altar'' \emph{is now the \temple} and that \emph{this altar is THE altar}. By doing this, the \chronicler has explicitly credited the figure of David to the establishment of \solomonstemple. It is not a coincidence, therefore, that the rest of 1 Chr 22 describes the preparations that David makes for the construction of the \temple, David's admonition to Solomon to build the \temple, and a command to the leaders of Israel to support Solomon in this endeavor. Thus the narrative of David's census has becomes the means by which the connection between the two major sites of memory in the book of \chronicles---David and the \temple---becomes explicit and within the social memory of the \chronicler, the two sites of memory have become more deeply entangled with one another.

The magnetic process was not limited to the mnemonic sites of David and the \temple. Later in \chronicles when Solomon is beginning to build the \temple, the \chronicler makes an explicit the connection between the site of the new \temple and ``Mt. Moriah'':
\begin{hebrewtext}
    \versenum{2 Chr 3:1}
    וַיָּחֶל שְׁלֹמֹה לִבְנוֹת אֶת־בֵּית־יְהוָה בִּירוּשָׁלִַם בְּהַר הַמּוֹרִיָּה אֲשֶׁר נִרְאָה לְדָוִיד אָבִיהוּ אֲשֶׁר הֵכִין בִּמְקוֹם דָּוִיד בְּגֹרֶן אָרְנָן הַיְבוּסִי׃
\end{hebrewtext}
\begin{translation}
    \versenum{2 Chr 3:1}
    Solomon began to build the \temple of \yahweh in Jerusalem on Mt. Moriah where [\yahweh]%
        \footnote{Following \lxx: 
        \greek{οὗ ὤφθη κύριος τῷ Δαυιδ πατρὶ αὐτοῦ}.
        The Hebrew does not make sense as written.}
    appeared to David his father;
    at the place where David had established; 
    at the threshing floor of Ornan the Jebusite.
\end{translation}
\noindent
Although this is the only reference to the \emph{mountain} of Moriah in the Hebrew Bible, the \emph{land} of Moriah is only mentioned one other time in the Hebrew Bible as the setting of the Aqedah (Gen 22). In the well-known story of the near-sacrifice of Isaac Abraham was instructed by \yahweh to bring his only son, Isaac,  to the land of Moriah and to sacrifice him (on a mountain!). This reference to Moriah appears to be another example of the magnetic processed between the core sites of memory within a community's social memory. In the same way that David's sacrificial acts at the threshing floor of Ornan---in the memory of the Chronicler---connected the idea of David as \temple-founder with the \jerusalemtemple, so too the near-sacrifice of Isaac, by its geographic association with ``Moriah'' converges on the site of the \jerusalemtemple and the near-sacrifice of Isaac becomes a prototype for the sacrificial cult.%
    \footnote{In fact, \vermes makes this point explicit and traces the tradition into early Christianity. See \cite[204--211]{vermes1961}. This connection has also been fruitfully analyzed by \cite{kalimi_htr1990}; \cite[190--191]{kalimi_jnes2009} and \cite{amit_brenner-polak2009}.}

Thus the story of David's census was not only reimagined within a different theological or ideological system than the parallel account in 2 Sam 24, but it was resituated within the social memory of the \chronicler as the crossroads of three major metanarrative arcs within remembered past of ancient Israel---the Davidic Monarchy, the \jerusalemtemple, and the Abrahamic covenant.


% >>> FIXME: START: >>>>>>>>>>>>>>>>>>>>>>>>>>>>>>>>>>>>>>
It is conspicuous to me, for example, that although all the accounts operate within the Persian administrative context, the effort to reconstruct the \temple in \chronicles and Ezra is instigated by Cyrus as a part of his benevolent edict. Because \temple-construction was thought to be one of the central responsibilities of kings in the ancient world, it is understandable that it is \emph{Cyrus} who gives the command to rebuild the \temple in Jerusalem.%
    \footnote{On \temple construction as a royal activity, see
        \cite{kapelrud_orientalia1963};
        \cite{petersen_cbq1974};
        \cite{laato_zaw1994}.

        It is also worth noting that the Persian Empire \emph{did} in fact commission the reconstruction of religious and cultural apparatuses, as evidenced by the Egyptian Udjahorresnet. See 
            \cite{lloyd_jea1982}. On the relationship of Udjahorresnet to Ezra and Nehemiah, see 
            \cite{blenkinsopp_jbl1987}.
        Whether or not this was a part of a broader practice of so-called imperial authorization of local customs remains a matter of debate. See especially
            \cite{frei_frei1984};
            \cite{frei_watts2001}.}
These accounts lack any hint of nationalistic aspirations for the return---the returnees are Persian subjects working at the behest of the benevolent and pious Persian king (2~Chr 36:22--23|| Ezra 1:1--4). Note further Zerubbabel's response in Ezra 4:3 to the \translit{ʕam hāʔāreṣ} who wished to assist in the \temple's reconstruction:
\begin{hebrewtext}
    \versenum{Ezra 4:3}
    וַיֹּאמֶר לָהֶם זְרֻבָּבֶל וְיֵשׁוּעַ וּשְׁאָר רָאשֵׁי הָאָבוֹת לְיִשְׂרָאֵל לֹא־לָכֶם וָלָנוּ לִבְנוֹת בַּיִת לֵאלֹהֵינוּ כִּי אֲנַחְנוּ יַחַד נִבְנֶה לַיהוָה אֱלֹהֵי יִשְׂרָאֵל כַּאֲשֶׁר צִוָּנוּ הַמֶּלֶךְ כּוֹרֶשׁ מֶלֶךְ־פָּרָס
\end{hebrewtext}
\begin{translation}
    \versenum{Ezra 4:3}
    And Zerubbabel and Joshua and the remaining heads of the families of Israel said to them, ``It is not your place, but ours, to build a \temple for our God. But we alone will build (it) for \yahweh, the God of Israel, \emph{as Cyrus the king of Persia commanded us}.'' (Emphasis added)
\end{translation}
\noindent
On the other hand, the accounts of Haggai and Zechariah, although not overtly nationalistic or anti-imperial, focus on the figure of Zerubbabel as a semi-royal, Davidic figure charged with the rebuilding of the \temple (along with the high priest, Joshua) \emph{by \yahweh}. In other words, this royal responsibility was taken on by Zerubbabel and Joshua (the high priest) and \emph{not} by the Persian king, which has lead some scholars to suggest that Zerubbabel was viewed as a royal-messianic figure. 


All of this is to say that during the \secondtemple period the \temple itself was overloaded with significance. The historical reality of Zerubbabel's failure to come into his kingship (if  indeed this was what Haggai and Zechariah allude to), meant that it was the \temple and not the kingship that provided continuity between the present and the remembered past. The significance of \solomonstemple in the memory of \secondtemple Judaism, is a reflection of the significance of the \secondtemple in the lived experience of the Golah community. This is not to suggest that \solomonstemple would not have been significant had the \secondtemple not been built, only that the \emph{presence} of the \secondtemple augmented and affected the memory of the former \temple in the memory of \secondtemple Judaism.
% >>> END: >>>>>>>>>>>>>>>>>>>>>>>>>>>>>>>>>>>>>>

% >>>>>>>>>>>>>>>> FIXME:
The questions that the book of \chronicles seeks to answer and the assumptions which it carries are different than that of Samuel--Kings or even Haggai and Zechariah and affect the way that the \chronicler not only read and interpreted his sources, but also the way that he situated various sites of memory with respect to one another. Thus David's role in the construction of the \temple is not isolated to the question of why he could not build it, but extends to the way that David, as site of memory, relates to the \temple \emph{as a site of memory}.

Although both David and the \temple may be thought of as discrete sites of memory, it is important to remember that they participate in a \emph{network} of symbolic social meaning. Thus, ``discrete'' here does not mean ``isolated.'' For example, as I have already demonstrated, the figure of Zerubbabel is connected both to the construction of the \secondtemple as well as to the figure of David, who himself is related to the construction of the \firsttemple, albeit not as its builder. Moreover, not all sites of memory carry the same weight of significance within a particular symbolic system. In other words, not all sites of memory are created equal; David is a much more prominent and potent node within the social memory of ancient Israel than was Shimei, his critic. Though they participate within the same discursive space---even in the Bible---David is a more significant symbol. Likewise the \temple's symbolic significance far outweighs that of the \translit{bāmôt}, despite the fact that---functionally---their social function was similar.
% <<<<< END

% >>>>>>>> FIXME: 
Moreover, both Haggai and Zechariah betray certain images that point toward some kind of semi-royal or messianic hope associated with him, most notably in Haggai's reference to \yahweh making Zerubbabel ``like a signet ring'' (Hag~2:23). Although the text does not make it explicit that Zerubbabel was viewed as a semi-royal figure, it is difficult to read Haggai's use of the term ``signet ring'' as anything but an allusion to Jer 22:24 which describes ``Coniah'' as a signet ring on \yahweh's hand which would be removed and cast into exile:
\begin{hebrewtext}
    \versenum{Jer 22:24}
    חַי־אָנִי נְאֻם־יְהוָה כִּי אִם־יִהְיֶה כָּנְיָהוּ בֶן־יְהוֹיָקִים מֶלֶךְ יְהוּדָה חוֹתָם עַל־יַד יְמִינִי כִּי מִשָּׁם אֶתְּקֶנְךָּ׃ 
    \versenum{25}
    וּנְתַתִּיךָ בְּיַד מְבַקְשֵׁי נַפְשֶׁךָ וּבְיַד אֲשֶׁר־אַתָּה יָגוֹר מִפְּנֵיהֶם וּבְיַד נְבוּכַדְרֶאצַּר מֶלֶךְ־בָּבֶל וּבְיַד הַכַּשְׂדִּים׃
\end{hebrewtext}
\begin{translation}
    \versenum{Jer 22:24}
    As I live---an utterance of \yahweh---even if Coniah, son of Jehoiakim, king of Judah, were a signet ring upon my right hand, even from there I would tear you off
    \versenum{25}
    and I would give you into the the hand of those who seek your life and into the hand of those you dread and into the hand of Nebuchadnezzar, the king of Babylon and into the hand of the Chaldeans.
\end{translation}
\noindent
In Haggai, this negative image of \yahweh removing and discarding the ``signet ring'' is used positively, ostensibly, to mark Zerubbabel as \yahweh's agent on earth and, probably, as a royal messianic figure.%
    \footnote{See
        \cite[71--103]{blenkinsopp2013};
        \cite[2:281--284]{vonrad1962};
        \cite[187]{redditt_interpretation2007}.}
The fact that Zerubbabel is referred to as the ``servant'' (Heb. \hebrew{עֶבֶד}) evokes the way that David was characterized as \yahweh's servant and supports this general conclusion.

%% <<<<<<<<<<<<< END

% >>>>>>>>>>> FIXME:
Thus when we consider the relationship between David and the \temple in the book of \chronicles, it should come as no surprise that these two large, highly connected nodes within the cultural memory of Persian Yehud have continued to entangle themselves with the major metanarratives and ideas of the \chronicler's society. Indeed, we have already discussed at length the way that David's role in the construction of the \temple expanded in the memory of the \chronicler \visavis Samuel--Kings. This expansion can be understood as a process of mnemonic magnetism whereby the most typologically significant and highly-connected node of memory about the remembered political kingdom of Israel (David) converges with the highly-connected and typologically significant center of Judahite identity during the \secondtemple period (the \temple). 
% <<<<<<<<<<<<<<<<<

% >>>>>>>>> FIXME: Find a place for this
Despite David's assertion that Solomon be the one who builds the \temple, the \chronicler credits David with making all the preparations and providing the bulk of the necessary building supplies for its construction. While Solomon would provide the labor, not only was the \emph{idea} of building the \temple David's, but he financed the operation \hebrew{בְעָנְיִי} ``with great pains'' (lit. ``in my oppression''; 1~Chr 22:14). In the mind of the \chronicler, therefore, Solomon may have been the one to build the \temple, but it was David who wrote, directed and produced the project. 
% <<<<<<<<<


\section{Conclusion}

Although traditional approaches to the book of \chronicles have tended to focus on the ideological and theological agenda of the the \chronicler and the changes that he made over-and-against his \vorlage as innovations, in this chapter I have shown how a memory approach both problematizes this characterization and offers a more robust understanding of the kinds of processes which affect how societies relate to their remembered past and how those systems of remembrance can change over time. I have argued that the concept of ``sites'' of memory offer a robust way to think about the way that specific ideas become locations for engaging with social discourses and that these sites operate within complex systems of symbolic meaning. Although such sites are not bound to reflect ``history'' in the modern sense, but the processes by which these sites of memory interact and evolve can be observed and reasoned about FIXME: \emph{historically}. The kinds of changes that social and cultural memory exhibits diachronically can be tied to historical social and cultural changes.

The book of \chronicles exhibits numerous changes over-and-against its putative \vorlage and those changes, I have argued, can be attributed not only to the genius of the \chronicler, but more fundamentally to the social context in which the \chronicler lived. The individual and idiosyncratic reinterpretations that the \chronicler provides are inextricably linked to the to his social frameworks. Characterizing this reimagining simply as an effort by one individual to ``smooth over'' problematic aspects of Samuel--Kings or as an effort to explain difficult sections of his \vorlage sells-short the social processes that contributed to the much larger reconfiguration of Israel's remembered past that the book of \chronicles represents.
