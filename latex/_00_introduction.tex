% !TeX root = ../dissertation.tex

\chapter*{Introduction}
\addcontentsline{toc}{chapter}{Introduction}


% Genesis of the project
The primary goal of this project is to introduce the theoretical framework of memory studies to the topic of \rwb and to illustrate the usefulness of social and cultural memory theory for describing the relationships between \rwb texts, their putative \vorlagen, and the societies that produced them. As I was considering topics for this dissertation, I recalled my dissatisfaction with the characterization of \rwb in much of the scholarly literature as focused primarily on biblical interpretation. In particular, I was dissatisfied with the way that many scholars characterized \rwb as primarily a method for ancient authors to explain the biblical text---as if \rwb texts were a form of early biblical commentary. At about this same time, I began working on a seminar paper which utilized memory theory, and it occurred to me that much of the language of memory theory could be usefully applied to the \rwb texts in such a way that both acknowledged the derivative nature of \rwb and also allowed for the creative license that is so characteristic of \rwb texts. So it was to my great delight that---while perusing the book room at SBL in Atlanta, GA---I stumbled upon George Brooke's (then just-published) article ``Memory, Cultural Memory and Rewriting Scripture'' which suggested that memory theory could be a fruitful avenue for the discussion of \rwb, although he made no attempt to apply the theory himself.%
    \autocite{brooke_zsengeller2014}

% What is the role that you seek to address? What about traditional approaches to rewritten Bible is not sufficient?
Thus this dissertation is a first step toward a broader theory of \rwb that frames \rwb texts by their participation in social discourses about their remembered past and broader cultural memory. Although I do not contest the idea that portions of \rwb texts participate in ``biblical interpretation,'' I argue that characterizing these texts as \emph{primarily} or \emph{essentially} focused on explaining the biblical text limits the ways that scholars can talk about how these texts may have \emph{functioned} in the ancient world. The characterization of \rwb texts as primarily focused on their \vorlagen is codified in the term \rwb itself, which implicitly assumes that the process of rewriting had something to do with the status of the underlying text as both a written \emph{text} and as ``Bible.'' This assumption is not lessened by substituting ``rewritten scripture'' or ``rewritten scriptural texts.'' Implicit in all these terms is the assumption that the rewriting process was fundamentally tied to the authority or sacredness of the underlying text \emph{as a text}. To put it another way, this approach assumes that the \rwb texts meant to tell its readers \emph{how to read the Bible}.

% How does Social/Cultural Memory address these concerns?
In the following chapters, I argue that treating \rwb texts as primarily focused on explaining the biblical text is an insufficient model for explaining not only \emph{what} \rwb texts are and \emph{how} they came about, but also \emph{why} they may have come about. As a means to address some of these questions, I suggest that memory studies provides a rich taxonomy and robust model for describing the production of \rwb texts as cultural products that participated in the construction of cultural memory through their writings by receiving and adapting traditions of Israel's remembered past (in part represented by the biblical text). Memory theory offers models to describe the reception of traditions into contemporary discourses, the effects of contemporary discourses on the remmebered past , and the construction of new memories that inform future generations. \rwb, I argue, can benefit from using this language to broaden our discussion about what roles \rwb texts might have played in antiquity. This approach can both take seriously the fact that \rwb texts utilized the biblical text as a major source of tradition without limiting our discussions of \rwb texts to their relationship with the Bible or its scriptural predecessors.


%%%%%%%%%%%%%%%%%%%
% SUMMARY SECTION %
%%%%%%%%%%%%%%%%%%%
% Summary of the arguments and how they advance your thesis

% Overview
Because this dissertation involves the intersection of two abstract ideas (\rwb and memory theory), I have included two introductory chapters: 1) a literature review summarizing the history of scholarship and the \emph{status quaestionis} of \rwb studies, and 2) an introduction to memory studies, including a history of scholarship that explains my particular approach to the topic of memory. These chapters also offer an explanation of how I think memory studies can fit into the discussion of \rwb and what shortcomings it can help to address. The core of this dissertation centers around three case studies, each of which focuses on a single so-called \rwb text: \chronicles, \ga and \jub, in that order. Each case study focuses on a different aspect of memory that seemed well suited to the particular text. Because of this approach, the kinds of observations that I have made about one text can---in all probability---be made about the others, even if I have not made the connection explicitly. For example, although I only discuss ``genre'' in the chapter on \ga, both \chronicles and \jub exhibit generic peculiarities that could be framed within a discussion of social memory. Likewise, although I only discuss the effect of memory on ``practice'' with respect to \jub, this should not be understood as a dismissal of any practical effects of memory in \chronicles or \ga. All the same, I have attempted to order the chapters in such a way that each case study builds methodologically on the former chapters.

% Summary: Chapter 1
Chapter one provides an overview of the \emph{status quaestionis} within \rwb studies and highlights some of the difficulties that still remain within the discussion, particularly with regard to whether the term should be applied as a genre, process, or something else. I offer an extended treatment of \vermes' seminal work \citetitle{vermes1961}, focusing on the way that \vermes used the term and tracing its development to the present, including its modified forms (Rewritten Scripture, Rewritten Scriptural Texts, etc.). The rest of the chapter is dedicated to addressing two current issues in \rwb scholarship, namely, the characterization of \rwb as either a category, process, or genre, and the definition of \rwb's boundaries. I argue that although few scholars use the term \rwb in the same way that \vermes did, \vermes' focus on the \emph{exegetical} qualities of \rwb has remained central to \rwb scholarship. The difficulties that this chapter addresses in defining \rwb, however, gives some reason for us to question whether such a focus is warranted. Thus I propose that memory theory may be one fruitful avenue to engage with these difficulties.

% Summary: Chapter 2
Chapter two introduces the core figures and ideas of social, collective, and cultural memory theory (I refer to these related concepts collectively as ``memory theory''). Beginning with an extended treatment of the works of \Halbwachs (the ``father'' of memory studies), I outline the major theoretical trajectories within the discipline, such as those championed by Yosef \yerushalmi, Pierre Nora, Jan and Aleida Assmann, and Barry Schwartz. I also discuss some of the recent criticisms of memory studies, particularly that of Kerwin Klein. I note that although memory studies has recently made inroads into the field of biblical studies (and more so within New Testament studies), most applications have focused on diachronic and historical issues (``cultural memory'') and rarely interact with the \emph{vast} secondary literature on memory. I propose that \rwb texts are prime candidates for analysis from a memory perspective, and that \rwb texts can be understood as a set of snapshots revealing the ways that the collective memory of \secondtemple Judaism was shaped by its remembered past and how contemporary frameworks, in turn, affected how the past was remembered and transmitted to the next generation.

% Summary: Chapter 3
In chapter three, I treat the book of \chronicles as an example of \rwb from the perspective of memory studies. I begin with \chronicles primarily because there has already been a considerable amount of work done on the book from a memory perspective. This fact is important because, although \chronicles is frequently treated as an exemplar of \rwb by those who study \rwb, the topic of \rwb is not frequently addressed by those who primarily study \chronicles. The fact that \chronicles has fruitfully been treated from the perspective of memory, therefore, bolsters my contention that other \rwb texts could benefit from the same. The primary purpose of this chapter is to illustrate two of the core ideas of memory theory: 1) that memory is \emph{constructed} through social processes and 2) that the remembered past is inextricably tied to the social frameworks of the present.  As such, this chapter is meant to represent a conventional approach to memory studies. In particular, I discuss the role that ``sites'' of memory play in evolving discourses about the remembered past and how social memory can be modeled as a complex network of symbolic meaning. Building on this idea, I suggest that these networks function like scale-free networks and that the principle of preferential attachment (a property of scale-free networks) may serve to explain the phenomenon of ``magnetism'' as described by Ehud Ben Zvi. Drawing heavily on the work of Ben Zvi and Pierre Nora, I argue that the book of \chronicles engages with the major sites of memory from Israel's remembered past and draws together its major metanarratives as a way to make sense of the events and circumstances of the \secondtemple period.

% Summary: Chapter 4
In chapter four, I begin my treatment of \rwb texts which have largely \emph{not} been approached from a memory perspective, beginning with \ga. In my treatment of \ga, I focus on the three-step cycle of memory reception and transmission: 1) the reception of cultural memory, 2) the reshaping of memory by contemporary social frameworks, and 3) the active construction, codification, and reintegration of memory for future transmission. Through an analysis of \ga's sources, I argue that the cultural memory that undergirds \ga was broader than just Genesis. This finding bears on the characterization of \ga as a rewriting of \emph{Genesis}. Instead, I suggest that \ga represents a rewriting of ``biblical memory,'' by which I mean the constellation of traditions that relate to the stories found in the Bible. Next, I suggest that \ga utilized specific generic forms (such as ``diaspora literature'' and ``court contest'') that were particular to the \secondtemple period and that those genres---as socially defined constructs---are examples of the ``social frameworks'' that \halbwachs discusses. Finally, I discuss \ga's \psgraphical qualities as a particular form of memory discourse that ``speaks into'' the received tradition in a way that was different from other third-person accounts. The purpose of this chapter is to show how the processes involved in the production of \rwb texts map onto the core processes of memory.

% Summary: Chapter 5
In chapter five, I discuss the book of \jub and the way that it engages with the figure of Moses as a site of memory. I argue that the book of \jub portrays itself as an authoritative revelation which invited the reader to incorporate new knowledge into their conception of the past in an effort to affect the behavior of its readers and to reinforce the practices of its remembering community. The author accomplished this goal by engaging with \Psgraphical discourses which located the putative events of his revelation at Sinai and placed the content of his revelation on the lips of \yahweh and his intercessor, the \ap. In discussing these rhetorical moves, I suggest that current approaches to \jub (especially that of Hindy Najman) dovetail well with a memory approach and can be augmented by memory theory. The primary objective of this chapter, however, is to demonstrate that the effects of memory construction do not remain in the abstract, intellectual sphere, but can affect \emph{practice}. To illustrate this point, I argue that the 364-day calendar, by virtue of its heptadic quality, reinforced the system of Sabbaths and holidays in practice during the \secondtemple period. Moreover, the super-annual system of weeks (seven-year cycles) and jubilees (49-year cycles), so characteristic of the book of \jub, were used to reimagine Israel's remembered past based on an epochal understanding of the jubilee. This system invites the reader to infer that the coming jubilee might bring with it not only release for those in debt-bondage, but also release from the bondage of foreign occupation and a renewed Israelite state during the \secondtemple period.

% Conclusion
What results, I think, is a reasonably compelling first step to making the case for a memory approach within \rwb scholarship. Although this dissertation is not an exhaustive treatment of the topic of \rwb through the lens of memory studies, it can be thought of as a ``sounding'' into the feasibility of using memory as a theoretical framework for contextualizing \rwb within a broader framework of cultural practice. Ultimately this dissertation shows the usefulness of a memory approach to reading \rwb texts, in particular as a means to explore the social and cultural significance of these texts that extend beyond traditional ``exegetical'' models. The language of memory  brings with it a taxonomy for discussing these literary products in terms that are also meaningful outside \secondtemple studies. In other words, the value of using memory language is not only that such language is descriptive of the processes that brought about \rwb texts, but also that it allows us to identify similar processes within other related kinds of literature. In this respect, I hope that this dissertation can serve as a point of departure for my own future research as well as a convenient starting point for others. 