% !TeX root = ../dissertation.tex

\chapter*{Introduction}
\addcontentsline{toc}{chapter}{Introduction}


% Genesis of the project
The primary goal of this project is to introduce the theoretical framework of memory studies to the topic of \rwb and to illustrate the usefulness of social and cultural memory theory for describing the relationships between \rwb texts, their putative \vorlagen, and the societies that produced them. As I was considering topics for this dissertation, I recalled my dissatisfaction with the characterization of \rwb in much of the scholarly literature as focused primarily on biblical interpretation. In particular, I was dissatisfied with the way that many scholars characterized \rwb as primarily a method for ancient authors to explain the biblical text---as if \rwb texts were a form of early commentary. Although there are---without question---elements of biblical interpretation that occur within \rwb texts, it struck me that characterizing these texts as \emph{primarily} or \emph{essentially} focused on explaining the biblical text limited the ways that scholars could talk about how these texts might have functioned in the ancient world. At about this same time, I began working on a seminar paper which utilized memory theory, and it occurred to me that much of the language of memory theory could be usefully applied to the \rwb texts in such a way that both acknowledged the derivative nature of \rwb, and also allowed for the creative license that is so characteristic of \rwb texts. So it was to my great delight that---while perusing the book room at SBL in Atlanta, GA---I stumbled upon George Brooke's (then just-published) article ``Memory, Cultural Memory and Rewriting Scripture'' which suggested that memory theory could be a fruitful avenue for the discussion of \rwb, although he made no attempt to apply the theory himself.%
    \autocite{brooke_zsengeller2014}


% What is the role that you seek to address? What about traditional approaches to rewritten Bible is not sufficient?

% How does Social/Cultural Memory address these concerns?

%%%%%%%%%%%%%%%%%%%
% SUMMARY SECTION %
%%%%%%%%%%%%%%%%%%%
% Summary of the arguments and how they advance your thesis

% Overview
Because this dissertation involves the intersection of two abstract ideas (\rwb and memory theory), I have included two introductory chapters: 1) a literature review summarizing the history of scholarship and the \emph{status quaestionis} of \rwb studies, ad 2) an introduction to memory studies, including a history of scholarship that explains my particular approach to the topic of memory. These chapters also offer an explanation of how I think memory studies can fit into the discussion of \rwb and what shortcomings it can help to address. The core of this dissertation, however, centers around three case studies, each of which focuses on a single so-called \rwb text: \chronicles, \ga and \jub, in that order. Each case study focuses on a different aspect of memory that seemed well-suited to the particular text. Because of this approach, the kinds of observations that I have made about one particular text very likely could be made about the others, even if I have not made the connection explicitly. For example, although I only discuss ``genre'' in the chapter on \ga, both \chronicles and \jub exhibit generic peculiarities that could be framed within a discussion of social memory. Likewise, although I only discuss the effect of memory on ``practice'' with respect to \jub, this should not be understood as a dismissal of any practical effects of memory in \chronicles or \ga. All the same, I have attempted to order the chapters in such a way that each case study builds methodologically on the former chapters.

% Summary: Chapter 1
Chapter one provides an overview of the \emph{status quaestionis} within \rwb studies and highlights some of the difficulties that still remain within the discussion, particularly with regard to whether the term should be applied as a genre, process, or something else. I offer an extended treatment of \vermes' seminal work \citetitle{vermes1961}, focusing on the way that \vermes' used the term and tracing its development to the present, including its various modified forms (Rewritten Scripture, Rewritten Scriptural Texts, etc.). The remainder of the chapter is dedicated to addressing two current issues in \rwb scholarship, namely, the characterization of \rwb as either a category, process, or genre, and the definition of \rwb's boundaries. I argue that although few scholars use the term \rwb in the same way that \vermes did, \vermes' focus on the \emph{exegetical} qualities of \rwb have remained front-and-center within \rwb scholarship. Given the difficulties that this chapter addresses in defining \rwb, however, gives some reason for us to question whether such a focus is warranted and I propose that memory theory may be one avenue to engage with these difficulties.

% Summary: Chapter 2
Chapter two introduces the core figures and ideas of social, collective, and cultural memory theory (I refer to these related concepts collectively as ``memory theory''). Beginning with an extended treatment of the works of \Halbwachs (the ``father'' of memory studies), I outline the various major theoretical trajectories within the discipline such as Yosef \yerushalmi, Pierre Nora, Jan and Aleida Assmann, and Barry Schwartz. I also discuss some of the recent criticisms of memory studies, particularly that of Kerwin Klein. I note that although memory studies have made some inroads into the field of biblical studies (and more so within New Testament studies), most applications have focused on diachronic and historical issues (``cultural'' memory) and rarely interact with the \emph{vast} secondary literature on memory. I propose that \rwb texts are prime candidates for analysis from a memory perspective, and that \rwb texts can be understood as a set of snapshots revealing the ways that the collective memory of \secondtemple Judaism was shaped by its remembered past and how contemporary frameworks in turn affected how the past was remembered and transmitted to the next generation.

% Summary: Chapter 3
In chapter three, I treat the book of \chronicles as an example of \rwb from the perspective of memory studies. I begin with \chronicles primarily because there has already been a considerable amount of work done on the book from a memory perspective. This fact is important because, although \chronicles is frequently treated as an exemplar of \rwb by those who study \rwb, the topic of \rwb is not frequently addressed by those who primarily study \chronicles. The fact that \chronicles has fruitfully been treated from the perspective of memory, therefore, bolsters my contention that other \rwb texts could benefit from the same. The primary purpose of this chapter, therefore, is to illustrate two of the core ideas of memory theory: 1) that memory is \emph{constructed} through social processes and 2) that the remembered past is inextricably tied to the social frameworks of the present.  As such, this chapter is meant to represent a fairly conventional approach to memory studies. In particular, I discuss the role that ``sites'' of memory play in evolving discourses about the remembered past and how social memory can be helpfully modeled as a complex network of symbolic meaning. Building on this idea, I suggest that these networks function like scale-free networks and that the principle of preferential attachment (a property of scale-free networks) may serve to explain the phenomenon of ``magnetism'' as described by Ehud Ben Zvi. Drawing heavily on the work of Ben Zvi and Pierre Nora, I argue that the book of \chronicles engages with the major sites of memory from Israel's remembered past and draws together its major metanarratives as a way to make sense of the events and circumstances of the \secondtemple period.

% Summary: Chapter 4
In chapter four, I begin my treatment of \rwb texts which have largely \emph{not} been approached from a memory perspective, beginning with \ga. In my treatment of \ga, I focus on the three-step cycle of memory reception and transmission: 1) the reception of cultural memory, 2) the reshaping of memory by contemporary social frameworks, and 3) the active construction, codification, and reintegration of memory for future transmission. Through an analysis of \ga's sources, I argue that the cultural memory that \ga received should be thought of as broader than just Genesis. This finding bears on the characterization of \ga as a rewriting of \emph{Genesis}. Instead I suggest that \ga represents a rewriting of ``biblical memory,'' by which I mean the constellation of traditions that relate to the stories found in Genesis. Next, I suggest that \ga utilized specific generic forms (such as ``diaspora literature'' and ``court contest'') that were particular to the \secondtemple period and that those genres---as socially defined constructs---are examples of the ``social frameworks'' that \halbwachs discusses. Finally, I discuss \ga's \psgraphical qualities as a particular form of memory discourse that ``speaks into'' the received tradition in a way that was different than other third-person accounts. The purpose of each section is to show how the processes involved in the production of \rwb texts map onto the core processes of memory.

% Summary: Chapter 5
In chapter five, I discuss the book of \jub and the way that it engages with the figure of Moses as a site of memory. I argue that the book of \jub portrays itself as an authoritative revelation which invited the reader to incorporate new knowledge into their conception of the past in an effort to affect the behavior of its readers and to reinforce the practices of its remembering community. The author accomplished this goal, as did \ga, engaging with \Psgraphical discourses and locating the putative events of his revelation at Sinai and on the lips of \yahweh and his intercessor, the \ap. The primary objective of this chapter is to demonstrate that the effects of memory construction do not remain in the abstract, intellectual sphere, but can effect \emph{practice}. To illustrate this point, I argue that the 364-day calendar, by virtue of its heptadic quality reinforced the system of Sabbaths and holidays in practice during the \secondtemple period. Moreover, the super-annual system of weeks (seven-year cycles) and jubilees (49-year cycles) so characteristic of the book of \jub were used to reimagine Israel's remembered past based on an epochal understanding of the jubilee. This system invites the reader to infer that the coming jubilee might bring with it not only release for those in debt-bondage, but release from the bondage of foreign occupation and a renewed Israelite state during the \secondtemple period.

% Summary: Conclusions Findings 
% What are the main claims of your dissertation?


% Conclusion
What resulted, I think, is a reasonably compelling first-step to making the case for a memory approach within \rwb scholarship. Although this dissertation is not an exhaustive treatment of the topic of \rwb through the lens of memory studies, it can be thought of as a ``sounding'' into the feasibility of using memory as a theoretical framework for contextualizing \rwb within a broader framework of cultural practice. In this respect, I hope that this dissertation can serve as a point of departure for my own future research as well as a convenient starting point for others. 