% !TeX root = ../dissertation.tex

\chapter*{Introduction}
\addcontentsline{toc}{chapter}{Introduction}

% Genesis of the project
The primary goal of this project is to introduce the theoretical framework of memory studies to the topic of \rwb and to illustrate the usefulness of social and cultural memory theory for describing the relationships between \rwb texts, their putative \vorlagen, and the societies that produced them. As I was considering topics for this dissertation, I recalled my dissatisfaction with the characterization of \rwb in much of the scholarly literature as focused primarily on biblical interpretation. In particular, I was dissatisfied with the way that many scholars characterized \rwb as primarily a method for ancient authors to explain the biblical text---as if \rwb texts were a form of early commentary. Although there are---without question---elements of biblical interpretation that occur within \rwb texts, it struck me that characterizing these texts as \emph{primarily} or \emph{essentially} focused on explaining the biblical text limited the ways that scholars could talk about how these texts might have functioned in the ancient world. At about this same time, I began working on a seminar paper which utilized memory theory, and it occurred to me that much of the language of memory theory could be usefully applied to the \rwb texts in such a way that both acknowledged the derivative nature of \rwb, and also allowed for the creative license that is so characteristic of \rwb texts. So it was to my great delight that---while perusing the book room at SBL in Atlanta, GA---I stumbled upon George Brooke's (then just-published) article ``Memory, Cultural Memory and Rewriting Scripture'' which suggested that memory theory could be a fruitful avenue for the discussion of \rwb, although he made no attempt to apply the theory himself.%
    \autocite{brooke_zsengeller2014}

What resulted, I think, is a reasonably compelling first-step to making the case for a memory approach within \rwb scholarship. Although this dissertation is not an exhaustive treatment of the topic of \rwb through the lens of memory studies, it can be thought of as a ``sounding'' into the feasibility of using memory as a theoretical framework for contextualizing \rwb within a broader framework of cultural practice. In this respect, I hope that this dissertation can serve as a point of departure for my own future research as well as a convenient starting point for others. 

I have structured this dissertation around three case studies, each of which focuses on a single so-called \rwb text: \chronicles, \ga and \jub, in that order. Each case study focuses on a different aspect of memory that seemed well-suited to the particular text. Because of this approach, the kinds of observations that I have made about one particular text very likely could be made about the others, even if I have not made the connection explicitly. For example, although I only discuss ``genre'' in the chapter on \ga, both \chronicles and \jub exhibit generic peculiarities that could be framed within a discussion of social memory. Likewise, although I only discuss the effect of memory on ``practice'' with respect to \jub, this should not be understood as a dismissal of any practical effects of memory in \chronicles or \ga. All the same, I have attempted to order the chapters in such a way that each case study builds methodologically on the former chapters. Before beginning on these case studies, however, I have included two introductory chapters: 1) a literature review summarizing the history of scholarship and the \emph{status quaestionis} of \rwb studies, ad 2) an introduction to memory studies, including a history of scholarship that explains my particular approach to the topic of memory. These chapters also offer an explanation of how I think memory studies can fit into the discussion of \rwb and what shortcomings it can help to address.