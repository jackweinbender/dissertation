% !TeX root = ../dissertation.tex

Over the past several decades, a dramatic increase in scholarly interest toward the topic of ``memory'' has swept throughout the social sciences and humanities. The precipitous rise in scholarly literature dealing with topics of memory coupled with its proliferation in popular media discourses has prompted some in the field to refer to a ``memory industry'' and to describe the ubiquity of memory discourses as a ``boom'' fast-approaching a bust.%
%
\footnote{\cite{rosenfeld_jmh2009}; \cite{winter2006}; \cite{berliner_aq2005}; \cite{confino_ahr1997}.}
%
Yet, as Olick et al.~make clear in their Introduction to \emph{The Collective Memory Reader}, there remain a significant number of scholars throughout the social sciences and humanities who continue to find memory to be a useful heuristic and a compelling theoretical basis for their various and sundry analytical applications.\autocite[3--6]{olick_olick-etal2011}

Central to the discussion of social and cultural memory (more on these terms below) is the interplay between the ways that memories are both ``stored'' and ``recalled'' and the impact that social structures have on these two processes at both the individual and societal level. One half of this equation deals with the how societies construct the memory of their shared experiences, how they ``commemorate'' people, places, events, etc. by imbuing such ideas with significance and social meaning. The other half of the equation deals with the way societies receive their own cultural memories into new contexts and how such new contexts affect the given society's understanding of their memories before being passed on to the next generation. Social and cultural memory, therefore, deal with the processes which one might otherwise call ``traditions.'' For the purposes of thinking about \rwb, it is the latter half of the equation that is of primary concern. From this perspective, \rwb texts can be viewed as textual products which represent the thorough recontextualization of received memories which were central to Jewish identity during the late \secondtemple period (e.g., the stories contained in Genesis). Moreover, the codification of these ``rewritten'' stories is also an example of the way that received memories themselves form the basis of commemoration for the next generation.

In this chapter I will outline the background and current state of memory studies with special attention to the work of Maurice \halbwachs and more recent contributions from Yosef \yerushalmi, Jan and Aleida Assmann, and Barry Schwartz in an effort to provide a theoretical foundation for my own discussion of \rwb.