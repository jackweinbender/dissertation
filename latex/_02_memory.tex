% !TeX root = ../dissertation.tex

\chapter{Memory Studies and the \RwB}
\label{chap:memory}

Over the past several decades, a dramatic increase in scholarly interest toward the topic of ``memory'' has swept throughout the social sciences and humanities. The precipitous rise in scholarly literature dealing with topics of memory coupled with its proliferation in popular media discourses has prompted some in the field to refer to a ``memory industry'' and to describe the ubiquity of memory discourses as a ``boom'' fast-approaching a bust.%
%
\footnote{\cite{rosenfeld_jmh2009}; \cite{winter2006}; \cite{berliner_aq2005}; \cite{confino_ahr1997}.}
%
Yet, as Olick et al.~make clear in their Introduction to \emph{The Collective Memory Reader}, there remain a significant number of scholars throughout the social sciences and humanities who continue to find memory to be a useful heuristic and a compelling theoretical basis for their various and sundry analytical applications.\autocite[3--6]{olick_olick-etal2011}

Central to the discussion of social and cultural memory (more on these terms below) is the interplay between the ways that memories are both ``stored'' and ``recalled'' and the impact that social structures have on these two processes at both the individual and societal level. One half of this equation deals with the how societies construct the memory of their shared experiences, how they ``commemorate'' people, places, events, etc. by imbuing such ideas with significance and social meaning. The other half of the equation deals with the way societies receive their own cultural memories into new contexts and how such new contexts affect the given society's understanding of their memories before being passed on to the next generation. Social and cultural memory, therefore, deal with the processes which one might otherwise call ``traditions.'' For the purposes of thinking about \rwb, it is the latter half of the equation that is of primary concern. From this perspective, \rwb texts can be viewed as textual products which represent the thorough recontextualization of received memories which were central to Jewish identity during the late \secondtemple period (e.g., the stories contained in Genesis). Moreover, the codification of these ``rewritten'' stories is also an example of the way that received memories themselves form the basis of commemoration for the next generation.

In this chapter I will outline the background and current state of memory studies with special attention to the work of Maurice \halbwachs and more recent contributions from Yosef \yerushalmi, Jan and Aleida Assmann, and Barry Schwartz in an effort to provide a theoretical foundation for my own discussion of \rwb.

\section{The Work of \Halbwachs: A Very Brief Overview}

Although the topic of memory has been of interest to philosophers and thinkers since the antiquity,\autocite{carruthers_radstone-schwarz2011} as Olick and Robbins note, modern social-scientific approaches (which concern this study) almost exclusively trace their genealogy to the early twentieth Century work of sociologist \Halbwachs.%
%
\footnote{\cites[106]{olick-robbins_ars1998}. It should be noted, however, that \halbwachs was not the first or only person to do work on memory or the impact of social structures on memory. See \cite[8--36]{olick_olick-etal2011}.}
%
Although \halbwachs's scholarly contributions were not limited to the topic of social memory (he also made contributions to statistics and probability theory, as well as sociological work on the topic of suicide and social morphology), the influence of his work in this area not only made a more lasting impact on the field of sociology than his other contributions, but it has also made a profound impact in a number of other fields such as history, anthropology, and biblical studies.\autocite[13--20]{coser_halbwachs1992}

\halbwachs published three primary works on the topic of memory, the first of which appeared in 1925 under the title \citetitle*{halbwachs1925}.\autocite{halbwachs1925} This monograph, along with the concluding chapter of his second monograph---dealing with the remembered geography of the Holy Land---was excerpted and translated by Lewis Coser in a single volume under the title \emph{On Collective Memory} in 1992.%
%
\footnote{Several of the most important chapters of \citetitle{halbwachs1925} were included in full. Likewise, the entirety of the conclusion of \citetitle{halbwachs1941} was included. \cites{halbwachs1992}; \cite{halbwachs1941}.}
%
His third and final contribution, entitled \citetitle*{halbwachs1980}, was published posthumously in 1950 and was translated into English in 1980 with an editorial introduction by Mary Douglas.\autocite{halbwachs1980} This work simultaneously represents some of \halbwachs's most developed ideas (responding to critics such as Charles Blondel) while evincing an incompleteness which posthumous publications often suffer.%
%
\footnote{%FIXME: Figure out why no Capital Ibid here.
\Cite{halbwachs1980}. As Coser observes, ``One may doubt that the author himself would have been willing to publish it in what seems to be an unfinished state. The book nevertheless contains many further developments of \halbwachs's thought in regard to such matters as the relation of space and time to collective memory as well as fruitful definitions and applications of the differences between individual, collective, and historical memory.'' \cite[2]{coser_halbwachs1992}.}

The central contribution of \halbwachs's work was the notion that human memory is intrinsically and inextricably tied to social frameworks.\autocite[37--38]{halbwachs1992} Humans are social beings and as such human activities, such as memory, are only usable within the context of a society. This focus on the \emph{social} dimensions of memory betrays the deep influence that Émile Durkheim's work had on \halbwachs.\autocite[8--9]{coser_halbwachs1992} Unlike Durkheim, however, \halbwachs's approach was tempered by his desire to identify the physical location of memory within to be within the \emph{individual}.%
%
\footnote{To clarify, \halbwachs was not at all interested in the biological processes or locating within the brain where memory is stored, only that ``collective memory'' \emph{is} stored biologically but, more importantly, is socially conditioned.}
%
Although the term ``collective memory'' evokes an ethereal or metaphysical idea, \halbwachs's use of the term was intended to ensure that any discussion of memory remained in the concrete. Collective memory is the sum total of those memories kept by \emph{individuals} within a society. An individual's ability to retrieve and utilize a particular memory, however, is inextricably entangled with the individual's social context. According to \halbwachs memories require social frameworks to function.\autocite[38]{halbwachs1992}  

To illustrate this point, \halbwachs begins \citetitle{halbwachs1925} by attempting to prove the negative. \emph{Without} a social framework, he argues, memories are always incomplete. Because humans---for all intents and purposes---always exist within a society, it is the dream state that most closely approximates the complete isolation of memory from society.\autocite[41--42]{halbwachs1992} Therefore, the way that the human brain deals with memories while dreaming can illustrate the (dis)function of memories lacking a social framework. Thus, he observes that ``dreams are composed of fragments of memory too mutilated and mixed up with others to allow us to recognize them.''\autocite[41]{halbwachs1992} Because the mind lacks the ability to ``check'' itself against anything external while in a dream state, dreams do not contain ``true memories.''\autocite[41]{halbwachs1992} This assertion is set against the ``purely individual psychology'' of Bergson and Freud which viewed \emph{memory} as a location of social isolation.%
%
\footnote{See \cite{ansellpearson_radstone-schwarz2011} and \cite{terdiman_radstone-schwarz2011}.}
%
Regarding the incompleteness of the dream state, he writes:  

\begin{quote}
    Almost completely detached from the system of social representations, {[}the dream state's{]} images are nothing more than raw materials, capable of entering into all sorts of combinations. They establish only random relations among each other---relations based on the disordered play of corporal modification.\autocite[42]{halbwachs1992}
\end{quote}  

The ``system of social representations'' that \halbwachs refers to is not limited, however, to macro structures such as familial, religious, or class groups. Although these structures certainly \emph{do} make up an important stratum of social frameworks, \halbwachs envisions something much more fundamental which betrays his broadly structuralist perspective. \halbwachs uses the phrase ``social representations'' to refer to a system of shared ``signs'' that encompassed not only these macro structures, but every aspect of a group's social framework---a sort of ``cultural \emph{langue}.'' Although, \halbwachs does not use the language of semiotics, the analogy is helpful. Just as Saussurian semiotics argues that the concrete arbitrary sign is given meaning only by participating in the broader, shared \emph{langue}, so too memories (read: ``signs'') require a framework to convey meaning, as do the concrete, individual expressions of remembrance (read: \emph{parole}).%
%
\footnote{This is my terminology with reference to \halbwachs. Of course, it is borrowed from Saussure. See \cite*{saussure1916}. For a brief overview of these terms see \cite[93--94]{smith-riley2009}.}
%
\halbwachs writes:  

\begin{quote}
    It is in this sense that there exists a collective memory and social frameworks for memory; it is to the degree that our individual thought places itself in these frameworks and participates in this memory that it is capable of the act of recollection.\autocite[38]{halbwachs1992}
\end{quote}  

Memories, therefore, cannot be understood in isolation from their social frameworks and therefore should not be analyzed without consideration to the social context of the rememberer.  

Of course, people participate in a plurality of social contexts, often simultaneously, and the experiences that are later to be recalled, too, must be situated within these frameworks. In order to bring these autobiographical memories to mind, according to \halbwachs, an individual attempts to mentally situate herself within those same frameworks.\autocite[38]{halbwachs1992} For instance, I find it much easier to recall whether a particular university course I have taken occurred in the Fall or Spring semester, rather than which month or even year it occurred. The social framework that is the ``academic year'' remains a potent framework for my own memories; I imagine the ``year'' beginning in the Fall, and often refer to ``next semester.'' On the other hand, my wife---who had the good sense to stop her formal education after one degree---no longer thinks in terms of semesters. Yet, when remembering events during her time at university, the semester once again becomes a useful framework for memory. It is for this reason that recent memories are more easy to call to mind: because the social frameworks that produced the memory (the people, places, customs, etc.) remain in close proximity for the rememberer and the effort required to situate the memory within the social frameworks that produced it is minimal.\autocite[52]{halbwachs1992} This notion is a central part of \halbwachs's thesis and provides a point of departure for his more in-depth studies of collective memory in the family, religion, and social classes.  

\subsection{A Note on \halbwachs's Terminology}

There is a grand tradition of imprecise and overlapping terminologies within memory studies going back to \halbwachs himself. For example, on page 40 of \emph{On Collective Memory}, \halbwachs uses each of the terms ``collective memory,'' ``social memory,'' ``social frameworks of memory,'' and ``collective frameworks of memory'' and it is not entirely clear how \halbwachs is distinguishing between them. The way that he is able to use the terms almost interchangeably has led some in the current discussion to treat them as synonyms. As Anthony Le Donne observes, ``In fact, {[}`social' and `collective' memory{]} are currently used synonymously with such frequency that their nuances vary from author to author.''\autocite[42 n.8]{ledonne2009} Yet, Le Donne points out, \halbwachs actually uses these terms with slightly different nuances. On the one hand, \halbwachs uses the term ``social'' memory when he is describing the way social structures affect memory, while on the other hand ``collective'' memory tends to refer to the content of memories which are transmitted between individuals and common to those of a particular group.  

In other words, when \halbwachs uses the term ``social'' memory, he is referring to the social frameworks in which individual memory participates, i.e.~how society provides the framework that makes individual memory possible.\autocite[180]{hubenthal_carstens-hasselbalch2012} On the other hand, when he uses the term ``collective'' memory, he tends to refer to shared memory, ``the shared cultural past to which individuals contribute and upon which they call; but ultimately a past that transcends individual memory.''%
%
\footnote{\cite[360]{keith_ec2015}. See also \cite[180]{hubenthal_carstens-hasselbalch2012}.}
%
The two ideas work together and mutually influence one another. As Hübenthal puts it, ``The difference [between social and collective memory] lies in the perspective: \emph{social memory} is using the framework, \emph{collective memory} is establishing it.''\autocite[180.]{hubenthal_carstens-hasselbalch2012} Hübenthal's use of the active verb ``establish'' is intentional: for \halbwachs, collective memory is not a passive social accretion, but an actively constructed part of the group's common identity which \emph{speaks to the concerns and needs of the community in the present}. Social frameworks shape the way that people remember. The retrieval of memories is shaped by those same frameworks, and as those frameworks shift, so too do the memories that are recalled in those societies.%
%
\footnote{For a modern assessments on the malleability of human memory and the effects of social networks on the formation of collective memory, see \cite{coman-etal_pnas2016}; \cite{yamashiro-hirst_jarmc2014}; \cite{coman-etal_yang-etal2012}.}
%

In his later work, \halbwachs distinguishes between two kinds of memory which can be identified by the experiential-relation of the rememberer to the object of memory: autobiographical and historical memory.\autocite[52]{halbwachs1980} Autobiographical memory refers to the sort of memories which are the result of individual, subjective experience, while historical memory refers to those which fall outside the experience of the individual. Elsewhere \halbwachs refers to these as ``internal'' and ``external'' memory. Autobiographical memory is rooted in the individual, sensory experiences which provide a full, ``thick'' memory---to borrow from Geertz\footnote{\cite[3--30]{geertz1973}. See also \cite[189--192]{smith-riley2009}.}--- while historical memory offers only a thin, schematic overview and by definition is never ``experienced'' by the rememberer.  

Although \halbwachs distinguished between these two forms of memory, he nevertheless emphasized their interrelatedness. In particular, \halbwachs notes that autobiographical memory necessarily is dependent upon historical memory, insofar as our lives participate in ``general history.''\autocite[52]{halbwachs1980} For example, memories of a more indirect nature are able to shape autobiographical memory by shaping the social frameworks which produced them and the frameworks into which they are recalled. The quintessential example for Americans of my age would be the events of September 11, 2001. Although comparatively few people directly witnessed the events (I was asleep on the West Coast when the first plane crashed), the impact that those events had (and continue to have) on the orientation of American national memory is unquestionably a part of many people's lived experience, including my own and would therefore constitute a part of America's current ``collective memory.'' Although the incoming undergraduate class at the University of Texas at Austin, many of whom will have been born after 2001, have \emph{no} autobiographical memory of these events, it is, nonetheless, a part of the collective memory of their society at large. On the other hand the War of 1812 is not a part of any living person's autobiographical memory and its impact on the collective memory of most Americans is likely restricted to a few popular media references, or localized to specific geographical regions with a close connection to major events in the conflict (e.g., New Orleans).\footnote{Such as Jimmy Driftwood's \emph{The Battle of New   Orleans}, best known as performed by Johnny Horton which topped   \emph{Billboard} charts in the US, Canada, and Australia in 1959 and   was recently acknowledged to be one of the Top 100 Western songs of   all time. See,   https://en.wikipedia.org/wiki/The\_Battle\_of\_New\_Orleans.}  

The memories of historical events, likewise, are shaped by the social frameworks of the rememberer. The events of September 11, 2001 in the memory of most Americans are now further shaped by the socio-political discourses surrounding the United States' continued military presence in the Middle East and its controversial pretexts for engagement in the region, especially with the invasion of Iraq in 2003. Likewise, although no living person has an autobiographical memory of the American Civil War, the construction of certain confederate monuments on the campus of the University of Texas at Austin during the Jim Crow era, and their subsequent removal in August of 2017, illustrates how historical memory can be (consciously, in this case) reshaped and restructured as the remembering society changes.\footnote{See https://www.nytimes.com/2017/08/21/us/texas-austin-confederate-statues.html.}  

It is the way that these remembered events change over time that makes social memory studies so interesting for the historian. \halbwachs's own work in the area of history is best seen in \emph{The Legendary Topography of the Holy Land}, where he focuses on the ways that memories relate to particular geographic sites. Notably, \halbwachs is not interested in ``doing'' history. Rather, \halbwachs's study focuses on the way that the geographical sites in and around the Galilee and Jerusalem were imbued with significance based on their putative connection with significant events related to Jesus, the Apostles, and early Christian communities. 

\halbwachs makes a number of observations about the way that memories are formed and the ways that they interact. His first observation comes in contrasting the portrayal of Jesus within the Gospels with what must have been the lived experience of the Apostles.\autocite[193--198]{halbwachs1992} The involvement of the apostles in the day-to-day life of Jesus in some sense would have prohibited them from achieving the kind of ``necessary detachment'' to write something like the Gospels. In other words (and to use \halbwachs's later terminology), the memory of Jesus as portrayed in the Gospels is almost necessarily informed by \emph{historical} rather than autobiographical memory.\autocite[194]{halbwachs1992} Indeed, \halbwachs rightly observes that the Gospels present Jesus and his ministry ``as if Jesus's whole life was but a preparation for his death, as if this was what he had announced in advance.''\autocite[198]{halbwachs1992} Although the religious significance of Jesus's death continues to be remembered as a central component of Christianity, surely Jesus's mother remembered the death of her son differently than the way the Church later commemorated it.%
%
\footnote{Regardless of whether \halbwachs's conception of Early Christianity would be considered sound today, the idea that the Gospels represent several collective remembrances of Jesus's life, ministry and death each bearing marks from their own \emph{Sitz im Leben} (to borrow from the form critics) seems relatively uncontroversial. A number of studies on the Jesus and early Christian memory have come about in the past several years. See \cite{ledonne2009}; \cite{rodriguez2010}. For an overview of the modern impact of \halbwachs (and memory studies more generally) on the field of Historical Jesus studies, see \cite{keith_ec2015} and \cite{keith_ec2015b}.}

\halbwachs, drawing on the Pauline epistles, observes that the earliest recollections of Jesus make no mention of the location of his death (Jerusalem) nor of his ministry (Galilee). He writes:  

\begin{quote}
    In the authentic epistles of Paul, we are told only that the son of God has come to earth, that he died for our sins, and that he was brought back to life again. There is no allusion to the circumstances of his life, except for the Lord's Supper, which, Paul says, appeared to him in a vision (and not through witnesses). There is no indication of locality, no question of Galilee, or of the preachings of Jesus on the shores of the lake of Gennesaret.\autocite[209]{halbwachs1992}\footnote{Notably, the only Jesus scholar with whom \halbwachs interacts is Ernest Renan, a figure whose work has survived mostly as a punching-bag for later scholars and as an example of overt anti-Semitism in biblical scholarship. See \cite[39]{heschel2008}}
\end{quote}  

\halbwachs's point is that within the narrative of the Gospels, the location of Jesus's death---by virtue of the social and political reality of the day---\emph{had} to occur in Jerusalem.\autocite[211]{halbwachs1992} Whether or not it actually did, or whether or not that information was explicitly handed down to the authors of the Gospels is irrelevant for the purposes of collective memory. Sacred places become sacred through the process of memory \emph{construction}, not simply through the transmission of autobiographical experience. They are spaces where significant ideas within the collective memory of a group can take concrete form. He writes, ``Sacred places thus commemorate not facts certified by contemporary witnesses but rather beliefs born perhaps not far from these places and strengthened by taking root in this environment.''\autocite[199]{halbwachs1992} Localizing historical memory, therefore, functions as a way to move abstract ideas into the real world and reinforce fundamental components of the group's collective memory.  

Perhaps more interesting is \halbwachs's treatment of the ability for memories to coalesce and split over time. \halbwachs makes the observation that, according to tradition (i.e., the collective memory of the Church), certain places in the Holy Land mark the location of \emph{several} significant events. From an historical perspective \halbwachs, obviously, doubts that these assertions are accurate---even assuming the events indeed occurred at all---but finds the clustering of these events to be more than just coincidence. For example, he writes:  

\begin{quote}
One is surprised to find on the shores of the lake Gennesaret, near the Seven Fountains, the place where apostles were chosen, the Sermon on the Mount, the appearance of Jesus on the waters after the Resurrection---all in the same place.\autocite[220]{halbwachs1992}
\end{quote}  

\halbwachs's assumption is that there was something about the location \emph{itself}, some ``earlier consecration,''\autocite[220]{halbwachs1992} which attracted these memories to particular locales. Extending this rationale further, we can appreciate the fact that for Christianity, the significance of Jerusalem is not limited to the significance of the city as the location of Jesus's death, but rather by the prior significance of the city for Judaism.%
%
\footnote{In addition, Jerusalem was the location of the leadership for the earliest Christian Church according to Gal 1:18 (and attested throughout Acts).}
%
Within the collective memory of Christian tradition, one might say that Jerusalem is not significant because it is the location of the Passion and resurrection of Jesus, but that the Passion and resurrection of Jesus happened in Jerusalem \emph{because Jerusalem was significant}. \halbwachs writes:  

\begin{quote}
The Christian collective memory could annex a part of Jewish collective memory only by appropriating part of the latter's local remembrance while at the same time transforming its entire perspective of historical space.\autocite[215]{halbwachs1992}%
%
\footnote{Because the earliest Christians were Jewish, it stands to reason that the collective memory of \emph{earliest} Christianity was rooted in broader Jewish memory. In later periods---especially during and after the so-called parting of the ways (however problematic this term has become)---\halbwachs is certainly correct. Regarding the current discussion on the Jewish--Christian schism see \cite[19--60]{burns2016}.}
\end{quote}  

One might object to this suggestion by noting that, supposing Jesus \emph{actually was} crucified in Jerusalem, one hardly needs to re-appropriate Jewish tradition or attribute this remembrance to some special process. Yet, it is worth pointing out in cases where the historical data are lacking (or, perhaps, where eyewitness accounts certainly did not exist), this same basic phenomenon occurred. For example, \halbwachs points to the birth narratives of the Gospels, in particular that of Matthew, where Jesus is described as being born as a descendant of David in the town of Bethlehem (Matt 1:20; 2:1). Although there is no reason to think that Jesus was \emph{actually} born in Bethlehem, \halbwachs rightly observes, ``the authors of the gospels seem entirely to have invented this poetic history which has occupied a considerable place in Christian History.''\autocite[214]{halbwachs1992} In fact, Jesus's entire portrayal in the Gospel of Matthew is an exercise in collective remembrance which is structured on the foundational narratives of the Hebrew Bible: the slaughter of innocents (Matt 2:16--17), and Jesus's portrayal as a lawgiver ``on the mount''(Matt 5:1--7:29), and framed as the fulfillment of Jewish prophecy (Matt 1:23; 2:6, 18 \emph{et passim}). 

The inverse of this phenomenon is also observable. According to \halbwachs while some events converge to particular locations, other events diverge among several sites. One expression of this process is the way that significant events are themselves divided providing the opportunity for each portion of the event to be separately localized. For example, \halbwachs notes how the memory of specific important events, such as the Passion, may be split and localized at a very fine level of detail:  

\begin{quote} Around Golgotha and the Holy Sepulcher, for example, we find the rock of anointing, the rock of the angel, the rock of the gardener, the place where Jesus was stripped, etc.\autocite[220]{halbwachs1992} \end{quote}  

The proliferation of these micro-sites of memory, according to \halbwachs, aide and reinforce the collective memory through repetition. Furthermore, the added detail serves in ``renewing and rejuvenating an ancient image.''\autocite[220]{halbwachs1992}  

%% TODO: Get ancient source citations
The same event may also be localized in multiple places. \halbwachs describes several traditional locations of the Cenacle (the ``Upper Room'' from the Gospels), including the Mount of Olives, Gethsemane, and the Grotto of Jesus's teaching. These traditions coexisted into the fourth century, yet, later, the site was moved to the Christian hill of Zion. Likewise, \halbwachs notes that there were two locations for Emmaus and two different mountains on which Jesus is said to have appeared in Galilee after his resurrection. While it runs counter to conventional modern conceptions of the past, that seemingly contradictory traditions are able to coexist within a society---or even within the memory of a single individual---is well documented.
% TODO: get citation.
\halbwachs points out that autobiographical memory, however, does not allow for this kind of fragmentation.
% TODO: get citation.
%
We all realize that the same event from our own past can not have happened in two locations simultaneously. Yet, \halbwachs points out that should that same person belong to two groups who disagree on a particular remembered event from history (one that the individual did not personally witness), individuals are generally able to hold such memories together (if in tension) without the need assert one or the other. The same is true of complex social entities such as religious groups who are themselves composed of smaller sects which may possess their own unique collective memory. \halbwachs writes:  

\begin{quote} A community must often accommodate itself to contradictions introduced by diverse groups so long as none of these groups prevails, or so long as the community itself does not find a new reason for decisively settling the issue. This is especially true when the community faces a controversy about its rites, which are an anchor for its component groups.\autocite[224]{halbwachs1992} \end{quote} 

%% TODO: Some kind of summary here?

\section{The Memory Boom}

\halbwachs's work, while not ignored, would not make its most significant impact until well after his death. It is frequently argued that the so-called ``memory boom,'' which began in the 1980's in the wake of the ``theory boom,'' picked up \halbwachs's terminology and central ideas in an attempt to deal with the perceived insufficiency of traditional historiography to deal with the sorts of major, traumatic events which characterized the mid twentieth Century.%
%
\footnote{\cite[1--2]{galinsky_galinsky2016}. See also \cite[29--36]{olick_olick-etal2011}. One cannot help but speculate that---at least in the English-speaking world---the translation of \emph{The Collective Memory} in 1980 contributed to the popularity of \halbwachs's terminology.}
%
Works such as Yosef \yerushalmi's \citetitle{yerushalmi1989} and Pierre Nora's \emph{Les Lieux de mémoire} are typically cited as the foundational works of the modern memory boom.\autocites[112--113]{klein2011}{yerushalmi1989}[Nora's massive project has been abridged and translated into English as][]{nora1996}  

In \citetitle{yerushalmi1989}, \yerushalmi is quick to identify the tension between what traditional cultures and societies remember about their past and how the discipline of history treats the past. For remembering groups, what is preserved in the collective memory is what is useful for the edification of that group---whether through religious ritual, family stories, or some other combination of received traditions. Of course, prior to the enlightenment, this was the default mode of understanding the past for most people, and remains so for many social groups, including those within modern, Western societies. In particular, \yerushalmi addresses this tension for the Jewish historian---a vocation which, he notes, is a recent phenomenon.\autocite[81--103]{yerushalmi1989} Although, ancient Israel and Judah, clearly, were concerned with matters of the perceived past---much of the Hebrew Bible is preoccupied with narrating events from the perceived past---these codified traditions are preserved in a plurality of socio-religious groups for a complex set of purposes spanning cultural, social, and theological modes of discourse which are fundamentally at odds with the discipline of history.%
%
\footnote{I am aware of the problematic nature of this statement. Contemporary approaches to historiography are emphatically \emph{not} attempting to write ``objective history.'' Yet, referring only to ``Modernist'' historiography does not give due consideration to the fact that common discourse around the idea of ``history'' is largely influenced by Modernism. Even taking into account that contemporary historiography has moved beyond discussions of ``objectivity'' the methodological underpinning of historical discourse remains fundamentally different, if only by the existence of its own meta-discourse. As Daniel Pioske puts it, ``from the recounting of a culture's sanctioned memories is consequently the historian's determination to isolate and compare disparate testimonies about the past with other past traces that may corroborate or discredit their claims.'' \cite[12--13]{pioske_bibint2015}.}
%
Thus, the biblical command to ``remember,'' is not a command to keep tedious notes of historically accurate events, but a cultural and theological imperative to maintain the foundational narratives of the community. \yerushalmi writes:  

\begin{quote} There the fact that history has meaning does not mean that everything that happened in history is meaningful or worthy of recollection. Of Manasseh of Judah, a powerful king, who reigned for fifty-five years in Jerusalem, we hear only that ``he did what was evil in the sight of the Lord'' (II Kings 21:2).\autocite[10]{yerushalmi1989} \end{quote}  

In other words, what was remembered about Manasseh by the biblical tradants were those details which were useful for their socio-religious projects. The rules and methods of this process---remembering what is important and forgetting what is not---are generally not explicit or transparent.  

The discipline of history, on the other hand, generally attempts to uphold a certain set of explicit methodological and theoretical criteria which---while not exempt from distortion by the subjectivity of the historian---can be corroborated or contradicted by evidence and argumentation.\autocite[12--13]{pioske_bibint2015} While the historian participates in the collective memory of her own society, her reconstruction of the past attempts to approach the topic from the outside. The historian, too, (re)constructs the past, but the goals of the historian are, as \yerushalmi puts it, to recreate ``an ever more detailed past whose shapes and textures memory does not recognize.''\autocites[94]{yerushalmi1989}[See also][532]{verovsek_pgi2016} Even the most theory-conscious historian cannot help but struggle in avoiding older discourses about ``what really happened,'' particularly when stated over and against memory in the form of received tradition. All of this is not to say that modern history writing is in any meaningful sense ``objective,'' nor that the historian is able to remove herself from her own socio-political context. So, although memory and history both offer reconstructions of the past, it is important to affirm that their modes of doing so are radically different and for different purposes.%
%
\footnote{See esp. \cite[497]{ricoeur2004}. As Pioske observes, ``The epistemological tension observed by Ricoeur between memory and history is thus understood as the outcome of two processes that, though having the similar intent of re-presenting former phenomena, nevertheless pursue and mediate the past through quite disparate means.'' \cite[12]{pioske_bibint2015}.}

Thus the memory ``boom'' has, in some circles, been viewed as anti-historical and an attempt at ``resacrilization of the past'' to counter the disenchantment brought about by modern historical consciousness.\autocite[282]{winter2006} Kerwin Klein, for example, traces the origins of scholarly interest surrounding memory and lists five narratives that others have offered as explanations for the origins of memory discourse in society generally:  

\begin{quote} We have, then, several alternative narratives of the origins of our new memory discourse. The first, following Pierre Nora, holds that we are obsessed with memory because we have destroyed it with historical consciousness. A second holds that memory is a new category of experience that grew out of the modernist crisis of the self in the nineteenth century and then gradually evolved into our current usage. A third sketches a tale in which Hegelian historicism took up pre-modern forms of memory that we have since modified through structural vocabularies. A fourth implies that memory is a mode of discourse natural to people without history, and so its emergence is a salutary feature of decolonization. And a fifth claims that memory talk is a belated response to the wounds of modernity.\autocite[134]{klein_klein2011} \end{quote}  

Although Klein finds none of these ``fully satisfying,'' it is noteworthy that the general trend among these narratives corroborates the thesis that memory represents a ``reaction'' against history in some form.  

Whatever combination of these causes may have ultimately brought about the memory boom, the problem remains, according to Klein, that memory has come to dominate historical discourse as a ``therapeutic alternative'' to history in place of a rigorous scientific methodology.\autocite[137]{klein2011} As Winter puts it, ``It is a fix for those who cannot stand the harshness of critical thinking or historical analysis.''\autocite[283 (summarizing Klein)]{winter2006} Although I think Klein under-appreciates the value of the memory discourse as a meaningful mode of inquiry, I am in fact, quite sympathetic to his critique overall. As methodologies for querying the past, memory and history operate on different sets of hermeneutical and epistemological foundations, which is, I think, one of \yerushalmi's main points. 
%
% TODO: See Keith's long note on p. 23
%
However, what Klein does not address is the way that, for modern Westerners, history \emph{is} our collective memory (or at least, heavily influences our collective memory). This is what Nora means when he says that ``We speak so much of memory because there is so little of it left.''\autocite[7]{nora_representations1989} And for Klein, this is a good thing---historical consciousness is uniquely valuable as a scientific endeavor and jettisoning this critical posture toward the past is tantamount to abandoning the enlightenment.  

For modern historians studying the cultural memory of other modern people, it is easy to conflate the historical consciousness of the historian subject and that of the object. Such historical work relies on court documents, news articles, eyewitness accounts, and other documentary evidence that operates within an historical consciousness that closely resembles that of the historian. As a result, the historian can utilize her own historical intuitions when interacting with her sources. In \halbwachs's terms, the social frameworks (in this case the understanding of the way ``history'' is done) of the historian and their object of study are quite similar. For example, reading news reports from the mid-twentieth century does not require the historian to dramatically reorient her understanding of what ``news'' is. On the other hand, when studying ancient history, the intellectual distance between the source and the historian is, often, much more pronounced. Reading ``historical'' texts from antiquity often requires a kind of hermeneutical suspicion that is different from that used by scholars reading texts from the recent past.%
%
\footnote{For example competent readers of modern newspapers know to bring a different set of suspicions to ``news'' articles versus editorials. Similarly, historians can read personal correspondence with a different kind of suspicion than monumental inscriptions.}

In fact, biblical scholars in particular have been dealing with this problem since the enlightenment. The tension between memory and history is played out clearly in both modern Jewish and Christian circles \visavis historical-critical study of the Bible. Insofar as the Bible forms a major portion of both Jewish and Christian collective memory, historical-critical approaches to the biblical text continue to be met with fervent opposition in more conservative traditions. Parallels to what Klein describes within the discipline of history can be seen within biblical studies as well. Consider, for example, the way that Brevard Childs's canonical approach attempted to ``overcome the long-established tension between the canon and criticism.''\autocite[45]{childs1979} For Childs, writing an introduction to the Old Testament in the traditional manner (i.e., as an historical-critical introduction) was insufficient for use in churches or synagogues because it bypassed a fundamental aspect of the biblical text, the canon. Although he does not use the language of memory in his discussion of canon (though, it should be noted he made an important early contribution to the idea of memory in the biblical tradition which, I imagine, is not a coincidence\autocite{childs1962}), here we can see that the various canons of scripture in use by Christians and Jews throughout the world nevertheless function as a form of collective memory by constructing and filtering what should and should not be remembered by the community.  

The tension between history and memory is most problematic---as evidenced by Childs---when the historian participates within the collective memory of the community under investigation. This is why both Childs and \yerushalmi express their discomfort and dissatisfaction while attempting to operate with one foot in each world. This is the central critique of Klein: historians operate from the outside looking in (an etic approach), while practitioners of memory operate from within (an emic approach). Yet, this etic/emic distinction only makes sense when memory is placed on equal footing with history as a means of interrogating the past. From this perspective, I wholeheartedly agree with Klein that such an approach undercuts the epistemological foundations of modern historical inquiry. However, Klein does not address memory as the \emph{object} of historical study. I think this is what makes \yerushalmi's approach so intriguing. Although he acknowledges his precarious position as a Jewish historian, \yerushalmi discusses memory \emph{as an historian} and it is this approach which I think is the most fruitful avenue of memory research. Thus, this dissertation will treat memory as a phenomenon which can be studied historically rather than as a source of information about the past.

\section{Memory, History, and the ``Actual Past''}

\halbwachs's did not see any reason to assume that the remembered past had any meaningful connection to the ``actual past.'' Because memory is always constructed in the present for use in the present, the ``actual past'' does not carry any meaningful influence on this (re)construction. It was in his \citetitle{halbwachs1941} that \halbwachs makes this case most forcefully, and I think he does so quite convincingly. \halbwachs's understanding of memory as a phenomenon of the present has thus earned him the label of ``presentist'' or ``constructivist'' over and against a number of more recent theorists who wish to attribute some normative force to the past.\autocite[27--30]{coser_halbwachs1992}  

\hypertarget{the-presentist-perspective}{%
\subsection{The Presentist Perspective}\label{the-presentist-perspective}}  

This presentist mantle has been donned by a number of more recent scholars, perhaps most notably by the German scholars Jan and Aleida Assmann.\autocites[See esp.][]{assmann_nikulin2015}{assmann2011}[and][]{a_assmann2011} Where \halbwachs distinguished between autobiographical and historical memory, Jan Assmann describes what he calls communicative and cultural memory (German: \emph{kommunikatives} and \emph{kulturelles} \emph{Gedächtnis}, respectively).\autocites[36]{assmann2011}[For a concise terminological crash-course, see][182--183]{hubenthal_carstens-hasselbalch2012} Rather than focus on the relationship of the rememberer to the experience (viz. whether the memory is ``autobiographical''), this terminology essentially distinguishes between synchronic and diachronic processes of memory. On the one hand, communicative memory represents a synchronic, or ``horizontal'' memory shared by a society at a particular chronological horizon based on direct communication between individuals. According to Assmann, this memory has a temporal horizon of 80--100 years---limited by spatial (where people are) and chronological (how long people live) factors. He writes:  

\begin{quote} A typical instance would be generational memory that accrues within the group, originating and disappearing with time or, to be more precise, with its carriers. Once those who embodied it have died, it gives way to a new memory.\autocite[36]{assmann2011} \end{quote}  

On the other hand, at the end of this crucial period, as particular memories begin to drop from current discourse and lose relevance; as those individuals with direct connections to the events, people, and places which the memories involve die off, the remembering community will either forget or transform the memory for long-term transmission in the form of \emph{cultural memory}. The canonization of memory at points during this period is a conscious, \emph{constructive} activity by a remembering group. \autocite[45]{assmann2011}  

Where \halbwachs's terminology took as its point of departure the psychological perspectives of Freud\autocite{terdiman_radstone-schwarz2011} and Bergson\autocite{ansellpearson_radstone-schwarz2011}, Assmann's taxonomy is rooted in ethnological research on oral traditions, specifically that of Jan Vansina and his notion of a ``floating gap'' between the present and the distant past.\autocite{vansina1985} Vansina observes that in oral cultures often there is an abundance of common knowledge about current goings-on and a similar abundance of shared knowledge about the distant past (esp.~with regard to origin stories and the like), but there often exists a gap for the not-so-distant past. The proportion of collective knowledge, therefore, is unevenly distributed between two chronological poles of memory, although members of the society in question may not perceive it as such in their own reconstructions of the past.%
%
\footnote{\cite[23--24]{vansina1985}. As Assmann observes \autocite[35]{assmann2011}, ``In the cultural memory of a group, both levels of the past merge seamlessly into one another.''}
%
In other words, from the perspective of the remembering society, often there exists a continuity between the distant past (often legendary or mythic) and the near-past (a few generations, at most) where in reality a good deal of the not-so-recent past has fallen from memory. Memory of the near-past---those things which, while not necessarily ``autobiographical'' to every rememberer, nevertheless are reinforced by those with autobiographical memory---is categorized as ``communicative'' because it is memory that it generated and spread in the present by those with direct access to the events in question. Those memories which are deemed significant enough to not be forgotten---those which will make up cultural memory---undergo a process by which they are transformed from ``factual into remembered history,'' and may take the form of myth or legend.\autocite[37--38]{assmann2011} Thus, according to Assmann, myth and legend cannot be distinguished from ``history'' as a part of cultural memory. The significance of an event is not tied to whether or not it is ``factual,'' but by its ``truth'' seen through its continued relevance to the remembering community in the present.\autocite[Paul Veyne offers a particularly stimulating discussion of the perception of the past and its relationship to myth. He concludes his book with the insightful quote, ``The theme of this book was very simple. Merely by reading the title, anyone with the slightest historical background would immediately have answered, `but of course they believed in their myths!' We have simply wanted also to make clear that what is true of `them' is also true of ourselves and to bring out the implications of this primary truth.''][128--129]{veyne1988} Assmann writes:  

\begin{quote} Myth is foundational history that is narrated in order to illuminate the present from the standpoint of its origins. The Exodus, for instance, regardless of any historical accuracy, is the myth behind the foundation of Israel; thus it is celebrated at Pesach and thus it is part of the cultural memory of the Israelites. Through memory, history becomes myth. This does not make it unreal---on the contrary, this is what makes it real, in the sense that it becomes a lasting, normative, and formative power.\autocite[38]{assmann2011} \end{quote}  

Assmann's understanding of the relationship of the actual past to a society's cultural memory, therefore is not concerned with the discussion of the historicity of cultural memory. Although Assmann does not dismiss cultural memory as a source for historical inquiry, like \halbwachs, his primary interest is in exploring the constructive, presentist aspects of memory.  

\hypertarget{the-continuity-perspective}{%
\subsection{The Continuity Perspective}\label{the-continuity-perspective}}  

Critics of \halbwachs's presentist posture (and the similar approaches of Jan and Aleida Assmann) agree that memory is malleable but argue that there are constraints placed upon memory which mitigate unbounded fictionalization of the remembered past. This so-called ``continuity'' (or ``essentialist'') perspective---primarily associated with the American sociologist Barry Schwartz---insists that the ``actual'' past carries some normative force in the shaping of collective memory in addition to the ``received'' past.\autocites[Schwartz has made numerous contributions to the field of memory studies. See esp.][]{schwartz_sf1982}{schwartz_asr1991}[and][]{schwartz2000}[Note also the SBL volume specifically interacting with his work:][]{thatcher2014} Critiquing \halbwachs, Schwartz writes:  

\begin{quote} Unfortunately, this [\halbwachs's presentist] perspective has problems of its own. It promotes the idea that our conception of the past is entirely at the mercy of current conditions, that there is no objectivity in events, nothing in history which transcends the peculiarities of the present.\autocite[376]{schwartz_sf1982} \end{quote}  

At the heart of the so-called ``continuity'' approach is the conviction that while memories are always conditioned by the present, there is a limit to the amount of distortion acceptable to the remembering community. As Michael Schudson puts it, ``The past is in some respects and under some conditions, highly resistant to efforts to make it over.''\autocite[107]{schudson_communication1989}  

In fact, I think the conceptual distance between the presentist and continuity perspectives is not as great as it may initially appear. Neither \halbwachs nor Assmann assert that there \emph{cannot} be any historical value to cultural/collective memory, nor that such memory cannot be used for historiographical purposes. For example, in \citetitle{halbwachs1941}, \halbwachs takes seriously that the figure of Jesus \emph{did} exist as an historical person while making it clear that he does not accept the Gospels at face value as historically accurate (he explicitly compares his basic approach toward the historicity of the Gospels as similar to that of Ernst Renan, which is problematic, as noted above).\autocite[205--206]{halbwachs1992} Throughout the work, \halbwachs does talk about the ``actual'' past and allows for the possibility that the Gospels do refer, at least partially, to real events. In other words, he does not make the argument that the Gospels were fabricated of whole-cloth and instead takes the position that the ``actual past'' is irretrievable and unknowable and that historical memory has no obligation to align with the actual past as such. On the other hand, Schwartz and the continuity perspective never argue that memory is \emph{accurate}, but instead that memory ought not be treated as \emph{entirely} arbitrary. In other words, the two perspectives both agree on the central premise that memory is shaped by society in the present, but they each approach the question of memory's connection to the actual past from opposite ends of the epistemological spectrum.  

This difference in perspective, I think, is attributable to the respective fields that Assmann and Schwartz deal with in their own research. As an Egyptologist dealing with literatures from the ancient Near East, Assmann necessarily is reliant on scant documentary evidence that may or may not have any supporting evidence whatsoever. The same can be said of other ancient fields such as biblical studies, Assyriology and Classics. Under these circumstances, the historian \emph{must} approach her sources with an appreciation for the intellectual gap that exists between the historian her source, particularly when not corroborated by an independent alternate source. On the other hand, Schwartz, as an Americanist, is able to marshal a plethora of contemporary sources for reconstructing the collective memory of the antebellum United States. What each scholar is able to assume about his sources speaks to their instincts toward the reliability of those sources. Furthermore, Schwartz deals with a comparatively disenchanted society whose historical consciousness more closely resembles our own, while Assmann deals with societies for whom myth and legend are not distinguished from history. Their historical methodologies may be the same, but the \emph{kinds} of sources that each field deals with creates a different set of scholarly instincts for dealing with the idea of memory and its relation to the actual past.  

Because this dissertation deals with the way that early Judaism interacted with its own received collective memory (rather than how it created those memories to begin with), I will tend to interact with the topic of collective memory from the perspective of \halbwachs and Assmann. This is not to say that I am entirely unsympathetic to Schwartz's critique of a purely presentist approach, only that the particulars of this project preclude the need to discuss the relationship between memory and the ``actual'' past.  


\section{Sites of Memory and Networks of Meaning}

Although most of us find it intuitive to talk about processes of individual memory, one of the main problems we have when talking about social and cultural memory is that we lack good language describe the structures and functions of those mnemonic systems at the level of society. As such, memory theorists have adopted analogies and terms to describe how societies remember and how individuals and groups interact with memory at the social level. It is important to remember that because social memory is a social construct we must not equate the remembered past with the events, experiences, and individuals which informed it. Where one might refer to an individual person having ``a memory'' of a particular event, there is no central repository---be it material or biological---of social memory.%
    \footnote{See especially \cite{brockmeier_cp2010} and \cite{wertsch_cp2011}.}  
As has been noted by numerous memory theorists, ``there is no such `thing' and social or collective memory.''%
    \footnote{\cite[14]{wilson2017} citing \cite[112]{olick-robbins_ars1998} and \cite[118--24]{wertsch_boyer-wertsch2009}.}
In other words, when we talk about social or cultural ``memory'' we are talking about a complex network of social processes and discourses which make up a society's understanding of the past.

These social processes and discourses tend to center around particular events, places, people, and ideas which the society has imbued with special mnemonic significance. These clusters of discourse are commonly referred to by memory theorists as ``sites'' of memory. The term ``site of memory'' is a translation of the French \emph{lieu de mémoire} was coined by Pierre Nora in the 1970's and has been adopted and adapted by numerous theorists since then.%
    \footnote{%
        The term was originally coined by Nora in the work
        \cite*{nora_goff-etal1978}, and used subsequently in 
        \cite*{nora1984} and 
        \cite*{nora_representations1989}. For a discussion of Nora's use of the term and its reception, see 
        \cite{szpociński_teksty-drugie2016}.}
Although Nora did not clearly define the term, a ``site of memory,'' as used by Nora, might better be translated as a ``place of remembrance,'' or a ``place where people remember.'' For Nora, modern-day ``sites'' of memory existed ``because there are no longer \emph{milieux de mémoire}, real environments of memory.''%
    \autocite[7]{nora_representations1989}
By Nora's reckoning, because modern historical consciousness has all but eradicated ``memory,'' the preservation of memory in the modern era has been relegated to particular ``sites'' of memory---monuments, structures, and practices whose purpose is to perpetuate memory. He writes:

\begin{quote}
    \emph{Lieux de mémoire} are simple and ambiguous, natural and artificial, at once immediately available in concrete sensual experience and susceptible to the most abstract elaboration. Indeed they are \emph{lieux} in three senses of the word---material, symbolic, and functional. Even an apparently purely material site, like an archive, becomes a \emph{lieu de mémoire} only if the imagination invests it with a symbolic aura. A purely functional site, like a classroom manual, a testament, or a veterans' reunion belong only inasmuch as it is also the object of a ritual. And the observation of a commemorative minute of silence, an extreme example of a strictly symbolic action, serves as a concentrated appeal to memory by literally breaking the temporal continuity.\autocite[18--19]{nora_representations1989}
\end{quote}

\noindent
Sites of memory, therefore, are not entirely abstract and intellectual, but bear on the practice and materiality of a society in addition to having symbolic significance.

Although Nora's original use of the term tended to focus especially on sites of memory which bear on so-called ``great traditions''%
    \footnote{As coined by Redfield in \cite*[41--42]{redfield1956}.}
of political and ideological importance such as national monuments and archives, the modern use of the term tends to be more abstract and to refer to any ``place'' where memory discourses occur within a society for the purpose of remembering. Such sites of memory may operate within any number of social/cultural spheres such as national memory (war memorials, national holidays, etc.), religious memory (religious holidays, symbolic ritual acts, etc.), or family memory (traditional foods, birthdays, anniversaries) and may be thought of as distinct, but connected ``nodes'' of symbolic meaning within a complex network of cultural symbols---what \halbwachs called the ``social frameworks of memory.''%
    \autocite[38]{halbwachs1992}  

Every edge and node within the graph of a society's collective memory is the product of memory construction. It is an abstraction. In much the same way that historiography offers a schematic narrative of past events which is necessarily selective and intentional about what specific events, people, and ideas are germane to the purpose of the historian, so too social and cultural memory is selective of the particulars which it preserves and constructive in how it presents people, events, and ideas within particular symbolic systems. Thus, sites of memory are social spaces where memory is constructed. For our purposes, and following a number of modern practitioners of memory studies, I will use the term ``site'' of memory to describe any discrete person, place, practice or idea where such discourses of memory occur.%
    \footnote{Within Hebrew Bible studies, see especially the work of Ehud Ben Zvi as well as his student Ian Wilson, esp. \cite[72--74]{benzvi_st2017} and \cite[25--26]{wilson2017}.}

The Hebrew Bible is replete with sites of memory---ideas, people, places, and practices which have been imbued with significance by numerous societies since antiquity and which form a central component to the identities and self-understanding of (especially) Jews and Christians throughout the world. Take, for example, the Exodus from Egypt. Regardless of the historical reality of such an event, the story of the Exodus as recounted in the Hebrew Bible is the central narrative undergirding the biblical rationale for Israel's possession of the Land. Likewise, the Israelites are told to be kind to strangers and sojourners within their community based on the memory of Israel's enslavement in Egypt. Similarly, the Torah could be understood as a distinct (and particularly potent) site of memory found in the Hebrew Bible; the same goes for the figure of Moses. Each of these sites of memory (the Exodus, Torah, and Moses) are distinct but they also exhibit clear relationships within the network of discourses which are found in the Hebrew Bible. And moreover, each site of memory also relates to and bears distinct significance for the various religious communities which hold the Hebrew Bible as a part of their tradition within their distinct systems of symbolic meaning. 

\section{Memory and \RwB}

Having laid the theoretical foundation of modern memory studies, we may now turn our attention to the particulars of this study, namely, addressing the way that social memory studies can meaningfully augment the scholarly discussion surrounding \rwb.\autocite[See also][]{brooke_zsengeller2014}  

At this point it should be fairly obvious how the Hebrew Bible may be convincingly framed as both the product and progenitor of collective and cultural memory during the late Persian and Early Hellenistic periods. In \halbwachs's terms, the biblical text represents the common, collective memory of \secondtemple Judaism which formed the basis for Jewish collective identity as a people of the land which \yahweh promised to Abraham and into which \yahweh lead the people of Israel, ``with a might hand and an outstretched arm'' after their miraculous escape from the land of Egypt and subsequent desert wanderings. Bracketing any discussions of the historicity of these biblical narratives, by the late \secondtemple period they would have been perceived as the true and central foundation narratives to any number of Jewish groups both in and out of Persian Yehud and Roman Palestine. In Assmann's terms, the biblical texts---and in particular the stories of the patriarchs, Exodus, and Conquest narratives---carried ``a lasting, normative, and formative power,''\autocite[38]{assmann2011} which can be observed concretely by their preservation both in antiquity (e.g., at \qumran as well in translation) and into the modern era.  

The process of textual interpretation, therefore, is itself a mnemonic process. Just as memories are recalled into and shaped by a set of social frameworks which may be alien to their original context, so too the interpretation of texts and traditions is shaped by the social frameworks of the interpreter. Because any single text or narrative represents only a sliver of the thick nexus of ideas that is collective memory, not only is a text always read into new social circumstances, but it is always read into a new literary context and discursive arena. No two readings of a given text will every be the same. Each reading is affected by the collective memory of the reader(s) which is constantly adapting and in flux as new memories are added and others are forgotten.  

\RwB, therefore, can be understood as a set of snapshots revealing the ways that the collective memory of \secondtemple Judaism was being used within Jewish communities (or, at the very least within some scribal circles) to shape remembering communities' identities. This shaping, however, was not a passive process, but elicited creative, constructive participation to not only ``read'' the past, but to rewrite it as well. These texts themselves would have contributed to the collective memories of their respective groups. The disparate ways that \rwb texts were passed on or jettisoned from various religious groups in antiquity illustrate the ways that new memories can be added to a group's cultural memory and be adopted as a part of its historical self-understanding. The three texts which I will treat in this dissertation each meet a different outcome. Chronicles---which I have framed (loosely) as a rewriting of Samuel--Kings---was adopted by both Jews and Christians in antiquity as a part of their cultural memory and became a part of both traditions' canon of scripture. The \ga, on the other hand, seems to have not survived within Judaism beyond the first century CE (although, it may have impacted some later traditions). Finally, \jub was not retained in Jewish circles, but \emph{was} passed on within certain segments of early Christianity and remains in liturgical use by the Ethiopian Orthodox Tewahedo Church.%
%
\footnote{\cite{baynes_mason-etal2012}; \cite{asale_bt2016}.}

Simply labeling these \rwb texts as examples of social or cultural memory, however, is rather uncontroversial. Such an assertion hardly requires a dissertation-length study and the task has already been sufficiently accomplished, to my mind, by Brooke.\autocite{brooke_zsengeller2014} Thus, this dissertation will attempt to go beyond simply labeling \rwb texts as exemplars of memory and instead attempt to offer a description of the processes by which \rwb texts functioned within the collective memory of \secondtemple Judaism(s). Many of these processes already exist within the scholarly discourse surrounding \rwb. For example, from the perspective of textual production, the topics of biblical interpretation, inner-biblical exegesis, and scribal culture are not new to the topic of biblical or \qumran studies. But each can provide valuable insights into the ways that groups of individuals understood and recalled their cultural memory and what in particular they found most valuable about their cultural memories. Approaching \rwb through the lens of social memory studies attempts to take a step back and address their \emph{function} as the means by which \secondtemple Judaism experienced its past in its present. Social memory studies, therefore, is not an alternative to more traditional modes of analysis, but a complement. 

Memory studies, therefore, provide a rich toolbox for thinking about and addressing the kinds of \emph{functional} questions which we raised in the first chapter. Discussing the ``purpose'' or ``function'' of a text is tantamount to discussing how a text can be both the product of and and participant in collective memory of its society. In other words, framing \rwb texts within the discourse of social and cultural memory means treating \rwb texts as more than creative ``exegetical'' works but also as cultural products which participate in social structures and discourses with concerns other than the explication of sacred texts. Thus, the following three chapters will offer readings of three \rwb texts understood from the perspective of memory studies with the aim of addressing questions of \emph{function} with respect to the pluriform collective and cultural memories of late \secondtemple Judaism(s).