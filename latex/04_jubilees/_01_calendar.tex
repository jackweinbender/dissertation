% !TEX root = dissertation.tex

\section{Restructuring the Past}

One of the most notable features of the Book of \jub is its preoccupation with the correct division of time---both with respect to a 364 day year as well as longer units encompassing multiple years. This concern with the proper division of time is reflected throughout the work and provides the central organizing principle for the book's rewriting of Gen 1--Exod 12.

The author of \jub makes it very clear that the proper division of time through a 364 day year is an essential practice for the correct observation of religious feasts and other holidays throughout the year. The pattern and significance of this 364 day cycle is explained to Moses after the Angel of the Presence retells the events of the Flood. The Angel explains the division of the year into four seasons, each beginning with a memorial day (\jub 6:23) and consisting of thirteen-weeks. The system as a whole yields a fifty-two week year (\jub. 6:29) and is presented as ``inscribed and ordained on the tablets of heaven'' (6:31; Eth.
    \eth{ተቄርፀ ፡ ወተሠርዐ ፡ ውስተ ፡ ጽላተ ፡ ሰማይ}
        {ta\qw{a}rḍa wa-tašarʕa wəsta ṣəllāta samāy}.
        % taqwarda       'engraved'     tG √qrḍ (Les. 440) 'to lacerate, tear away, etc.'
        % tašarʕa        'established'  tG √šrʕ (Les. 532) 'to establish, ordain'
        % ṣəllāta        'tablet' ṣəllē/ā - pl. ṣəllāt (Les. 554) 

The 364 day year is considered ``complete'' (Eth. \eth{}{TODO:}) by the Angel such that proper observance maintains synchrony year-over-year. In other words, adding or subtracting days from this calendar renders a ``floating'' calendar \visavis the absolute reference of the heavenly tablets.\footnote{TODO: citation for floating calendar} By comparison, the Angel of the Presence warns against the use of a lunar calendar because the lunar year is too short. \jub 6:36--37 reads:

\begin{ethiopictext}
    \versenum{6:36}
    ወይከውኑ ፡ እለ ፡ ያስተሐይጹ ፡ ወርኀ ፡ በሑያጼ ፡ ወርን ።
    ዕስመ ፡ ትማስን ፡ ይእቲ ፡ ጊዜያተ ፡ ወትቀድም ፡ እምዓመታት ፡ ለዓመት ፡ ዐሡረ ፡ ዕለተ ።
    \versenum{37}
    በእንተዝ ፡ ይመጽእ ፡ ዓመታተ ፡ ሎሙ ፡ እንዘ ፡ ያማስኑ ፡ ወይገብሩ ፡ ዕለተ ፡ ስምዕ ፡ ምንንተ ።
    ወዕለተ ፡ ርኵስተ ፡ በዓለ ፡ ወኵሉ ፡ ይዴምር ፡ ወማዋዕላ ፡ 
        ቅዱሳተ ፡ ርኩሰ ፡ ወዕለተ ፡ ርኵስተ ፡ ለዕለት ፡ ቅድስት ።
    ዕስመ ፡ ይስሕቱ ፡ አውራኀ ፡ ወሰንበታተ ወብዓላተ ፡ ወኢዩቤለ ።
\end{ethiopictext}
\begin{transliteration}
    % TODO: Normalize this
    \versenum{6:36}
    wa-yekawwenu ʔella yāstaḥayyeṣu warḥa ba-ḥuyāṣē warḫ
    % yekawwenu     G impf. √kwn 'to be' 
    % yāstaḥayyeṣu  CGt impf. √ḥyṣ 'to perceive, observe (closely)' (Les. 252) 
    % warḫa         warḥ pl. ʔawrāḫa 'moon, month' (Les. 617) 
    % ḥuyāṣē        'observation'
    ʕesma temās(s)en yeʔeti gizēyāta wa-teqaddem ʔem-ʕāmatāt la-ʕāmat ʕašur ʕelata
    % esma          'because'
    % temās(s)en    L impf. √msn 'decay, corrupt' (Les. 366)
    % yeʔeti        'she, it' (here, the moon)
    % gizēyāta      gizē pl. gizēyāt 'time, season' (Les. 210)
    % teqaddem      G impf. 3fs √qdm 'to precede, go before'
    % ʕāmatāt       ʕām pl. ʕāmatāt 'year' (Les. 62)
    % ʕašur         'the tenth day, ten days' (Les. 73)
    % ʕelata        'day' (Les. 603)
    \versenum{37}
    ba-ʔenta-ze yemaṣeʔ ʕāmatāta lomu ʔenza yāmāsenu wa-yegabru ʕelata semʕ menent
    wa-ʕelata re\kw{e}sta ba-ʕāla wa-\kw{e}llu yedēmer wa-māwāʕelā 
        qedusāta rekusā wa-ʕelata re\kw{e}sta laʕlat qedust
    ʕesma yeseḥetu ʔawrāḫa wa-sanbatāta wa-beʕālāta wa-ʔiyobēla
    % ʔewrāḫa         'months' warḥ op. cit.
\end{transliteration}
\begin{translation}
    % TODO:
    \versenum{6:36}
    [36] VanderKam: There will be people who carefully observe the moon with lunar observations 
    because it is corrupt (with respect to) the seasons and is early from year to year by ten days. 
    \versenum{37}
    [37] Therefore years will come about for them when they will disturb (the year) 
    and make a day of testimony something worthless and a profane day a festival. 
    Everyone will join together the holy days with the profane and the profane day with the holy day, 
    for they will err regarding the months, the sabbaths; the festivals, and the jubilee.
\end{translation}
\noindent
The contrast drawn to the lunar calendar combined with the fact that a 364 day calendar more closely approximates the actual period of Earth's orbit around the sun (approx. 365.24 days) led most early interpreters of \jub to call the 364 day calendar a ``solar'' calendar.\footnote{TODO: Get Charles, maybe?} Because some of the early Israelite festivals were tied to the agricultural year (for example, \emph{Shavuot} was celebrated after the wheat harvest, see Exod 34:22),
a solar calendar would indeed keep the calendar from drifting backward every year. Because the lunar (synodic) month%
    \footnote{The synodic month is derived from the length of time it takes the moon to process through its full cycle and is distinct from the period of the moon's \emph{orbit}.}
averages approximately 29.5 days, a lunar year (twelve synodic months) lasts approximately 354 days. Without any intercalation the calendar would drift back 11.24 days per year (a so-called ``revolving year''). Within a matter of only two-or-three years, the correlation between agricultural activity and cultic practice would break down.%
    \footnote{The major advantage of the lunar system is the ability for anybody to make reasonably accurate observations about when months begin and end. By contrast, the solar year requires a more subtle and long-term set of measurements. Most cultures which utilize a lunar calendar account for the discrepancy through the intercalation of an additional month every few years to bring the solar and lunar calendars into alignment. Most ``lunar'' calendars, therefore, are really lunisolar calendars, though exceptions (such as the Islamic calendar) do exist. See \cite[214, 238]{glessmer_flint-vanderkam1999}; \cite[37--38]{horowitz_janes1996}.}

More recent treatments of the 364-day calendar, however, have eschewed the ``solar'' label in most cases.\footnote{\cite[231]{glessmer_flint-vanderkam1999}\cite[80]{bendov_steele2011}; \cite[438]{jacobus_brooke-hempell2018}.} The rationale for doing so is two-fold. First, although a 364-day year is \emph{close} to the actual period of Earth's orbit around the sun, the 1.24 day discrepancy is large enough that after fifty years, the calendar would have floated backward a full two-months.\footnote{Specifically, 62 days. This would be the equivalent of celebrating the new year near Halloween.} In other words, although a 1.24 day drift may not be noticeable from one year to the next, the difference is significant \emph{enough} to be noticeable within the average lifespan of an individual and would certainly conflict with agriculturally contingent festivals.\footnote{\cite[28--37]{wacholder-wacholder_huca1995}. This assumes, of course, that the various festivals continued to be connected to the agricultural cycle and not a purely utopian construct as Wacholder and Wacholder suggest.} Second, while the Angel of the Presence expresses concern with the ``corruption'' of the yearly cycle, the rationale for the 364-day year is nowhere explicitly connected to the solar year. In other words, when the Angel of the Presence decries the deficiencies of the lunar year, it does so with respect to the 364-day year and \emph{not} with respect to the solar year. Instead, the concern with a 354-day year, according to the Angel of the presence is that the holidays, months, sabbaths, festivals, and jubilees will fall on the wrong days \emph{according to the 364-day calendar}. This rationale is, essentially, circular. The 364-day year is an absolute measure of a ``year'' according to the book of \jub---it is inscribed on the ``heavenly tablets'' as such---and is not contingent or defined with reference to the sun or the moon.

% Do the Jaubert thing

% Calendars were really important to some people in the second Temple period.
% What was the Calendar system of Jub?
% Was the calendar of Jub new and novel?
% The broader mileiu
% Reaction to luni-solar calendar; imposed by Antiochus? (so: Vanderkam)
% Which is more ancient?
% What about concurrent calendars?
% What was the purpose of the system (present)?
% What effect did it have on Gen-Ex?
% How does this restructuring engage with memory?
