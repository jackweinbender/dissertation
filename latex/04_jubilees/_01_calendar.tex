% !TEX root = dissertation.tex

\section{Restructuring the Past}

One of the most notable features of the Book of \jub is its preoccupation with the correct division of time---both with respect to a 364 day year as well as longer units encompassing multiple years. Although neither the 364-day year nor the larger 7 and 49 year units (``weeks'' of years and ``jubilees,'' respectively) are unique to the book of \jub, the proper division of time is into these units provides the central organizing principle for the book's rewriting of Gen 1--Exod 12.

The author of \jub makes it very clear that the proper division of time through a 364 day year is an essential practice for the correct observation of religious feasts and other holidays throughout the year. The pattern and significance of this 364 day cycle is explained to Moses after the Angel of the Presence retells the events of the Flood. The Angel explains the division of the year into four seasons, each beginning with a memorial day (\jub 6:23) and consisting of thirteen-weeks. The system as a whole yields a fifty-two week year (\jub. 6:29) and is presented as ``inscribed and ordained on the tablets of heaven'' (6:31; Eth.
    \eth{ተቄርፀ ፡ ወተሠርዐ ፡ ውስተ ፡ ጽላተ ፡ ሰማይ}
        {ta\qw{a}rḍa wa-tašarʕa wəsta ṣəllāta samāy}.
        % taqwarda       'engraved'     tG √qrḍ (Les. 440) 'to lacerate, tear away, etc.'
        % tašarʕa        'established'  tG √šrʕ (Les. 532) 'to establish, ordain'
        % ṣəllāta        'tablet' ṣəllē/ā - pl. ṣəllāt (Les. 554) 

The 364 day year is considered ``complete'' (Eth. \eth{}{TODO:}) by the Angel such that proper observance maintains synchrony year-over-year. In other words, adding or subtracting days from this calendar renders a ``floating'' calendar \visavis the absolute reference of the heavenly tablets.%
    \footnote{TODO: citation for floating calendar}
By comparison, the Angel of the Presence warns against the use of a lunar calendar because the lunar year is too short. \jub 6:36--37 reads:

% !TEX root = dissertation.tex
\begin{ethiopictext}
    \versenum{6:36}
    ወይከውኑ~፡ እለ~፡ ያስተሐይጹ~፡ ወርኀ~፡
    በሑያጼ~፡ ወርን~። እስመ~፡ ትማስን~፡ ይእቲ~፡ ጊዜያተ~፡ ወትቀድም~፡
    እምዓመታት~፡ ለዓመት~፡ ዐሡረ~፡ ዕለተ~።
    \versenum{37}
    በእንተዝ~፡ ይመጽእ~፡
    ዓመታተ~፡ ሎሙ~፡ እንዘ~፡ ያማስኑ~፡ ወይገብሩ~፡ ዕለተ~፡ ስምዕ~፡
    ምንንተ~። ወዕለተ~፡ ርኵስተ~፡ በዓለ~፡ ወኵሉ~፡ ይዴምር~፡ ወማዋዕላ~፡ 
    ቅዱሳተ~፡ ርኩሰ~፡ ወዕለተ~፡ ርኵስተ~፡ ለዕለት~፡ ቅድስት~። እስመ~፡
    ይስሕቱ~፡ አውራኀ~፡ ወሰንበታተ ወብዓላተ~፡ ወኢዩቤለ~።
\end{ethiopictext}
\begin{transliteration}
    \versenum{6:36}
    wa-yekawwenu ʔella yāstaḥayyeṣu warḥa 
    % yekawwenu     G impf. √kwn 'to be' 
    % yāstaḥayyeṣu  CGt impf. √ḥyṣ 'to perceive, observe (closely)' (Les. 252) 
    % warḫa         warḥ pl. ʔawrāḫa 'moon, month' (Les. 617) 
    ba-ḥuyāṣē warḫ ʔesma temās(s)en yeʔeti gizēyāta wa-teqaddem 
    % ḥuyāṣē        'observation'
    % esma          'because'
    % temās(s)en    L impf. √msn 'decay, corrupt' (Les. 366)
    % yeʔeti        'she, it' (here, the moon)
    % gizēyāta      gizē pl. gizēyāt 'time, season' (Les. 210)
    % teqaddem      G impf. 3fs √qdm 'to precede, go before'
    ʔem-ʕāmatāt la-ʕāmat ʕašur ʕelata
    % ʕāmatāt       ʕām pl. ʕāmatāt 'year' (Les. 62)
    % ʕašur         'the tenth day, ten days' (Les. 73)
    % ʕelata        'day' (Les. 603)
    \versenum{37}
    ba-ʔenta-ze yemaṣṣeʔ 
    % yemaṣṣeʔ          g impf 3ms √mṣʔ 'come, happen, arise, overtake'
    ʕāmatāta lomu ʔenza yāmās(s)enu wa-yegabberu ʕelata semʕ 
    % yāmās(s)enu       CL impf 3mp √msn 'decay, be corrupt' Les. 366
    % yegabberu         g impf 3mp √gbr 'act, do'
    % semʕ              √smʕ 'rumor, news, witness, testimony'
    mennenta wa-ʕelata re\kw{e}sta baʕāla wa-\kw{e}llu yedēmmer wa-māwāʕelā 
    % mennenta           D √mnn 'despise, reject, renounce'
    % rekwest           adj. √rkws 'to be unclean, impure' Les. 470
    % baʕāla            'festival' Les. 83
    % yedēmmer          D impf. √dmr 'to insert, add, mix, mingle, multiply (arithmetic)
    qedusāta rekusā wa-ʕelata re\kw{e}sta laʕlat qedust ʔesma
    % rekusa            adj. √rkws 'to be unclean, impure' Les. 470
    % laʕlat            
    yeseḥetu ʔawrāḫa wa-sanbatāta wa-beʕālāta wa-ʔiyobēla
    % yeseḥetu          √sḥt 'wound, harm, violate'
    % ʔewrāḫa           'months' warḥ op. cit.
    % sanbatāta         'sabbath'
\end{transliteration}
\begin{translation}
    \versenum{6:36}
    There will be those who watch the moon closely with lunar observations
    because it is deficient (concerning) the seasons and is premature from year to year by ten days. 
    \versenum{37}
    Therefore
    years will come about for them when they decay. And they will make a day of
    testimony despised (make) and a profane day a festival. All will mingle holy days
    (with) the profane and a profane day with a holy day, for
    the months will err along with the sabbaths and the festivals and the jubilee.
\end{translation}
\noindent
The contrast drawn to the lunar calendar combined with the fact that a 364 day calendar more closely approximates the actual period of Earth's orbit around the sun (approx. 365.24 days) led most early interpreters of \jub to call the 364 day calendar a ``solar'' calendar.%
    \footnote{TODO: Get Charles, maybe?}
Because some of the early Israelite festivals were tied to the agricultural year (for example, \emph{Shavuot} was celebrated after the wheat harvest, see Exod 34:22), a solar calendar would indeed keep the calendar from drifting backward every year. Because the lunar (synodic) month%
    \footnote{The synodic month is derived from the length of time it takes the moon to process through its full cycle and is distinct from the period of the moon's \emph{orbit}.}
averages approximately 29.5 days, a lunar year (twelve synodic months) lasts approximately 354 days. Without any intercalation the calendar would drift back 11.24 days per year (a so-called ``revolving year''). Within a matter of only two-or-three years, the correlation between agricultural activity and cultic practice would break down.%
    \footnote{The major advantage of the lunar system is the ability for anybody to make reasonably accurate observations about when months begin and end. By contrast, the solar year requires a more subtle and long-term set of measurements. Most cultures which utilize a lunar calendar account for the discrepancy through the intercalation of an additional month every few years to bring the solar and lunar calendars into alignment. Most ``lunar'' calendars, therefore, are really lunisolar calendars, though exceptions (such as the Islamic calendar) do exist. See \cite[214, 238]{glessmer_flint-vanderkam1999}; \cite[37--38]{horowitz_janes1996}.}

Recent treatments of the 364-day calendar, however, have eschewed the ``solar'' label in most cases.%
    \footnote{\cite[231]{glessmer_flint-vanderkam1999}\cite[80]{bendov_steele2011}; \cite[438]{jacobus_brooke-hempel2018}.}
The rationale for doing so is two-fold: first, although a 364-day year is \emph{close} to the actual period of Earth's orbit around the sun, the 1.24 day discrepancy is large enough that after fifty years, the calendar would have floated backward a full two-months.%
    \footnote{Specifically, 62 days. This would be the equivalent of celebrating the new year near Halloween.}
In other words, although a 1.24 day drift may not be noticeable from one year to the next, the difference is significant \emph{enough} to be noticeable within the average lifespan of an individual and would certainly conflict with agriculturally contingent festivals.%
    \footnote{\cite[28--37]{wacholder-wacholder_huca1995}. This assumes, of course, that the various festivals continued to be connected to the agricultural cycle and not a purely utopian construct as Wacholder and Wacholder suggest.}
Second, while the Angel of the Presence expresses concern with the ``corruption'' of the yearly cycle, the rationale for the 364-day year is not explicitly connected to the solar year. In other words, when the Angel of the Presence decries the deficiencies of the lunar year, it does so with respect to the 364-day year and \emph{not} with respect to the solar year. Instead, the problem with a 354-day (lunar) year, according to the Angel of the presence is that the holidays, months, sabbaths, festivals, and jubilees will fall on the wrong days \emph{according to the 364-day calendar}. This rationale is, essentially, circular. The 364-day year is an absolute measure of a ``year'' according to the book of \jub---it is inscribed on the ``heavenly tablets'' as such---and is not contingent or defined with reference to the sun or the moon. Instead, the author of \jub seems more concerned with the proper and even division of \emph{seasons} (defined as three months) and \emph{weeks} (a so-called heptadic structure) without the need for intercalation.%
    \footnote{\cite[125]{bendov-saulnier_cbr2008}.}

According to most reconstructions of \jub's 364-day calendar, the year was divided into four seasons consisting of exactly thirteen weeks (91 days). Each season was also divided into three months, though, because 91 does not divide evenly by 30, the third month in each season was counted as 31 days. Thus, each season was composed of two months of 30 days and one month of 31 days. Because these seasons' lengths divide evenly by seven, every season began on the same day of the week and followed an identical structure.%
    \footnote{In other words, every season began on the same day of the week, and the ``nth'' day of any given season was the same day of the week as the nth day of any other season.}
Thus, the ``memorial'' days  prescribed in \jub 6:23 would always fall on the same day of the week. In the same way, because the whole year also divides evenly by seven, every day of the year (in every year) implicitly referred to a particular day of the week.%
    \footnote{Thus, if a person were born on a Tuesday, every subsequent birthday would also fall on a Tuesday. Likewise, there would be no need to buy a new calendar every year, since every year is the same ``shape.'' See esp. \cite[253]{jaubert_vt1953}.}

Although the mechanics of this calendar are reasonably well understood, its purpose and antiquity remain matters of debate and the seminal work of Annie Jaubert (building on Barthélemy) during the mid-20th century remains the \emph{Ausganspunkt} for most discussions of the topic.%    
    \footnote{See especially \cite{jaubert_vt1953}; \cite{jaubert_vt1957}; \cite{jaubert1957}. The final work was translated into English as \cite*{jaubert1965}.}
Her thesis took as its point of departure Barthélemy's theory that the Jewish 364-day year began on Wednesday, the day that the sun and moon were created, according to the Priestly creation account in Genesis 1:14--19.%
    \footnote{\cite{barthelemy_rb1952}; \cite[250]{jaubert_vt1953}; \cite[24--25]{jaubert1965}.}
To prove this idea, she began by noting that the book of \jub specifically prohibits beginning a journey on the sabbath (50:8, 12) and infers that, therefore, the various travel narratives in \jub ought to obey this rule, e.g, when Abram travelled, he would not have done so on the Sabbath according to \jub. She worked backwards through the descriptions of such journeys in \jub to confirm that, indeed, the only possible situation where the patriarchs would not have traveled on the sabbath, as described in \jub demands that the first day of the year be a Wednesday.%
    \footnote{\cite[252--254]{jaubert_vt1953}; \cite[25--27]{jaubert1965}.}
Jaubert further hypothesized that the 364-day calendar utilized by the author of \jub was, in fact, quite ancient and reflected the same views of the latest Priestly strata of the Hexateuch by applying the same method to the Hexateuch and yielding an identical result.%
    \footnote{\cite[258]{jaubert_vt1953}; \cite[33]{jaubert1965}.}
Thus, according to Jaubert, the 364-day calendar was the calendar of \secondtemple Judaism and it was not until later---at the time of Ben Sira---that the lunar modifications known from the Rabbinic period were instituted.%
    \footnote{\cite[254--258; 262--264]{jaubert_vt1953}; \cite[47--51]{jaubert1965}.}

Jaubert's thesis has been challenged and modified over the past several decades, but the publication of a number of important calendrical texts from Qumran have---at least partially---served to support the broad strokes of her thesis that the 364 day calendar was in broad use during the late \secondtemple period (though the more specific claims remain controversial).%
    \footnote{Early reactions to her thesis were mixed. In particular, she was critiqued by Baumgarten (\cite*{baumgarten_tarbiz1962} translated into English as \cite*{baumgarten_baumgarten1977}) and more recently by Wacholder \& Wacholder (\cite*{wacholder-wacholder_huca1995}) and Ravid (\cite*{ravid_dsd2003}). Her thesis was adopted and slightly modified by Morgenstern (who made the first month of the quarter 31 days, rather than the last month; \cite*{morgenstern_vt1955}), at least partially supported by \vanderkam (\cite*[410--411]{vanderkam_cbq1979}) and still retains broad support generally, if at times (seemingly) by virtue of its ubiquity. See \cite[142]{bendov-saulnier_cbr2008}.}
What seems apparent from the more recently discovered evidence from Qumran is that the system of keeping time during the \secondtemple period was not a monolith. And while the book of \jub clearly participates in a tradition which privileged the 364-day year, the particulars of the \jub calendar and its theological and ideological underpinnings do not necessarily align with other advocates for the 364-day year (such as the Astronomical Book and the other calendrical texts from Qumran).%
    \footnote{See \cite[,159]{bendov-saulnier_cbr2008}. Although the calendar of \jub is distinct from other 364 day calendars inferred from the Qumran texts, many of the more general observations about their function apply to all such calendars and are frequently discussed together. The early discussions of Barthélemy and Jaubert mostly focused on \jub, as most of the Qumran scrolls had either not been discovered or not published at the time of writing. See \cite{barthelemy_rb1952} and \cite{jaubert_vt1957}.}
Thus, although the Astronomical Book (\firstenoch 72--82), the Aramaic Levi Document (), the Temple Scroll (), MMT (), and other calendrical () and liturgical () texts from Qumran all utilize a 364-day calendar, they do not all seem to agree on \emph{why} they follow it.%
    \footnote{For a concise summary of the calendrical issues in these texts, see \cite{vanderkam1998}; \cite[233--268]{glessmer_flint-vanderkam1999}; \cite[127--135]{bendov-saulnier_cbr2008}; and \cite{jacobus_brooke-hempel2018}.}


% Calendars were really important to some people in the second Temple period.
% What was the Calendar system of Jubilees?
% Was the calendar of Jubilees new and novel?
% The broader milieu
% Reaction to luni-solar calendar; imposed by Antiochus? (so: Vanderkam)
% Which is more ancient?
% What about concurrent calendars?
% What was the purpose of the system (present)?
% What effect did it have on Gen-Ex?
% How does this restructuring engage with memory?
