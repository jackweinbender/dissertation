% !TEX root = dissertation.tex

\section{Restructuring the Past}

One of the most nable features of the Book of \jub is its preoccupation with the correct division of time---both with respect to a 364 day year as well as longer units encompassing multiple years. This concern with the proper division of time is reflected throughout the work and provides the central organizing principle for the book's rewriting of Gen 1--Exod 12.

The author of \jub makes it very clear that the proper division of time through a 364 day year is an essential practice for the correct observation of religious feasts and other holidays throughout the year. The pattern and siginifcance of this 364 day cycle is explained to Moses after the Angel of the Presence retells the events of the Flood. The angel explains the division of the year into four seasons, each begining with a memorial day (Jub 6:23) and consisting of thirteen-weeks. The sysyem as a whole yeilds a fifty-two week year (Jub. 6:29) and is presented as ``inscribed and ordained on the tablets of heaven'' (6:31; Eth.
    \eth{ተቄርፀ ፡ ወተሠርዐ ፡ ውስተ ፡ ጽላተ ፡ ሰማይ}
        {ta\qw{a}rḍa wa-tašarʕa wəsta ṣəllāta samāy}.
        % taqwarda       'engraved'     tG √qrḍ (Les. 440) 'to lacerate, tear away, etc.'
        % tašarʕa        'established'  tG √šrʕ (Les. 532) 'to establish, ordian'
        % ṣəllāta        'tablet'               (Les. 554) ṣəllē/ā - pl. ṣəllāt

The 364 day year is considered ``complete'' (Eth. \eth{}{}) by the Angel such that proper observence maintains synchrony year-over-year. In other words, adding or subtracting days from this calendar renders a ``floating'' \visavis the absolute reference of the heavenly tablets. By comparison, the Angel of the Presence warns agains the use of a lunar calendar because the lunar year is too short. \jub 6:36--37 reads:

\begin{ethiopictext}
    \versenum{6:36}
    ወይከውኑ ፡ እለ ፡ ያስተሐይጹ ፡ ወርኀ ፡ በሑያጼ ፡ ወርን ።
    ዕስመ ፡ ትማስን ፡ ይእቲ ፡ ጊዜያታ ፡ ወትቀድም ፡ እምዓመታት ፡ ለዓመት ፡ ዐሡረ ፡ ዕለተ ።
    \versenum{37}
    በእንተዝ ፡ ይመጽእ ፡ ዓመታተ ፡ ሎሙ ፡ እንዘ ፡ ያማስኑ ፡ ወይገብሩ ፡ ዕለተ ፡ ስምዕ ፡ ምንንተ ።
    ወዕለተ ፡ ርኵስተ ፡ በዓለ ፡ ወኵሉ ፡ ይዴምር ፡ ወማዋዕላ ፡ 
        ቅዱሳተ ፡ ርኩሰ ፡ ወዕለተ ፡ ርኵስተ ፡ ለዕለት ፡ ቅድስት ።
    ዕስመ ፡ ይስሕቱ ፡ አውራኀ ፡ ወሰንበታተ ወብዓላተ ፡ ወኢዩቤለ ።
\end{ethiopictext}
\begin{transliteration}
    % TODO: Normalize this
    \versenum{6:36}
    wa-yekawwenu ʔella yāstaḥayyeṣu warḫa ba-ḥuyāṣē warḫ
    % yekawwenu     'to be' G impf. √kwn
    % yāstaḥayyeṣu  'to percieve, observe (closely)' CGt impf. √ḥyṣ (Les. 252) 
    % warḫa         'moon' (Les. 617) warḥ 'moon, month' pl. ʔawrāḫa
    % ḥuyāṣē        'observation'
    ʕesma temāsen yeʔeti gizēyātā wateqadum ʔemʕāmatāt laʕāmat ʕašur ʕelata
    \versenum{37}
    ba-ʔenta-ze yemaṣeʔ ʕāmatāta lomu ʔenza yāmāsenu wa-yegabru ʕelata semʕ menent
    wa-ʕelata re\kw{e}sta ba-ʕāla wa-\kw{e}llu yedēmer wa-māwāʕelā 
        qedusāta rekusā wa-ʕelata re\kw{e}sta laʕlat qedust
    ʕesma yeseḥetu ʔawrāḫa wa-sanbatāta wa-beʕālāta wa-ʔiyobēla
    % ʔewrāḫa         'months' warḥ op. cit.
\end{transliteration}
\begin{translation}
    \versenum{6:36}
    \versenum{37}
    % VanderKam: There will be people who carefully observe the moon with lunar observations because it is corrupt (with repsect to) the seasons and is early from year to year by ten days. // Therefore years will come about for them when they will disturb (the year) and make a day of testimony something worthless and a profane day a festival. Everyone will join together oth holy days with the profane and the profane day with eth eholy day, for they will err regarding the months, the sabbaths,; the festivals, and the jubilee.
\end{translation}

Although 364 days appraoches the actual period of Earth's orbit around the sun (approx. 365.24 days), the earlier characterizations of the Jublilees calendar as ``solar''  are generally not accepted by mondern scholars. 



% Calendars were really important to some people in the second Temple period.
% What was the Calendar system of Jub?
% Was the calendar of Jub new and novel?
% The broader mileiu
% Reaction to luni-solar calendar; imposed by Antiochus? (so: Vanderkam)
% Which is more ancient?
% What about concurrent calendars?
% What was the purpose of the system (present)?
% What effect did it have on Gen-Ex?
% How does this restructuring engage with memory?
