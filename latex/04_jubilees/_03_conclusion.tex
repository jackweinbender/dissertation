% !TEX root = dissertation.tex

\section{Bad Conclusion}
% TODO: Make this not awful
In this chapter I have attempted to differentiate the \emph{manner} that \rwb texts may have engaged with the cultural memory. I will focused on the ways that \jub portrayed itself as an authoritative revelation and how that posture affected concrete \emph{practice}. The book of \jub engages with cultural memory in a distinct fashion \visavis \ga by utilizing so-called ``authority conferring strategies'' which more directly engages the received tradition and calls on the reader to integrate her understanding of the past with the new information presented in the book. I have further argued that this distinction is significant because it illustrates the way that memory not only affects the intellectual conceptions of the past, but also carries with it \emph{concrete practical effects which can be concretely observed}. Drawing on Hindy Najman's work on Mosaic Discourse I discussed the ways that \jub portrays itself as authoritative literature, how this portrayal may have been understood in antiquity and how it could, in some sense, both authorize and rewrite the Torah with halakhic implications. Finally, to illustrate the point, I discussed the calendrical and chronological system of the book of \jub as an example of the concrete ways that constructing memory can impact one's understanding of the past and present and ultimately affect the praxis of the reader.