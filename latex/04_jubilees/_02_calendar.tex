% !TEX root = dissertation.tex

\section{Restructuring the Past}
% Intro to Calendars and Memory
% TODO:

% 364 Day Year
One of the most notable features of the Book of \jub is its preoccupation with the correct division of time---both with respect to a 364 day year as well as longer units encompassing multiple years. Although neither the 364-day year nor the larger 7 and 49 year units (``weeks'' of years and ``jubilees,'' respectively) are unique to the book of \jub, the proper division of time is into these units provides the central organizing principle for the book's rewriting of Gen 1--Exod 12.

The author of \jub makes it very clear that the proper division of time through a 364 day year is an essential practice for the correct observation of religious feasts and other holidays throughout the year. The pattern and significance of this 364 day cycle is explained to Moses after the \ap retells the events of the Flood. The \ap explains the division of the year into four seasons, each beginning with a memorial day (\jub 6:23) and consisting of thirteen-weeks. The system as a whole yields a fifty-two week year (\jub. 6:29) and is presented as ``inscribed and ordained on the tablets of heaven'' (6:31; Eth.
    \eth{ተቄርፀ ፡ ወተሠርዐ ፡ ውስተ ፡ ጽላተ ፡ ሰማይ}
        {ta\qw{a}rḍa wa-tašarʕa wəsta ṣəllāta samāy}.
        % taqwarda       'engraved'     tG √qrḍ (Les. 440) 'to lacerate, tear away, etc.'
        % tašarʕa        'established'  tG √šrʕ (Les. 532) 'to establish, ordain'
        % ṣəllāta        'tablet' ṣəllē/ā - pl. ṣəllāt (Les. 554) 

The 364 day year is considered ``complete'' (Eth. \eth{ፍጹመ}{fəṣṣuma}) by the \ap such that proper observance maintains synchrony year-over-year. In other words, adding or subtracting days from this calendar renders a ``revolving'' calendar \visavis the absolute reference of the heavenly tablets.%
    \footnote{For an overview of calendar systems in the ancient world, including a discussion of ``revolving calendars,'' see \cite[214]{glessmer_flint-vanderkam1999}.}
By comparison, the \ap warns against the use of a lunar calendar because the lunar year is too short. \jub 6:36--37 reads:

% !TEX root = dissertation.tex
\begin{ethiopictext}
    \versenum{6:36}
    ወይከውኑ~፡ እለ~፡ ያስተሐይጹ~፡ ወርኀ~፡
    በሑያጼ~፡ ወርን~። እስመ~፡ ትማስን~፡ ይእቲ~፡ ጊዜያተ~፡ ወትቀድም~፡
    እምዓመታት~፡ ለዓመት~፡ ዐሡረ~፡ ዕለተ~።
    \versenum{37}
    በእንተዝ~፡ ይመጽእ~፡
    ዓመታተ~፡ ሎሙ~፡ እንዘ~፡ ያማስኑ~፡ ወይገብሩ~፡ ዕለተ~፡ ስምዕ~፡
    ምንንተ~። ወዕለተ~፡ ርኵስተ~፡ በዓለ~፡ ወኵሉ~፡ ይዴምር~፡ ወማዋዕላ~፡ 
    ቅዱሳተ~፡ ርኩሰ~፡ ወዕለተ~፡ ርኵስተ~፡ ለዕለት~፡ ቅድስት~። እስመ~፡
    ይስሕቱ~፡ አውራኀ~፡ ወሰንበታተ ወብዓላተ~፡ ወኢዩቤለ~።
\end{ethiopictext}
\begin{transliteration}
    \versenum{6:36}
    wa-yekawwenu ʔella yāstaḥayyeṣu warḥa 
    % yekawwenu     G impf. √kwn 'to be' 
    % yāstaḥayyeṣu  CGt impf. √ḥyṣ 'to perceive, observe (closely)' (Les. 252) 
    % warḫa         warḥ pl. ʔawrāḫa 'moon, month' (Les. 617) 
    ba-ḥuyāṣē warḫ ʔesma temās(s)en yeʔeti gizēyāta wa-teqaddem 
    % ḥuyāṣē        'observation'
    % esma          'because'
    % temās(s)en    L impf. √msn 'decay, corrupt' (Les. 366)
    % yeʔeti        'she, it' (here, the moon)
    % gizēyāta      gizē pl. gizēyāt 'time, season' (Les. 210)
    % teqaddem      G impf. 3fs √qdm 'to precede, go before'
    ʔem-ʕāmatāt la-ʕāmat ʕašur ʕelata
    % ʕāmatāt       ʕām pl. ʕāmatāt 'year' (Les. 62)
    % ʕašur         'the tenth day, ten days' (Les. 73)
    % ʕelata        'day' (Les. 603)
    \versenum{37}
    ba-ʔenta-ze yemaṣṣeʔ 
    % yemaṣṣeʔ          g impf 3ms √mṣʔ 'come, happen, arise, overtake'
    ʕāmatāta lomu ʔenza yāmās(s)enu wa-yegabberu ʕelata semʕ 
    % yāmās(s)enu       CL impf 3mp √msn 'decay, be corrupt' Les. 366
    % yegabberu         g impf 3mp √gbr 'act, do'
    % semʕ              √smʕ 'rumor, news, witness, testimony'
    mennenta wa-ʕelata re\kw{e}sta baʕāla wa-\kw{e}llu yedēmmer wa-māwāʕelā 
    % mennenta           D √mnn 'despise, reject, renounce'
    % rekwest           adj. √rkws 'to be unclean, impure' Les. 470
    % baʕāla            'festival' Les. 83
    % yedēmmer          D impf. √dmr 'to insert, add, mix, mingle, multiply (arithmetic)
    qedusāta rekusā wa-ʕelata re\kw{e}sta laʕlat qedust ʔesma
    % rekusa            adj. √rkws 'to be unclean, impure' Les. 470
    % laʕlat            
    yeseḥetu ʔawrāḫa wa-sanbatāta wa-beʕālāta wa-ʔiyobēla
    % yeseḥetu          √sḥt 'wound, harm, violate'
    % ʔewrāḫa           'months' warḥ op. cit.
    % sanbatāta         'sabbath'
\end{transliteration}
\begin{translation}
    \versenum{6:36}
    There will be those who watch the moon closely with lunar observations
    because it is deficient (concerning) the seasons and is premature from year to year by ten days. 
    \versenum{37}
    Therefore
    years will come about for them when they decay. And they will make a day of
    testimony despised (make) and a profane day a festival. All will mingle holy days
    (with) the profane and a profane day with a holy day, for
    the months will err along with the sabbaths and the festivals and the jubilee.
\end{translation}
\noindent
The contrast drawn to the lunar calendar combined with the fact that a 364 day calendar more closely approximates the actual period of Earth's orbit around the sun (approx. 365.24 days) led most early interpreters of \jub to call the 364 day calendar a ``solar'' calendar.%
    \footnote{Some recent contributions retain this designation such as \cite[10]{stern2001}.}
Because some of the early Israelite festivals were tied to the agricultural year (for example, \emph{Shavuot} was celebrated after the wheat harvest, see Exod 34:22), a solar calendar would indeed keep the calendar from drifting backward every year. Because the lunar (synodic) month%
    \footnote{The synodic month is derived from the length of time it takes the moon to process through its full cycle and is distinct from the period of the moon's \emph{orbit}.}
averages approximately 29.5 days, a lunar year (twelve synodic months) lasts approximately 354 days. Without any intercalation the calendar would drift back 11.24 days per year (a so-called ``revolving year''). Within a matter of only two-or-three years, the correlation between agricultural activity and cultic practice would break down.%
    \footnote{The major advantage of the lunar system is the ability for anybody to make reasonably accurate observations about when months begin and end. By contrast, the solar year requires a more subtle and long-term set of measurements. Most cultures which utilize a lunar calendar account for the discrepancy through the intercalation of an additional month every few years to bring the solar and lunar calendars into alignment. Most ``lunar'' calendars, therefore, are really lunisolar calendars, though exceptions (such as the Islamic calendar) do exist. See
        \cite[214, 238]{glessmer_flint-vanderkam1999}; 
        \cite[37--38]{horowitz_janes1996}.}

Recent treatments of the 364-day calendar, however, have eschewed the ``solar'' label in most cases.%
    \footnote{%
        \cite[231]{glessmer_flint-vanderkam1999};
        \cite[80]{bendov_steele2011};
        \cite[438]{jacobus_brooke-hempel2018}.}
The rationale for doing so is two-fold: first, although a 364-day year is \emph{close} to the actual period of Earth's orbit around the sun, the 1.24 day discrepancy is large enough that after fifty years, the calendar would have floated backward a full two-months.%
    \footnote{Specifically, 62 days. This would be the equivalent of celebrating the new year near Halloween.}
In other words, although a 1.24 day drift may not be noticeable from one year to the next, the difference is significant \emph{enough} to be noticeable within the average lifespan of an individual and would certainly conflict with agriculturally contingent festivals.%
    \footnote{\cite[28--37]{wacholder-wacholder_huca1995}. This assumes, of course, that the various festivals continued to be connected to the agricultural cycle and not a purely utopian construct as Wacholder and Wacholder suggest.}
Second, while the \ap expresses concern with the ``corruption'' of the yearly cycle, the rationale for the 364-day year is not explicitly connected to the solar year. In other words, when the \ap decries the deficiencies of the lunar year, it does so with respect to the 364-day year and \emph{not} with respect to the solar year. Instead, the problem with a 354-day (lunar) year, according to the \ap is that the holidays, months, sabbaths, festivals, and jubilees will fall on the wrong days \emph{according to the 364-day calendar}. This rationale is, essentially, circular. The 364-day year is an absolute measure of a ``year'' according to the book of \jub---it is inscribed on the ``heavenly tablets'' as such---and is not contingent or defined with reference to the sun or the moon. Instead, the author of \jub seems more concerned with the proper and even division of \emph{seasons} (defined as three months) and \emph{weeks} (a so-called heptadic structure) without the need for intercalation.%
    \footnote{\cite[125]{bendov-saulnier_cbr2008}.}

According to most reconstructions of \jub's 364-day calendar, the year was divided into four seasons consisting of exactly thirteen weeks (91 days). Each season was also divided into three months, though, because 91 does not divide evenly by 30, the third month in each season was counted as 31 days. Thus, each season was composed of two months of 30 days and one month of 31 days. Because these seasons' lengths divide evenly by seven, every season began on the same day of the week and followed an identical structure.%
    \footnote{In other words, every season began on the same day of the week, and the ``nth'' day of any given season was the same day of the week as the nth day of any other season.}
The advantage of such a system is its consistency year-over-year. Because the whole year divides evenly by seven, every day of the year (in every year) implicitly referred to a particular day of the week. Thus any scheduled event would fall on the same day of the week the following year, preventing the undesirable situation where a holiday would accidentally fall on a Sabbath (such as the memorial feasts prescribed in \jub 6:23).%
    \footnote{\cite[233]{bergsma2007}. So, if a person were born on a Tuesday, every subsequent birthday would also fall on a Tuesday. Likewise, there would be no need to buy a new calendar every year, since every year is the same ``shape.'' See esp. \cite[253]{jaubert_vt1953}.}

Although the mechanics of this calendar are reasonably well understood, its purpose and antiquity remain matters of debate. The seminal work of Annie Jaubert (building on Barthélemy) during the mid-20th century, despite numerous criticisms, remains the \emph{Ausganspunkt} for most discussions of the topic.%    
    \footnote{See especially
        \cite{jaubert_vt1953};
        \cite{jaubert_vt1957};
        \cite{jaubert1957}.
        The final work was translated into English as
        \cite*{jaubert1965}.}
Her thesis took as its point of departure Barthélemy's theory that the Jewish 364-day year began on Wednesday, the day that the sun and moon were created, according to the Priestly creation account in Genesis 1:14--19.%
    \footnote{%
        \cite{barthelemy_rb1952};
        \cite[250]{jaubert_vt1953};
        \cite[24--25]{jaubert1965}.}
To prove this idea, she began by noting that the book of \jub specifically prohibits beginning a journey on the sabbath (50:8, 12) and infers that, therefore, the various travel narratives in \jub ought to obey this rule, e.g, when Abram travelled, he would not have done so on the Sabbath according to \jub. She worked backwards through the descriptions of such journeys in \jub to confirm that, indeed, the only possible situation where the patriarchs would not have traveled on the sabbath, as described in \jub demands that the first day of the year be a Wednesday.%
    \footnote{%
        \cite[252--254]{jaubert_vt1953};
        \cite[25--27]{jaubert1965}.}
Jaubert further hypothesized that the 364-day calendar utilized by the author of \jub was, in fact, quite ancient and reflected the same views of the latest Priestly strata of the Hexateuch by applying the same method to the Hexateuch and yielding an identical result.%
    \footnote{%
        \cite[258]{jaubert_vt1953};
        \cite[33]{jaubert1965}.}
Thus, according to Jaubert, the 364-day calendar was the calendar of \secondtemple Judaism and it was not until later---at the time of Ben Sira---that the lunar modifications known from the Rabbinic period were instituted.%
    \footnote{%
        \cite[254--258; 262--264]{jaubert_vt1953};
        \cite[47--51]{jaubert1965}.}

Jaubert's thesis has been challenged and modified over the past several decades, but the publication of a number of important calendrical texts from Qumran have---at least partially---served to support the broad strokes of her thesis that the 364 day calendar was in broad use during the late \secondtemple period (though the more specific claims remain controversial).%
    \footnote{Early reactions to her thesis were mixed. In particular, she was critiqued by Baumgarten 
    \autocite*{baumgarten_baumgarten1977} and more recently by Wacholder \& Wacholder 
    \autocite*{wacholder-wacholder_huca1995} and Ravid 
    \autocite*{ravid_dsd2003}. Her thesis was adopted and slightly modified by Morgenstern who suggested that the first month of the quarter was 31 days, rather than the last month; 
    \autocite*{morgenstern_vt1955}, at least partially supported by \vanderkam 
    \autocite*[410--411]{vanderkam_cbq1979} and still retains broad support generally, if at times (seemingly) by virtue of its ubiquity. See 
    \cite[142]{bendov-saulnier_cbr2008}.}
What seems apparent from the more recently discovered evidence from Qumran is that the system of keeping time during the \secondtemple period was not a monolith. As \vanderkam notes, among the Qumran texts the festivals were generally dated based on the 364-day calendar but there still remain cases where 354-day ``lunar'' year was used for more general purposes.\autocite[1:45]{vanderkam2018} And while the book of \jub clearly participates in a tradition which privileged the 364-day year, the particulars of the \jub calendar and its theological and ideological underpinnings do not necessarily align with other advocates for the 364-day year (such as the Astronomical Book and the other calendrical texts from Qumran).%
    \footnote{See \cite[,159]{bendov-saulnier_cbr2008}. Although the calendar of \jub is distinct from other 364 day calendars inferred from the Qumran texts, many of the more general observations about their function apply to all such calendars and are frequently discussed together. The early discussions of Barthélemy and Jaubert mostly focused on \jub, as most of the Qumran scrolls had either not been discovered or not published at the time of writing. See \cite{barthelemy_rb1952} and \cite{jaubert_vt1957}.}
In other words, one of the major observations from the most recent scholarship on the 364-day calendar tradition is that their commonalities are complimented by significant variation. So, although the Astronomical Book (\firstenoch 72--82), the Aramaic Levi Document, the Temple Scroll, MMT, \q{4}{252}{} and other astronomical (e.g., \q{4}{317}{}; \q{4}{318}{}), liturgical (Songs of the Sabbath Sacrifice; \q{11}{Psalms}{a}; \q{4}{503}{}; \q{4}{334}{}) and many formally calendrical texts%
    \footnote{Ben Dov and Saulnier lists several dozen texts and fragments of these calendrical texts in their recent summary. See \cite[132--133]{bendov-saulnier_cbr2008}.}
from Qumran tend to prefer a 364-day calendar, they do not all seem to agree on \emph{why} they follow it.%
    \footnote{For a concise summary of the calendrical issues in these texts, see 
        \cite{vanderkam1998};
        \cite[233--268]{glessmer_flint-vanderkam1999};
        \cite[127--135]{bendov-saulnier_cbr2008}; and 
        \cite{jacobus_brooke-hempel2018}.}
This diversity leaves open the question of what the purpose and significance of the 364-day calendar was for the author of the book of \jub and raises new questions about its polemical underpinnings.

The larger super-annual chronological cycles which concern the author of \jub also follow a heptadic structure. Throughout the work, the author refers to ``weeks'' of years (a seven-year interval) and the length of time known as a ``Jubilee'' (seven ``weeks'' of years, or 49 years) both of which are heptadic units which reflect the same concern with sabbath cycles as the intra-annual divisions.%
    \footnote{Indeed, as cited above in the prologue, the work is concerned with the ``the testimony for the event[s] of the years; for their weeks, for their jubilees in all the years of the world.''}
In fact, as \vanderkam has observed, while the calendar (364-day year) is only mentioned in \jub 6, the chronological system (7-year ``weeks'' and jubilees) is a pervasive and first-order literary device for the author's adaptation of Israel's past.%
    \footnote{\cite[522]{vanderkam-b_vanderkam2000}. He credits Wiesenberg with this observation as well who writes, ``His chronology, not his calendar, is the object of primary interest to the writer of the Book of Jubilees.'' See \cite[4]{wiesenberg_rev-qumran1961}.}

    %% TODO: This (following) paragraph should focus on Weeks first and their connection to the sabbath cycles, followed by a discussion of the Jubilee
The heptadic quality of the entire system of \jub's calendar and chronological systems is rooted in the traditions surrounding the sabbath and slave laws, which themselves show considerable development within the Hebrew Bible itself. The Sabbath and Jubilee legislation of Leviticus 25 likely draws from and adapts the earlier slave and fallow laws from the Covenant Code (Exod 21:1--11 and 23:10--11, respectively) and bears similarities with other ancient Near Eastern practices such as the \translit{mīšarum} and \translit{andurārum} known from Mesopotamia.%
    \footnote{For the ostensible antecedents of the biblical Jubilee see \cite[1--51]{bergsma2007}. Other major publications on the idea of the biblical Jubilee include 
        \cite{north1954};
        \cite{fager1993} and 
        \cite{lefebvre2003}.}
At the core of the Jubilee tradition in Leviticus 25 is an abstraction of the idea of sabbath ``rest'' on the seventh day of the week to longer seven-year units of time: the manumission of slaves, the forgiveness of debts, reallocation of ancestral lands, and letting the land lie fallow all occur in the seventh year, just as people were to rest on the sabbath day. Seven sets of these ``weeks'' completed a full cycle, which was then followed by a Jubilee year (year 50).%
    \footnote{\cite[85--92]{bergsma2007}.}

Within the book of \jub, however, the term Jubilee is used to delineate a period of 49 years, rather than to specify the 50th year.%
    \footnote{%
        \cite[524--525]{vanderkam-b_vanderkam2000};
        \cite[234]{bergsma2007}.}
Thus, when the author of \jub describes an event occurring in the \emph{nth} jubilee, he is referring to the the event occurring within a particular 49-year span and not in the \emph{nth} ``jubilee year.'' The term ``week'' or ``week of years,'' on the other hand, retains its traditional denotation.

% TODO: Needs a Transition

\subsection{Restructuring the Past}
As I have alluded to, frameworks for ordering the past are not neutral and the use of particular systems bears on ones interpretation of the past and understanding of the present. In other words, chronological systems can have a profound impact on processes of memory. Thus, the ordering of time with respect to the 364-day year, sabbath and jubilee traditions should be understood as not simply as the alignment of the past with an idiosyncratic numbering system, but as a reinterpretation and commemoration of Israel's past within a discrete social and ideological framework. 

The insistence of the author of \jub that the 364-day year be maintained and his sharp rebuke of those who ``closely observe the moon'' (\jub 6:36) point toward the likelihood that calendar conflicts were a point of contention between the author of \jub and some of his contemporaries. This apparently polemical tone used by the author has prompted speculation about the possible causes of such polemic. \vanderkam, for example has suggested that the impetus for the calendar dispute was Antiochus IV Epiphanes' imposition of a Hellenistic luni-solar calendar in-or-around \bce{167}. According to \vanderkam's theory the 364-day calendar was the calendar in use by the Jerusalem temple in the late Persian and early \secondtemple periods (generally following the argument of Jaubert). As evidence for Antiochus IV's calendrical changes, \vanderkam cites the numerous and infamous decrees made by Antiochus IV recounted in the books of Daniel and 1 \& 2 Maccabees. Although he concedes that none of these texts demand a calendrical change (only that the decrees prohibited certain festivals) \vanderkam reads Dan 7:25 to mean that the Seleucids did not only proscribe certain Jewish practices, but may have imposed a different calendar system.%
    \footnote{\cite[59--60; 68--69]{vanderkam_jsj1981}}
Daniel 7:23--25 reads:

\begin{aramaictranslation}
    \versenum{Dan 7:23}
    ‏כֵּן אֲמַר חֵיוְתָא רְבִיעָיְתָא מַלְכוּ רְבִיעָיָא תֶּהֱוֵא בְאַרְעָא דִּי תִשְׁנֵא מִן־כָּל־מַלְכְוָתָא וְתֵאכֻל כָּל־אַרְעָא וּתְדוּשִׁנַּהּ וְתַדְּקִנַּהּ׃ ‎
    \versenum{24}
    ‏ וְקַרְנַיָּא עֲשַׂר מִנַּהּ מַלְכוּתָה עַשְׂרָה מַלְכִין יְקֻמוּן וְאָחֳרָן יְקוּם אַחֲרֵיהוֹן וְהוּא יִשְׁנֵא מִן־קַדְמָיֵא וּתְלָתָה מַלְכִין יְהַשְׁפִּל׃ ‎
    \versenum{25}
    ‏ וּמִלִּין לְצַד עִלָּיָא יְמַלִּל וּלְקַדִּישֵׁי עֶלְיוֹנִין יְבַלֵּא וְיִסְבַּר לְהַשְׁנָיָה זִמְנִין וְדָת וְיִתְיַהֲבוּן בִּידֵהּ עַד־עִדָּן וְעִדָּנִין וּפְלַג עִדָּן׃
\end{aramaictranslation}

\begin{translation}
    \versenum{Dan 7:23} 
    Thus he said, ``As for the fourth beast, there will be a fourth kingdom on the earth which will be different from all the other kingdoms and it will consume the whole earth and trample it and crush it.
    \versenum{24}
    As for the ten horns---from it [the kingdom] ten kings will rise up and another will rise up after them and that one will be different from the previous ones and will bring down three kings.
    \versenum{25}
    And he will speak words against the Most High and he will wear-out the Holy Ones of the Most High and he will try to change the times and the Law and they will be given into his hand for a time, two times, and half a time.''
\end{translation}

\noindent
\vanderkam suggests that the Aramaic term \aramaic{זִמְנִין} in v. 25 may be equivalent to Hebrew \hebrew{מוֹעֲדִים} or \hebrew{עִתִּים} and thus may be referring to particular appointed times and festivals.%
    \footnote{\cite[59--60]{vanderkam_jsj1981}.}
\vanderkam further argues that 1 Macc 1:59 and 2 Macc 6:7a allude to the practice of celebrating the king's birthday with a sacrifice on a monthly basis (every \emph{nth} day of the month) which would have demanded that the Jerusalem temple to adopt the Seleucid calendar. Thus, he reasons, this may be the time when the traditional 364-day calendar was replaced by the Hellenistic lunisolar calendar in the Jerusalem temple. When the Maccabees took power, however, they did not, apparently, revert back to the older calendar. The conservative ``Essene'' group which later formed the Qumran community opposed this innovation and separated themselves from the Jerusalem priesthood. Thus, \vanderkam suggests that the calendar change/crisis may have been one of the major precipitating factors for the schism between the Qumran community and the Jerusalem temple authorities.\autocite[52]{vanderkam_jsj1981} \vanderkam's theory, however, has been met with some resistance, particularly from scholars such as Philip Davies, among others.%
    \footnote{%
        \cite{davies_cbq1983};
        \cite{wacholder-wacholder_huca1995};
        \cite{stern_lim-etal2000};
        \cite{stern_zpe2000};
        \cite[29 n. 136]{stern2001}.
        The core of these criticisms boil down to the fact that \vanderkam's theory is quite speculative and lacking in concrete \emph{positive} evidence of his historical reconstruction. The theory provides a clean explanation for a pressing historical question, but is perhaps a bit over-simplified. Ben Dov and Saulnier observe that \vanderkam's theory tends to be more popular among scholars who specifically study Essenes, while it is generally rejected by historians of the \secondtemple period more generally. See \cite[142]{bendov-saulnier_cbr2008}.}

For our purposes, the putative calendrical conflict to which Jubilees alludes points toward the \emph{significance} of such traditions for everyday practice. For the author of Jubilees (and, perhaps for the Qumran community) the calendar was not simply a mundane system for bookkeeping, but was intimately tied to liturgical  and cosmological order. Such a system aligns with God's created order which takes the seven-day week as its fundamental unit (as described in Gen 1). Such a system, one presumes, ought to respect the sanctity of the sabbath and prevent the overlap of holidays with the sabbath. The book of \jub does not appeal to observation or ``science'' but instead asserts the absolute fact of the 364-day year, as established by God and recorded on the heavenly tablets. 

Although the book of \jub portrays the 364-day year as a principle \emph{predicated on} a seven-day week and related numerical properties, in fact, from the perspective of memory construction and reinforcement, the opposite is the case. By insisting on the utilization of a calendar whose distinguishing characteristic is its protection of sabbath laws (i.e., that no holidays will ever conflict with the sabbath), and the consistently of memorial days \visavis the day of the week, the calendar reinforces the practices of observing the sabbath and the other holidays. It is a system which (though not, perhaps, designed for the purpose) reinforces some of the fundamental practices of early Judaism.

The larger cycles of weeks and jubilees likewise carry significance beyond their simple numerical values.
    % the jubilee cycle is used as a ``metaphor'' (key? frame?) as a way of understanding and interpreting the past of Israel

    % Jubilee cycles—connect this with Seleucid calendar as a framework for memory


% Conclusion of Section