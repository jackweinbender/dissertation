% !TEX root = dissertation.tex

\section{Constructing Authority in Jubilees}

The book of \jub engages directly in prescriptive discourses---\jub has a view of the world which it presents as an official, correct, understanding which is divinely ordained . \jub deals with legal and halakhic matters directly---it gives instructions about how and when to celebrate festivals and directly critiques the sinful behavior of Israel. It stands to reason, therefore, that the purpose of \jub was not simply religious entertainment or vaguely edifying storytelling; \jub does not mince words about what is right or wrong. While, the \ga may have \emph{implicitly} endorsed particular ideologies and halakhic practices through linking them with the foundational figures of Genesis (Lamech, Noah, and Abram), the book of \jub at times engages in direct imperative and presents itself as an authoritative text whose content comes directly from God, incised in the Heavenly Tablets, mediated by God's chief angelic being (the \ap), and ultimately recorded by Israel's most authoritative legal figure, Moses. Yet, the book of \jub makes clear that the Torah, too, is from God. For example, God tells Moses in \jub 1:9--10 that the people will stray from the covenant in part by ``forgetting'' God's commandments and neglecting proper cultic activities. Furthermore, the persecution of those who study the law is included in a catena of evil deeds that Israel will perpetrate:

% !TEX root = dissertation.tex
\begin{ethiopictext}
\end{ethiopictext}
\begin{transliteration}
\end{transliteration}
\begin{translation}
    \versenum{Jubilees 1:9}
    VanderKam: For they will forget all my commandments---everything that I command them---and will follow the nations, their impurities, and their shame. They will serve their gods and (this) will prove an obstacle for them---an affliction, a pain, and a trap.
    \versenum{10}
    Many will be destroyed. They will be captured and will fall into the enemy's control because they abandoned my statutes, my commandments, my covenantal festivals, my sabbaths, my holy things which I have hallowed for myself among them, my tabernacle, and my temple which I sanctified for myself in the middle of the land so that I could set my name on it and that it could live (there)
    \versenum{11}
    They  made for themselves high places, (sacred) groves, and carved images; each of them prostrated himself before his own in  order to go astray. They will sacrifice children to demons and to every product (conceived by) their erring minds.
    \versenum{12}
    I will send witnesses to them so that I may testify to them, but they will not listen and will kill the witnesses. they will persecute those too who study the law diligently. They will abrogate everything and will begin to do evil in my presence.

    \emph{**VanderKam's translation (in the interest of time). I will provide my own text and translation in subsequent drafts}
    % FIXME: through 14?
\end{translation}

\noindent
Thus, \jub at once affirms the centrality of the Torah, while, in some sense, circumventing it by providing its own idiosyncratic account of Gen 1--Exod 12. The juxtaposition of deference toward Torah while simultaneously circumventing its claim to primacy yields a sort of ``\psgraphical paradox.'' It is not immediately clear how a \psgraphical author, knowingly writing under a false name, can simultaneously endorse one text, while offering novel embellishments and changes to its interpretation. At least to the modern reader, this practice seems foreign and disingenuous by the \psgraphical author. The question should be raised, therefore, whether \jub \emph{was in fact} intending to supersede or circumvent the authority of the Torah (as some scholars suggest) or whether some other relationship existed between the texts.%
    \footnote{wacholder, for example, understands \jub and the \templescroll to be a single unit and a work which was meant to supersede the Pentateuch. See \cite{wacholder_kampen-etal1997}. His theory has not been widely accepted.}

Although there is some question whether the book of \jub attained the status of ``scripture'' in antiquity, it is generally agreed that \psgraphical texts such as \jub were not intended as replacements for the more well-known scriptures (especially the Torah).%
    \footnote{This position undoubtedly represents the majority opinion, though it is not unanimous. For the opposing opinion, see especially \cite{wacholder_kampen-etal1997}.}
Of course, the reality is that we do not know for certain what kinds of categories ancient readers used to classify their literature; most likely, however, they were not static nor consistent across time and differed by social group. All the same, the special place that the Torah had for a number of Jewish sects---even in antiquity---seems to me to preclude the idea that \psgraphical texts such as \jub would be placed on-par with the Pentateuch, even if a work carried a potent practical authority (see below). What \emph{can} be said about the book of \jub, however, is that \emph{it presents itself} as a unique revelation that claims for itself the same kind of divine source as the Torah.%
    \footnote{As a matter of clarification, I am assuming a distinction between 1) the author's intent, 2) the way the work presents itself, and 3) the way the work was understood by its readers. Thus the text may present itself as ``on-par'' with the Torah without either the author or audience treating it as such.}

\vanderkam has offered a concise summary and analysis of this ``\psgraphical paradox'' and comes to the conclusion that the book of \jub functions as a vehicle for its author to proffer his own interpretation of Gen 1--Exod 12. \vanderkam addresses the problem of \jub's author both acknowledging the existence and authority of the Torah while simultaneously offering his own original material, writing:

\begin{quote}
    [W]e could say differences in interpreting the Pentateuch had arisen by his time and that the author wanted to defend his own reading as the correct one. But he wished to find a way to package his case more forcefully than that, presumably within the limits of what was acceptable in his society.\autocite[28]{vanderkam_metso-etal2010}
\end{quote}

\noindent
According to \vanderkam, therefore, the project of the author of \jub was primarily one of \emph{exegesis}. The book of \jub is an expression of the author's understanding of Gen 1--Exod 12; it offers explicit teachings about specific ambiguities and difficulties in the text of Genesis and Exodus. He had a particular understanding of how the Pentateuch should be understood, and he used the common rhetorical technique of \psy to ``more forcefully'' get his point across.%
    \autocite[28]{vanderkam_metso-etal2010}

\vanderkam argues that the author of \jub intentionally located the setting of his work in the Exod 24:12 ascent for a rhetorical advantage. He argues for three such advantages: first, by locating the story during Moses's ascent, he is able to draw on the \emph{character} of Moses. The author, therefore was able to imbue his work with the gravitas of Israel's most famous lawgiver. Second, setting the work as a part of the first forty-day period that Moses was on Mt. Sinai grounds the author's interpretation of Torah in the original revelation of the Law (prior to even Deuteronomy). These events putatively took place at the same time that Moses received the first set of stone tablets from God. While the stone tablets were broken and had to be rewritten, the account provided in \jub is prior even to those ``copies'' of the decalogue. Any subsequent interpretation of the Torah is secondary by virtue of its relative lateness. Finally, because Moses himself is presented as the author of \jub, there is no question of the chain of transmission. God revealed the contents of \jub to Moses by having the \ap dictate to him the contents of the Heavenly Tablets. God is supreme, the tablets are eternal, and Moses is reliable.

Moses, therefore, received more from God on Mt. Sinai than is recorded in the Torah. The claim made by \jub is that it contains the additional information given to Moses, and that the subject of this additional revelation is the sacred history of Israel schematized according to the absolute heavenly reckoning of time (364-day years, weeks of years, and jubilees).

The tradition that God told Moses more on Mt. Sinai than he recorded in the Torah is not unique to the book of \jub. \vanderkam points toward the later rabbinic tradition that Moses received the Oral Torah during his time atop Mt. Sinai.%
    \footnote{\cite[28--31]{vanderkam_metso-etal2010}.}
For example, \vanderkam cites b. Berakot 5a, which references the specific time during which \jub is set (Exod 24:12):%
    \footnote{Translations of all rabbinic texts are my own.}

\begin{aramaictext}
    מאי דכתיב ואתנה לך את לחת האבן והתורה והמצוה אשר כתבתי להורתם לחת אלו עשרת הדברות תורה זה מקרא והמצוה זו משנה אשר כתבתי אלו נביאים וכתובים להרתם זה תלמוד מלמד שכולם נתנו למשה מסיני: 
\end{aramaictext}

\begin{translation}
    What is [the meaning where] it is written, \emph{I will give you the tablets of stone and the Torah and the commandments which I have written so that you might teach them} (Exod 24:12)?\\
    \-\hspace{2em}`the tablets' --- these are the ten commandments\\
    \-\hspace{2em}`the Torah' --- this is scripture\\
    \-\hspace{2em}`the commandments' --- this is Mishnah\\
    \-\hspace{2em}`that which I have written' --- these are the Prophets and the Writings\\
    \-\hspace{2em}`that you might teach them' --- this is Talmud\\\~
    [This] teaches that all of them were given to Moses on at Sinai.
\end{translation}

\noindent
The tradition here, therefore, asserts that the decalogue, the full Torah, its interpretation, the rest of the Tanakh, and the Talmud were all revealed to Moses on Sinai. Similarly, Sifra Beḥuqqotay 8, citing Lev 26:46:
\begin{aramaictext}
    אלה החקים והמשפטים והתורֹת: החוקים אלו המדרשות והמשפטים אלו הדינים והתורות מלמד ששתי תורות ניתנו להם לישראל אחד בכתב ואחד בעל פה
\end{aramaictext}
\begin{translation}
    \emph{These are the statutes and ordinances and Torahs} (Lev 26:46):\\
    \-\hspace{2em} `the statutes' --- this is midrash.\\
    \-\hspace{2em} `and the judgments' --- this is the legal rulings.\\
    \-\hspace{2em} `and the Torahs' --- [this] teaches that two Torahs were given to Israel: one in writing, the other by mouth.
\end{translation}

\noindent
The rhetorical function of asserting that later interpretive material was revealed to Moses is essentially the same as it is for \jub.

Thus, for \vanderkam, the book of \jub upholds the authority of the Torah by offering its own interpretation of its contents in a similar fashion to the way that the oral Torah, too, rooted its authenticity in the Sinai revelation. \jub, therefore asserts itself as a correct and authoritative interpretation of the Torah by claiming that it is the interpretation that Moses himself received from God; as \vanderkam puts it, according to the book of \jub, ``[t]he message of \jub is verbally inerrant.''%
    \footnote{
        \cite[33]{vanderkam_metso-etal2010}.
        Although the book of \jub is not generally thought to be the product of the Qumran community (it likely predates the settlement), it is worth noting that within the community, it was accepted that the community not only possessed the correct interpretation of its scriptures, but also that the community received a special revelation which the rest of Israel did not receive. As Fraade notes, this idea is quite different than supposing that additional material had been revealed \emph{to Moses}. See 
        \cite[67]{fraade_jjs1993}.}

While \vanderkam makes a number of useful observations, his characterization of \jub as exegesis, I think, ignores the question of how readers would have understood the work. This is where the analogy to the Oral Torah breaks down. While rabbinic claims that the Oral Torah was revealed to Moses, rabbinic discourse self-consciously acknowledges its work as exegetical---the rabbis offer explanations and instruction on how to understand the texts that they are commenting on. Although the rabbis may claim that an interpretation goes back to Moses, it is not the same as claiming to speak \emph{for} Moses or \emph{as} Moses. Thus \vanderkam's assertion that the purpose of writing pseudonymously and claiming that a work is the result of direct divine revelation goes beyond simply advocating for one's own interpretation ``more forcefully.'' The fact that \vanderkam leaves the particulars of this phrase ambiguous, I think, indicates ambiguity in his own thinking about \emph{how specifically} ancient readers may have understood \jub \visavis other so-called authoritative works, in particular, the Torah.

A more nuanced approach to this topic has been offered by Hindy Najman who, similarly has argued that the author of \jub utilized several ``modes of self-authorization'' in order to bolster its audience's perception of the work's authority.%
    \footnote{\cite[380]{najman_jsj1999}.}
Building on the work of Florentino García Martínez,\autocite{martinez_najman-tigchelaar2012} Najman argues that the book of \jub utilized (at least) four such ``authority conferring strategies,'' which I have reproduced in full:
    % FIXME: Paraphrase this?
    \begin{quote}
        1. \jub repeatedly claims that it reproduces material that had been written long before the ``heavenly tablets,'' a great corpus of divine teachings kept in heaven.

        2. The entire content of the book of \jub was dictated by the angel of the presence at God's own command. Hence, it is itself the product of divine revelation.

        3. \jub was dictated to Moses, the same Moses to whom the Torah had been given on Mount Sinai. Thus the book of \jub is the co-equal accompaniment of the Torah; both were transmitted by the same true prophet.

        4. \jub claims that its teachings are the true interpretation of the Torah. thus, its teachings also derive their authority from that of the Torah; that its interpretations match the Torah's words resolve all interpretive problems further substantiates its veracity.%
        \autocite[380]{najman_jsj1999}
    \end{quote}
\noindent
Her ultimate conclusion is that texts such as \jub which interpret and rewrite portions of the Bible do so to ``[respond] to both the demand for interpretation and the demand for demonstration of authority.''\autocite[408]{najman_jsj1999} Thus the purpose of the book of \jub, according to Najman, is to provide an ``interpretive context'' for reading the Torah---to make explicit a particular tradition of interpretation that guides the Torah-reader away from spurious or otherwise heterodox readings. 

This idea is similar to, but importantly distinct from \vanderkam's understanding of \jub. Whereas \vanderkam envisioned \jub as an exegetical \emph{product} of Gen 1--Exod 12, Najman understands \jub as a kind of ``background'' text which is meant as an aid \emph{for reading} Torah. The difference is subtle, but significant, especially for our understanding of \jub within the framework of cultural memory. \vanderkam's characterization of \jub as a sort of ``official'' interpretation of the Torah is problematic because it does not leave room for Torah going forward. If \jub portrays itself as \emph{the} meaning of Gen 1--Exod 12---the inerrant interpretation of this portion of Torah---what need is there for the Torah? Najman's model, on the other hand, assumes that readers are cued into the genre. Rather than  characterizing \jub as an authoritative, but idiosyncratic, interpretation of Torah, Najman's approach understands \jub as something that could be read \emph{before} the Torah in order to quash potentially errant readings of Torah when the reader finally reaches them.\autocite[408]{najman_jsj1999}

In her subsequent book, Najman builds on this thesis by introducing the idea of ``Mosaic Discourse'' into the discussion of Early Jewish and Christian literary production. She traces the practice of pseudonymous engagement with the Mosaic legal tradition through literary production back to the book of Deuteronomy.\autocite[48]{najman2003} She identifies four features of Mosaic discourse, which she extrapolates from the way that Deuteronomy draws from, augments, and affirms earlier legal traditions (such as the Covenant Code). The way that the author of Deuteronomy was able to both modify/reinterpret the legal tradition of the Covenant Code while retaining the traditions of the Covenant Code served as a model for later tradants (such as the author of \jub, but also the \templescroll and others) to repeat the process by engaging with and developing both the message of Moses and the idea of Moses as an author. This is what she refers to as ``Mosaic Discourse.'' With this term, Najman builds on a Foucauldian understanding of the Author which is neither static, nor bound by any historical or literary factors. She writes:

\begin{quote}
    As Foucault reminds us, it is not only \emph{texts} that develop over time. The connected \emph{concepts} of the authority and authorship of texts \emph{also} have long and complex histories. Both models of anonymity and of pseudonymity can be found in the Hebrew Bible and in the extra-biblical texts of the Second Temple period. But even when an author is identified in a biblical text, it is unclear if that identification is to be considered \emph{the same} as what moderns would characterize as \emph{the author function}.%
        \footnote{%
            \cite[9--10]{najman2003}. Here she is referencing
            \cite[213]{foucault_essential-foucault_2}.}
\end{quote}
\noindent
Najman suggests that when ancient writers participated in pseudonymous writing, the purpose was not to deceive their readers so much as to honor the tradition of the Author under whose name they wrote.%
    \footnote{Najman notes a number of classical authors who seem to have practiced a form of pseudonymity where a student writes in the name of their master. In particular, she cites Iamblichus the Pythagorean who claims that it was ``more honorable and praiseworthy'' to use Pythagorus' name, rather than one's own name when publishing (De Vita Pythagorica 98). She also quotes Tertullian who suggests that certain New Testament work sought to be ascribed to Paul and Peter because the works in question were written by their disciples (Marc. 6.5). Likewise, she notes that Plato wrote under the name of his master, Socrates. See \cite[13]{najman2003}.}
Historically speaking, of course, unless one posits that a real figure named Moses established the legal tradition of Israel, \emph{all} Mosaic attribution is, in effect, \psgraphical and an expansion of Moses the Author. The tradition of Moses the Author grew in step with the ``writings'' of Moses.

The book of \jub, therefore, can be understood as participating within this tradition of Mosaic attribution which serves to faithfully augment the body of Mosaic teaching through the use of \psy. The interpretation of the Torah by the writer of \jub is not meant to be understood as the ``actual words'' of Moses, but as a representation of ``authentic teaching'' which aligns with the function of Moses as an Author as an aide to reading the Torah.\autocite[13]{najman2003}

% \subsection{Memory, Mosaic Discourse, and Practice}
% Reframe mosaic discourse in terms of Memory
Unsurprisingly, Najman's approach to \jub and Mosaic Discourse dovetails quite well with the idea of social and cultural memory theory. The way that Najman describes the growth and development of the Author extending beyond the historical and literary bounds of the ``real'' author is evocative of the process of memory construction. In fact, from my perspective, what Najman describes as Mosaic Discourse \emph{is} a process of memory construction, though she does not use the terminology. What she describes as Mosaic Discourse is the same set of processes which enabled the author(s) of the Enochic works to expand on and speculate about the Watchers and the Flood, and which enabled the \ga to draw from those traditions in its rewriting.%
    \footnote{The \ga, of course, may have also drawn from the book of \jub, which only goes to further this point.}
Given the fascination with the figure of Enoch in the \secondtemple period (as evidenced by the plethora of texts which evoke the character), we could just as easily talk about ``Enochic Discourse'' when we discuss the various and sundry texts which draw from, expand, and reframe the enigmatic antedeluvian figure. Furthermore, we can easily identify additional Discourses about the figures of Abram, Daniel, and David, all of whom are the subjects of expanding bodies of literary production in the \secondtemple period, albeit not all as \psa, and not all with the same foundational significance as that of Moses.

% ALL DIZ SHIT IS MEMORY
The ability to also talk about these other discourses, I think, signals to the broader applicability of Najman's ideas. From my perspective, the discussion can be further augmented by including language of cultural and social memory which brings with it a taxonomy for discussing the processes in sociological terms. Najman's terminology is able to describe \emph{that} these various texts are participating in a particular discourse, but it does not \emph{describe} the discourse nor the \emph{social influences} or \emph{social effects} of the discourses. 

% FIXME: You need to clarify the above and figure out whether she actually does any of this. She's usually talking about "reading strategies" I think, so that will be an angle from which to depart.

% WHAT IS INTERESTING ABOUT JUB is the posture toward MEMORY: PRESCRIPTIVE, PRACTICE
Although the \ga and \jub generally participate in different sets of discourses (with notable exceptions, such as the division of the world sections), they engage with them in qualitatively different ways, despite the fact that both can be characterized as \psa. The \ga, although written largely in the first-person, takes a broadly \emph{de}scriptive approach to its memory construction through rewriting. Although there are parts of the text which portray its characters in ways that betray the author's own social frameworks (See chapter 3), the \ga resists readings which could be characterized as didactic or halakhic. \jub, on the other hand, includes a framing narrative which encourages the reader to not only reshape the way that they think about the characters within their rewriting, but also encourage particular kinds of \emph{practices}. In this sense \jub can be thought of as engaging with memory in more \emph{pre}scriptive discourses.

This shift from more \emph{descriptive} rewriting (like the \ga) to rewritings which incorporate \emph{prescriptive} discourses (like \jub) illustrates the social dimensions of talking about these texts as \emph{memory}. Even if the procedural, technical processes of interpretation and rewriting are identical, the social outcomes and concrete purposes of texts effect memory differently. For example, although the \ga utilizes the literary form of a \emph{waṣf}, this does not bear meaningfully on the concrete social effects of the \ga on memory. Even supposing the readers of \ga believed \ga to be the authentic ``historical''%
    \footnote{Here again, I am referring to the fact that, for the ancient readers of \jub, the character of Lamech/Noah/Abram was likely perceived as a real person from the distant past.}
accounts of Lamech, Noah, and Abram, the \ga simply \emph{does not ask} the reader to accept its account over and against any others. Its authenticity may be implied by its first-person rhetoric, but \ga does not exhibit the same kinds of authority conferring strategies that we see in \jub. 

On the other hand, the claim made by the book of \jub---that Moses received more information atop Mt. Sinai than is recorded in the books of the Pentateuch---is characterized as a kind of authoritative,revelatory literature which invites the reader to incorporate this new knowledge into their conception of the past in an act of cultural memory construction. The effect of this change on the remembered past does not remain in the abstract, however, but rather alters the way that the reader perceives the past with real-life, concrete, practical outcomes. 
