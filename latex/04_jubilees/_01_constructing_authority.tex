% !TEX root = dissertation.tex

Like the \ga, the book of \jub engages in a form of rewriting which participates in the construction of (biblical) memory through \psgraphical discourse. In the book of \jub Moses is repeatedly commanded to ``write down'' everything that he hears both from God (1:5, 7, 26) and from the \ap (2:1), thus the author presents the work itself as the record of the experience of Moses. Although not framed as a first-person account, \jub portrays itself as the product of first-hand experience, as indicated by the prologue.%
    \footnote{Notably, the prologue and first chapter of the book of \jub are preserved among the Qumran fragments (\q{4}{216}{}) and were, therefore, certainly a part of the work in antiquity---we have no reason to doubt that the prologue/superscription and framing narrative of the work were not a part of the most ancient versions of the work. See 
        \cite[1:125]{vanderkam2018};
        \cite[25]{vanderkam_metso-etal2010}.}
 Thus the rewritten account of ``biblical history'' from Gen 1--Exod 12 is, like \ga, \emph{drawn from} biblical memory and \emph{speaks back into} biblical memory through the process of rewriting.%
    \footnote{Here, again, I am using the term ``biblical memory'' to refer to the constellation of traditions which would have informed a reader's understanding of the Gen 1--Exod 12, and therefore \jub. Like \ga, \jub seems to draw from traditions also known from the Enochic corpus, including its use of a 364-day calendar.}
In the case of the \ga, this process left open the question of how, specifically, readers were intended to understand the ``authority'' of the account.

The book of \jub, however, engages more directly in prescriptive discourses, and it stands to reason that the purpose of \jub was not simply religious entertainment or vaguely ``edifying'' storytelling. Where, the \ga may have implicitly endorsed particular ideologies and halakhic practices through linking them with the foundational figures of Genesis (Lamech, Noah, and Abram), the book of \jub engages at times in direct imperative and presents itself as an ``authoritative'' text whose content comes directly from God, mediated by God's chief angelic being (the \ap), and ultimately through Israel's most authoritative legal figure, Moses. Yet, the book of \jub makes clear that the Torah, too, is from God, and that studying it is a good thing. Thus, \jub at once seems to assert the authority of the Torah, while subverting (or perhaps circumventing) the Torah by providing an alternate account of Gen 1--Exod 12. The juxtaposition of deference toward Torah while simultaneously circumventing its claim to primacy yields a sort of ``\psgraphical paradox.'' How can a text uphold the authority of a text while (in some sense) contradicting it (or at least tacitly admitting its inadequacy)? Was \jub trying to supersede the Torah?

Although there is some question whether the book of \jub attained the status of ``scripture'' in antiquity, it is generally agreed that \psgraphical texts such as \jub were not intended as replacements for the more well-known scriptures (especially the Torah).%
    \footnote{This position undoubtedly represents the majority opinion, though it is not unanimous. For the opposing opinion, see especially \cite{wacholder_kampen-etal1997}.}
Of course, the reality is that we do not know for certain what kinds of categories the ancients used to classify their literature. Most likely such categories were not static nor consistent across time or social groups. All the same, the special place that the Torah had---even in antiquity---seems to me to preclude the idea that \psgraphical  texts such as \jub would be placed on-par with them, even if they carried a potent practical authority (see below). What \emph{can} be said about the book of \jub, however, is that it presents itself as a unique revelation that claims for itself the same kind of divine source as the Torah.%
    \footnote{As a matter of clarification, I am assuming a distinction between 1) the author's intent, 2) the way the work presents itself, and 3) the way the work was understood by its readers. Thus the text may present itself as ``on-par'' with the Torah without either the author or audience treating it as such.}

\vanderkam has offered a concise summary and analysis of this ``\psgraphical paradox'' and comes to the conclusion that the book of \jub functions as a vehicle for its author to proffer his own interpretation of Gen 1--Exod 12. \vanderkam addresses the problem of \jub's author both acknowledging the existence and authority of the Torah while simultaneously offering his own original material, writing:

\begin{quote}
    [W]e could say differences in interpreting the Pentateuch had arisen by his time and that the author wanted to defend his own reading as the correct one. But he wished to find a way to package his case more forcefully than that, presumably within the limits of what was acceptable in his society.\autocite[28]{vanderkam_metso-etal2010}
\end{quote}

According to \vanderkam, therefore, the project of the author of Jubilees was primarily one of exegesis. He had a particular understanding of how the Pentateuch should be understood, and he used the common rhetorical technique of \psy to ``more forcefully'' get his point across.%
    \autocite[28]{vanderkam_metso-etal2010}
\vanderkam argues for three specific rhetorical advantages of setting the book at the Sinai revelation and on the lips of Moses. First, simply by using the figure of Moses, the author associated his work with the famous lawgiver of Israel. Second, setting the work as a part of the first forty-day period that Moses was on Mt. Sinai grounds the author's interpretation of Torah in the original revelation of the Law (prior to even Deuteronomy). Thus any subsequent interpretation of the Torah is, by virtue of its relative lateness, secondary. Finally, because Moses himself is presented as the author of \jub, there is no question of the chain of transmission. God revealed the contents of \jub to Moses by having the \ap dictate to him the contents of the Heavenly Tablets. God is supreme, the tablets are eternal, and Moses is reliable.

By way of analogy, \vanderkam points toward the later rabbinic tradition that Moses received more from God than he wrote in the Torah.%
    \footnote{\cite[28--31]{vanderkam_metso-etal2010}.}
For example, \vanderkam cites b. Berakot 5a, which references Exod 24:12:%
    \footnote{Translations of all rabbinic texts are my own.}

\begin{aramaictranslation}
    מאי דכתיב ואתנה לך את לחת האבן והתורה והמצוה אשר כתבתי להורתם לחת אלו עשרת הדברות תורה זה מקרא והמצוה זו משנה אשר כתבתי אלו נביאים וכתובים להרתם זה תלמוד מלמד שכולם נתנו למשה מסיני: 
\end{aramaictranslation}

\begin{translation}
    What is [the meaning where] it is written, ``I will give you the tablets of stone and the Torah and the commandments which I have written so that you them'' (Exod 24:12)?\\
    \-\hspace{2em}`the tablets' --- this is the ten commandments\\
    \-\hspace{2em}`the Torah' --- this is scripture\\
    \-\hspace{2em}`the commandments' --- this is Mishnah\\
    \-\hspace{2em}`that which I have written' --- this is the Prophets and the Writings\\
    \-\hspace{2em}`that you teach them' --- this is Talmud\\\~
    [This] teaches that all of them were given to Moses on at Sinai.
\end{translation}

\noindent
Similarly, Sifra Beḥuqqotay 8, citing Lev 26:46:
\begin{aramaictranslation}
    אלה החקים והמשפטים והתורֹת: החוקים אלו המדרשות והמשפטים אלו הדינים והתורות מלמד ששתי תורות ניתנו להם לישראל אחד בכתב ואחד בעל פה
\end{aramaictranslation}
\begin{translation}
    These are the statutes and ordinances and Torahs (Lev 26:46):\\
    \-\hspace{2em} `the statutes' --- this is midrash.\\
    \-\hspace{2em} `and the judgments' --- this is the legal rulings.\\
    \-\hspace{2em} `and the Torahs' --- [this] teaches that two Torahs were given to Israel: one in writing, the other by mouth.
\end{translation}

\noindent
Thus, for \vanderkam, the book of \jub upholds the authority of the Torah by offering its own interpretation of its contents in a similar fashion to the way that the oral Torah, too, claimed a similar primacy. \jub, therefore asserts itself as a correct and authoritative interpretation of the Torah by claiming that it is the interpretation that Moses himself received from God; as \vanderkam puts it, according to the book of \jub, ``[t]he message of \jub is verbally inerrant.''%
    \footnote{
        \cite[33]{vanderkam_metso-etal2010}.
        Although the book of \jub is not generally thought to be the product of the Qumran community (it likely predates the settlement), it is worth noting that within the community, it was accepted that the community not only possessed the correct interpretation of its scriptures, but also that the community received a special revelation which the rest of Israel did not receive. As Fraade notes, this idea is quite different than supposing that additional material had been revealed \emph{to Moses}. See 
        \cite[67]{fraade_jjs1993}.}

Hindy Najman, too, has argued that the author of \jub utilized several ``modes of self-authorization'' in order to bolster its audience's perception of the work's authority.%
    \footnote{\cite[380]{najman_jsj1999}.}
Building on the work of Florentino García Martínez,\autocite{martinez_najman-tigchelaar2012} Najman argues that the book of \jub utilized (at least) four such ``authority conferring strategies,'' which I have reproduced in full:
    \begin{quote}
        1. Jubilees repeatedly claims that it reproduces material that had been written long before the ``heavenly tablets,'' a great corpus of divine teachings kept in heaven.

        2. The entire content of the book of \jub was dictated by the angel of the presence at God's own command. Hence, it is itself the product of divine revelation.

        3. \jub was dictated to Moses, the same Moses to whom the Torah had been given on Mount Sinai. Thus the book of \jub is the co-equal accompaniment of the Torah; both were transmitted by the same true prophet.

        4. \jub claims that its teachings are the true interpretation of the Torah. thus, its teachings also derive their authority from that of the Torah; that its interpretations match the Torah's words resolve all interpretive problems further substantiates its veracity.%
        \autocite[380]{najman_jsj1999}
    \end{quote}
\noindent
Her ultimate conclusion is that texts such as \jub which interpret and rewrite portions of the Bible do so to ``[respond] to both the demand for interpretation and the demand for demonstration of authority.''\autocite[408]{najman_jsj1999} Thus the purpose of the book of \jub, according to Najman is to provide an ``interpretive context'' for reading the Torah---to make explicit a particular tradition of interpretation that guides the Torah-reader away from spurious or otherwise heterodox readings. 

In her subsequent book, Najman builds on this thesis by introducing the idea of ``Mosaic Discourse'' into the discussion of Early Jewish and Christian literary production. She traces the practice of pseudonymous engagement with the Mosaic legal tradition through literary production back to the book of Deuteronomy.\autocite[48]{najman2003} She identifies four features of Mosaic discourse, which she extrapolates from the way that Deuteronomy draws from, augments, and affirms earlier legal traditions (such as the Covenant Code).
The way that the author of Deuteronomy was able to both modify/reinterpret the legal tradition of the Covenant Code while (apparently) retaining the  with later, similar \psgraphical texts such as \jub and the \templescroll. 

Rather than treat these \psgraphical texts simply as ``pious forgeries,'' Najman builds on a Foucauldian understanding of the Author which is neither static, nor bound by the actual, historical author. She writes:

\begin{quote}
    As Foucault reminds us, it is not only \emph{texts} that develop over time. The connected \emph{concepts} of the authority and authorship of texts \emph{also} have long and complex histories. Both models of anonymity and of pseudonymity can be found in the Hebrew Bible and in the extra-biblical texts of the Second Temple period. But even when an author is identified in a biblical text, it is unclear if that identification is to be considered \emph{the same} as what moderns would characterize as \emph{the author function}.%
        \footnote{%
            \cite[9--10]{najman2003}. Here she is referencing
            \cite[213]{foucault_essential-foucault_2}.}
\end{quote}
\noindent
Najman suggests that when ancient writers participated in pseudonymous writing the purpose was not to deceive their readers so much as to honor the tradition of the Author under whose name they wrote.%
    \footnote{Najman notes a number of classical authors who seem to have practiced a form of pseudonymity where a student writes in the name of their master. In particular, she cites Iamblichus the Pythagorean who claims that it was ``more honorable and praiseworthy'' to use Pythagorus' name, rather than one's own name when publishing (De Vita Pythagorica 98). She also quotes Tertullian who suggests that certain New Testament work sought to be ascribed to Paul and Peter because the works in question were written by their disciples (Marc. 6.5). Likewise, she notes that Plato wrote under the name of his master, Socrates. See \cite[13]{najman2003}.}

The book of \jub, therefore, can be understood as participating within a tradition of Mosaic attribution which serves to faithfully augment the body of Mosaic teaching through the use of \psy. The interpretation of the Torah by the writer of \jub is not meant to be understood as the ``actual words'' of Moses, but as a representation of ``authentic teaching'' which aligns with the function of Moses as an Author.\autocite[13]{najman2003}

The fact that \jub offers a \emph{particular} tradition of interpretation, however, assumes the possibility that other interpretations may exist, and indeed, that other competing interpretations \emph{did} exist in antiquity. After all, there would be no need to offer a clarifying interpretation if the source text were immune from ``misreading,'' or if the problem of misreading or heteropraxis were not a reality for the author. 

The reality of other interpretations, moreover, sheds light on the practical problem of discussing ``authoritative texts'' because ``texts,'' in fact, do not exercise authority. Texts must be interpreted (texts cannot speak), and authority is exercised within a society \emph{by people} (texts cannot \emph{do} anything). Thus when one speaks of ``authoritative texts,'' one may speak not only of the abstract ``status'' of a text (whether it is ``scripture'' or not), but also more concretely the way that interpretations of texts are used socially to affect behavior and exert power.\autocite{foucault_ci1982} To be sure, the perception of a text (whether it is sacred scripture or otherwise intrinsically distinct) can affect its potency in bringing about practical effects, however, such perceptions are always \emph{socially} defined and are not truly intrinsic to the text. Societies imbue texts with significance. In other words issues of ``textual authority'' include both how texts are perceived in the abstract as well as how they affect concrete \emph{praxis}, but both qualities are socially defined and enforced.%
    \footnote{This falls under the category of what Brooke calls ``acted authority.'' See
        \cite[519--523]{brooke_rev-qumran2012} and the discussion in 
        \cite[475]{debel_jsj2014}.}

