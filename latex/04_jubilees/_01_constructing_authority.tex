% !TEX root = dissertation.tex

Like the \ga, the book of \jub engages in a form of rewriting which participates in the construction of (biblical) memory through \psgraphical discourse. In the book of \jub Moses is repeatedly commanded to ``write down'' everything that he hears both from God (1:5, 7, 26) and from the \ap (2:1), thus the author presents the work itself as the record of the experience of Moses. Although not framed as a first-person account, \jub portrays itself as the product of first-hand experience, as indicated by the prologue.%
    \footnote{Notably, the prologue and first chapter of the book of \jub are preserved among the Qumran fragments (\q{4}{216}{}) and were, therefore, certainly a part of the work in antiquity---we have no reason to doubt that the prologue/superscription and framing narrative of the work were not a part of the most ancient versions of the work. See \cite[1:125]{vanderkam2018}; \cite[25]{vanderkam_metso-etal2010}.}
 Thus the rewritten account of ``biblical history'' from Gen 1--Exod 12 is, like \ga, \emph{drawn from} biblical memory and \emph{speaks back into} biblical memory through the process of rewriting.%
    \footnote{Here, again, I am using the term ``biblical memory'' to refer to the constellation of traditions which would have informed a reader's understanding of the Gen 1--Exod 12, and therefore \jub. Like \ga, \jub seems to draw from traditions also known from the Enochic corpus, including its use of a 364-day calendar.}
In the case of the \ga, this process left open the question of how, specifically, readers were intended to understand the ``authority'' of the account.

The book of \jub, however, engages more directly in prescriptive discourses, and it stands to reason that the purpose of \jub was not simply religious entertainment or vaguely ``edifying'' storytelling. Where, the \ga may have implicitly endorsed particular ideologies and halakhic practices through linking them with the foundational figures of Genesis (Lamech, Noah, and Abram), the book of \jub engages at times in direct imperative and presents itself as an ``authoritative'' text whose content comes directly from God, mediated by God's chief angelic being (the \ap), and ultimately through Israel's most authoritative legal figure, Moses. Yet, the book of \jub makes clear that the Torah, too, is from God, and that studying it is a good thing. Thus, \jub at once seems to assert the authority of the Torah, while subverting (or perhaps circumventing) the Torah by providing an alternate account of Gen 1--Exod 12. 

The juxtaposition of deference toward Torah while simultaneously circumventing its claim to primacy yields a sort of \psgraphical paradox. How can a text uphold the authority of a text while (in some sense) contradicting it? Moreover, why would an author do this? 

\vanderkam has recently offered a concise summary and analysis of this problem and comes to the conclusion that the book of \jub functions as a vehicle for its author to proffer his own interpretation of Gen 1--Exod 12. \vanderkam addresses the problem of \jub's author both acknowledging the existence and authority of the Torah while simultaneously offering his own original material, writing:

\begin{quote}
    [W]e could say differences in interpreting the Pentateuch had arisen by his time and that the author wanted to defend his own reading as the correct one. But he wished to find a way to package his case more forcefully than that, presumably within the limits of what was acceptable in his society.\autocite[28]{vanderkam_metso-etal2010}
\end{quote}

According to \vanderkam, therefore, the project of the author of Jubilees was primarily one of exegesis. He had a particular understanding of how the Pentateuch should be understood, and he used the common rhetorical technique of \psy to ``more forcefully'' get his point across.%
    \autocite[28]{vanderkam_metso-etal2010}
\vanderkam argues for three specific rhetorical advantages of setting the book at the Sinai revelation and on the lips of Moses. First, simply by using the figure of Moses, the author associated his work with the famous lawgiver of Israel. Second, setting the work as a part of the first forty-day period that Moses was on Mt. Sinai grounds the author's interpretation of Torah in the original revelation of the Law (prior to even Deuteronomy). Thus any subsequent interpretation of the Torah is, by virtue of its relative lateness, secondary. Finally, because Moses himself is presented as the author of \jub, there is no question of the chain of transmission. God revealed the contents of \jub to Moses by having the \ap dictate to him the contents of the Heavenly Tablets. God is supreme, the tablets are eternal, and Moses is reliable.

Hindy Najman, too, has described the ways that the author of \jub bolstered the perceived authority of his text, and enumerated four so-called ``authority conferring strategies'' used in \jub.%
    \footnote{\cite[380]{najman_jsj1999}.}

By way of analogy, \vanderkam points toward the later rabbinic tradition that Moses received more from God than he wrote in the Torah.\footnote{\cite[28--31]{vanderkam_metso-etal2010}.} For example, \vanderkam cites b. Berakhot 5a, which references Exod 24:12:%
    \footnote{Translations of all rabbinic texts are my own.}

\begin{aramaictranslation}
    מאי דכתיב ואתנה לך את לחת האבן והתורה והמצוה אשר כתבתי להורתם לחת אלו עשרת הדברות תורה זה מקרא והמצוה זו משנה אשר כתבתי אלו נביאים וכתובים להרתם זה תלמוד מלמד שכולם נתנו למשה מסיני: 
\end{aramaictranslation}

\begin{translation}
    What is [the meaning where] it is written, ``I will give you the tablets of stone and the Torah and the commandments which I have written so that you them'' (Exod 24:12)?\\
    \-\hspace{2em}`the tablets' --- this is the ten commandments\\
    \-\hspace{2em}`the Torah' --- this is scripture\\
    \-\hspace{2em}`the commandments' --- this is Mishnah\\
    \-\hspace{2em}`that which I have written' --- this is the Prophets and the Writings\\
    \-\hspace{2em}`that you teach them' --- this is Talmud\\\~
    [This] teaches that all of them were given to Moses on at Sinai.
\end{translation}
\noindent
Similarly, Sifra Beḥuqqotay 8, citing Lev 26:46:
\begin{aramaictranslation}
    אלה החקים והמשפטים והתורֹת: החוקים אלו המדרשות והמשפטים אלו הדינים והתורות מלמד ששתי תורות ניתנו להם לישראל אחד בכתב ואחד בעל פה
\end{aramaictranslation}
\begin{translation}
    These are the statutes and ordinances and Torahs (Lev 26:46):\\
    \-\hspace{2em} `the statutes' --- this is midrash.\\
    \-\hspace{2em} `and the judgments' --- this is the legal rulings.\\
    \-\hspace{2em} `and the Torahs' --- [this] teaches that two Torahs were given to Israel: one in writing, the other by mouth.
\end{translation}

\noindent
Thus, for \vanderkam, the book of \jub upholds the authority of the Torah by offering its own interpretation of its contents in a similar fashion to the way that the oral Torah, too, claimed a similar primacy. \jub, therefore asserts itself as a correct and authoritative interpretation of the Torah by claiming that it is the interpretation that Moses himself received from God; as \vanderkam puts it, according to the book of \jub, ``[t]he message of \jub is verbally inerrant.''\autocite[33]{vanderkam_metso-etal2010}

The fact that \jub offers a \emph{particular} interpretation, however, assumes the possibility that other interpretations may exist. The reality of other interpretations, moreover, sheds light on the practical problem of discussing ``authoritative texts'' because ``texts,'' in fact, cannot wield authority. Texts must be interpreted, and authority is exercised within a society \emph{by people}.%
    \footnote{See especially the discussions in \cite[475]{debel_jsj2014} and \cite{brooke_rev-qumran2012}.}
Thus when one speaks of ``authoritative texts,'' one mat speak not only of the abstract ``status'' of a text (whether it is ``scripture'' or not), but also more concretely the way that interpretations of texts are used socially to affect behavior and exert power.\autocite{foucault_ci1982} In other words the issues of ``textual authority'' include how texts are perceived and how they affect \emph{praxis}.

