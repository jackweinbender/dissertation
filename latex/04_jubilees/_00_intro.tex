% !TEX root = dissertation.tex



% Discuss "discovery" 
In 1844 Heinrich Ewald published a description of an Ethiopian manuscript which had been preserved in Classical Ethiopic (\geez) under the title \eth{መጽሐፈ ኩፋሌ}{masḥafa kufālē}.%
\footnote{All translations are my own. \geez citations are from \vanderkam's critical edition, \cite*{vanderkam1989}.}
Because the name followed the common convention using a work's first few (key) words as its title (in this case, \eth{ዝንቱ፡ነገረ፡ኩፋሌ}{zentu nabara kufālē}), Ewald suggested that this manuscript may have been a copy of the work known from antiquity as both \gk{τά Ἰωβηλαϊα}, ``the Jubilee,'' and \gk{Λεπτὴ Γίνεσις}, the ``Little Genesis.''\autocite[176--179]{ewald_zkm1844} Although the work had been in continuous use within Ethiopian Christianity since antiquity, European scholarship only knew of the work through secondary references in a few classical sources.%
\footnote{\vanderkam offers a concise summary of the various late-antique citations and allusions in his commentary, most notably in the works of Epiphanius (The Panarion, Measures and Weights) and Syncellus (Chronography). \cite[1:10--14]{vanderkam2018}. See also \cite{reed_kister-etal2015} and \cite{kreps_ch2018}.%
% TODO: Get proper citations for these classical texts
}%

The work was published and supplemented by additional manuscripts by August Dillmann in 1859\autocite{dillmann1859} and R.~H. Charles in 1895.\autocite{charles1895} More recently, \vanderkam's 1989 edition utilized twenty-seven copies of the text\autocite[1:xiv--xvi]{vanderkam1989} and since its publication over twenty more copies have been catalogued and imaged.%
%
\footnote{%
\cite{erho_bsoas2013}.
\vanderkam helpfully lists the twenty-seven manuscripts he used for his critical edition in the introduction of his commentary where he also notes the additional manuscripts photographed since its publication. See \cite[1:14--16]{vanderkam2018}.
}

The Ethiopic text remains to this day the only tradition to preserve \jub in its entirety. Save for the rediscovery of the text itself, the most significant find for the study of \jub was the discovery of several Hebrew fragments of the work among the \dss which attest to the work's antiquity and likely original language of composition. Although the Hebrew and Ethiopic versions are---to the degree that we can tell---very close to one another, the Ethiopic text appears to be a granddaughter translation of the Hebrew through a Greek daughter translation, though no such text has been found.\footnote{See especially \vanderkam's treatment of the textual history of \jub in \cite*[1--18]{vanderkam1977}.} This fact was convincingly demonstrated by Dillmann who observed several Greek forms preserved as transliterations in the Ethiopic text.\footnote{Specifically: \gk{δρῦς}, \gk{βάλανος}, \gk{λίψ}, \gk{σχῖνος}, \gk{φάραγξ}. \cite[88]{dillamnn_jbw1850}. Charles later added \gk{ἡλιου} to the list. \cite[xxx]{charles1902}.} By the end of the 19th century, partial copies of \jub had been uncovered in Latin translation which similarly appear to be daughter translations of the Greek text. Finally, although no manuscript evidence has been found, \jub scholars posit that a Syriac translation of \jub was made in antiquity based on what appeared to be a number of Syriac citations of \jub which lacked any apparent influence from Greek.%
%
\footnote{%
See especially \cite[231--232]{tisserant_rb1921} and \cite[xxix]{charles1902} but also \cite[2:ix--x]{ceriani1861} and \cite[x]{charles1895}.}
%% Ethiopic Stuff
%% Qumran additions
%% Recent additions and other extant versions

% Content and Character
% As RwB
% Thesis on MEMORY

\nocite{dillamnn_jbw_kleine}