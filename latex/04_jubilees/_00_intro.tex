% !TEX root = dissertation.tex

\nocite{dillamnn_jbw_kleine}
\nocite{ewald_zkm1844}

%%%%%%%%%%%%%%%%%%%%%%%%%%%%%%%%%%%%%%%%%%%%%%%%%%%%%%%%%
% MEMORY CONSTRUCTION is the key Idea for this chapter. %
%%%%%%%%%%%%%%%%%%%%%%%%%%%%%%%%%%%%%%%%%%%%%%%%%%%%%%%%%

\section{Introduction}
% TODO: Introduction to the chapter as a whole

% ``Constructing Authority: Restructuring and reinforcing the Past''
% Begin with the exercise of authority and the way that Jubilees reappropriates Moses
% Show how that authority is wielded through the imposition of the calendar and chronological system
% use of *written* things to reinforce practice

\subsection{Discovery and Publication}
% Discuss "discovery" 
In 1844 Heinrich Ewald published a description of an Ethiopian manuscript which had been preserved in Classical Ethiopic (\geez) under the title \eth{መጽሐፈ ኩፋሌ}{masḥafa kufālē}.%
        \footnote{All translations are my own. \geez citations are from \vanderkam's critical edition, \cite*{vanderkam1989}.}
Because the name followed the common convention using a work's first few (key) words as its title (in this case, \eth{ዝንቱ፡ነገረ፡ኩፋሌ}{zentu nabara kufālē}), Ewald suggested that this manuscript may have been a copy of the work known from antiquity as both \greek{τά Ἰωβηλαϊα}, ``the Jubilee,'' and \greek{Λεπτὴ Γίνεσις}, the ``Little Genesis.''\autocite[176--179]{ewald_zkm1844} Although the work had been in continuous use within Ethiopian Christianity since antiquity, European scholarship only knew of the work through secondary references in a few classical sources.%
        \footnote{\vanderkam offers a concise summary of the various late-antique citations and allusions in his commentary, most notably in the works of Epiphanius (\emph{Panarion}, \emph{Measures and Weights}) and Syncellus (\emph{Chronography}).
                \cite[1:10--14]{vanderkam2018}. See also 
                \cite{reed_kister-etal2015} and 
                \cite{kreps_ch2018}.%
        % TODO: Get proper citations for these classical texts
}%

The work was published and supplemented by additional manuscripts by August Dillmann in 1859\autocite{dillmann1859} and R.~H. Charles in 1895.\autocite{charles1895} More recently, \vanderkam's 1989 edition utilized twenty-seven copies of the text\autocite[1:xiv--xvi]{vanderkam1989} and since its publication over twenty more copies have been cataloged and imaged.%
        \footnote{%
                \cite{erho_bsoas2013}.
                \vanderkam helpfully lists the twenty-seven manuscripts he used for his critical edition in the introduction of his commentary where he also notes the additional manuscripts photographed since its publication. See 
                \cite[1:14--16]{vanderkam2018}.}

Save for the rediscovery of the text itself, the most significant find for the study of \jub was the discovery of several Hebrew fragments of the work among the \dss which attest to the work's antiquity and likely original language of composition. Although the Hebrew and Ethiopic versions are---to the degree that we can tell---very close to one another, the Ethiopic text appears to be a granddaughter translation of the Hebrew through a Greek daughter translation, though no such text has been found.\footnote{See especially \vanderkam's treatment of the textual history of \jub in \cite*[1--18]{vanderkam1977}.} This fact was convincingly demonstrated by Dillmann who observed several Greek forms preserved as transliterations in the Ethiopic text.\footnote{Specifically: \greek{δρῦς}, \greek{βάλανος}, \greek{λίψ}, \greek{σχῖνος}, and \greek{φάραγξ}. \cite[88]{dillamnn_jbw1850}. Charles later added \greek{ἡλιου} to the list. \cite[xxx]{charles1902}.} By the end of the 19th century, partial copies of \jub had been uncovered in Latin translation which similarly appear to be daughter translations of the Greek text. Finally, although no manuscript evidence has been found, \jub scholars posit that a Syriac translation of \jub was made in antiquity based on what appeared to be a number of Syriac citations of \jub which lacked any apparent influence from Greek.%
        \footnote{%
                See especially
                \cite[231--232]{tisserant_rb1921} and 
                \cite[xxix]{charles1902} but also 
                \cite[2:ix--x]{ceriani1861} and 
                \cite[x]{charles1895}.}
Despite all of these finds, however, the Ethiopic text remains the only tradition to preserve \jub in its entirety. Thus, in my treatment of \jub I will be relying primarily on the Ethiopic text and will be supplementing from the Hebrew where available.

\subsection{Content and Character}
The book of \jub offers a rewriting of the book of Genesis and the first part of Exodus (Gen 1--Exod 12). The work is presented as a revelation from \yahweh given to Moses atop Mt. Sinai, framed by a brief prologue and epilogue.\autocite[1:17]{vanderkam2018} The prologue gives a short description of the work as an account concerned with the division of time into units of years, weeks, and jubilees:

% !TEX root = dissertation.tex
\begin{ethiopictext}
        \versenum{Prologue}
        ዝንቱ ፡ ነገረ ፡ ኩፋሌ ፡
        መዋዕላተ ፡ ሕግ ፡ ወለስምዕ ፡
        ለግብረ ፡ ዓመታት ፡ ለተሳብዖቶሙ ፡ 
        ለኢዮቤልውሳቲሆሙ ፡ ውስተ ፡ ኲሉ ፡ ዓመታተ ፡ ዓለም ፡
        በከመ ፡ ተናገሮ ፡ ለሙሴ ፡ በደብረ ፡ ሲና ፡
        አመ ፡ ዐርገ ፡ ይንሣእ ፡ ጽላተ ፡ እብን ፡ ሕግ ፡ ወትእዛዝ ፡ 
        በቃለ ፡ አግዚአብሔር ፡ በከመ ፡ ይቤሎ ፡ ይዕርግ ውስተ ፡ ርእሰ ፡ ደብር ።
\end{ethiopictext}

\begin{transliteration}
        \versenum{Prologue}
        zəntu nagara kufālē
        % kufālē                division
        mawāʕəlāta [la-]ḥegg wa-la-səmʕ
        % mawāʕel           'period, era, time' √mʕl 'to pass the day' Les. 603
        % səmʿ               testimony
        la-gəbra ʕāmatāt la-tasābəʕotomu
        % tasābeʿot             tGL perf from √sbʕ; not in the dictionary, but √sbʿ is seven, so… weeks
        la-ʔiyyobēləwəsātihomu wəsta \kw{ə}llu ʕāmatāta ʕālam
        % ˀiyyobēlwelātihomu    ʾiyyobēl is Jubilee, the rest (-welāt) some extended plural?
        ba-kama tanāgaro la-Musē ba-dabra Sinā
        % ba-kama               Just as
        % tanāgaro              Glt perf 3ms + 3ms
        ʔama ʕarga yenšāʔ ṣəllāta ʔəbn---ḥəgg wa-təʔzāz---%
        % ʕarga                 √ʕrg G pf 3ms 'go up'
        % yenšāʔ                √nšʔ G subj 3ms 'raise, accept, receive*' 
        % ṣellē                 pl. ṣellāt    'tablet'
        ba-qāla ʔagziʔabḥēr ba-kama yəbēlo yəʕrəg wəsta rəʔsa dabr.
        % yebēlo                G perf + 3ms
        % yeˤreg                G subj 
\end{transliteration}

\begin{translation}
        \versenum{Prologue}
        These are the words%
        \footnote{Lit. ``This is the word.'' I've chosen to follow VanderKam and others by rendering this construction in the plural based on the probable underlying Hebrew \he{אלה הדברים}. See \cite[125]{vanderkam2018}}
        of the division 
        of the days for the law and for the testimony
        for the event[s] of the years; for their weeks,
        for their Jubilees in all the years of the world
        just as he spoke (them) to Moses on Mount Sinai 
        when he went up to receive the tablets of stone---the law and the commandment---%
        at the command of God, as he had said to him 
        that he should ascend to the top of the mountain.
\end{translation}%
\noindent
Following this prologue, the setting of the story is established as the during the ``first year of the Israelites' exodus from Egypt, in the third month, on the sixteenth of the month'' when \yahweh called Moses atop Mt. Sinai.

The bulk of the work (\jub 2:1--50:13) is dedicated to the recounting of Jewish history, following the basic narrative provided by Gen 1--Exod 12, with special concern for halakhic matters and the division of time according to a 364 day calendrical system.\footnote{TODO: Reference} The particulars of the revelation are mediated by the ``\ap'' (Eth. \ethiopic{መልአከ ገጽ} [\emph{mal'aka gaṣṣ}]) who dictates its contents to Moses, the fastidious scribe. The treatment of Moses as a scribe places him within a chain of tradition---along with Enoch and Noah---which emphasizes writing and written works as essential sources of tradition and revelation.\footnote{TODO: Get refs and say something here} The work closes with a terse statement declaring ``Here the account of the division of time is ended'' (\jub 50:13; Eth. 
    \eth{ተፈጸመ ፡ በዝየ ፡ ነገር ፡ ዘኩፋሌ ፡ መዋዕል ።}
        {tafaṣṣama ba-zeyya nagar za-kufālē mawāʕel}).
        % tafaṣṣama     tD fṣm 'to complete'
        % ba-zeyya      here
        % nagar         account, speech, etc.
        % kufālē        division(s)
        % mawāʕel       √mʕl 'to pass the day' here: period, era, time Les. 603

\subsection{As RwB}


\subsection{Thesis on Memory}
% MEMORY CONSTRUCTION is the key Idea for this chapter. 
%% Calendar is the schema for a new conception of the past
%% Authority conferring strategies as memory-making strategies
%% Writing and language as reinforcement of tradition
