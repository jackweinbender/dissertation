% !TeX root = ../dissertation.tex

\section{\ga as \psa}

% Intro about third level of discourse
While the \ga can be seen engaging with its reveived cultural memory through its sources and engaging with its contemporary social memory at the level of literary form and genre, the \ga also participates in the construction of cultural memory going forward. Although all literary and cultural products can participate in constructing cultural memory, in this section, I will argue that \ga's \psgraphic form participates in this constructive act differently than other forms of literature, in particular the Biblical text.\footnote{I continue to reiterate that although the term ``biblical'' is anachronistic for the late \secondtemple period, it is a usefully concise term for my purposes.}

\subsection{On Pseudepigrapha}

Writing in the voice of these early biblical figures formally places \ga into the literary category of \psy and so we should take a moment to clearly state the way that I will use the ``\psy,'' ``\psa,'' and related terms.\autocites[The topic of \psy has received a large amount of very sophisticated attention in recent years. See especially][]{mroczek2016}{tigchelaar_tigchelaar2014}{reed_towsend-moulie2011}{reed_jts2009}{reed_ditomasso-turcescu2008}{najman_hilhorst-puech2007}{najman2003} In the simplest terms, \psa are texts which are fictively purported to be written by figures (typically) from the ancient past. 

The ancient use of the term \psa denoted spurious texts which Church leaders believed to be intentionally misleading about their authorship.\footnote{See esp.~Hist. Eccl. 6.12.2 where the Bishop of   Antioch, Serapion, refers to the * Gospel of Peter* among the a number   of works ``falsely attributed'': \gk{γάρ, ἀδελφοί, καὶ Πέτρον καὶ τοὺς   ἄλλους ἀποστόλους ἀποδεχόμεθα ὡς Χριστόν, τὰ δὲ ὀνόματι αὐτῶν   ψευδεπίγραφα ὡς ἔμπειροι παραιτούμεθα, γινώσκοντες ὅτι τὰ τοιαῦτα οὐ   παρελάβομεν}. ``For we, brothers, accept both Peter and the other   apostles as Christ, but we skillfully reject those falsely ascribed   writings, knowing that they were not handed down to us.''} The number of (esp.~Jewish) \psgraphical texts discovered within the past century provide good reason to question the assumption that pseudonymous authors's intentions were to deceive their readers.\autocites[53--58]{mroczek2016}[See also][]{reed_jts2009} Thus, I wish to eschew the value judgments of this ancient usage. At the other end of the spectrum, in some scholarly discourse, the term ``\psa'' has become generalized to encompass any text written in around the turn of the era which did not make it into the canon of rabbinic Judaism or early Christianity. Bernstein observes, for example, that although the first volume of James Charlesworth's two-volume \emph{Old Testament Pseudepigrapha} contains formally \psgraphic works, the second volume includes many which do not meet the formal definition of \psa.\autocites[2]{bernstein_chazon-etal1999}{charlesworth_OTP} This expansive practice, likewise, is not particularly helpful for clarifying the term and so I will attempt to restrict my useage to a more clearly defined set of criteria.

Thus I will use the terms \psy and \psa to refer to texts (or practices) which seem to actively construct a fictive author whose identity would have been well-known and meaningful to its reader and who (typically) would have lived in the distant past.

\subsection{The Hebrew Bible as a Baseline}

The vast majority of the Hebrew Bible is narrated in the third-person omniscient and is formally anonymous. There are, of course, exceptions to this generalization, most notably within the prophetic corpus (such as Isa 6--8), the so-called Nehemiah Memoir (Neh 11--13), and perhaps works such as Deuteronomy and Song of Songs. But for the lion's share of the biblical text, the the author (and narrator) operates invisibly.

The rhetorical force of this particular authorial voice, as observed by Erhard Blum, is significant for the function of the Hebrew Bible's participation in the collective memory of the communities that claim it as their own. Although the implied author does occasionally engage directly with the reader by offering explanatory observations (for example where the author inserts phrases like ``this is why\ldots{}'' or ``\ldots{}until this day''), for all intents and purposes, the author presents as both \emph{reliable} and \emph{authoritative} without a hint of subjectivity. As Blum puts it, ``In this sense the narrative does not distinguish the depiction from the depicted.''\autocite[33]{blum_barton-etal2007} Put another way, the text does not acknowledge that it \emph{has} an author, it simply \emph{is}. The rhetorical effect of this invisible, omniscient author is to collapse the knowledge gap between the reader and the events narrated by removing the author from view. This move, according to Blum, allows the text to convey ``an unmediated truth claim which is not based on the author's distinguishable critical judgments and convictions.''\autocite[33]{blum_barton-etal2007} The effectiveness of this implied author, according to Blum, is tied to the pragmatics of the text, that is, tied to the context of the biblical narratives as scripture (though, Blum does not refer to ``scripture'' \emph{per se}). The implied audience of the biblical narratives by-and-large can be understood as group-insiders for whom the biblical text worked to reinforce group identity.

Of course, the ``unmediated truth claims'' of the biblical text \emph{were}, in fact, mediated and reinforced by those who (orally or otherwise) transmitted the tradition from one generation to another.\autocite[33]{blum_barton-etal2007} Individuals within the community---teachers and religious leaders and even parents---become the voice of the biblical text as it is passed on. In other words, one might say that the narrator of the biblical text is the community itself---its collective memory. Blum writes:

\begin{quote}
If we assume that the traditional literature was primarily transmitted through oral means, than the narrator who is speaking supplies the material with a personal presence; he is not present as an author who judges and evaluates his sources from a critical distance, but as a `transmitter' who participates in the tradition itself and is able to lend it credence through his own personality, his standing, and/or his office.\cite[33]{blum_barton-etal2007}
\end{quote}

In other words the authoritative claims of ``biblical'' texts are actually made by their communities and not by the text itself. Thus the way biblical texts participate in the collective memory is determined by their \emph{use}---how their \emph{readers} frame their function and how the text relates to the collective memory. 

\subsection{Pseudepigrapha and Cultural Memory}

If we take seriously Blum's characterization of the way that the biblical text may have engaged with the collective memory of Israel based on the formal, narratological features of the text, it stands to reason that the \ga as first-person \psy would engage that collective memory in a different way than the biblical text despite the relative similarity of the textual content. The \psgraphic quality of \ga shapes the way that the text engages with the remembered past by describing the biblical story through the mouths of important figures.%
%
\footnote{Here ``story'' refers to the abstract sequence of actions which the narrative describes. The \emph{way} a story is recounted, on the other hand, is referred to by narratologists as \emph{narrative discourse.} Thus the \ga's change from third-person omniscient to a \psgraphical first-person narrative can be understood as a change in \emph{narrative discourse} which, broadly, retains the same \emph{story} as that of the biblical text. See \cite[TODO: PAGE NUMBER]{abbott2008}.}
%
Pseudepigraphic texts provide extra clues for how the reader should understand the text \visavis the broader collective memory by making explicit from whose perspective the text was (supposedly) written. This explicitness changes the way that the reader understands how the text fits into the collective memory by shifting the claim to authority onto its putative author. 

Moshe Bernstein, in his discussion of the phenomenon of \psy distinguishes between ``authoritative'' \psy and ``decorative'' \psy.\footnote{He also identifies a third form, ``convenient'' \psy which is located somewhere between the two. I do not find this category as helpful. \autocite[3--7]{bernstein_chazon-etal1999}.} By ``authoritative'' \psy, Bernstein refers to texts that \emph{portray themselves} as being written by a particular figure (such as the memoirs in the \ga) while by ``decorative'' \psy, he refers to texts which were later \emph{attributed to} some ancient figure.

The \ga falls squarely within the category of ``authoritative'' \psy along with, for example, portions of 1 Enoch (in particular the latter three books, Astronomical Writings [72--82], Dream Visions [83--90], and the Epistle of Enoch [91--107]) which present themselves as if they were written by Enoch himself. Psalm 23, on the other hand, although attributed to David, was presumably not \emph{actually} written by David. Moreover, whoever did write Ps 23, (again, presumably) did not intend to write it \emph{as if} it had been written by David. Rather, the Psalm was simply \emph{attributed} to David, along with many others, in part due to the tradition od David being a musician.\footnote{CITATION NEEDED} Thus, the difference between ``authoritative'' and ``decorative'' \psy can, in some sense, be boiled down to the notoriously difficult issue of authorial intent.

Less clear-cut examples, however, require a more nuanced treatment. For example, Deuteronomy is not generally referred to as among the \psa, yet, from a literary perspective, it is framed as \he{הדברִם אשׁר דבר משׁה אל־כל־ישׂראל}
``the words which Moses spoke to all Israel'' (Deut 1:1a). Although the whole narrative is not written in the first person, longs sections of the book are treated as verbatim recountings of Moses' speech. Was Moses the author of Deuteronomy? Most critical scholars, of course, have traditionally dated Deuteronomy to the late monarchic period and thus have eschewed the traditional attribution. But whether Deuteronomy was \emph{written} as \psa or just attributed to Moses after the fact is difficult to say with certainly. What we \emph{can} say is that there are concrete literary cues within Deuteronomy which make the attribution to Moses easier. Framing Deuteronomy as ``the words which Moses spoke,'' while not formally ``\psa'' participates in the construction of memory in a similar fashion as \psa proper by, again, shifting the source of the text's authority onto a particular figure in antiquity.

% \subsection{The \ga as \psy}

\subsection{Conclusions}
The question can therefor be asked, from the perspective of the memory, is there a meaning ful difference between these disparate forms of ``\psy?'' Although the manner of the attribution differs among \psgraphic texts, once the attribution has been made, the text becomes a part of the collective memory and informs the way that individuals within the remembering communities understood the relationship of the text to the collective memory.