% !TeX root = ../dissertation.tex

\section{\ga as \psa}

% Intro about third level of discourse
While the \ga can be seen engaging with its received cultural memory through its sources and engaging with its contemporary social memory at the level of literary form and genre, the \ga also participates in the construction of cultural memory going forward. Although \emph{all} literary and cultural products can participate in constructing cultural memory, in this section, I will argue that \ga's \psgraphic form participates in this constructive act differently than other forms of literature, in particular the biblical text.\footnote{I continue to reiterate that although the term ``biblical'' is anachronistic for the late \secondtemple period, it is a usefully concise term for my purposes.}

\subsection{The Hebrew Bible as a Baseline}

The vast majority of the Hebrew Bible is narrated in the third-person omniscient and is formally anonymous. There are, of course, exceptions to this generalization, most notably within the prophetic corpus (such as Isa 6--8), the so-called Nehemiah Memoir (Neh 11--13), and perhaps works such as Deuteronomy and Song of Songs. But for the lion's share of the biblical text, the the author (and narrator) operates invisibly.

The rhetorical force of this particular authorial voice, as observed by Erhard Blum, is significant for the function of the Hebrew Bible's participation in the collective memory of the communities that claim it as their own. Although the implied author does occasionally engage directly with the reader by offering explanatory observations (for example where the author inserts phrases like ``this is why\ldots{}'' or ``\ldots{}until this day''), for all intents and purposes, the author presents as both \emph{reliable} and \emph{authoritative} without a hint of subjectivity. As Blum puts it, ``In this sense the narrative does not distinguish the depiction from the depicted.''\autocite[33]{blum_barton-etal2007} Put another way, the text does not acknowledge that it \emph{has} an author, it simply \emph{is}. The rhetorical effect of this invisible, omniscient author is to collapse the knowledge gap between the reader and the events narrated by removing the author from view. This move, according to Blum, allows the text to convey ``an unmediated truth claim which is not based on the author's distinguishable critical judgments and convictions.''\autocite[33]{blum_barton-etal2007} The effectiveness of this implied author, according to Blum, is tied to the pragmatics of the text, that is, tied to the context of the biblical narratives as scripture (though, Blum does not refer to ``scripture'' \emph{per se}). The implied audience of the biblical narratives by-and-large can be understood as group-insiders for whom the biblical text worked to reinforce group identity.

Of course, the ``unmediated truth claims'' of the biblical text \emph{were}, in fact, mediated and reinforced by those who (orally or otherwise) transmitted the tradition from one generation to another.\autocite[33]{blum_barton-etal2007} Individuals within the community---teachers and religious leaders and even parents---become the voice of the biblical text as it is passed on. In other words, one might say that the narrator of the biblical text is the community itself---its collective memory. Blum writes:

\begin{quote}
If we assume that the traditional literature was primarily transmitted through oral means, than the narrator who is speaking supplies the material with a personal presence; he is not present as an author who judges and evaluates his sources from a critical distance, but as a `transmitter' who participates in the tradition itself and is able to lend it credence through his own personality, his standing, and/or his office.\cite[33]{blum_barton-etal2007}
\end{quote}

In other words the authoritative claims of ``biblical'' texts are actually made by their communities and not by the text itself. Thus the way biblical texts participate in the collective memory is determined by their \emph{use}---how their \emph{readers} frame their function and how the text relates to the collective memory. 

\subsection{On \Psy and the \Psa}

Because significant portions of the \ga are written in the first person as though written by Lamech, Noah, and Abram, \ga may be formally included in the literary category of \psy. Before moving on, however, it is worth taking a moment to clearly define what is meant by ``\psy,'' ``\psa,'' and related terms.\autocites[The topic of \psy has received a large amount of very sophisticated attention in recent years. See especially][]{mroczek2016}{tigchelaar_tigchelaar2014}{reed_towsend-moulie2011}{reed_jts2009}{reed_ditomasso-turcescu2008}{najman_hilhorst-puech2007}{najman2003} In the simplest terms, \psa are texts which are fictively purported to be written by figures (typically) from the ancient past. 

The ancient use of the term \psa denoted spurious texts which Church leaders believed to be intentionally misleading about their authorship.\footnote{See esp.~Hist. Eccl. 6.12.2 where the Bishop of Antioch, Serapion, refers to the \emph{Gospel of Peter} among the a number of works ``falsely attributed'': \gk{γάρ, ἀδελφοί, καὶ Πέτρον καὶ τοὺς ἄλλους ἀποστόλους ἀποδεχόμεθα ὡς Χριστόν, τὰ δὲ ὀνόματι αὐτῶν   ψευδεπίγραφα ὡς ἔμπειροι παραιτούμεθα, γινώσκοντες ὅτι τὰ τοιαῦτα οὐ   παρελάβομεν}. ``For we, brothers, accept both Peter and the other apostles as Christ, but we skillfully reject those falsely ascribed writings, knowing that they were not handed down to us.''} Thus, the term has tended to carry a somewhat negative connotation, even when such a connotation is not warranted. Implicit in the negative use of the term is the assumtion that ``false'' attribution was malicious, or at the very least intentionally misleading. Yet, the number of (esp.~Jewish) \psgraphical texts discovered within the past century provide good reason to question the assumption that pseudonymous authors's intentions were to deceive their readers.\autocites[53--58]{mroczek2016}[See also][]{reed_jts2009} On the contrary, the sheer number of \psgraphical works now known to us suggests that the historical reality and social function of \psgraphical works was not simply a matter of being ``falsely attributed.''

At the other end of the spectrum, because so many early Jewish texts seem to fall into the category of \psa, in some scholarly discourse, the term ``\psa'' has become generalized to encompass any text written in around the turn of the era which did not make it into the canon of rabbinic Judaism or early Christianity. Bernstein observes, for example, that although the first volume of James Charlesworth's two-volume \emph{Old Testament Pseudepigrapha} contains a number of formally \psgraphic works, the second volume includes many which do not meet the formal definition of \psa.\autocites[2]{bernstein_chazon-etal1999}{charlesworth_OTP} This expansive practice is not particularly helpful for clarifying the term and so I will attempt to restrict my useage to a more clearly defined set of criteria.

Moshe Bernstein, in his discussion of the phenomenon of \psy distinguishes between ``authoritative'' \psy and ``decorative'' \psy.\footnote{He also identifies a third form, ``convenient'' \psy which is located somewhere between the two. \cite[3--7]{bernstein_chazon-etal1999}.} By ``authoritative'' \psy, Bernstein refers to texts that \emph{portray themselves} as being written by a particular figure. Portions of 1 Enoch (in particular the latter three books, Astronomical Writings [72--82], Dream Visions [83--90], and the Epistle of Enoch [91--107]), which present themselves as if they were written by Enoch himself, are prime examples of ``authoritative'' \psy. Psalm 23, on the other hand, although attributed to David, was presumably not \emph{actually} written by David. Moreover, whoever did write Ps 23, (again, presumably) did not intend to write it \emph{as if} it had been written by David. Rather, the Psalm was simply \emph{attributed} to David, along with many others, in part due to the tradition od David being a musician.\footnote{CITATION NEEDED} Thus, Ps 23 could be classified as ``decorative'' \psy. Thus the difference between ``authoritative'' and ``decorative'' \psy can, in some sense, be boiled down to the notoriously difficult issue of authorial intent---whether a text was \emph{intended} to be read as \psa or whether the work was anonymous, and later attributed to an explicit author.

Less clear-cut examples, however, require a more nuanced treatment. For example, Deuteronomy is not generally referred to as among the \psa, yet, from a literary perspective, it is framed as \he{הדברִם אשׁר דבר משׁה אל־כל־ישׂראל} ``the words which Moses spoke to all Israel'' (Deut 1:1a). Although the whole narrative is not written in the first person, long sections of the book are treated as verbatim recountings of Moses' speech. Was Moses the author of Deuteronomy? Traditionally, most critical scholars have dated Deuteronomy to the late monarchic period and thus have eschewed the traditional attribution. But whether Deuteronomy was \emph{written} as \psa or just attributed to Moses after the fact is difficult to say with certainly and the matter is further complicated by the editorial processes that the book likely underwent through the centuries.\autocite[143--172]{toorn2007} What we \emph{can} say is that there are concrete literary cues within Deuteronomy which make the attribution to Moses easier. Framing Deuteronomy as ``the words which Moses spoke,'' while not formally ``\psa'' participates in the construction of memory in a similar fashion as \psa proper.

\subsection{Pseudepigrapha, \ga, and Memory Construction}

If we take seriously Blum's characterization of the way that the anonymous, third-person omniscient biblical text may have engaged with the collective memory of Israel based on formal, narratological features within the text, it stands to reason that the \ga as first-person \psy would engage that collective memory in a different way, despite the fact that the stories within the \ga are found in the book of Genesis. In other words the literary form of the \ga affects how it relates \emph{back} to the biblical memory, and how it can be used in the further \emph{construction of} that memory.

The \psgraphic quality of \ga shapes the way that the text engages with the remembered past by describing the biblical story through the mouths of important figures.%
%
\footnote{Here ``story'' refers to the abstract sequence of actions which the narrative describes. The \emph{way} a story is recounted, on the other hand, is referred to by narratologists as \emph{narrative discourse.} Thus the \ga's change from third-person omniscient to a \psgraphical first-person narrative can be understood as a change in \emph{narrative discourse} which, broadly, retains the same \emph{story} as that of the biblical text. See \cite[13--27, esp. 18--19]{abbott2008}.}
%
This explicitness changes the way that the reader understands how the text fits into the collective memory by shifting the locus of authenticity onto the text's putative author and away from the mediating figures within the community. In other words, as an example of \psy, the \ga can be thought of as a set of fictional \emph{primary sources} that bypass the received tradition. As these sources are used and enter into the discourse of the broader biblical memory they are able to function not simply as ``alternate'' versions of events but as qualitatively distinct contributions to the tradition as it is passed on to the next generation.\footnote{On analogy to Hindy Najman's notion of ``Mosaic Discourse,'' here I am saying that the \ga is participating in a broader ``biblical'' discourse insofar as it participates in discourses surrounding Lamech, Noah, and Abram. See \cite[GET PAGE]{najman2003}.}

Of course, referring to \psy as ``fictional primary sources'' may overstate these texts' importance or otherwise misunderstand how ``authentic'' these texts were thought to be by various and sundry religious groups in antiquity. On the one hand, it could be that readers understood that such novel fictional adaptations took certain artistic license with their biblicial \emph{Vorlagen}. By way of analogy, modern adaptations of biblical narratives into film are expected to deviate to a certain degree from their source material, despite the fact that the Hebrew Bible remains a sacred, authoritative text for many modern Jews and Christians. Such adaptations are not, typically, understood to be superceding the Bible because viewers understand intuitively that there is a qualitative difference between their scriptures and a movie. On the other hand, there certainly are examples of \psgraphical texts which ultimately \emph{did} become authoritative for certain religious groups.%
%
\footnote{For example, the Ethiopian Orthodox Church includes 1 Enoch among its scriptures. Tobit, too may, under certain rubrics, be considered \psa, which is included within the Roman Catholic and Eastern Orthodox deuterocanon. Insofar as deutero- and trito- Isaiah were penned as is written by Isaiah, they too could be considered \psa. And, of course, a number of the so called ``disputed'' pauline letters within the Christian New Testament likely were not penned by Paul and are properly \psgraphical.}
%
My point here is not to suggest that there were multiple ways to understand \psgraphical writing in antiquity so much as to point out that discussions of ``false'' or ``authentic'' attribution are generally from later periods and do not tell us anything meaningful about \emph{why} such a text was written or \emph{how} it would have been understood by its original readers.

The \ga, of course, was never considered ``scripture'' so far as we know, but that does not mean that it did not participate in the broader biblical memory, even if only in the popular imagination. But even at the level of the popular imagination---even as an entertaining fiction---the \ga participated in how its society conceived of the Genesis narratives. Regardless of whether the memoirs in \ga were thought to be ``authentic,'' they represent both an interpretive understanding of biblical memory and an original contribution to that memory.