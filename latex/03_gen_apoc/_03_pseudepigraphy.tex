% !TeX root = ../dissertation.tex

\section{\ga as \psa}

% Intro about third level of discourse
While the \ga can be seen engaging with its reveived cultural memory through its sources and engaging with its contemporary social memory at the level of literary form and genre, the \ga also participates in the construction of cultural memory going forward. Although all literary and cultural products can participate in constructing cultural memory, in this section, I will argue that \ga's \psgraphic form participates in this constructive act differently than other forms of literature, in particular the Biblical text.\footnote{I continue to reiterate that although the term ``biblical'' is anachronistic for the late \secondtemple period, it is a usefully concise term for my purposes.}


\subsection{The Hebrew Bible as a Baseline}

The vast majority of the Hebrew Bible is narrated in the third-person omniscient and is formally anonymous. There are, of course, exceptions to this generalization, most notably within the prophetic corpus (such as Isa 6--8), the so-called Nehemiah Memoir (Neh 11--13), and perhaps works such as Deuteronomy and Song of Songs. But for the lion's share of the biblical text, the the author (and narrator) operates invisibly.

The rhetorical force of this particular authorial voice, as observed by Erhard Blum, is significant for the function of the Hebrew Bible's participation in the collective memory of the communities that claim it as their own. Although the implied author does occasionally engage directly with the reader by offering explanatory observations (for example where the author inserts phrases like ``this is why\ldots{}'' or ``\ldots{}until this day''), for all intents and purposes, the author presents as both \emph{reliable} and \emph{authoritative} without a hint of subjectivity. As Blum puts it, ``In this sense the narrative does not distinguish the depiction from the depicted.''\autocite[33]{blum_barton-etal2007} Put another way, the text does not acknowledge that it \emph{has} an author, it simply \emph{is}. The rhetorical effect of this invisible, omniscient author is to collapse the knowledge gap between the reader and the events narrated by removing the author from view. This move, according to Blum, allows the text to convey ``an unmediated truth claim which is not based on the author's distinguishable critical judgments and convictions.''\autocite[33]{blum_barton-etal2007} The effectiveness of this implied author, according to Blum, is tied to the pragmatics of the text, that is, tied to the context of the biblical narratives as scripture (though, Blum does not refer to ``scripture'' \emph{per se}). The implied audience of the biblical narratives by-and-large can be understood as group-insiders for whom the biblical text worked to reinforce group identity.

Of course, the ``unmediated truth claims'' of the biblical text \emph{were}, in fact, mediated and reinforced by those who (orally or otherwise) transmitted the tradition from one generation to another.\autocite[33]{blum_barton-etal2007} Individuals within the community---teachers and religious leaders and even parents---become the voice of the biblical text as it is passed on. In other words, one might say that the narrator of the biblical text is the community itself---its collective memory. Blum writes:

\begin{quote}
If we assume that the traditional literature was primarily transmitted through oral means, than the narrator who is speaking supplies the material with a personal presence; he is not present as an author who judges and evaluates his sources from a critical distance, but as a `transmitter' who participates in the tradition itself and is able to lend it credence through his own personality, his standing, and/or his office.\cite[33]{blum_barton-etal2007}
\end{quote}

In other words the authoritative claims of ``biblical'' texts are actually made by their communities and not by the text itself. Thus the way biblical texts participate in the collective memory is determined by its \emph{use}.

\subsection{On Pseudepgraphy}

Pseudepigraphic texts, on the other hand, 

% Contrast with the Genesis Apocryphon
    % Define Kinds of Pseudepigraphy
    % Does it matter?

% Discuss GA as Psa

% Conclusions