\section{\GA, First Person Narrative, and \Psy}

 One of the most striking features of the \ga when compared to other \rwb texts is its pervasive use of the first-person voice to narrate the events of its story. This quality sets \ga apart from the majority of narrative material in the Hebrew Bible which, with notable exceptions, usually maintains an omniscient third-person voice. The \ga's use of first person must be nuanced, however, by the fact that the work presents itself as a collection of first person ``memoirs'' from three of the Patriarchs from Genesis (Lamech, Noah, and Abram). So, while each memoir does indeed utilize the first person, the narrator itself changes throughout the course of the text.
 
Speaking of the \ga as a single text, however, should not be taken for granted. Therefore, it is necessary for the moment to consider whether the \ga should be considered ``a'' text or whether instead it should be treated as a collection of ``texts.''  Although the composite character of the \ga was noted in the \emph{editio princeps} by Avigad and Yadin, they maintained that \ga functioned as a single literary unit though allowing that it was made up of several literary sources.\autocite[38]{avigad-yadin1956} Since then, however, the unity of the work has been further interrogated and analyzed from a number of perspectives and the unity of \ga is more tenuous than ever.

Perhaps the most compelling argument for the genuinely composite nature of \ga has been offered by Moshe Bernstein, who has pointed out that discrete units of the \ga utilize distinct titles and epithets for \yahweh.\footnote{\cite{bernstein_jbl2009}. It is delightfully reminiscent of the classical formulations of the Documentary Hypothesis. Attributed to Graf and Wellhausen.  See also~\cite{bernstein_as2010} and~\cite{weigold_as2010}}

From a structural standpoint, it is not at all clear whether these three ``memoirs'' are meaningfully related and any thematic consistency is readily explained through the collating process itself.\footnote{In other words, we could easily suppose that the reason that three ``texts'' such as these would be grouped together on a single scroll was that they shared certain common themes or formal characteristics.} One could certainly imagine the \ga as a collection of fictional patriarchal memoirs collected onto a single scroll, and organized roughly by each text's correspondence to the chronology of the biblical narrative. 

In fact, this is precisely how the \ga presents itself. Although the beginnings of the Lamech and Abram accounts are lost due to damage to the scroll, col. 5.29 seems to offer a superscription for the account of Noah, reconstructed as [\emph{pršgn}] \emph{ktb mly nwḥ} ``[a copy of] the book of the words of Noah.'' Based on this superscription, it is reasonable to suppose that the accounts of Lamech and Abram, too, had such superscriptions, although at least two complicating factors should be taken into account. First, while the extant portions of \ga present themselves as distinct units, both the very beginning and the very end of the scroll have been lost. From a structural standpoint, this fact should elicit caution because it is at the beginning and end of texts where such features as ``framing narratives'' and other explanatory material is often located. Without definitive evidence of the presence or absence of such features one must be extra cautious when making observations about the rhetorical purpose of the macro structure of a particular text. Second, although all three accounts are \emph{generally} written in the first-person, as noted above, none of them are rigorously committed to maintaining the voice. As Loren Stuckenbruck notes, each of the three ``documents,'' at one point or another, falls into some kind of third-person voice: Lamech in 5.24--25, Noah in 16.14--17.19, and Abram in 21.23--22.34. Curiously, Stuckenbruck includes the superscription(s) as examples of this inconsistency and does not distinguish between instances where the narrator moves into the third person \emph{within} the narrative and cases where one might suppose the presence of an editorial voice.

The problem of whether to understand the superscription(s) as ``internal'' to the work is a good example of how the macro-structure of the work is important for this kind of analysis. By treating the superscription as a contribution of the ``author'' of a unified \ga, Stuckenbruck understands the superscription to be ``in the third person'' and would (apparently) treat each first-person account as an embedded narrative within a larger framing narrative (of which the superscription would be a part). For example, if the beginning of the scroll gave a brief framing narrative, describing a young man who discovered three scrolls in the desert and thus proceeded to provide ``a copy of the book of the words of X,'' Stuckenbruck would be absolutely correct. However, if one understands the superscription to be an editorial insertion, it does not make sense to include it as an example of third-person discourse for the same reasons it does not make sense to say that Ps 23 uses third-person discourse by beginning with \emph{mizmôr lə-dāwid}.\footnote{\cite[315--316]{stuckenbruck_roitman-etal2011}. See also \cite[15--16]{bernstein_chazon-etal1999}. Even supposing a single author for \ga, as Stuckenbruck and others imply, I am still inclined to consider the superscriptions separately from the former examples because they would exist outside the frame of each embedded narrative.}
In other words, the way that the \ga presents itself, I believe, should best be understood as a \emph{collection} of memoirs compiled by an editor who would have, putatively, supplied a set of paratextual superscriptions.

The fact that \ga presents itself as a collection of disparate texts, however, does not demand that the work cannot be treated as a whole.
 
 
\subsection{Pseudepigraphy and the Implied Author}



The term ``implied author'' also deserves a clear definition; I have adopted that of H. Porter Abbott:

\begin{quote} Neither the real \emph{author} nor the \emph{narrator}, the implied author is the idea of the author constructed by the reader as she or he reads the \emph{narrative}. In an \emph{intentional reading}, the implied author is that sensibility and moral intelligence that the reader gradually constructs to infer the intended meanings and effects of the narrative. The implied author might as easily (and with greater justice) be called the ``inferred author'' \autocite[235]{abbott2008} \end{quote}

Because the implied author is a construction of the reader, it is frequently not desirable to talk about this construct as a literary feature so much as a heuristic for intentional reading. Part of the advantage of basing the definition of \psy on the idea of the implied author is that it mitigates any prejudice toward the intention of the author to deceive (maliciously or otherwise) his reader. Thus, the question of ``who'' the implied author \emph{is} generally misses the point. In the case of \psy, however, the central formal characteristic of the work seems to be the \emph{intentional construction of a known implied author}. Therefore, one could make the case that what is characteristic about \psgraphic texts is the intentionality on the part of the real author to shape the processes by which their readers's construct an implied author. Where Abbott notes that the term ``inferred author'' may be a better term than ``implied author'' (the connotation shifts the active role to the reader) when discussing \psy, ``implied'' still fits quite well (since we can attribute some intentionality to the real author). The implied author, therefore, provides an important point of contact between the reader and the (real) author.

... and further to consider that their readers were aware of and participants in the authorial fiction. In such a case, the implied author would elicit an entirely different set of sensibilities for the reader to ``infer the intended meanings ad effects of the narrative.''\autocite[235]{abbott2008}

 \subsection{Scripture, \Psy and Memory Construction}
 ...

 In contrast to the omniscient implied biblical author, the \ga frames itself as a collection of first-person accounts which formally fall into the category of \psy. 
 
  

 Approaching these questions from the perspective of Social Memory Studies asks us to think about the way that differing social frameworks and cognitive contexts may have allowed for or demanded presenting this material in a different form than its \emph{Vorlage}.

 One of the difficulties in dealing with \psy is the apparently divergent ways that the authors and original readers may have understood \psy as compared to the way that later groups (e.g., Church Fathers, modern scholars) treat it. The crux of the issue, it seems to me, is less to do with whether the author intended to ``deceive'' his audience, and more to do with whether the readers understood themselves to be reading something ``authentic'' or were willing participants in an authorial fiction. Yet, even language of ``authenticity'' or ``fiction'' presupposes that such terms were a meaningful part of the discourse surrounding ``scripture'' during the late \secondtemple period.