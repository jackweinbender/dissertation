\section{`Say You Are My Sister': \ga and the Biblical Tradition}

% Abram and Sarai in Egypt
The longest, mostly complete, portion of the \ga is found in cols. 19--22 which retells the stories of Abram and Sarai's sojourn in Egypt (||~Gen 12:10--20), Abram's subsequent conflict with Lot (||~Gen 13:1--18), the Elamite campaign (||~Gen 14:1--24), and the beginning of Abram's vision (||~Gen 15:1--4). \ga's retelling of these stories follow the chronology of Gen 12--15 very closely, but \ga embelleshes and augments the narrative throughout. The similarity of these retellings to their putative biblical \emph{Vorlage}%
%
\footnote{Of course, we cannot be certain that the \emph{Vorlage} of \ga was, in fact, the same as the MT. That said, relative stability of (especially) the Torah texts during the Second Temple Period is widely accepted. For the purposes of this section, I will work under the assumption that the MT represents a very close approximation to the text that the authors/editors of Genesis Apocryphon were familiar with.}

prompted some (esp. early) scholars to suggest that \ga represents a sort of prototype for the later Pentateuchal Targums.\autocite[193]{black1983} While this may be a fair assessment for the very end of the scroll (which offers, at times, a near word-for-word translation of the MT), the freer sections that preceed it, and especially the earliest portions that deal with Lamech and Noah, really cannot reasonably be considered even ``paraphrases.'' Thus most treatments of \ga have tended to discuss the the work, following \vermes, in terms of ``midrash.''%
%
\footnote{}

% Notably, although narrated in the first-person voice starting in col. 19, \ga switches to the more familiar thrid-persion voice in the middle of col. 21 when the narrative shifts focus away from Abram to introduce the Mesopotamian kings in line 23.\footnote{Line 23 seems to begin a new paragraph, following \emph{vacat} of the previous line. Fitzmyer adds the caveat that \ga returns to the first person at theend of col. 22}

%% Rewriting of the biblical account


%% What are the points of contact with the Bible and other contemporary sources? (Jubilees, Enoch, etc.)
%% What is GA's relationship *to* the Bible
    %% Conflating Pharoah and Abimelek is "interpretation" collapsing to a single story
    %% Cedar and Date Palm? Dafuck is going on here?

% (re)frame this example as 'memory.'
    %% A synthesis of traditions about Abe and Sarai
    %% Likely some innovative stuff
    %% But it's not beholden to the tradition. That is, it is participating in the creation of tradition and that tradition may be able to be traced as continuiing into the Gen. Rab. with the Cedar and DatePalm stuff (?)
