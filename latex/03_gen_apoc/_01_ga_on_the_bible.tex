\section{Genesis Apocryohon and the Bible: Palm and Cedar}

% Briefly describe the traditional style of approaching ga as 'midrash' or whatever

% Give a solid example, such as the palm/cedar stuff from my other paper.

% (re)frame this example as 'memory.'


The name given to the \ga in the \emph{editio princeps} set the agenda for scholarly inquiry on the work into the modern era by connetcing it to the biblical book of Genesis while simultaneously categorizing the it as apocryphal. As already noted, \vermes's treatment of \ga focused on the role that it played in showing the continuity between the interpretation of Jewish scripture during the \secondtemple period and the aggadic traditions of early rabbinic Judaism. In \cite*{vermes1961}, \vermes treats in detail the relationship between Gen 12:8--15:4 and \ga cols. 19--22, ultimately declaring \ga to be ``the most ancient midrash of all'' and the ``lost link between the biblical and the Rabbinic midrash.''\autocite[124]{vermes1961} The result of this framing (whether one considers it appropriate or not) has been that much of the scholarly attention paid to \ga has focused on its relationship to Genesis and especially how its author(s) may have been addressing exegetical issues found within the (later) biblical work. 

However, the treatment of \ga as primarily exegetical (or in the case of \vermes, as midrash) tacitly implies that the purpose of \ga was to explain or interpret Genesis. Put another way, \ga is often treated as if its purpose in antiquity was to say something \emph{about how to read Genesis}. Placing \ga under the rubric of ``biblical interpretation,'' for example, does not, to my mind, adequately appreciate the potential for \ga to be a creative work in its own right. This is not to say, of course, that 