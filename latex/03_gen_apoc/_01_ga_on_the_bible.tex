\section{Genesis and the Genesis Apocryphon}

The name given to the \ga by scholars after its discovery set the agenda for scholarly inquiry on the work into the modern era by connetcing it to the biblical book of Genesis while simultaneously categorizing the it as apocryphal. The result of this framing (whether one considers it appropriate or not) has been that much of the scholarly attention paid to \ga has focused on its relationship to Genesis and especially how its author(s) may have been addressing exegetical issues found within the (later) biblical work. As already noted, \vermes's treatment of \ga focused on the role that it played in showing the continuity between the interpretation of jewish scripture during the \secondtemple period and the aggadic traditions of early rabbinic Judaism. In \cite*{vermes1961}, \vermes treats in detail the relationship between Gen 12:8--15:4 and cols. 19--22 declaring it to be ``the most ancient midrash of all''\autocite[124]{vermes1961} and the ``lost link between the biblical and the Rabbinic midrash.'' \autocite[124]{vermes1961}