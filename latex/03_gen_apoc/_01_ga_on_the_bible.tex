% !TeX root = ../dissertation.tex

\section{\ga and Biblical Memory}

% Section introduction

Although it is anachronistic to suggest that the ``Bible''  existed during the late \secondtemple period, insofar as the texts and traditions that later formalized as the ``Bible''---especially those contained in the Pentateuch---were certainly present in a reasonably stable and even privileged state, I think it is a mistake to jettison any discussion of \rwb texts as they relate to the texts that would later become the Hebrew Bible. On the other hand, restricting our discussion to those later biblical texts would likewise not do justice to the wide variety of texts and traditions in existence during the \secondtemple period which undoubtedly influenced \ga. In an effort to strike a middle ground, therefore, I have opted to refer to ``biblical memory,'' by which I simply mean the confluence of stories and traditions which relate to those later formalized in the Hebrew Bible.%
%
\footnote{I would like to emphasize that I am not suggesting that ``biblical memory'' represents a qualitatively unique form of memory, only that the scope of the traditions under consideration relate to texts that later became the Bible, and, in all likelihood, held at least some sort of special privilege within the memory of many \secondtemple Jews.}
%
In this section, therefore, I would like to discuss the ways that the \ga participated in biblical memory.

% What was GA rewriting: Sources for the parts of GS
\subsection{What was the \ga Rewriting?}

Although the \ga is generally touted as one of the more clear-cut examples of the \rwb, it is noteworthy that its relationship to the biblical text is not, in fact, entirely uniform.\footnote{\cite[333]{bernstein_berthelot-etal2010}.}


% First Columns (Lamech Memoir: Summary and relation to 1 Enoch)
\subsubsection{The Lamech Memoir (Cols. 0--5)}

The earliest columns of the \ga (cols. 0--5), which are narrated from the perspective of Lamech (the ``Lamech Memoir'' by my terminology), Noah's father, essentially offer a rewriting of 1 \enoch 106--107.\footnote{\cite[174]{nickelsburg2005}. The birth of Noah seems to have been a matter of some interest; a number of other texts likewise discuss the exceptional qualities of Noah at his birth. See \q{4}{534}{}[\q{4}{BNoah}{a-d}], \q{1}{Noah} as well as \cite{vanderkam_kapera1992}. Note also \cite{stuckenbruck_berthelot-etal2010}.} In this section, Lamech, recounts the birth of Noah and Lamech's fear that his wife, Bitenosh, had conceived Noah by means of the \he{עירין} "Watchers." Despite Bitenosh's assurances, Lamech petitions his father, Methusaleh to ask \emph{his} father, \enoch, for further assurance, which he ultimately gives. Although this section is fragmentary, its close resemblance to 1 \enoch 106--107 makes the scholarly reconstruction of the missing sections quite plausible. While it may be tempting to suggest that this section of \ga represents a variant edition of 1 \enoch 106--107, rather than a rewriting, the fact that the version of the story preserved in \ga is told in the first-person from the point of view of Lamech, while 1 \enoch 106--107 is told in the third-person, makes this suggestion highly unlikely. Moreover, because both 1 \enoch and \ga were composed in Aramaic, the differences between the two tellings cannot be attributed to translational issues. In other words, although cols. 0--5 deal, nominally, with events in Genesis 5:28--29, for all intents and purposes, the story recounted in these columns is a retelling of events known from the Enochic tradition.%
%
\footnote{It is not clear what the precise relationship between the Enochic traditions and the \ga actually were. Here I have more-or-less assumed the priority of 1 \enoch, but I wish to leave ambiguous whether \ga represents a rewriting of the \emph{text} of 1 \enoch, or wether they simply draw on a common tradition. Thus, I have chosen to refer to the tradition ``known from'' 1 \enoch, rather than 1 \enoch itself. See Stuckenbruck's treatment of these traditions in \cite*{stuckenbruck_berthelot-etal2010}; Nickelsburg's concise but thorough treatment of the similarities and differences in of these texts is also quite helpful. See \cite[173--174]{nickelsburg2005} as well as \cite[122--123]{fitzmyer2004}.} 


% Second Section (Noah Memoir)
\subsubsection{The Noah Memoir (Cols. 5--17)}
The second major section of \ga begins with a superscription identifying What follows as a \he{[פרשגן] כתב מלי נוח} or ``[A copy of] the Book of the Words of Noah" (5.29) and continues through col 17 (and, likely, onto the beginning of 18).%
%
\footnote{\cite[174--175]{nickelsburg2005}; Regarding the superscription, see \cite{steiner_dsd1995}. On the topic of the existence of a so-called ``book of Noah'' see \cite{dimant_vanderkam-etal2006} and \cite{werman_chazon-etal1999}.}

% Summary
Although this section accounts for the bulk of the scroll, significant portions are missing or unreadable. This ``Noah Memoir'' begins with a description of Noah's righteousness\footnote{vanderkam:righteousness-of-noah} (affirmed even in-utero) and his early family life (5.29--6.9), followed by a vision predicting the flood (6.9--7.9) which comes about due to the evil behavior of the Nephilim. Cols. 7--8 are highly fragmentary, but most likely described the events of the flood, while cols. 9--12 (which are slightly less fragmentary) describe the Ark's putting in on Mt. Ararat, God's instructions to  and blessing of Noah (including the prohibition of consuming blood), and Noah's subsequent interest in viticulture. Cols 13--15 recount a dream-vision in which Noah is depicted as a cedar tree with shoots representing his sons, including a fragmentary explanation of the dream. Finally, cols. 16--17 describe the division of the land by Noah to his sons.

% "Sources" and Relationship of NM to Jubilees & 1 Enoch?
As with the Lamech Memoir, the Noah Memoir clearly draws from traditions outside of those preserved in Genesis. This fact was acknowledged even from the scroll's initial publication.\autocite[38]{avigad-yadin1956} Although the flood account in Gen 6:9--9:17 is a longer and more developed story in its own right than is the account of Noah's birth (which the Lamech Memoir takes as its point of departure), characterizing cols 6--17 of \ga as \emph{primarily} a rewriting of the Genesis flood story does not give due consideration to the additional traditions which influenced its composition. The mention of the Watchers (Aram: \aram{עירין}) and the Nephilim in cols. 6--7 especially bear a thematic resemblance to the Book of Watchers in 1 \enoch 6--11.\footnote{\cite[174]{nickelsburg2005}.} and the explicit reference to the ``the [Book] of the Words of Enoch'' in col. 19.25 suggests that the \ga was familiar with 1 \enoch, or at the very least a tradition of enochic writings.%
%
\footnote{It is worth noting, of course, that this reference occurs in the latter Abram section which some have argued originates in a different source than the first two memoirs. See esp. \cite{bernstein_berthelot-etal2010} and \cite{bernstein_as2010}.}

More plain, however, is the Noah Memoir's connection to the book of \jub, which seems to offer a consistent point of contact with this section of the \ga.\autocite[20]{fitzmyer2004} In fact, it was the explicit identification of Lamech's wife Bitenosh which first prompted Trevor's initial identification of the (unopened) scroll with the so-called Book of Lamech.\autocite{trevor_basor1949} Although an exhaustive treatment of the parallels between \jub and \ga is outside the scope of this chapter, it will suffice to note a few of the most significant points of contact between the Noah Memoir and \jub. James VanderKam has recently offered a detailed, yet concise, summary of these similarities and differences, which, while too long to reproduced in full, can be summarized as follows:%
\footnote{See \cite[374--376]{vanderkam_feldman-etal2017}. For additional treatments of this topic, see also \cite{machiela2009} and \cite[305--342]{kugel2012} previously published as \cite{kugel_roitman-etal2011}} 

\begin{enumerate}
    \item Several personal and geographic\footnote{%
        Mahaq Sea (16.9; Jub. 8.22), Tina River (16.15; Jub. 8.12), Mount Lubar (12.13; Jub. 5.28), Erythrean/Red Sea (17.7; Jub. 8.21), and Gadeira (16.11; Jub 8.26).}%
        %
        names which are never mentioned in the Bible show up in both \ga and \jub (including Batenosh, which is a part of the Lamech Memoir).
    \item Both \jub and \ga utilize ``Jubilees'' as significant chronological unit (\ga to a lesser degree than \jub).
    \item Several shared stories, themes, and phrases such as 1) ``in the days of Jared,'' 2) Enoch remains accessible after his departure from normal terrestrial life, 3) Noah makes atonement for the ``whole earth,'' and 4) stories about Noah and his vineyard.
    \item The ``division of the earth,'' while different in several specifics are strikingly similar and offer, perhaps, the most compelling case for a direct, genetic relationship between the two texts.\footnote{See also Machiela's extensive treatment of this section where he argues for the theory that both texts could be drawing from a shared cartographical source in \cite*[105--130]{machiela2009}. See also \cite{alexander_jjs1982}.}
\end{enumerate}

The striking similarities between the Noah Memoir and \jub  (and to a lesser degree, 1 \enoch) over and against the biblical text, again complicates the characterization of \ga as \rwb or strictly exegetical in nature. In other words if \ga drew from \jub (or if they drew from some common source) I think it is fair to scrutinize whether this section of \ga should be considered a rewritting of \emph{Genesis} or of some other set of traditions.\footnote{Of course, if \ga is the earliest (as Avigad and Yadin as well as Vermes supposed), we would simply be asking the same questions about the book of \jub with the same basic implications.}

% Third Section (Abram) Description
\subsubsection{The Abram Memoir (Cols. 19--22)}

The final surviving columns of the scroll, cols. 19--22, represent the longest and most complete sustained narrative preserved in \ga, here referred to as the ``Abram Memoir.'' More so than the previous sections, the Abram Memoir maps very closely onto the evens narrated in Genesis. These columns parallel Genesis 12:10--15:14, retelling the stories of Abram and Sarai's sojourn in Egypt (||~Gen 12:10--20), Abram's subsequent conflict with Lot (||~Gen 13:1--18), the Elamite campaign (||~Gen 14:1--24), and the beginning of Abram's vision (||~Gen 15:1--4). \ga's retelling of these stories follows the chronology of Gen 12--15 very closely, but embellishes and augments the narrative throughout. Like the Lamech and Noah Memoirs, this section of the \ga is largely written as a first-person narrative, this time in Abram's voice. The transition between the Noah Memoir and the Abram memoir is missing, so there is no superscription or title for this section, however, the phrase ``I, Abram'' shows up a number of times, making it clear who the narrator is. This fact is complicated, however, by the fact that, although the narrative begins the in the first-person, beginning in 21.23 the narrator transitions to the third person and remains so through the end of the surviving portion of the scroll.\footnote{It is worth pointing out that the final surviving sheet of parchment was not the final sheet of the scroll originally. Avigad and Yadin note that although only four sheets of the work were present, the seem between the fourth and (what would be) the fifth sheets is visible on the edge of the fourth sheet. \cite*[14]{avigad-yadin1956}.} This inconsistency, perhaps more than any other feature of \ga, has complicated its generic classification.

% First Half (Midrash)
The earlier portions of the Abram Memoir strike a balance between fidelity and innovation with regard to the \emph{biblical} text that the other sections lack. For example, the narrative of Abram and Sarai's descent into Egypt is clearly and recognizably built from the story preserved in the Hebrew Bible. The events and chronology of the story map directly onto Gen 12:10--20, but the \ga offers---in addition to the first-person point of view---a number of expansions that seem plainly to be innovative or, as Vermes would put it and example or prototype of ``midrash.''%
%
\footnote{\cite[124]{vermes1961} Notably, the characterization of \ga as \rwb is typically based on an analysis of the Abram Memoir. Although the earlier portions of the scroll were known, \vermes's treatment of \ga only dealt with cols. 19--22. Together with the fact that these are the best-preserved and most complete columns, this reality has, I think, impacted the characterization of \ga as a whole, perhaps unfairly. On the characterization pre-rabbinic texts as ``midrash,'' see \cite[GET PAGE RANGE]{mandel2017}; \cite{mandel_bakhos2006}.}
% 
Numerous small additions and emendations occur throughout the retelling such as making explicit how long Sarai and Abram lived in Egypt prior to Sarai's notice by Pharoah's princes, how long Sarai was with Pharoah, numerous geographical and personal names, etc. A number of these details, as with earlier sections of \ga, are also found in \jub, though, again, the direction of dependence is not clear (if present). More noticeable are the larger expansions present in the \ga such as Abrams portentous dream (19.14--17), the \emph{waṣf} put on the lips of Pharoah's princes about Sarai (20.2--8), Abram's prayer following Sarai's abduction (20.12--16), the details of Pharoah's afflictions(20.16--21), Harkenosh's discussion with Lot (20.21--20.24), and Abram's intervention on Pharoah's behalf (20.24--32).%
%
\footnote{Other changes from later in the memoir include a description of Abram walking the length and width of the land as well as a notable abbreviation of Abram and Lot's conflict in Gen 13:5--12.}

The explanation of these expansions, according to \vermes---which has been adopted by most treatments of \ga---is as a means of ``correcting'' or otherwise supplementing the biblical text in order to engage the reader and to \emph{explain} the biblical text.\autocite[126]{vermes1961} \vermes writes:

\begin{quote}
The author of GA does indeed try, by every means at his disposal, to make the biblical story more attractive, more real, more edifying, and above all more intelligible. Geographic data are inserted to complete biblical lacunae or to identify altered  place names, and various descriptive touches are added to give the story substance\dots To this work of expansion and development Genesis Apocryphon adds another, namely, the reconciliation of unexplained or apparently conflicting statements in the biblical text in order to allay doubt and worry.\autocite[125]{vermes1961}
\end{quote}

% Latter Half (Targum)
By contrast, the latter portion of the Abram Memoir, beginning at 21.23 at times borders on a word-for-word translation of Genesis into Aramaic with comparatively few significant changes. This quality provided occasion for a number of (especially early) scholars to compare \ga with the Targums.\footnote{\cite[193]{black1983}. Though, he notably amended his opinion later \cite*{black_black-fohrer1968}.} Although the change from first-person to third-person is, perhaps, the most significant literary shift that occurs in the \ga, other literary features of the Abram Memoir agree against the Lamech and Noah Memoirs in such a way that gives reason to suppose the Abram Memoir makes up a literary unit.\footnote{Specifically, Moshe Bernstein has noted based on the divine names that are use throughout the work that the primary division is between the Lamech/Noah Memoirs and the Abram Memoir; the earlier sections utilizing a specific set of divine titles and the latter section(s) using a different set. See \cite{bernstein_jbl2009}; See also \cite[97]{falk2007}. Regarding the genre(s) and unity of \ga more generally see Bernstein's later work \cite*{bernstein_berthelot-etal2010} and \cite*{bernstein_as2010}.} It is not clear, however, why there seems to be such a dramatic difference in narrative voice beginning in 21.23.

\subsection{Exegesis and Memory}

Thus, modern treatments of the \ga have tended to speak about the work as ``Rewritten Bible'' as a third category somewhere between Targum and Midrash, with a preference to the latter.%
%
\footnote{\cite{evans_revq1988}; \cite[19]{fitzmyer2004}. Esther Eshel has proposed the term ``narative midrash,'' but I am in agreement with Harrington and Bernstein in eschewing later categories such as ``midrash'' for these pre-rabbinic soruecs. See \cite[182]{eshel_roitman-etal2011}; Cf. \cite[242]{harrington_kraft-nickelsburg1986}; \cite[327 n. 33; 328--329]{bernstein_berthelot-etal2010}.}

Yet, as I have illustrated, although portions of the \ga relate clearly to the text of Genesis (notably, the Abram Memoir), much of the earlier portions of the scroll only nominally relate to Genesis, and instead show an affinity to the traditions associated with 1 \enoch and \jub. Thus, characterizing the work as a whole as focused primarily on the explanation of Genesis (as \vermes suggests), seems to me to be ill-founded. Indeed, the disjunction between the various parts of \ga have been observed by numerous scholars, even by those who broadly accept the \ga to be a literary unity, but such discussions still seem to focus on generic classification, which, I think is a methodological dead-end for thinking about \ga.\footnote{Notably \cite{bernstein_as2010} and \cite{falk2007}. Cf. \cite{eshel_roitman-etal2011}.}

To illustrate this difficulty, I would like to focus on Moshe Bernstein's treatment of the ``Genre(s)'' of the \ga.\autocite[As argued in][]{bernstein_berthelot-etal2010} Bernstein's basic thesis is to note that the \ga, as a composite work, must be treated as multi-generic, rather than simply as ``rewritten Bible'' or ``parabiblical'' or the like because, as noted above, the \ga does not relate uniformly to the biblical text. The difficulty, for Bernstein, comes when one must decide how to characterize the work as a whole. While works such as \jub and Pseudo-Philo could be viewed as works that have been uniformly ``rewritten'' (that is, that the entirety of the work is a single rewriting), works such as \ga (he also includes the \templescroll) could be viewed as ``a series of mini-rewritings of limited scope.''%
%
\footnote{\cite[336]{bernstein_berthelot-etal2010}. I am reminded here of Nickelsburg's similar sentiment regarding the ways that 1 \enoch rewrites the flood story several times, arguing that the phenomenon of rewriting moved from smaller units of rewriting to larger, more systematic rewritings. See \cite[TODO: Get Pages]{nickelsburg_stone1984}.}
%
According to such a characterization, Bernstein writes, ``we have no choice but to refer to Part I [the Lamech and Noah Memoirs] as `parabiblical' and Part II [the Abram Memoir] as `rewritten Bible''' based on the fact that, while the Abram Memoir rewrites portions of Genesis, the Lamech and Noah Memoirs really only take Genesis as a point of departure for their stories (and may, in fact, be rewriting other texts).\autocite[337]{bernstein_berthelot-etal2010} To refer to the entirety of \ga as \rwb or as two different kinds of \rwb is, according to Bernstein, unacceptably imprecise. While I am happy to accept a multigeneric characterization of \ga (and any number of other texts), I think Bernstein has sidestepped a more fundamental question by suggesting that the relationship between the \ga and its sources is best addressed as an issue of genre. The assumption made by Bernstein is that there was a qualitative difference between the sources utilized by \ga\footnote{While I am sympathetic to viewing \ga as secondary to \jub and 1 \enoch, here, I am simply stating this as Bernstein's position.} which forms the basis of his characterization of \ga as ``multigeneric.'' This pluriformity is in tension with his larger assertion affirming the unity of the work. 

However, it seems to me that the situation may be better analysed in reverse, namely that the genre of \ga is consistent and it its the assumed qualitative distinction between its sources that should be interrogated. In other words, rather than characterizing \ga as a work that utilized both ``biblical'' and ``non-biblical'' sources, it is just as reasonable to begin with the assumption that \ga's method is consistent and that the use of ``non-biblical'' sources actually points to the possibility that \jub and 1 \enoch were just as legitimate sources as Genesis. One possible inference from this observation could be that these other works may have been on equal footing as Genesis and enjoyed some special ``scriptural'' (or otherwise authoritative) position for the author of \ga or that such categories were  not operative at this time.\footnote{SOMETHIGN SOMETHING Eva Mrozeck.}




% Relationship to Jubilees and 1 Enoch and the Bible (Also, other texts [See Vermes])
% Although the scholarly consensus since the initial publication of \ga has been that 1 \enoch, Jubilees, and \ga all participate in overlapping or adjacent traditions,\footnote{\cite[38]{avigad-yadin1956}; \cite[20--22]{fitzmyer2004}; \cite[110--116]{crawford2008}; \cite[8--19]{machiela2009}.} what remains unclear is the nature and directionality (if any) of these relationships. While Avigad and Yadin suspected that \ga was a source for 1 \enoch and \jub,\autocite[38]{avigad-yadin1956} it is now widely acknowledged that no definitive evidence has yet been assembled to argue one way or another.\footnote{At the risk of over-simplifying the issue, Fitzmyer, Kugel, VanderKam, and Nickelsburg tend to see \ga as secondary to \jub and \ga, while Machiela and Segal have argued the reverse. See \cite{vanderkam_feldman-etal2017}, \cite[]{fitzmyer2004}, \cite[174]{nickelsburg2005}, \cite[305--342]{kugel2012}. Cf. \cite{segal_as2010}, \cite[140--142]{machiela2009}.}


%% What are the points of contact with the Bible and other contemporary sources? (Jubilees, Enoch, etc.)
%% What is GA's relationship *to* the Bible
    %% Conflating Pharoah and Abimelek is "interpretation" collapsing to a single story
    %% Cedar and Date Palm? Dafuck is going on here?

% (re)frame this example as 'memory.'
    %% A synthesis of traditions about Abe and Sarai
    %% Likely some innovative stuff
    %% But it's not beholden to the tradition. That is, it is participating in the creation of tradition and that tradition may be able to be traced as continuing into the Gen. Rab. with the Cedar and DatePalm stuff (?)




% \vermes goes so far as to say that the omission of Gen 12:16's assertion that things went well for Abram \heb{בעבורה} ``on account of her (Sarai)'' was do to ``an apologetic preoccupation and a desire to avoid scandal'' because the ``retention of the passage as it stands would offend pious ears.''\autocite[125]{vermes1961}