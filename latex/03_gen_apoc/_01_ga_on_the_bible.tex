
\section{\ga and Biblical Memory}

% Section introduction

Although it is anachronistic to suggest that the ``Bible''  existed during the late \secondtemple period, insofar as the texts and traditions that later formalized as the ``Bible''---especially those contained in the Pentateuch---were certainly present in a reasonably stable and even privileged state, I think it is a mistake to jettison any discussion of \rwb texts as they relate to the texts that would later become the Hebrew Bible. On the other hand, restricting our discussion to those later biblical texts would likewise not do justice to the wide variety of texts and traditions in existence during the \secondtemple period which undoubtedly influenced \ga. In an effort to strike a middle ground, therefore, I have opted to refer to ``biblical memory,'' by which I simply mean the confluence of stories and traditions which relate to those later formalized in the Hebrew Bible.%
%
\footnote{I would like to emphasize that I am not suggesting that ``biblical memory'' represents a qualitatively unique form of memory, only that the scope of the traditions under consideration relate to texts that later became the Bible, and, in all likelihood, held at least some sort of special privilege within the memory of many \secondtemple Jews.}
%
In this section, therefore, I would like to discuss the ways that the \ga participated in biblical memory.

% What was GA rewriting: Sources for the parts of GS
\subsection{What was the \ga Rewriting?}

Although the \ga is generally touted as one of the more clear-cut examples of the \rwb, it is noteworthy that its relationship to the biblical text is not, in fact, entirely uniform.\footnote{\cite[333]{bernstein_berthelot-etal2010}.}

% First Columns (Lamech)
The earliest columns of the \ga (cols. 0--5), which are narrated from the perspective of Lamech, Noah's father, essentially offer a rewriting of 1 Enoch 106--107.\footnote{\cite[174]{nickelsburg2005}. The birth of Noah seems to have been a matter of some interest; a number of other texts likewise discuss the exceptional qualities of Noah at his birth. See \q{4}{534}{}[\q{4}{BNoah}{a-d}], \q{1}{Noah} as well as \cite{vanderkam_kapera1992}.} In this section, Lamech, recounts the birth of Noah and Lamech's fear that his wife, Bitenosh, had conceived Noah by means of the \he{עירין} "Watchers." Despite Bitenosh's assurances, Lamech petitions his father, Methusaleh to ask \emph{his} father, Enoch, for further assurance, which he ultimately gives. Although this section is fragmentary, its close resemblance to 1 Enoch 106--107 makes the scholarly reconstruction of the missing sections quite plausible. While it may be tempting to suggest that this section of \ga represents a variant edition of 1 Enoch 106--107, rather than a rewriting, the fact that the version of the story preserved in \ga is told in the first-person from the point of view of Lamech, while 1 Enoch 106--107 is told in the third-person, makes this suggestion highly unlikely. Moreover, because both 1 Enoch and \ga were composed in Aramaic, the differences between the two tellings cannot be attributed to translational issues. In other words, although cols. 0--5 deal, nominally, with events in Genesis 5:28--29, for all intents and purposes, the story recounted in these columns is a retelling of events known from the Enochic tradition.%
%
\footnote{It is not clear what the precise relationship between the Enochic traditions and the \ga actually were. Here I have more-or-less assumed the priority of 1 Enoch, but I wish to leave ambiguous whether \ga represents a rewriting of the \emph{text} of 1 Enoch, or wether they simply draw on a common tradition. Thus, I have chosen to refer to the tradition ``known from'' 1 Enoch, rather than 1 Enoch itself. See Nickelsburg's concise but thorough treatment of the similarities and differences in \cite[173--174]{nickelsburg2005} as well as that of \cite[122--123]{fitzmyer2004}.} 


% Second Section (Noah)
The second major section of \ga begins with a superscription identifying What follows as a \he{[פרשגן] כתב מלי נוח} or ``[A copy of] the Book of the Words of Noah" (5.29) and continues onto the beginning of col. 18.%
%
\footnote{\cite[174--175]{nickelsburg2005}; On the topic of the so-called ``book of Noah'' see \cite{dimant_vanderkam-etal2006}.}

%
\footnote{See also \cite[174]{nickelsburg2005} who suggests the Lamech narrative may be a rewriting of 1 Enoch 106--107.}

% Third Section (Abram)
At the other end of the spectrum (and the scroll), cols. 19--22 represent the longest and most complete sustained narrative preserved in \ga. Moreover this section of the text maps very closely to Genesis 12:10--15:14 which retells the stories of Abram and Sarai's sojourn in Egypt (||~Gen 12:10--20), Abram's subsequent conflict with Lot (||~Gen 13:1--18), the Elamite campaign (||~Gen 14:1--24), and the beginning of Abram's vision (||~Gen 15:1--4). \ga's retelling of these stories follow the chronology of Gen 12--15 very closely, but \ga embellishes and augments the narrative throughout. In particular, the narrative portion beginning in col. 21.23 and continuing to the end of the (extant) scroll at times borders on a word-for-word translation of Genesis into Aramaic.

The similarity of these retellings to their putative biblical \emph{Vorlage}%
\footnote{Of course, we cannot be certain that the \emph{Vorlage} of \ga was, in fact, the same as the MT. That said, relative stability of (especially) the Torah texts during the Second Temple Period is widely accepted. For the purposes of this section, I will work under the assumption that the MT represents a very close approximation to the text that the authors/editors of Genesis Apocryphon were familiar with.}

prompted some (esp. early) scholars to suggest that \ga represented a sort of prototype for the later Pentateuchal Targums.\autocite[193]{black1983} While this may be a fair assessment for the very end of the scroll, the freer sections that precede it, and especially the earliest portions that deal with Lamech and Noah, really cannot reasonably be considered even ``paraphrases.'' Thus most treatments of \ga have tended to discuss the the work, following \vermes, in terms of its relationship to the genre of ``midrash.''\autocite{vermes1961}%
%
\footnote{}

%% Rewriting of the biblical account


%% What are the points of contact with the Bible and other contemporary sources? (Jubilees, Enoch, etc.)
%% What is GA's relationship *to* the Bible
    %% Conflating Pharoah and Abimelek is "interpretation" collapsing to a single story
    %% Cedar and Date Palm? Dafuck is going on here?

% (re)frame this example as 'memory.'
    %% A synthesis of traditions about Abe and Sarai
    %% Likely some innovative stuff
    %% But it's not beholden to the tradition. That is, it is participating in the creation of tradition and that tradition may be able to be traced as continuing into the Gen. Rab. with the Cedar and DatePalm stuff (?)
