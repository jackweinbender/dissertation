\section{\ga and the Biblical Tradition}

Although the \ga is generally touted as one of the more clear-cut examples of the \rwb, it is notworthy that its relationship to the biblical text is not, in fact, entirely uniform. The earliest columns of the \ga (cols. 0--5)---insofar as they are preserved---take very few queues from Genesis, if any, and instead deal with the more fantastical stories found in 1 Enoch and Jubilees.%
%
\footnote{\cite[333]{bernstein_berthelot-etal2010}. See also \cite[174]{nickelsburg2005} who suggests the Lamech narrative may be a rewriting of 1 Enoch 106--107.}
% Abram and Sarai in Egypt
At the other end of the spectrum (and the scroll), cols. 19--22 represent the longest and most complete sustained narrative preserved in \ga. Moreover this section of the text maps very closely to Genesis 12:10--15:14 which retells the stories of Abram and Sarai's sojourn in Egypt (||~Gen 12:10--20), Abram's subsequent conflict with Lot (||~Gen 13:1--18), the Elamite campaign (||~Gen 14:1--24), and the beginning of Abram's vision (||~Gen 15:1--4). \ga's retelling of these stories follow the chronology of Gen 12--15 very closely, but \ga embelleshes and augments the narrative throughout. In particular, the narrative portion beginning in col. 21.23 and continuiing to the end of the (extant) scroll at times borders on a word-for-word translation of Genesis into Aramaic.

The similarity of these retellings to their putative biblical \emph{Vorlage}%
%
\footnote{Of course, we cannot be certain that the \emph{Vorlage} of \ga was, in fact, the same as the MT. That said, relative stability of (especially) the Torah texts during the Second Temple Period is widely accepted. For the purposes of this section, I will work under the assumption that the MT represents a very close approximation to the text that the authors/editors of Genesis Apocryphon were familiar with.}

prompted some (esp. early) scholars to suggest that \ga represented a sort of prototype for the later Pentateuchal Targums.\autocite[193]{black1983} While this may be a fair assessment for the very end of the scroll, the freer sections that preceed it, and especially the earliest portions that deal with Lamech and Noah, really cannot reasonably be considered even ``paraphrases.'' Thus most treatments of \ga have tended to discuss the the work, following \vermes, in terms of its relationship to the genre of ``midrash.''\autocite{vermes1961}%
%
\footnote{}

%% Rewriting of the biblical account


%% What are the points of contact with the Bible and other contemporary sources? (Jubilees, Enoch, etc.)
%% What is GA's relationship *to* the Bible
    %% Conflating Pharoah and Abimelek is "interpretation" collapsing to a single story
    %% Cedar and Date Palm? Dafuck is going on here?

% (re)frame this example as 'memory.'
    %% A synthesis of traditions about Abe and Sarai
    %% Likely some innovative stuff
    %% But it's not beholden to the tradition. That is, it is participating in the creation of tradition and that tradition may be able to be traced as continuiing into the Gen. Rab. with the Cedar and DatePalm stuff (?)
