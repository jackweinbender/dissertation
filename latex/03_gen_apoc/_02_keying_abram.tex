% !TeX root = ../dissertation.tex

\section{Abram in the Diaspora: The Literary Frameworks of \GA}

Having dealt with the \ga as the product of cultural memory in terms of its relationship to its inherited biblical memory (including the traditions which were ancillary to Genesis proper), we may now turn our attention to the ways that the \ga addressed its audience at the level of \emph{social} memory. In this section I will address the way that the \ga \emph{speaks to} its audience and the ways that the \ga changes and adapts its cultural memory into a meaningful piece of literature for \secondtemple Judaism.

As I have already noted, the narrative of the \ga is not simply a straight-forward retelling of Genesis from the perspectives of Lamech, Noah, and Abram, but participates more broadly in the ``biblical memory'' of Genesis. However, what is most compelling about \rwb texts very often is the \emph{ways} that they adapt biblical memory. These adaptations can come at the level of story---by adding, removing, or rearranging events---or at the level of narrative discourse by describing events differently or with different emphases. In the case of \ga, and in particular in the account of Abram in cols. 19--22, the biblical narrative has been recast as a (first-person) Hellenistic novella in a similar vein to other well-known Second Temple Jewish works such as the narrative portions of Daniel (including the Greek additions), Esther, Tobit, and (arguably) the so-called Joseph novella of Genesis 37 and 39--50.\footnote{See especially Lawrence Wills work on the Jewish novels and novellas in antiquity: \cite*{wills2002} as well as his important earlier works \cite*{wills1995} and \cite{wills1990}.}

The reading of \ga 19--20 as a Hellenistic Jewish novella has recently been very thoroughly explicated by Blake Jurgens, who has further argued that the utilization of Hellenistic literary motifs and structures in \ga altered the overall purpose the the pericope for the purpose of edifying Jews living in the Hellenistic world in the shadow of empire.\autocite{jurgens_jsj2018} Although much of Jurgens's paper is based on long-established observations about the literary influences on \ga, he makes the important discursive turn toward the audience by claiming that the \ga was meant to be useful to readers:

\begin{quote}
By imbuing its story with literary tropes and techniques similar to those found in Dan 1--6, Esther, and other Jewish texts arising out of the Hellenistic period, the author successfully attends to the narratival ambiguities of Gen 12:10--20 through interpretive expansion upon the latent exegetical links of the text while concurrently modifying the narrative to appeal to contemporary literary expectations.\autocite[27]{jurgens_jsj2018} \end{quote}

Thinking in terms of social memory, however, we can appreciate the way that the stories that the \ga retells are ``remembered into'' the social context of Hellenistic Judaism and are fitted into contemporary social frameworks by the utilization of common literary techniques. In other words, the changes that Jurgens identifies as authorial decisions intended to engage with readers can also be framed as \emph{determined by} the social location of the author and the literary tools available to him.


\subsection{Abram in the Diaspora}

One of the primary features of Jewish Hellenistic novellas is their setting. Jurgens notes that, typically, these Jewish novellas are set in the diaspora, which invariably place the Jewish (or, in Tobit and Judith's case, Israelite) protagonist under the hegemony of a foreign power. In the case of \ga, although not properly ``diaspora,'' Abram is a sojourner in a foreign land and is under foreign hegemony. Moreover, from a modern perspective, these stories have a tendency to commit rather egregious factual errors about certain historical particulars such as the names of rulers (Judith 1:1; Dan 4; Tobit) and geographic items (Tobit 5:6). Likewise, \ga seems to utilize details which almost certainly were inventions of the author (or an earlier tradant) such as referring to ``Pharaoh Zoan'' (we know of no such figure) and Herqanos, a name popular in the Ptolemaic period, but not attested otherwise as well as referring to the ``Karmon River'' (probably the Kharma canal), as the one of the seven heads of the Nile river, which it is not.\autocites[7]{jurgens_jsj2018}[See also][50--59]{machiela_as2010}[197--199]{fitzmyer2004} These details, according to Jurgens, are meant to create a sense of verisimilitude and authenticity within the narrative. Thus, although the story of Abram's sojourn in Egypt as narrated in the biblical text engages with discourses of the \emph{foundation} of Israel, the narrative of the \ga seems to be turning the story to engage with the contemporary discourses around the idea of \emph{diaspora}. In other words the way that Abram's sojourn in Egypt was remembered in the \secondtemple period, at least in part, took on new meaning for those sojourning in the diaspora and for those living in the land under foreign hegemony.

\subsection{Abram in the Court of a Foreign King: Literary Genre as Social Framework}

If we place the pericope of Abram's journey into Egypt in \ga under the rubric of diaspora literature, the final scene in the pericope bears a striking resemblance to the so called court contest narratives well-known from (especially) the book of Daniel.\footnote{Other court contest narratives include the Joseph Cycle (Gen 41)} Such narratives, as observed by Collins and others, follow particular narrative progressions with common features.%
%
\footnote{\cite[TODO: pages]{collins1993}; \cite{humphreys_jbl1973}; \cite{collins_jbl1975}; \cite{wills1990}. See also \cite{niditch-doran_jbl1977}.}
%
Jurgens has convincingly shown that the \ga's retelling of Abram's sojourn in Egypt fits such a progression by comparing this pericope to to Dan 2, 4, and 5 as well as Gen 41. Although based on the earlier work of Collins and Humprheys, Jurgens offers his own outline, which can be summarized as follows\autocite[21]{jurgens_jsj2018}:

\begin{SingleSpace}
\begin{itemize}
    \item The foreign king has a problem that he is unable to solve.
    \item The king's own personnel are charged with solving the problem
    \item The king's personnel are unable to solve the problem
    \item The Jewish protagonist is asked to solve the problem
    \item The Jewish protagonist is able to solve the problem
    \item The Jewish protagonist is rewarded by the king
\end{itemize}
\end{SingleSpace}

It is easy to imagine how the author of \ga would conceive of Abram's interaction with Pharoah in Gen 12 as analogous to other well-known court contests from Israel's biblical memory. The biblical account, however, offers a rather anemic description of the events, but leaves open the specifics of how Pharoah came to know about Abram and how the monarch was relieved from the plagues. 

From an innerbiblical perspective, the \ga's description of Abram and Pharaoh's interaction might be thought of as a synthesis or exegetical harmonization with the Abimelech doublet in Gen 20, which offers a much more detailed account of the Abimelech's confrontation with Abram/Abraham. While the Gen 12 account is very terse, the Gen 20 account includes a dream-revelation, specifies that the plagues that afflicted the monarch impeded his sexual activities (specifically with Sarah), and describes Abraham praying over Abimelech and his household to heal them. Similar details are given in the \ga's account which likewise includes a dream-revelation, notes that the plague were sexual in nature, and describes Abram praying over Pharoah and his household for healing. 

However, while these similarities may indeed represent some kind of literary conflation between the two accounts,%
%
\footnote{From a memory perspective I would prefer to account for the \ga's adoption of certain details from Gen 20 in more passive terms where the specifics of the Gen 12 story are, where absent, supplied from another well-known, typologically similar, source.} 
%
at the level of genre and structure, conflation with Gen 20 cannot account for the \ga's reframing as a court-contest. For example. the dream-revelation in Gen 20 is given to Abimelech, rather than to Abram as in \ga. Moreover although Abraham prays for healing for Abimelech and his household in a very similar fashion to the way he is portrayed in \ga praying for Pharoah, in Gen 20, he does so only after Abimelech effectively ``pays him off.'' It is the revelation given to Abimelech in a dream which causes him to ``repent'' in Gen 20, while in \ga, the miraculous healing of Pharoah and his household functions as the sign and catalyst for Pharoah's rich rewarding of Abram. Although this difference may seem subtle, the primary feature of the court-contest is the demonstration of God's power through the protagonist which leads to the foreign king's repentance/conversion and the rationale for his rewarding of the protagonist. In other words, while it may have been that the \ga used details from Gen 20 to supplement the account from Gen 12, \ga's framing of Abram's contest with Pharoah cannot be solely attributed to a harmonization of the Gen 12/20 doublet. Thus, drawing on details from, or perhaps just inspired by, the Abimelech doublet in Gen 20  the author of \ga was able to reframe this portion of the Abram narrative to conform to the common court-contest pattern, which, as Jurgens rightly notes, surely would have been an effective and entertaining adaptation by comparison to the account from Genesis.

\subsection{Other Literary features and Motifs}
A number of other generic and literary motifs which diverge from the Genesis account, but which are at home in the \secondtemple period can be identified in this portion of the \ga as well.

\subsubsection{Abram as Oracle}
Although the Abimelech story in Gen 20 includes a dream-revelation, it is noteworthy that in \ga, Abram himself is given the dream as a means of warning him about how the Egyptians would attempt to kill him on account of Sarai's beauty. Where in the biblical text credits Abram's intuition for anticipating the Egyptians' desire for Sarai (though, we are left to wonder whether he would have been killed had the ruse not been realized) the \ga describes Abram receiving a portentous dream vision characteristic of other \secondtemple literature.%
%
\footnote{\cite{gevirtz_maarav1992}; \cite[184]{fitzmyer2004}; \cite{dacy_tzoref2013}}
%

Although dream-visions are not unique to the \secondtemple period, their ubiquity within Jewish literature from the \secondtemple period is indisputable. In his treatment of the Dream-Visions among the Aramaic \dss, Andrew Perrin describes both Abram and Noah as being ``recast as a dreamer[s]'' within the \ga.\footnote{\cite[52--57]{perrin2015}. See also \cite{eshel_klostergaard-etal2009} and \cite{machiela_falk-etal2010}.} While Noah is not described as a dreamer within the biblical text, within the \ga, he seems to have been the recipient of as many as five dream-visions.\footnote{\cite[53]{perrin2015}. Elsewhere in the Enochic literature dreams weight heavily in the events surrounding the flood, if not always given to Noah (\firstenoch, Book of Giants, etc.).} Restricting the discussion to Abram, however, Perrin suggests that the insertion of a dream-vision into the story on the eve of Abram and Sarai's descent into Egypt functioned as part of a larger project to ``extent Abram's prophetic credentials in light of Gen 20:7.''\autocite[55]{perrin2015} 

Fitzmyer notes that the component parts of this dream---``cedar" (Aramaic: \aram{ארז}) and ``date-palm" (Aramaic: \aram{תמרא})---are derived from Ps 92, which declares ``the righteous will flourish like the date palm (Hebrew: \he{כַּתָּמָר}); like a cedar (Hebrew: \he{כְּאֶרֶז}) in Lebanon he will grow" (Ps 92:13). The identification of Abram and Sarai with the cedar and date-palm, respectively, is plain enough by the parallel to what happens next in the narrative and supported both by the grammatical gender of \aram{ארז} (masc.) and \aram{תמרא} (fem.) as well as the use of the name ``Tamar'' by a number of women in the Bible (Gen 38:6; 2 Sam 13:1; 14:27).

The dream itself provides an allegorical vision that credits the date-palm (Sarai) with saving the cedar (Abram) from the people seeking to destroy it. Although there is a question whether the beginning of 19.14--15 should read \aram{והא אזר חד ותמרא כחדא חמחו מן שרש חד} ``a cedar and a date-palm \emph{growing from a single root}" (so DJD), or \aram{ והא אזר חד ותמרא חדא יאיא שגי} ``a cedar and a date-palm [which was] \emph{very beautiful}" (so, Fitzmyer), all editions understand 19.16 to read \aram{ארי תרינא מן שרש חד זמחנא} ``for the two of us grow from a single root." Thus, the purpose of the dream  is to show Abram the way that he should avoid being ``cut down and uprooted" by the Egyptians, namely, by claiming that he and Sarai ``sprung from the same root," viz. are siblings. This interpretation is also offered by Abram himself in 19.19--21.

Significantly, the later Genesis Rabbah connects the this section of Genesis with Ps 92 and utilizes the cedar/date-palm imagery as well, albeit in a different manner. Specifically, during its treatment of the description of the plagues which God inflicted on Pharoah, Genesis Rabbah begins with a citation of Ps 92, ``The righteous will flourish like the date-palm (Hebrew: \he{תמר}); like a cedar (Hebrew: \he{ארז}) in Lebanon he will grow" and this comparison to date-palms and cedars makes several digressions. First, building on the idea of righteousness, Genesis Rabbah observes that both cedars and date-palms are "straight" trees, largely without crooks and crotches.\footnote{Although the Psalm uses the typical term for "righteousness" (Hebrew: \he{צָדִיק}), another common biblical term for a person who acts in an upright manner is "straight" (Hebrew: \he{יָשָׁר}) which seems to be what the Genesis Rabbah is playing off of. Furthermore, being tall trees, they cast longs shadows representing the fact that their reward will come later.} The second digression focuses on the ability of date-palms to produce fruit (including through grafting) and the usefulness of, especially date-palms for all manner of practical concerns. Genesis Rabbah then extends the comparison to the whole of Israel:

\begin{quote}
As no part of the palm has any waste, the dates being eaten, the branches used for Hallel, the twigs for covering [booths], and bast for ropes, the leaves for besoms, and the planed boards for ceiling rooms, so are there none worthless in Israel, some being versed in Scripture, some in Mishnah, some in Talmud, others Haggadah. (Gen. Rabbah 41.1)
\end{quote}

The final comparison makes the claim that, like dealing with Israel, climbing these tall trees is perilous. The proof, for Genesis Rabbah, brings us back to the verse at hand. That Pharaoh was plagued by Yahweh when he took Sarai for himself demonstrates the danger in engaging with Israel as an adversary.

For our purposes, what is significant is that the authors of both the \ga and the Genesis Rabbah both connect Ps 92 with this section of Genesis, but importantly, they use this connection for their own purposes. Although the rationale for connecting Ps 92 to Gen 12 is not entirely clear,%
%
\footnote{%
\begin{SingleSpace}Perhaps based on the Psalm's later reference to bearing children in one's old age:
\begin{quote}
    Planted in the house of Yahweh; they will flourish in the courts of our God\\
    They will still bring forth fruit in old age; they will be full of sap and green\\
    (Ps 92:14--15)
\end{quote}
\end{SingleSpace}}
%
\emph{that} the two passages were juxtaposed in both \ga and Genesis Rabbah seems more than a coincidence and it is notable that the two works use the connection for dramatically different purposes. Thus, while \ga's use of the cedar/date-palm imagery may rely on some previous tradition, the dream revelation itself is best understood as an example of the author of \ga utilizing the literary tropes of his own time and place. 

% \subsubsection{Abram as Sage}
% [TODO: A Brief discussion of Abram being depicted as a sage (sought for knowledge; reads from the ``book of the words of Enoch,'' etc)]

% \subsubsection{A \emph{waṣf} about Sarai}
% [TODO: Talk about Waṣf]

\subsection{Conclusions}

The recasting of Abram's sojourn as ``diaspora,'' his conflict with Pharaoh as a court-contest along and the portrayal of Abram and Noah as dreamers can be accounted for in terms of social memory as the author of \ga pressing the stories of Genesis into existing literary genres. Insofar as ``genres" can be understood as commonly understood literary conventions---a social ``contract'' of expectations between the author and her audience---they are socially defined and, for our purposes, function as what \halbwachs would call ``social frameworks.'' As Abram's sojourn in Egypt could be understood in new ways within the context of the diaspora, so too the common trope of the court-contest, well-known from the book of Daniel, provided a new framework into which the story of Gen 12 could be read. Thus, Jurgens' basic premise---that these stories are ``updated'' for a new audience---takes for granted what the memory approach makes explicit: \secondtemple Jews had their own ways of thinking about the way that God interacted with the ancients, and how pious Jews acted in particular circumstances. These social frameworks provided new structures for understanding the stories that they inherited from the biblical tradition. Thus, rather than only thinking about how the author was trying to ``fix'' the biblical account, from the memory perspective we can imagine the author of \ga not only interpreting the biblical tradition, but making efforts to contextualize it within his own literary frame of reference.