\section{Abram in the Diaspora: Keying and Framing in \GA}

While retelling portions of Genesis as first-person narrative reorients the way that the story engages with the received tradition and collective memory at the macro-level the narrative of the \ga is not simply a straight-forward retelling of Genesis from the perspectives of Lamech, Noah, and Abram. Indeed, what is most compelling about \rwb texts very often is the ways that they depart from the biblical narrative. These departures can come at the level of story by adding, removing, or rearranging events or at the level of narrative discourse by describing events differently or with different emphases. In the case of \ga, and in particular in the account of Abram in cols. 19--22, the biblical narrative has been recast as a (first-person) Hellenistic novella in a similar vein to other well-known Second Temple Jewish works such as the narrative portions of Daniel (including the Greek additions), Esther, Tobit, and (arguably) the so-called Joseph novella of Genesis 37 and 39--50.\autocites*[See especially Lawrence Wills work on the Jewish novels and novellas in antiquity:][]{wills2002}[as well as his important earlier works:][]{wills1995}[and][]{wills1990}

The reading of \ga 19--20 as a Hellenistic Jewish novella has recently been very thoroughly explicated by Blake Jurgens, who has further argued that the utilization of Hellenistic literary motifs and structures in \ga altered the overall purpose the the pericope for the purpose of edifying Jews living in the Hellenistic world in the shadow of empire.\autocite{jurgens_jsj2018} Although much of Jurgen's paper is based on long-established observations about the literary influences on \ga, he makes the important discursive turn toward the audience by claiming that the \ga was meant to be useful to readers:

\begin{quote}
By imbuing its story with literary tropes and techniques similar to those found in Dan 1--6, Esther, and other Jewish texts arising out of the Hellenistic period, the author successfully attends to the narratival ambiguities of Gen 12:10--20 through interpretive expansion upon the latent exegetical links of the text while concurrently modifying the narrative to appeal to contemporary literary expectations.\autocite[27]{jurgens_jsj2018} \end{quote}

The process of this transcription, which he terms ``fictionalization,'' is described by Jurgens in the six distinct narrative units within cols. 19-20 of \ga which describe Abram and Sarai's sojourn in Egypt: the decent into Egypt, Abram's dream vision, the banquet Scene, praise of Sarai's beauty, Abram's prayer, and the final court contest. In each section Jurgens notes the ways that the \ga utilizes literary structures common to it broader Hellenistic mileu to rewrite the the events of this story. Jurgens offers a very thorough description of the ways that the \ga utilizes these literary structures and makes a plausible claim that these changes were meant to engage readers in familiar style. Thinking in terms of social memory, however, we can appreciate the way that the story of Abram and Sarai in Egypt is ``remembered into'' the social context of Hellenistic Judaism and is fitted into the contemporary social frameworks (read: literary conventions) therein.

While \halbwachs addressed the fact that memories are shaped by the present, recently Barry Schwartz has attempted to more clearly articulate this process. One of the most important contributions of Schwartz's work in this area is his conviction that the interactions between the remembered past and the present are not unidirectional. Where \halbwachs limits his discussion to the describing the ways that memories of the past are shaped by the present, Schwartz sees the past as a potent force in the present as well. In other words, not only does the present influence the way the past is remembered, but the past \emph{itself} (that is, both the remembered and ``actual'' past) also effects the present.

Schwartz employs two terms, ``keying'' and ``framing,'' to describe this additional dimension to the way that the past impacts the present. On the one hand the idea of ``keying'' can be understood as way of


\subsection{Setting}

One of the primary features of these novellas is their setting. Jurgens notes that, typically, these Jewish novellas are set in the diaspora, which invariably place the Jewish (or, in Tobit and Judith's case, Israelite) protagonist under the hegemony of a foreign power. In the case of \ga, although not properly ``diaspora,'' Abram is a sojourner in a foreign land and is under foreign hegemony. Moreover, from a modern perspective, these stories have a tendency to commit rather egregious factual errors about certain historical particulars such as the names of rulers (Judith 1:1; Dan 4; Tobit) and geographic items (Tobit 5:6). Likewise, \ga seems to utilize details which almost certainly were inventions of the author (or an earlier tradant) such as referring to ``Pharaoh Zoan'' (we know of no such figure) and Herqanos, a name popular in the Ptolemaic period, but not attested otherwise as well as referring to the ``Karmon River'' (probably the Kharma canal), as the one of the seven heads of the Nile river, which it is not.\autocites[7]{jurgens_jsj2018}[See also][50--59]{machiela_as2010}[197--199]{fitzmyer2004} These details, according to Jurgens, are meant to create a sense of verisimilitude and authenticity within the narrative. Thus, although the story of Abram's sojourn in Egypt as narrated in the biblical text engages with discourses of the \emph{foundation} of Israel, the narrative of the \ga seems to be turning the story to engage with the contemporary discourses around the idea of \emph{diaspora}.

\subsection{Abram in the Court of a Foreign King}
\subsection{Abram the Sage}
\subsection{Abram the Oracle}