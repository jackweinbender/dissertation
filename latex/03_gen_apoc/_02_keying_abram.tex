% !TeX root = ../dissertation.tex

\section{Abram in the Diaspora: The Literary Frameworks of \GA}

Having discussed how the \ga functioned as the \emph{recipient} of a cultural memory that was broader than the text of Genesis, we may now turn our attention to the ways that the \ga was adapted to address its audience within the frameworks of their \emph{social} memory. This section will focus on the way that the \ga was shaped by the social frameworks of its compositional milieu, specifically the ways that contemporary cultural discourses and literary forms molded the biblical memory of Genesis (specifically, the Abram narrative) into a meaningful participant in the collective memory of \secondtemple Judaism.

As I have already noted, the narrative of the \ga is not simply a straight-forward retelling of Genesis from the perspectives of Lamech, Noah, and Abram, but participates more broadly in the ``biblical memory'' of Genesis (which includes related tradition like those represented in \firstenoch and \jub). However, what is most compelling about \rwb texts very often is the \emph{ways} that they adapt biblical memory. These adaptations can come at the level of story---by adding, removing, or rearranging events---or at the level of narrative discourse by describing events differently or with different emphases. In the case of \ga, and in particular in the account of Abram's descent into Egypt in \cols{19}{20}, the biblical narrative has been recast as a (first-person) Hellenistic novella in a similar vein to other well-known Second Temple Jewish works such as the narrative portions of Daniel (including the Greek additions), Esther, Tobit, and (arguably) the so-called Joseph novella of Genesis 37 and 39--50.\footnote{See especially Lawrence Wills work on the Jewish novels and novellas in antiquity: \cite*{wills2002} as well as his important earlier works \cite*{wills1995} and \cite{wills1990}.}

% For the convenience of the reader, I have provided the text and my translation of \q{1}{apGen ar}{} cols. \rom{19}, 10--\rom{20}, 33:

% [[TODO: Add text and my translation]]
% % TODO: Add  Analysis of Cols. 19-20?

The reading of \ga \cols{19}{20} as a Hellenistic Jewish novella has recently been very thoroughly explicated by Blake Jurgens, who has further argued that the utilization of Hellenistic literary motifs and structures in \ga altered the overall presentation of the pericope for the purpose of edifying Jews living in the Hellenistic world in the shadow of empire.\autocite{jurgens_jsj2018} Although much of Jurgens's paper is based on long-established observations about the literary influences on \ga, he makes an important discursive turn toward the audience by claiming that the \ga was meant to be useful to readers:

\begin{quote}
By imbuing its story with literary tropes and techniques similar to those found in Dan 1--6, Esther, and other Jewish texts arising out of the Hellenistic period, the author successfully attends to the narratival ambiguities of Gen 12:10--20 through interpretive expansion upon the latent exegetical links of the text while concurrently modifying the narrative to appeal to contemporary literary expectations.\autocite[27]{jurgens_jsj2018} \end{quote}

Thinking in terms of social memory, however, we can appreciate the way that the stories that the \ga retells are ``remembered into'' the social context of Hellenistic Judaism and are fitted into contemporary social frameworks by the utilization of common literary techniques. In other words, the changes that Jurgens identifies as authorial decisions intended to engage with readers can also be framed as \emph{determined by} the social location of the author and the literary tools available to him. In other words, as societies change over time, new kinds of literary forms overtake older ones and older stories take on new meanings for new contexts. This adaptation into new forms and meanings is socially determined and should be understood as an example of how social frameworks (in this case, literary frameworks), shape memory \emph{in the present}.

\subsection{Abram in the Diaspora}

One of the primary features of Jewish Hellenistic novellas is their setting. Jurgens notes that, typically, these Jewish novellas are set in the diaspora, which invariably place the Jewish (or, in Tobit and Judith's case, Israelite) protagonist under the hegemony of a foreign power. In the case of \ga, although not properly ``diaspora,'' Abram is a sojourner in a foreign land and is under foreign hegemony. Moreover, from a modern perspective, these stories have a tendency to commit rather egregious factual errors about certain historical particulars such as the names of rulers (Judith 1:1; Dan 4; Tobit) and geographic items (Tobit 5:6). Likewise, \ga seems to utilize details which almost certainly were inventions of the author (or an earlier tradant) such as referring to ``Pharaoh Zoan'' (we know of no such figure) and Herqanos, a name popular in the Ptolemaic period, but not attested otherwise as well as referring to the ``Karmon River'' (probably the Kharma canal), as the one of the seven heads of the Nile river, which it is not.\autocites[7]{jurgens_jsj2018}[See also][50--59]{machiela_as2010}[197--199]{fitzmyer2004} These details, according to Jurgens, are meant to create a sense of verisimilitude and authenticity within the narrative. Thus, although the story of Abram's sojourn in Egypt as narrated in the biblical text engages with discourses of the \emph{foundation} of Israel, the narrative of the \ga seems to be turning the story to engage with the contemporary discourses around the idea of \emph{diaspora}. In other words the way that Abram's sojourn in Egypt was remembered in the \secondtemple period, at least in part, took on new meaning for those sojourning in the diaspora and for those living in the land under foreign hegemony.

\subsection{Abram in the Court of a Foreign King: Literary Genre as Social Framework}

If we place the pericope of Abram's journey into Egypt in \ga under the rubric of diaspora literature, the final scene in the pericope bears a striking resemblance to the so called court contest narratives well-known from (especially) the book of Daniel.\footnote{Other court contest narratives include the Joseph Cycle (Gen 41)} Such narratives, as observed by Collins and others, follow particular narrative progressions with common features.%
%
\footnote{\cite[38--52]{collins1993}; \cite{humphreys_jbl1973}; \cite{collins_jbl1975}; \cite{wills1990}. See also \cite{niditch-doran_jbl1977}.}
%
Jurgens has convincingly argued that the \ga's retelling of Abram's sojourn in Egypt fits such a progression by comparing this pericope to to Dan 2, 4, and 5 as well as Gen 41. The resemblance is quite striking. Although based on the earlier work of Collins and Humprheys, Jurgens offers his own outline, which can be summarized as follows\autocite[21]{jurgens_jsj2018}:

\begin{SingleSpace}
\begin{itemize}
    \item The foreign king has a problem that he is unable to solve.
    \item The king's own personnel are charged with solving the problem
    \item The king's personnel are unable to solve the problem
    \item The Jewish protagonist is asked to solve the problem
    \item The Jewish protagonist is able to solve the problem
    \item The Jewish protagonist is rewarded by the king
\end{itemize}
\end{SingleSpace}

The biblical account, however, offers a rather anemic description of the events, but leaves open the specifics of how Pharoah came to know about Abram and how the monarch was relieved from the plagues. 
It is easy to imagine how the author of \ga would conceive of Abram's interaction with Pharoah in Gen 12 as analogous to other well-known court contests from Israel's biblical memory, despite the fact that the biblical version offers almost no detail. Gen 12:17--20 reads: 

\subsubsection{Genesis 12:17--20}
\begin{hebrewtext}
    \versenum{12:17}
    ‏וַיְנַגַּע יְהוָה אֶת־פַּרְעֹה נְגָעִים גְּדֹלִים וְאֶת־בֵּיתוֹ עַל־דְּבַר שָׂרַי אֵשֶׁת אַבְרָם׃
    \versenum{18}
    וַיִּקְרָא פַרְעֹה לְאַבְרָם וַיֹּאמֶר מַה־זֹּאת עָשִׂיתָ לִּי לָמָּה לֹא־הִגַּדְתָּ לִּי כִּי אִשְׁתְּךָ הִוא׃
    \versenum{19}
    לָמָה אָמַרְתָּ אֲחֹתִי הִוא וָאֶקַּח אֹתָהּ לִי לְאִשָּׁה וְעַתָּה הִנֵּה אִשְׁתְּךָ קַח וָלֵךְ׃ 
    \versenum{20}
    וַיְצַו עָלָיו פַּרְעֹה אֲנָשִׁים וַיְשַׁלְּחוּ אֹתוֹ וְאֶת־אִשְׁתּוֹ וְאֶת־כָּל־אֲשֶׁר־לוֹ׃
\end{hebrewtext}

\begin{translation}
    \versenum{Gen 12:17}
    Now, \yahweh afflicted Pharoah and his house with terrible plagues on account of Sarai, the wife of Abram. 
    \versenum{18}
    And Pharoah called for Abram and he said, ``What is this that you have done to me? Why did you not tell me that she was your wife? 
    \versenum{19}
    Why did you say, `she is my sister' such that I took her as a wife? Anyhow, here is your wife. Take her and go. 
    \versenum{20}
    And Pharoah gave his men orders about him and they sent him away along with his wife away and everything he owned. (My~translation)
\end{translation}

From an innerbiblical perspective, the \ga's description of Abram and Pharaoh's interaction might be thought of as a synthesis or exegetical harmonization with the Abimelech doublet in Gen 20, which offers a much more detailed account of the Abimelech's confrontation with Abram/Abraham (Genesis 20:8--18):

\begin{hebrewtext}
    \versenum{Gen 20:8}
    וַיַּשְׁכֵּם אֲבִימֶלֶךְ בַּבֹּקֶר וַיִּקְרָא לְכָל־עֲבָדָיו וַיְדַבֵּר אֶת־כָּל־הַדְּבָרִים הָאֵלֶּה בְּאָזְנֵיהֶם וַיִּירְאוּ הָאֲנָשִׁים מְאֹד׃
    \versenum{9}
    וַיִּקְרָא אֲבִימֶלֶךְ לְאַבְרָהָם וַיֹּאמֶר לוֹ מֶה־עָשִׂיתָ לָּנוּ וּמֶה־חָטָאתִי לָךְ כִּי־הֵבֵאתָ עָלַי וְעַל־מַמְלַכְתִּי חֲטָאָה גְדֹלָה מַעֲשִׂים אֲשֶׁר לֹא־יֵעָשׂוּ עָשִׂיתָ עִמָּדִי׃
    \versenum{10}
    וַיֹּאמֶר אֲבִימֶלֶךְ אֶל־אַבְרָהָם מָה רָאִיתָ כִּי עָשִׂיתָ אֶת־הַדָּבָר הַזֶּה׃
    \versenum{11}
    וַיֹּאמֶר אַבְרָהָם כִּי אָמַרְתִּי רַק אֵין־יִרְאַת אֱלֹהִים בַּמָּקוֹם הַזֶּה וַהֲרָגוּנִי עַל־דְּבַר אִשְׁתִּי׃
    \versenum{12}
    וְגַם־אָמְנָה אֲחֹתִי בַת־אָבִי הִוא אַךְ לֹא בַת־אִמִּי וַתְּהִי־לִי לְאִשָּׁה׃
    \versenum{13}
    וַיְהִי כַּאֲשֶׁר הִתְעוּ אֹתִי אֱלֹהִים מִבֵּית אָבִי וָאֹמַר לָהּ זֶה חַסְדֵּךְ אֲשֶׁר תַּעֲשִׂי עִמָּדִי אֶל כָּל־הַמָּקוֹם אֲשֶׁר נָבוֹא שָׁמָּה אִמְרִי־לִי אָחִי הוּא׃
\end{hebrewtext}

\begin{translation}
    \versenum{Gen 20:8}
    Abimelech rose early in the morning and called all his servants and spoke all these things in their ears. And the men were very frightened. 
    \versenum{9}
    So Abimelech summoned Abraham, and said to him, ``What have you done to us? How have I sinned against you such that you have brought terrible guilt upon me and my kingdom? You have done things to me which should not be done.''
    \versenum{10}
    And Abimelech said to Abraham, “What were you thinking that made you do this thing?” 
    \versenum{11}
    Abraham said, ``I thought `Surely there is no fear of God at all in this place, and they will kill me because of my wife. 
    \versenum{12} 
    Also, honestly, she is my sister; the daughter of my father but not the daughter of my mother and she \emph{became} my wife.
    \versenum{13}
    And when God caused me to wander from my father's house, I said to her, `This is the kindness you must do for me: at every place when we enter there, say  ``He is my brother.'''''
\end{translation}

\begin{hebrewtext}
    \versenum{Gen 20:14}
    וַיִּקַּח אֲבִימֶלֶךְ צֹאן וּבָקָר וַעֲבָדִים וּשְׁפָחֹת וַיִּתֵּן לְאַבְרָהָם וַיָּשֶׁב לוֹ אֵת שָׂרָה אִשְׁתּוֹ׃
    \versenum{15}
    וַיֹּאמֶר אֲבִימֶלֶךְ הִנֵּה אַרְצִי לְפָנֶיךָ בַּטּוֹב בְּעֵינֶיךָ שֵׁב׃
    \versenum{16}
    וּלְשָׂרָה אָמַר הִנֵּה נָתַתִּי אֶלֶף כֶּסֶף לְאָחִיךְ הִנֵּה הוּא־לָךְ כְּסוּת עֵינַיִם לְכֹל אֲשֶׁר אִתָּךְ וְאֵת כֹּל וְנֹכָחַת׃
    \versenum{17}
    וַיִּתְפַּלֵּל אַבְרָהָם אֶל־הָאֱלֹהִים וַיִּרְפָּא אֱלֹהִים אֶת־אֲבִימֶלֶךְ וְאֶת־אִשְׁתּוֹ וְאַמְהֹתָיו וַיֵּלֵדוּ׃
    \versenum{18}
    כִּי־עָצֹר עָצַר יְהוָה בְּעַד כָּל־רֶחֶם לְבֵית אֲבִימֶלֶךְ עַל־דְּבַר שָׂרָה אֵשֶׁת אַבְרָהָם׃  
\end{hebrewtext}

\begin{translation}
    \versenum{Gen 20:14}
    Then Abimelech took sheep and cattle, and male and female slaves, and gave them to Abraham, and restored his wife Sarah to him.
    \versenum{15}
    Abimelech said, ``See, my land is before you. Settle where it seems good to you.'' 
    \versenum{16}
    But to Sarah he said, ``Look, I have given your brother a thousand pieces of silver; it is your exoneration before all who are with you; you are completely vindicated.'' 
    \versenum{17}
    Then Abraham prayed to God; and God healed Abimelech, and also healed his wife and female slaves so that they bore children. 
    \versenum{18}
    For \yahweh had completely closed all the wombs of the house of Abimelech because of the word of Sarah, Abraham's wife. (My~translation)
\end{translation}

While the Gen 12 account is very terse, the Gen 20 account includes a dream-revelation (20:6--7), specifies that the plagues that afflicted the monarch impeded his sexual activities (specifically with Sarah), and describes Abraham praying over Abimelech and his household to heal them. Similar details are given in the \ga's account which likewise includes a dream-revelation, notes that the plague were sexual in nature, and describes Abram praying over Pharoah and his household for healing. 

However, while these similarities may indeed represent some kind of literary conflation between the two accounts,%
%
\footnote{From a memory perspective I would prefer to account for the \ga's adoption of certain details from Gen 20 in more passive terms where the specifics of the Gen 12 story are, where absent, supplied from another well-known, typologically similar, source. It is also, perhaps, worth noting that this doublet has traditionally been attributed source critically to different hands (Gen 12 = J; Gen 20 = E) See \cite[15]{driver1956} .} 
%
at the level of genre and structure, conflation with Gen 20 cannot account for the \ga's reframing as a court-contest. For example. the dream-revelation in Gen 20 is given to Abimelech, rather than to Abram as in \ga. Moreover although Abraham prays for healing for Abimelech and his household in a very similar fashion to the way he is portrayed in \ga praying for Pharoah, in Gen 20, he does so only after Abimelech effectively ``pays him off.'' It is the revelation given to Abimelech in a dream which causes him to ``repent'' in Gen 20, while in \ga, the miraculous healing of Pharoah and his household functions as the sign and catalyst for Pharoah's rich rewarding of Abram. Although this difference may seem subtle, the primary feature of the court-contest is the demonstration of God's power through the protagonist which leads to the foreign king's repentance/conversion and the rationale for his rewarding of the protagonist. In other words, while it may have been that the \ga used details from Gen 20 to supplement the account from Gen 12, \ga's framing of Abram's contest with Pharoah cannot be solely attributed to a harmonization of the Gen 12/20 doublet. Thus, drawing on details from, or perhaps just inspired by, the Abimelech doublet in Gen 20, the author of \ga was able to reframe this portion of the Abram narrative to conform to the common court-contest pattern, which, as Jurgens rightly notes, surely would have been an effective and entertaining adaptation by comparison to the account from Genesis.

\subsection{Other Literary features and Motifs}
A number of other generic and literary motifs which diverge from the Genesis account, but which are at home in the \secondtemple period can be identified in this portion of the \ga as well.

\subsubsection{Abram as Oracle}
Although the Abimelech story in Gen 20 includes a dream-revelation, it is noteworthy that in \ga, Abram himself is given the dream as a means of warning him about how the Egyptians would attempt to kill him on account of Sarai's beauty. Where the biblical text credits Abram's intuition for anticipating the Egyptians' desire for Sarai (though, we are left to wonder whether he would have been killed had the ruse not been realized), the \ga describes Abram receiving a portentous dream vision characteristic of other \secondtemple literature.%
%
\footnote{\cite{gevirtz_maarav1992}; \cite[184]{fitzmyer2004}; \cite{dacy_tzoref2013}}
%

Although dream-visions are not unique to the \secondtemple period, their ubiquity within Jewish literature from the \secondtemple period is indisputable. In his treatment of the Dream-Visions among the Aramaic \dss, Andrew Perrin describes both Abram and Noah as being ``recast as a dreamer[s]'' within the \ga.\footnote{\cite[52--57]{perrin2015}. See also \cite{eshel_klostergaard-etal2009} and \cite{machiela_falk-etal2010}.} While Noah is not described as a dreamer within the biblical text, within the \ga, he seems to have been the recipient of as many as five dream-visions.\footnote{\cite[53]{perrin2015}. Elsewhere in the Enochic literature dreams weight heavily in the events surrounding the flood, if not always given to Noah (\firstenoch, Book of Giants, etc.).} Restricting the discussion to Abram, however, Perrin suggests that the insertion of a dream-vision into the story on the eve of Abram and Sarai's descent into Egypt functioned as part of a larger project to ``extend Abram's prophetic credentials in light of Gen 20:7.''\autocite[55]{perrin2015} 

Fitzmyer notes that the component parts of this dream---``cedar'' (Aramaic: \aram{ארז}) and ``date-palm'' (Aramaic: \aram{תמרא})---are derived from Ps 92, which declares ``the righteous will flourish like the date palm (Hebrew: \he{כַּתָּמָר}); like a cedar (Hebrew: \he{כְּאֶרֶז}) in Lebanon he will grow'' (Ps 92:13). The identification of Abram and Sarai with the cedar and date-palm, respectively, is plain enough by the parallel to what happens next in the narrative. The interpretation is supported by the grammatical gender of the terms \aram{ארז} (masc.) and \aram{תמרא} (fem.) which correspond to the gender of the characters. Although grammatical gender does not correspond to natural gender, the identification of Sarai with the date-palm is supported further by the use of ``Tamar'' as a feminine personal name within the Bible (Gen 38:6; 2 Sam 13:1; 14:27).

The dream itself provides an allegorical vision that credits the date-palm (Sarai) with saving the cedar (Abram) from the people seeking to destroy it. Although there is a question whether the beginning of \col{19}, 14--15 should read \aram{והא אזר חד ותמרא כחדא חמחו מן שרש חד} ``a cedar and a date-palm \emph{growing from a single root}'' (so DJD), or \aram{ והא אזר חד ותמרא חדא יאיא שגי} ``a cedar and a date-palm [which was] \emph{very beautiful}'' (so, Fitzmyer), all editions understand \col{19}, 16 to read \aram{ארי תרינא מן שרש חד זמחנא} ``for the two of us grow from a single root.'' Thus, the purpose of the dream  is to show Abram the way that he should avoid being ``cut down and uprooted'' by the Egyptians, namely, by claiming that he and Sarai ``sprung from the same root,'' viz. are siblings. This interpretation is also offered by Abram himself in \col{19}, 19--21.

Significantly, the later \GenRabbah connects the this section of Genesis with Ps 92 and utilizes the cedar/date-palm imagery as well, albeit in a different manner. Specifically, during its treatment of the description of the plagues which God inflicted on Pharoah, \GenRabbah begins with a citation of Ps 92, ``The righteous will flourish like the date-palm (Hebrew: \he{תמר}); like a cedar (Hebrew: \he{ארז}) in Lebanon he will grow'' and this comparison to date-palms and cedars makes several digressions. First, building on the idea of righteousness, \GenRabbah observes that both cedars and date-palms are ``straight'' trees, largely without crooks and crotches.\footnote{Although the Psalm uses the typical term for ``righteousness'' (Hebrew: \he{צַדִיק}), another common biblical term for a person who acts in an upright manner is ``straight'' (Hebrew: \he{יָשָׁר}). The author of \GenRabbah seems to be playing off of this association. Furthermore, according to \GenRabbah, tall trees cast long shadows; the length of these shadows represent the fact that the reward for such righteous people will only come later.} The second digression focuses on the ability of date-palms to produce fruit (including through grafting) and the usefulness of, especially date-palms for all manner of practical concerns. \GenRabbah then extends the comparison to the whole of Israel:

\begin{quote}
As no part of the palm has any waste, the dates being eaten, the branches used for Hallel, the twigs for covering [booths], and bast for ropes, the leaves for besoms, and the planed boards for ceiling rooms, so are there none worthless in Israel, some being versed in Scripture, some in Mishnah, some in Talmud, others Haggadah. (\genrab{41.1})
\end{quote}

The final comparison makes the claim that, like dealing with Israel, climbing these tall trees is perilous. The proof, for \GenRabbah, brings us back to the verse at hand. That Pharaoh was plagued by Yahweh when he took Sarai for himself demonstrates the danger in engaging with Israel as an adversary. What is significant here is that the authors of both \ga and \GenRabbah connect Ps 92 with this section of Genesis, but importantly, they do so with different sets of interpretive ``evidence.'' In other words, the two texts agree on the \emph{fact} of the connection between Ps 92 and the story in Gen 12, but they arrive at their interpretations through different means.
%
\footnote{%
\begin{SingleSpace}Perhaps based on the Psalm's later reference to bearing children in one's old age:
\begin{quote}
    Planted in the house of Yahweh; they will flourish in the courts of our God\\
    They will still bring forth fruit in old age; they will be full of sap and green\\
    (Ps 92:14--15)
\end{quote}
\end{SingleSpace}}
%
Yet, the fact that Ps 92 and Gen 12 are explicitly connected in both \ga and \GenRabbah seems more than a coincidence. Thus, while \ga's use of the cedar/date-palm imagery may rely on some previous tradition, the dream revelation itself is best understood as an example of the author of \ga utilizing the literary tropes of his own time and place. 

% \subsubsection{Abram as Sage}
% [TODO: A Brief discussion of Abram being depicted as a sage (sought for knowledge; reads from the ``book of the words of Enoch,'' etc)]

% \subsubsection{A \emph{waṣf} about Sarai}
% [TODO: Talk about Waṣf]

\subsection{Conclusions}

The recasting of Abram's sojourn as ``diaspora,'' his conflict with Pharaoh as a court-contest along with the portrayal of Abram and Noah as dreamers can be understood in terms of social memory as the author of \ga pressing the stories of Genesis into existing literary genres. Insofar as ``genres'' can be understood as commonly understood literary conventions---a social ``contract'' of expectations between the author and her audience---they are socially defined and, for our purposes, function as what \halbwachs would call ``social frameworks.'' As Abram's sojourn in Egypt could take on new meanings within the context and social framework of diaspora Judaism during the \secondtemple period, so too the common trope of the court-contest---well-known from the book of Daniel---provided a new framework into which the story of Gen 12 could be read. Thus, Jurgens's basic premise---that these stories are ``updated'' for a new audience---takes for granted what the memory approach makes explicit: \secondtemple Jews had their own ways of thinking about the way that God interacted with the ancients, and how pious Jews acted in particular circumstances. These social frameworks provided new structures for understanding the stories that they inherited from the biblical tradition. Thus, rather than only thinking about how the author was trying to ``fix'' the biblical account, from the memory perspective we can imagine the author of \ga not only interpreting the biblical tradition, but making efforts to contextualize it within his own literary frame of reference.