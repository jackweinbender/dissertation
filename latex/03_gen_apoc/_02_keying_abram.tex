% !TeX root = ../dissertation.tex

\section{Abram in the Diaspora: The Literary Frameworks of \GA}

Having dealt with the \ga as the product of cultural memory in terms of its relationship to its inherited biblical memory (including the traditions which were ancillary to Genesis proper), we may now turn our attention to the ways that the \ga addressed its audience at the level of \emph{social} memory. In this section I will address the way that the \ga \emph{speaks to} its audience and the ways that the \ga changes and adapts its cultural memory into a meaningful piece of literature for \secondtemple Judaism.

As I have already noted, the narrative of the \ga is not simply a straight-forward retelling of Genesis from the perspectives of Lamech, Noah, and Abram, but participates more broadly in the ``biblical memory'' of Genesis. However, what is most compelling about \rwb texts very often is the \emph{ways} that they adapt biblical memory. These adaptations can come at the level of story---by adding, removing, or rearranging events---or at the level of narrative discourse by describing events differently or with different emphases. In the case of \ga, and in particular in the account of Abram in cols. 19--22, the biblical narrative has been recast as a (first-person) Hellenistic novella in a similar vein to other well-known Second Temple Jewish works such as the narrative portions of Daniel (including the Greek additions), Esther, Tobit, and (arguably) the so-called Joseph novella of Genesis 37 and 39--50.\footnote{See especially Lawrence Wills work on the Jewish novels and novellas in antiquity: \cite*{wills2002} as well as his important earlier works \cite*{wills1995} and \cite{wills1990}.}

The reading of \ga 19--20 as a Hellenistic Jewish novella has recently been very thoroughly explicated by Blake Jurgens, who has further argued that the utilization of Hellenistic literary motifs and structures in \ga altered the overall purpose the the pericope for the purpose of edifying Jews living in the Hellenistic world in the shadow of empire.\autocite{jurgens_jsj2018} Although much of Jurgen's paper is based on long-established observations about the literary influences on \ga, he makes the important discursive turn toward the audience by claiming that the \ga was meant to be useful to readers:

\begin{quote}
By imbuing its story with literary tropes and techniques similar to those found in Dan 1--6, Esther, and other Jewish texts arising out of the Hellenistic period, the author successfully attends to the narratival ambiguities of Gen 12:10--20 through interpretive expansion upon the latent exegetical links of the text while concurrently modifying the narrative to appeal to contemporary literary expectations.\autocite[27]{jurgens_jsj2018} \end{quote}

Thinking in terms of social memory, however, we can appreciate the way that the stories that the \ga retells are ``remembered into'' the social context of Hellenistic Judaism and are fitted into the contemporary social frameworks by the utilization of common literary techniques. In other words, the changes that Jurgens identifies as authorial decisions intended to engage with readers can also be framed as \emph{determined by} the social location of the author and the literary tools available to him.


\subsection{Abram in the Diaspora}

One of the primary features of Jewish hellenistic novellas is their setting. Jurgens notes that, typically, these Jewish novellas are set in the diaspora, which invariably place the Jewish (or, in Tobit and Judith's case, Israelite) protagonist under the hegemony of a foreign power. In the case of \ga, although not properly ``diaspora,'' Abram is a sojourner in a foreign land and is under foreign hegemony. Moreover, from a modern perspective, these stories have a tendency to commit rather egregious factual errors about certain historical particulars such as the names of rulers (Judith 1:1; Dan 4; Tobit) and geographic items (Tobit 5:6). Likewise, \ga seems to utilize details which almost certainly were inventions of the author (or an earlier tradant) such as referring to ``Pharaoh Zoan'' (we know of no such figure) and Herqanos, a name popular in the Ptolemaic period, but not attested otherwise as well as referring to the ``Karmon River'' (probably the Kharma canal), as the one of the seven heads of the Nile river, which it is not.\autocites[7]{jurgens_jsj2018}[See also][50--59]{machiela_as2010}[197--199]{fitzmyer2004} These details, according to Jurgens, are meant to create a sense of verisimilitude and authenticity within the narrative. Thus, although the story of Abram's sojourn in Egypt as narrated in the biblical text engages with discourses of the \emph{foundation} of Israel, the narrative of the \ga seems to be turning the story to engage with the contemporary discourses around the idea of \emph{diaspora}. In other words the way that Abram's sojourn in Egypt was remembered in the \secondtemple period, at least in part, took on new meaning for those sojourning in the diaspora and for those living in the land under foreign hegemony.

\subsection{Abram in the Court of a Foreign King: Literary Genre as Social Framework}

If we place the pericope of Abram's journey into Egypt in \ga under the rubric of diaspora literature, the final scene in the pericope bears a striking resemblance to the so called court contest narratives well-known from (especially) the book of Daniel.\footnote{Other court contest narratives include the Joseph Cycle (Gen 41)} Such narratives, as observed by Collins and others, follow particular narrative progressions with features common .\footnote{\cite[TODO: pages]{collins1993}; \cite{humphreys_jbl1973}; \cite{collins_jbl1975}; \cite{wills1990}. See also \cite{niditch-doran_jbl1977}.} Jurgens has convincingly shown that the \ga's retelling of Abram's sojourn in Egypt fits such a progression by comparing this pericope to to Dan 2, 4, and 5 as well as Gen 41. Although based on the earlier work of Collins and Humprheys, Jurgens offers his own outline, which can be summaried as follows\autocite[21]{jurgens_jsj2018}:

\begin{itemize}
    \item The foreign king has a problem that he is unable to solve.
    \item The king's own personnel are charged with solving the problem
    \item The king's personnel are unable to solve the problem
    \item The Jewish protagonist is asked to solve the problem
    \item The Jewish protagonist is able to solve the problem
    \item The Jewish protagonist is rewarded by the king
\end{itemize}

It is easy to imagine how the author of \ga would conceive of Abram's interaction with Pharoah in Gen 12 as analogous to other well-known court contests from Israel's biblical memory. The biblical account, however, offers a rather anemic description of the events, but leaves open the specifics of how Pharoah came to know about Abram and how the monarch was relieved from them. 

From an innerbiblical perspective, the \ga's description of Abram and Phraoah's interaction might be thought of as a synthesis or exegetical harmonization with the Abimelek doublet in Gen 20, which offers a much more detailed account of the Abimelech's confrontation with Abra(ha)m. While the Gen 12 account is very terse, the Gen 20 account includes a dream-revelation, specifies that the plagues that afflicted the monarch impeded his sexual activities (specifically with Sarah), and describes Abraham praying over Abimelech and his household to heal them. Similar details are given in the \ga's account which likewise includes a dream-revelation, notes that the plague were sexual in nature, and describes Abram praying over Pharoah and his household for healing. 

However, while these similarities may indeed represent some kind of literary conflation between the two accounts,%
%
\footnote{From a memory perspective I would prefer to account for the \ga's adoption of certain details from Gen 20 in more passive terms where the specifics of the Gen 12 story are, where absent, supplied from another well-known, typologically similar, source.} 
%
at the level of genre and structure, conflation with Gen 20 cannot account for the \ga's reframing as a court-contest. For example. the dream-revelation in Gen 20 is given to Abimelech, rather than to Abram as in \ga. Moreover although Abraham prays for healing for Abimelech and his household in a very similar fashion to the way he is portrayed in \ga praying for Pharoah, in Gen 20, he does so after Abimelech effectively buys him off. It is the revelation given to Abimelech in a dream which causes him to ``repent'' in Gen 20, while in \ga, the miraculous healing of Pharoah and his household functions as the sign and catalyst for Pharoah's rich rewarding of Abram. Although this small difference may seem subtle, the primary feature of the court-contest is the demonstration of God's power through the protagonist which leads to the foreignh king's repentance/conversion and the rationale for his rewarding of the protagonist. In other words, while it amy have been that the \ga used details from Gen 20 to supplement the account from Gen 12, \ga's framing of Abram's contest with Pharoah cannot be solely attributed to a harmonization of the Gen 12/20 doublet. Thus, drawing on details from, or perhaps just inspired by, the Abimelech doublet in Gen 20  the author of \ga was able to reframe this portion of the Abram narrative to conform to the common court-contest pattern, which, as Jurgens rightly notes, surely would have been an effectve and entertaining adaptation by comparison to the account from Genesis.

\subsection{Abram the Oracle}



% [TODO: A Brief discussion of Abram being depicted as a dreamer/oracle who recieves the dream from God about his coming troubles]

\subsection{Abram the Sage}
% [TODO: A Brief discussion of Abram being depicted as a sage (sought for knowledge; reads from the ``book of the words of Enoch,'' etc)]
\subsection{[Section] Conclusions}

Jurgens' basic premise is that these stories are ``updated'' for its audience. Thinking in terms of memory, we can account for this ``updating'' in terms of category mismatches: second temple Jews had their own structures and ways of thinking about the way that God interacted with the ancients, and how pious Jews acted in particular circumstances. These social frameworks provided the structure for understanding the stories that they inherited from the biblical tradition. Thus, rather than thinking about how the author was trying to ``fix'' the biblical account, from the memory perspective we should imagine the author interpreting the biblical tradition and fitting it into his own social and literary mileu.