Introduction:

% Describe the basic pupose of the chapter
% Provide a roadmap

\section{The Genesis Apocryphon}

As one of the first seven scrolls discovered in the judean desert beginning in 1947, the \ga is one of the more well-studied works among the Dead Sea Scrolls. When the scroll was initially analyzed by scholars, it could not be fully unrolled and only a small portion of the outer layer of the scroll could be read. These visible portions, however, written in Aramaic, referenced the ante-deluvian Lamech, the father of Noah, and his wife, Batenosh, known from the book of Jubilees. The text appeared to be written in the first-person from the perspective of Lamech leading Trevor to conclude that the scroll was a copy of the so-called ``Book of Lamech" listed as an apocryphal work by a 7th century CE Greek canon list.\footnote{This fact led Trevor to refer to the scroll as the ``Ain Feshkha Lamech Scroll'' and J.T. Milik to refer to it as the Apocalypse of Lamech for the publication of the fragment in DJD 1. See \cite[9--10]{trevor_basor1949} and ``Apocalypse de Lamech" in \cite[86--87]{djd_1}} Once the scroll was completely unrolled, however, it became obvious that the scope of the scroll contained more than just a first-person account from Lamech and instead contained additional first-person accounts from figures found in the Genesis stories including Noah and Abram. Thus, the more descriptive title, \citetitle{avigad-yadin1956}, was given to the scroll by Avigad and Yadin in 1956 for the publication of its \emph{editio princeps} in 1956.

\footnote{Hebrew ...}

    %  See \cite{avigad-yadin1956}. While the name \ga remains in wide use, it is notable that the name has been criticized and a number of, perhaps more descriptive, titles have been suggested: ``Book of the Patriarchs" (Mazar), ``Memoirs of the Patriarchs" (Gaster), \heb{ספר אבות} (Flusser), and \aram{כתב אבהן} (Milik; note also Fitzmyer's emended \aram{כתב אבהתא}).


Although much of the scroll was very badly damaged, illegible, or missing, 

% Describe the scroll and its significance.

% Brief history of scholarship
% Issues outstanding
% Refer to problems in classification going back to the RwB Chapter


This chapter has three parts. 

First, I will describe the ways that the \ga is typically treated with respect to the Bible and how this approach, fits into the paradigm of "memory."

Second, I will describe recent work on the \ga with respect to literary approaches to the work and how these, too, can be incorperated into a holistic approach to \ga

Finally, I will discuss \ga as an exmple of \psy and argue that this to fits into a memory paradigm.

The advantage that Memory Studies has, in this case is that it offers a way to talk about the manifold ways that \ga both builds from its social location and speaks back into it at a number of discursive levels and into a number of discursive spaces.