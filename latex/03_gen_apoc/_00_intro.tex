Introduction:

% Describe the basic pupose of the chapter
% Provide a roadmap

\section{The Genesis Apocryphon}
% Describe the scroll and its significance.

% Brief history of scholarship
% Issues outstanding
% Refer to problems in classification going back to the RwB Chapter


This chapter has three parts. 

First, I will describe the ways that the \ga is typically treated with respect to the Bible and how this approach, fits into the paradigm of "memory."

Second, I will describe recent work on the \ga with respect to literary approaches to the work and how these, too, can be incorperated into a holistic approach to \ga

Finally, I will discuss \ga as an exmple of \psy and argue that this to fits into a memory paradigm.

The advantage that Memory Studies has, in this case is that it offers a way to talk about the manifold ways that \ga both builds from its social location and speaks back into it at a number of discursive levels and into a number of discursive spaces.