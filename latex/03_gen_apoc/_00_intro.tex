% !TeX root = ../dissertation.tex

%% TODO: Remove most subheads
%% TODO: Include primary texts and translations
%% TODO: Be consistent with GA/Jubilees/Enoch ordering

Since its initial discovery and publication, the Aramaic text known as the \ga (\q{1}{apGen ar}{}) has been associated in various ways with the book of Genesis. As one of the first seven scrolls discovered in the Judean desert beginning in 1947, the \ga is also one of the more well-studied works among the Dead Sea Scrolls. When the scroll was initially analyzed by scholars, it could not be fully unrolled and only a small portion of the outer layer of the scroll could be read. These visible portions, written in Aramaic, referenced the antedeluvian Lamech, the father of Noah, and his wife, Bitenosh, known from the book of \jub. The text appeared to be written in the first-person from the perspective of Lamech leading Trevor to conclude that the scroll was a copy of the so-called ``Book of Lamech'' listed as an apocryphal work by a 7th century CE Greek canon list.\footnote{This fact led Trevor to refer to the scroll as the ``Ain Feshkha Lamech Scroll'' and Milik to refer to it as the ``Apocalypse of Lamech'' for the publication of the fragment in DJD 1. See \cite[9--10]{trevor_basor1949} and ``Apocalypse de Lamech'' in \cite[86--87]{djd_1}} Once the scroll was completely unrolled, it became obvious that the scroll contained more than just a first-person account from Lamech and instead contained additional first-person accounts from figures found in the Genesis stories including Noah and Abram. Thus, the more descriptive title, \citetitle{avigad-yadin1956}, was given to the scroll by Avigad and Yadin for the publication of its \emph{editio princeps} in 1956.\footnote{Hebrew: \he{מגילה חיצונית לבראשית}. See \cite{avigad-yadin1956}.} While the name \ga remains in wide use, it is notable that the name has been criticized and a number of alternative titles have been suggested; most notably: ``Book of the Patriarchs''\autocite[Hebrew: \he{ספר אבות}. As suggested by Mazar in][379 n. 2]{flusser_ks1956}, ``Memoirs of the Patriarchs''\autocite[358]{gaster1976}, and \aram{כתב אבהן}\footnote{\cite[14 n. 1.]{milik1959}.  Fitzmyer suggests \aram{כתב אבהתא} would be, perhaps, even more suitable. See \cite[16]{fitzmyer2004}.} In this chapter, I will retain the traditional title, \ga.

Although much of the scroll was very badly damaged, illegible, or missing, enough survived for Avigad and Yadin to make the generalized observations that \ga followed the basic order and events of Genesis from the Flood (Gen 6) into the Abram narrative (ending in Gen 15). The events are generally (though, not exclusively) narrated in a series of first person accounts---what I will refer to as ``memoirs''%
%
\footnote{I will use the term ``memoir'' throughout this chapter as a way of referring to the distinct (mostly) first-person narratives found in the \ga. The term is meant to highlight the formal characteristic of being written in the first person voice without any reference to the authenticity of the work and in alignment with the convention of referring to first-person narratives in the Bible as ``memoirs'' (e.g., the ``Nehemiah Memoir'' or the ``Isaiah Memoir'').}%
---by Lamech, Noah, and Abram, respectively and show a clear affinity with the roughly contemporaneous works of \firstenoch and \jub.\autocite[16--37]{avigad-yadin1956} The literary relationship of \ga to both \firstenoch and (especially) \jub remains a matter of debate, with Avigad and Yadin suggesting that \ga more probably preceded \jub, while the recent prevailing opinion seems to prefer the opposite.%
%
\footnote{\cite[38]{avigad-yadin1956}; cf. \cite[20--21]{fitzmyer2004}. Fitzmyer cites Hartman's suggestion, building on Fitzmyer's own work, treating the similarity between \ga's and Jubilee's chronology of Abram's life. Because the chronology seems to have been closely tied to Jubilee's uniquely structured calendar, it follows that \ga drew from \jub. See \cite[497]{hartman_cbq1966}.}

The name given to the \ga in the \emph{editio princeps} set the agenda for scholarly inquiry on the work into the modern era by connecting it to the biblical book of Genesis while simultaneously categorizing it as apocryphal. Much of the attention given to the \ga, therefore, has focused on its literary genre and its relationship to the Bible and resemblance of the Targums and later midrashic works. As already noted, \vermes's treatment of \ga focused on the role that it played in showing the continuity between the interpretation of Jewish scripture during the \secondtemple period and the aggadic traditions of early rabbinic Judaism. In \emph{Scripture and Tradition}, \vermes treats in detail the relationship between Gen 12:8--15:4 and \ga \cols{19}{22}, ultimately declaring \ga to be ``the most ancient midrash of all'' and the ``lost link between the biblical and the Rabbinic midrash.''\autocite[124]{vermes1961} The result of this framing (whether one considers it appropriate or not) has been that much of the scholarly attention paid to \ga has focused on its relationship to Genesis and especially how its author(s) may have been addressing exegetical issues found within the (later) biblical work. Yet, as Fitzmyer observes, the roots of biblical midrash are now generally accepted to be found within the Hebrew Bible itself.\autocite[20]{fitzmyer2004} Together with the fact that a number of Targums have been found at Qumran makes the presence of targumic and midrashic qualities in \ga less remarkable and frees us from any obligation to try and fit it cleanly within either category.

Although few scholars insist on rigidly defining \ga as either targum or midrash, the treatment of \ga as primarily \emph{exegetical} tacitly implies that the purpose of \ga was to explain or interpret Genesis. Put another way, the discussions surrounding \ga are often preoccupied with gleaning information about how \secondtemple Jews read \emph{Genesis}---treating \ga from the perspective of ``biblical interpretation.'' While there is no question that such an approach has been fruitful, treating \ga as \emph{only} or even \emph{primarily} an example of biblical interpretation, I think, cannot offer a comprehensive reading of the work. In this chapter, therefore, I will focus on the ways that the author of \ga engages with a  constellation of discourses surrounding events and characters \emph{known from} the book of Genesis, as well as those from other texts not as biblical \emph{interpretation}, but as processes of \emph{memory}.%
%
\footnote{What remains uncertain about the \ga is what its function may have been for its original audience. I am in agreement with Fitzmyer that it seems unlikely that \ga would have been used liturgically and that the general character of the work is ``for a pious and edifying purpose.''\textcite[20]{fitzmyer2004}. Yet, I can not help but feel somewhat dissatisfied with this answer. How might \ga have edified its readers? Works such as \jub and \firstenoch, perhaps, have more obvious rhetorical aims, but for all its similarities to these texts, \ga maintains a different character which has generally eluded commentators. While I have no illusions that I will be able to offer a satisfactory answer to the question of \ga's specific purpose, approaching \ga as an object of cultural memory, I believe, is a good place to start. The advantage that a memory approach has in addressing this problem is that it offers a way to talk about the manifold ways that \ga both builds from its cultural memory and speaks back into it.}

From the perspective of cultural memory, therefore, \ga operates within a stream of traditions and participates in discourses surrounding early foundational figures in Jewish tradition: Lamech, Noah, and Abram. As such, it is both the recipient and progenitor of cultural memory whose participation in the mnemonic process affected the memory itself. Thus, building on the theoretical framework of chapter two, \ga may be understood to have taken part in three discrete mnemonic processes: 1) the reception of cultural memory, 2) the reshaping of memory by contemporary social frameworks, and 3) the active construction, codification, and reintegration of memory for future transmission. These three distinct processes are observable within the text of \ga. In this chapter, therefore, I have chosen to frame the discussion of \ga around these processes. First, and as a point of departure, I will discuss the ways that the \ga functions as the recipient of cultural memory through its engagement with what I refer to as  ``biblical memory.'' Second, I will discuss the ways that \ga was shaped by the social frameworks which inherited it through a discussion of literary genre and shared formal characteristics with contemporary texts. Finally, I will discuss how \ga participated in the construction of cultural memory through its use of \psgraphical discourse.
