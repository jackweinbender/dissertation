\section{Conclusions}

Although the preceding discussion has dealt with the idea of memory at three distinct levels of discourse and engaged with several discrete modes of analysis, my ultimate goal is to bring those discrete parts into conversation to advance a more comprehensive portrait of \ga through the lens of memory studies.

First, in discussing the way the \ga ``looks backward'' toward its sources and toward what I have called ``biblical memory,'' I argued that \ga drew from much more than just the biblical text. Although the relative chronology of 1 Enoch, Jubilees, and \ga is not secure, what is clear is that the cultural memory that surrounded the book of Genesis---the biblical memory of Genesis---was more broad at the time that the \ga was composed than simply the text of Genesis. The cultural memory from which \ga drew included additional traditions adjacent to the text of Genesis that we know from \jub and 1 \enoch whether or not \ga itself drew from the \emph{texts} of \jub and 1 \enoch and vice versa.

But the presentation of these traditions was not a straight-forward synthesis of their content. The author of the \ga utilized generic and thematic elements common to the social location in which it was written. Although the account of Abram's encounter with Pharoah in Gen 12 is a rather anemic narrative, the \ga does not simply fill-out missing details but recasts the final confrontation as a court-contest in the tradition of Daniel and Joseph. Even the Abraham/Abimelech doublet in Gen 20, although a more detailed narrative, can not account for this transformation. Instead, I have proposed that the utilization of the court-contest (as well as the depiction of Abram as a dreamer and his sojourn as diaspora) was a way for the author of the \ga to not only make his narrative entertaining, but to fit it into the extant social frameworks (read: genres) of the late \secondtemple period.

Finally, by participating in \psy, the author of the \ga reentered the discursive space of biblical memory by presenting the text of the \ga as a first person narrative. Although we cannot know specifically how the \ga was received by its audience in antiquity, the fact that it presents as a ``primary source'' for the stories of Genesis was meant as a queue to the reader for how to understand the \ga's claim to authority, whether that claim was minimal (as with a modern film-adaptation where the audience expects certain artistic license) or genuinely intended coopt the authority of its pseudonymous author (as with Paul's disputed letters).

These three levels of discourse map onto the basic premise of memory studies that memory is 1) inherited, 2) reshaped by extant social frameworks, and 3) constructive. As a product of memory, the \ga fits this schema well. Treating \ga simply or even primarily as a way of explaining the book of Genesis does not do justice to the complex and varied processes and traditions that informed the production of \ga nor adequately account for the plurality of purposes for which the \ga could have been intended. Instead, memory studies offers a way to talk about how \ga is able to engage with, draw from, and adapt the received traditions about Genesis and account for the reception of the resulting text and how it could further shape the memory of the society that produced it.