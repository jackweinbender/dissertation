\section{Conclusions}

As I have demonstrated, \ga may be understood to have taken part in three discrete mnemonic processes: 1) the reception of cultural memory, 2) the reshaping of memory by contemporary social frameworks, and 3) the active construction, codification, and reintegration of memory for future transmission. 

First, \ga functions as the recipient of cultural memory through its engagement with what I refer to as  ``biblical memory.'' I argued that \ga drew from more than just the biblical text and instead drew from a whole constellation of traditions and stories surrounding the early figures of Lamech, Noah, and Abram. Although the nature of the relationship(s) between \firstenoch, \jub, and \ga is not well understood, what is clear is that the cultural memory that surrounded the book of Genesis---the biblical memory of Genesis---was more broad at the time that the \ga was composed than simply the text of Genesis. The cultural memory from which \ga drew included additional traditions adjacent to the text of Genesis that we know from \jub and \firstenoch whether or not \ga itself drew from the \emph{texts} of \jub and \firstenoch and vice versa.

% Second, I will discuss the ways that \ga was shaped by the social frameworks which inherited it through a discussion of literary genre and shared formal characteristics with contemporary texts.

But the presentation of these traditions was not a straight-forward synthesis of their content. The author of the \ga utilized generic and thematic elements common to the social location in which it was written. Although the account of Abram's encounter with Pharoah in Gen 12 is a rather anemic narrative, the \ga does not simply fill-out missing details but recasts the final confrontation as a court-contest in the tradition of Daniel and Joseph. Even the Abraham/Abimelech doublet in Gen 20, although a more detailed narrative, cannot account for this transformation. Instead, I have proposed that the utilization of the court-contest (as well as the depiction of Abram as a dreamer and his sojourn as diaspora) was a way for the author of the \ga to not only make his narrative entertaining, but to fit it into the extant social frameworks (read: genres) of the late \secondtemple period.

Finally, I discussed how \ga participated in the construction of cultural memory through its use of \psgraphical discourse. By participating in the genre of \psy, the author of the \ga engaged in the further construction of biblical memory by presenting the text of the \ga as a first person narrative. Although we cannot know specifically how the \ga was received by its audience in antiquity, the fact that it presents as a ``primary source'' for the stories of Genesis (or, more precisely, the stories which participate in the biblical memory of Genesis) was meant as a queue to the reader for how to understand the \ga's claim to authority, whether that claim was minimal (as with a modern film-adaptation where the audience expects certain artistic license) or genuinely intended co-opt the authority of its pseudonymous author (as with Paul's disputed letters).

As a product of memory, the \ga fits this three-fold schema well. Treating \ga simply or even primarily as a way of explaining the book of Genesis does not do justice to the complex and varied processes and traditions that informed the production of \ga nor adequately account for the plurality of purposes for which the \ga could have been intended. Instead, memory studies offers a way to talk about how \ga was able receive, recontextualize, and codify the received traditions about Genesis (remembered) for himself and his contemporaries.

%% Is "Remembered Genesis" a good term?

% TODO: You need to elaborate a little more on this last point.
%% TODO: Note that this approach better deals with GA as a *whole* rather than just cols. 19ff.