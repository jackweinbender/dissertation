% !TeX root = ../dissertation.tex

\chapter{The \rwb}
\label{chap:rwb}

In his seminal work \citetitle*{vermes1961}, \Vermes introduced the term ``\rwB'' into the discussion of \secondtemple Jewish literature as part of a larger project to trace the development of haggadic traditions from the late \secondtemple period into the rabbinic period. \vermes used the term \rwB to describe narrative texts that closely follow biblical narratives and that augment, elide, and emend those narratives in ways which produced new literary works in their own right. According to \vermes, through this exegetical process, ``the midrashist inserts haggadic development into the biblical narrative'' in order to ``anticipate questions, and to solve problems in advance.''%
    \footnote{%
        \Cite[95]{vermes1961}. See also
        \cite{vermes_zsengeller2014}.}
\vermes traced these interpretive traditions historically and attempted to demonstrate an interpretive continuity between the \secondtemple period and nascent rabbinic Judaism and Christianity. Although the formal characteristics of these narratives differed from later midrash, \rwB texts displayed the same kinds of ``midrashic'' tendencies. In \vermes's conception, therefore, the authors of \rwB texts \emph{implicitly} made use of interpretive traditions that later works such as the Talmud and Mishnah expressed \emph{explicitly}. 

Since the publication of \citetitle{vermes1961}, \vermes's concept of \rwB (hereafter ``\rwb'') has taken on a life of its own and developed into a discreet area of study as scholars from related disciplines have reused, reinterpreted, and redefined the term.%
    \footnote{%
        See especially the early discussions in 
        \cite{alexander_carson-williamson1988}; 
        \cite{nickelsburg_stone1984}; and \cite{harrington_kraft-nickelsburg1986}.}
The discussion of \rwb has become especially fruitful within the field of \qumran studies, where new texts from the \secondtemple period continued to be published throughout the late twentieth century, and where new material discoveries continue to this day. However, the idea of biblical rewriting has also been fruitfully applied to texts that have long been known to scholars such as \jub, Deuteronomy, Chronicles, and even the Synoptic Gospels.%
    \footnote{%
        On reading the Gospels as \rwb, see 
        \cite{müller_back-kankaanniemi2012}; 
        \cite{malan_hts2014}; and more recently 
        \cite{allen_jsnt2018}.}

Although the scope and nuance of the term \rwb has shifted in the intervening years, the trajectory set by \vermes nearly sixty years ago has remained reasonably consistent. By focusing on the relationships between \rwb texts and their scriptural \vorlagen, studies on \rwb have tended to discuss the topic primarily through the lens of biblical or scriptural interpretation by focusing on how the authors or editors of \rwb texts retained, emended, or excised material \emph{from the biblical text}. While these treatments are often very good, this preoccupation with the ``biblical'' text (or a particular ``scriptural'' text, using the more common terminology) has impeded the study of these texts as participants in a broader cultural discourse that extends beyond ``biblical interpretation.''

In this chapter I will trace the emergence and evolution of the concept of \rwb from \vermes's use in \citetitle{vermes1961} to the present focusing on three key questions and ideas which have shaped the scholarly discourse around \rwb studies: 1) the terminology surrounding \rwb, 2) what works should fall under the rubric of \rwb and 3) whether \rwb constitutes a literary genre, a process, or some combination of the two. Taking these considerations into account, I will then offer my own suggestions on how the \rwb conversation can be augmented by the treatment of \rwb texts as participants in a broader cultural discourse through the lens of cultural and social memory studies.

\section{Scripture and Tradition}

The primary purpose of \citetitle{vermes1961} was not to offer a clear definition of the term ``\rwb,'' but to lay the groundwork for the historical, diachronic study of aggadic traditions, of which \rwb makes only a small part.%
    \footnote{%
        \vermes specifically was concerned with \emph{narrative} texts and did not consider legal works as a part of the category \rwb. See 
        \cite[3]{vermes_zsengeller2014}.}
As \vermes recounts, prior to the mid-twentieth century, the prevailing approach to the study of aggadic exegesis was to treat the aggadah as originating during the Tannaitic period. The aggadah were viewed as ``the result of the adoption, and anonymous repetition, of popular interpretations by favourite preachers,''%
    \autocite[3]{vermes1961}
the earliest of which were from the \ce{second century} and were represented by targums Onkelos and Jonathan. Furthermore, studies of ancient Jewish literature at this time focused on texts which modern Judaism considered authentic. As a result, a good number of earlier texts---for example, the apocrypha, pseudepigrapha, and sectarian texts---were often categorically excluded from discussions of the origins of aggadic exegesis.%
    \autocite[2]{vermes1961} 

A series of publications and discoveries beginning in the 1930's, however, began to undermine the notion that these early exegetical traditions began in the \ce{second century}. \vermes credits this broadening of aggadic studies to a series of major studies and discoveries such as Rappaport's \emph{Agada und Exegese bei Flavius Josephus},%
    \autocite{rappaport1930}
Paul Kahle's Schweich Lectures at the British Academy on the Cairo Geniza (given in 1941, published 1947),% 
    \autocite{kahle1947}
Kisch's new text edition of Ps. Philo's \lab (1949),%   
    \autocite{kisch1949}
the discovery of the Dead Sea Scrolls (1948) and Codex Neofiti (1956), as well as (and perhaps especially) Renée Bloch's work on midrash.%
    \footnote{%
        \Cite{bloch1954};
        \cite{bloch1955_repr};
        \cite[3--7]{vermes1961}.}
The overarching theme among these works was the evidence for continuity between biblical interpretive traditions prior to the \ce{second century} and later aggadah. For example, \vermes notes that Rappaport's work on \antiquities identified substantial overlaps between Josephus's text and rabbinic aggadah and suggested, therefore, that Josephus had drawn from an already living tradition of interpretation. His suggestion implies that the aggadah of the \ce{second century} were not novel exegetical works, but were themselves products of earlier exegetical traditions. 

Building on these recent advancements, the explicitly stated purpose of \emph{Scripture and Tradition} was to push the field beyond synchronic analysis of aggadah toward diachronic, historical analyses to trace the development of these exegetical traditions%
    \footnote{%
        \Cite[1]{vermes1961}. See also
        \cite{bloch1955_repr}.}
The book is eight chapters long and is divided into four parts. 

The first section of \citetitle{vermes1961}, entitled ``The Symbolism of Words,'' is composed of three chapters which attempt to explain some of the processes by which localized symbolic interpretations of particular texts were able to affect the interpretation of other, nominally related texts.
%
In the first chapter, \vermes notes the divergent treatment of Gen~44:18--19 among ancient commentators through a synoptic study of this passage as represented in the Fragmentary Targum, Targum Neofiti, and the Tosefta of Targum Yerushalmi. He argues for a relative chronology based on their use of shared interpretive traditions and concludes that the Fragmentary Targum represents the most primitive work, whose interpretive strategy is essentially inner-biblical. He then argues that the Tosefta of Targum Yerushalmi depends on the Fragmentary Targum, but offers a distinct interpretive stance and, further, that the Targum Neofiti represents a later combination of these two traditions.
%
In his second chapter, \vermes examines the symbolic use of the term ``Lebanon'' in the Hebrew Bible and other Jewish literature as a reference to Jerusalem, \thetemple, and how those symbolic meanings came to be. He identifies the Song of Songs as the intermediary text which helped to establish this tradition within post-exilic Judaism and suggests that it occupies a unique position as the only biblical text which clearly uses the name Lebanon symbolically for \thetemple. Importantly, \vermes shows that this symbolic use of Lebanon is rooted in \emph{biblical} exegesis. This is a key idea for \vermes because it establishes a continuity between the production of the biblical text and its later interpretation.
%
In Chapter Three, \vermes examines other words which, like ``Lebanon,'' take on symbolic meaning in later Jewish texts---``lion,'' ``Damascus,'' ``\emph{Meḥoqeq},'' and ``Man''---and attempts to show that a similar process took place among the \dss texts as well as the targumic and midrashic materials. 

It is in the second part of \emph{Scripture and Tradition}---entitled ``The \rwb''---that the topic of \rwb is first addressed directly. The section is composed of two chapters (four and five), both of which focus on the figure Abraham and the aggadic traditions surrounding his life. The purpose of these two chapters is to demonstrate a continuity of interpretive traditions from the late \secondtemple period through to the early rabbinic period and beyond. 

\vermes approaches the topic from both ends of the chronological spectrum in what he refers to as ``retrogressive'' and ``progressive'' historical studies. In Chapter Four, \vermes embarks on what he calls a ``retrogressive historical study,'' by which he means beginning with later, more developed traditions and working back toward their origins. In this case, \vermes begins with the \ce{eleventh-century} text \emph{Sefer ha-Yashar}, working backward to identify sections of the text which exhibit earlier traditions, most notably those in the targums, Josephus, \jub, and (Ps.) Philo. This chapter demonstrates that even late texts can contain valuable information about earlier methods of exegesis. As \vermes puts it, ``[Sefer ha-Yashar] manifests a direct continuity with the corresponding tradition of the time of the \secondtemple, but reflects also the influence of the haggadah of the Tannaim and Amoraim.''%
    \autocite[95]{vermes1961}
On the other hand, in Chapter Five, \vermes proceeds with a ``progressive historical study,'' beginning with the oldest materials and working forward. Still focusing on the figure Abraham, \vermes treats in detail the relationship between Gen 12:8--15:4 and \cols{19}{22} of \ga. Notably, \vermes treats \ga as ``the most ancient midrash of all,''%
    \autocite[124]{vermes1961}
and rather dramatically declares it to be the ``lost link between the biblical and the rabbinic midrash.''%
    \autocite[124]{vermes1961}
For \vermes, \ga occupies a unique position just one step removed from inner-biblical exegesis. Accordingly, \vermes believed that the author of \ga was attempting ``to make the biblical story more attractive, more real, more edifying, and above all more intelligible'' and to ``[reconcile] unexplained or apparently conflicting statements in the biblical text in order to allay doubt and worry.''%
    \autocite[126]{vermes1961}
According to \vermes, \ga's interpretation of Genesis was ``organically bound'' to the text of Genesis, arguing that the additions that \emph{were} made participated in a longer tradition of biblical interpretation, and did not emerge \emph{sui generis} from the mind of the author. Where texts like \jub sought to systematically advance a theological vision, according to \vermes, the author of \ga intended to simply ``explain the biblical text,'' calling it illustrative of ``the unbiased rewriting of the Bible.''%
    \footnote{%
        \Cite[126]{vermes1961}. I think this statement is demonstrably false, as I will argue in \hyperref[chap:ga]{chapter four}. \GA utilizes traditions tangential to Genesis, which are not themselves contained within the biblical work. In fairness to \vermes, the early columns of \ga were not available to him when he published \citetitle{vermes1961}, and it is in the earlier columns where this reliance on extra-biblical material is most easily seen.}

The third part of \citetitle{vermes1961} is titled ``Bible and Tradition'' and consists of a single chapter engaging in a lengthy analysis of the traditions surrounding the seer Balaam from Numbers 22--24. \vermes observes that while the majority of post-biblical texts treat Balaam as a villain, in \lab he is treated as a sort of tragic hero.%
    \autocite[173]{vermes1961}
The more traditional portrayal of Balaam as a wicked prophet began within the nexus of biblical tradition itself. The various documentary strata of the Balaam story cast the prophet in differing lights, and it is the final stratum, the Priestly layer, which got the final say.
%
% TODO: as evinced by which texts. You need to elaborate on this point briefly.
%
\vermes points out, however, that ignoring the Priestly additions yields a story somewhat similar to that of \lab. Thus, \vermes concludes that the exegetical traditions found in the later targums and rabbinic works are simply the continuation of the exegetical strategies employed within the Bible itself, which he calls ``biblical midrash or haggadah.''%
    \autocite[176]{vermes1961}

The last two chapters make up the fourth and final section of \vermes's study, titled ``Theology and Exegesis,'' and push the discussion to include early Christianity. Chapter seven is entitled ``Circumcision and Exodus 4:24--26'' but offers a subtitle of ``Prelude to the Theology of Baptism,'' which gives some hint at the ultimate, if tacit, goal of the chapter. Discussing the topic of circumcision in Exod 4:24--26 and its treatment among the early exegetes, \vermes's primary observation is simply that the theology of circumcision and the exegetical traditions which surrounded it were affected by historical forces and theological ideologies. For instance, he claims that \jub omitted the rather odd statement that God was going to kill Moses---who was saved by the circumcision of his son by Zipporah---because ``[i]t was impossible for its author to accept that God tried to kill Moses as it was for him to believe that Moses neglected to circumcise his son on the eighth day after his birth.''%
    \autocite[185]{vermes1961}
Similarly, he notes that after the Bar Kokhba rebellion, the practice of circumcision was outlawed. He writes:
\begin{quote}
    [I]t is not surprising, therefore, to find the spiritual authorities of Palestinian Judaism emphasizing the greatness and necessity of this essential rite, and explaining away \ldots{} every possible biblical excuse for delaying the circumcision of their children.%
        \autocite[189]{vermes1961}
\end{quote}
\noindent
He ends the chapter by suggesting that the early Christian association of baptism with circumcision (citing Rom 4:3--4 and Col 2:11--12) was enabled by the traditional Jewish association of circumcision with blood sacrifice (``the Blood of the Covenant'').%
    \autocite[190]{vermes1961}
That Paul associated baptism with circumcision was ``not due, therefore, to his own insight, but springs directly from the contemporary Jewish doctrine of circumcision which he adopted and adapted.''%
    \autocite[191]{vermes1961} 

\vermes makes a similar move in Chapter Eight, entitled ``Redemption and Genesis XXII: The Binding of Isaac and the Sacrifice of Jesus.'' In it, he compares a number of ancient works' treatment of the Akedah and demonstrates how the (near-)sacrifice of Isaac became a prototype for the entire sacrificial system in later Judaism. The sacrifice of animals in \thetemple functioned as a ``reminder'' to God of the faithfulness of Abraham. Furthermore, he shows the ways the tradition grew to focus on the willingness of Isaac to be sacrificed and his function as a proto-martyr. Thus, he ends the chapter by addressing the New Testament's portrayal of Jesus as a willing sacrifice to God and its putative relationship to the Akedah. \vermes makes the case that the redemptive theology of the NT---typically attributed to Paul---was not original to him. He writes: 

\begin{quote}
    For although [Paul] is undoubtedly the greatest theologian of the Redemption, he worked with inherited materials and among these was, by his own confession, the tradition that ``Christ dies for us according to the Scriptures.''%
    \autocite[221]{vermes1961}
\end{quote}
\noindent
He then proceeds to push the origin of this theology back further into the \ce{first century}, and, in rather dramatic fashion, suggests that the introduction of the Akedah motif into Christian theology---by means of the Suffering Servant---may have been by Jesus himself.%
    \autocite[223]{vermes1961}

\vermes concludes the chapter by discussing the Akedah and the Eucharist. Just as the whole sacrificial system pointed back toward the binding of Isaac in targumic exegesis, the eucharistic rite likewise was intended---according to \vermes---to point back to Jesus's redemptive sacrifice. Thus, he concludes: 

\begin{quote}
    Although it would be inexact to hold that the Eucharistic doctrine of the New Testament, together with the whole Christian doctrine of Redemption, is nothing but a Christian version of the Jewish Akedah theology, it is nevertheless true that in the formation of this doctrine the targumic representation of the Binding of Isaac has played an essential role. 
    
    Indeed, without the help of Jewish exegesis it is impossible to perceive any Christian teaching in its true perspective.%
    \autocite[227]{vermes1961}
\end{quote} 

The arc of \vermes's study, therefore, is meant to establish a continuity between the earliest traditions of biblical interpretation and the later traditions of both rabbinic Judaism and early Christianity, and to trace the evolution of those traditions historically. Rather than viewing the early rabbinic interpretations as \emph{sui generis}, \vermes's larger purpose is to establish \emph{continuity} between the earliest examples of biblical interpretation---even innerbiblical interpretation---and the exegetical work of the rabbis. According to \vermes, therefore, \rwb represents an intermediary phase between innerbiblical interpretation and later explicit commentaries, all of which can be viewed on a single interpretive continuum.

\subsection{\vermes's Use of \rwb} 

The fact that \vermes spent so little time explaining precisely what he meant by the term \rwb bears witness to the fact that \vermes thought the term was self-explanatory. \vermes makes this sentiment clear in his short retrospective on the origins of the term, expressing shock over the debate that his term prompted---coined over fifty years prior---and the scholarly confusion surrounding it.%
    \footnote{%
        The only other works in which \vermes addresses the topic of \rwb (to my knowledge) are in 
        \cite{vermes_eretz-israel1989} and his contributions to Emil Schürer's multi-volume history, 
        \cite{schurer1986}.}
He writes: 
\begin{quote}
    The notion [of \rwb], which over fifty years ago I thought was quite clear, seemed to the majority of the more recent practitioners nebulous and confused, and lacked methodological precision.%
    \autocite[3]{vermes_zsengeller2014}
\end{quote} 
\noindent
%
Only a few scholars, according to \vermes, managed to remain true to his original vision.%
    \footnote{%
        He specifically references
        \cite{alexander_carson-williamson1988} and 
        \cite{bernstein_textus2005}.}
%
Instead, he writes, many subsequent studies ``moved the goalposts'' to better ``suit the interest of their inquiry.''%
    \autocite[4]{vermes_zsengeller2014}

One cannot help but push back against \vermes here, though. It is an entirely reasonable impulse for scholars to narrow the scope of terms that are ill-defined. After all, \vermes's use of \rwb covers texts written in several languages, across centuries, in no particular geographical region, and, while all the texts are ``narratives,'' the formal similarities between \ga, \antiquities, \jub, and the \pTarg stop there. Moreover, if we accept non-narrative texts, such as the \templescroll, we are left with very few unifying formal features between these texts. \vermes specifically laments the narrowing of the term \rwb to focus primarily on the \dss texts. Of course, when \citetitle{vermes1961} was first published in 1961 (\vermes notes that the manuscript, in fact, was submitted for publication in 1959), only a small portion of the scrolls were published or accessible to more than a few specific scholars.%
    \footnote{%
        In particular, I wonder whether \vermes's insistence on \rwb as a narrative genre would have been affected if he had known about the \templescroll. Similarly, I wonder how his characterization of \ga might have been different had Avigad and Yadin published the earlier columns of \ga.}
But the field's subsequent preoccupation with the \qumran material, he suggests, is misguided. I am sympathetic to what \vermes perceived as ``moving the goalposts''---I think the context and purpose of how he used the term \rwb is often ignored---but it is worth pointing out that the reason the term \rwb is so often applied to the \qumran texts likely has less to do with a conscious, scholarly effort, and more to do with the fragmentation of the various fields that deal with the texts in question. A scholar with a background primarily focused on the New Testament or Hebrew Bible may not be as familiar with the texts and traditions of rabbinic Judaism that \vermes discusses in \citetitle{vermes1961}. Perhaps ironically, it was this sort of fragmentation that \citetitle{vermes1961} was written---at least in part---to overcome.

I think his characterization that the discussion has focused too much on texts from \qumran, however, is a bit over-blown. On the one hand, \ga and the \templescroll receive a lot of scholarly attention, but \jub and \antiquities do as well. Even so, whatever narrowing of the discussion of \rwb has occurred toward the \qumran scrolls is likely symptomatic of the ``methodological [im]precision'' attributed to \citetitle{vermes1961} and the fact that \vermes did not clearly state what he meant when he used the term in the first place. For example, \vermes's inclusion of the medieval \sefer muddies the waters for those who wish to discuss \rwb as a process of scriptural interpretation which can be situated historically. On the other hand, his inclusion of the Palestinian Targums makes sense diachronically, but formally, the targums are translations and not ``new compositions'' in the same sense that \jub or \ga are.%
    \footnote{%
        For example, the practice of alternating the reading of the Hebrew Bible with the targums served to reinforce the fact that they represented the ``same'' text and that their meanings were synonymous, while at the same time, reinforcing the boundaries between them.}

Within \citetitle{vermes1961}, \vermes treats these texts with due care and nuance---in the case of \sefer, he endeavors to show that traditions preserved in the text can be traced back to the \secondtemple period---but the fact that \vermes sought to situate haggadic developments diachronically while implementing a category that spanned such broad socio-religious (\qumran, early Christian, rabbinic, medieval), chronological (\bce{2nd century} -- \ce{12th century}), and literary (translations, narrative, revelatory/apocalyptic, history?) horizons has given some scholars a reasonable challenge when attempting to use the term in their own work. Thus, simply because \vermes set the ``goalposts'' (to suit his \emph{own} thesis, I might add), does not mean that others cannot or should not move them when appropriate, though hopefully along with a well-reasoned explanation for the change. 

\section{\rwb Or Rewritten Scripture?}

Since \vermes coined the term \rwb, a number of scholars have suggested that the term be modified to more accurately reflect the (now, well established) fact that there was no ``Bible'' in the late \secondtemple period and that many of the works that would eventually make up the Hebrew Bible did not have stable textual witnesses that could be meaningfully ``rewritten.'' Because of these difficulties, scholars have, in recent years, suggested alternate designations for the phenomenon under investigation, the most widely used of which is ``rewritten \emph{scripture}.''%
    \footnote{%
        For a concise overview of the debate, see
        \cite{zahn_weissenberg-pakkala2011}.}
\vermes's original term \rwb was a product of its time. It took for granted the existence of a canonical ``Bible'' that more-or-less resembled the Bible used by the rabbis in the \ce{early centuries}. The term ``rewritten scripture'' was intended to correct what scholars perceived as an anachronistic reference to this canon of scripture during the late \secondtemple period.%
    \footnote{%
        \Cite[58--59]{campbell_zsengeller2014}. See also
        \cite{ulrich_mcdonald-sanders2002}; 
        \cite{ulrich_zsengeller2014}.}

Apart from the anachronistic reference to a ``Bible,'' one of the primary objections to the use of the term \rwb is the implicit assertion that \rwb texts necessarily fall outside the Bible.%
    \autocite[61]{campbell_zsengeller2014}
The notion that a rewritten biblical text by definition could not be considered ``Bible'' itself runs contrary to, on the one hand, texts such as Chronicles and Deuteronomy which---for all intents and purposes---``rewrite'' their biblical \vorlagen, but are themselves a part of ``the Bible.'' On the other hand, texts such as \jub and the \templescroll were likely were considered ``scripture'' among certain groups in antiquity. For instance, the book of \jub continues to be in use by the Ethiopian Orthodox Tewahedo Church.%
    \footnote{%
        Although it is not quite accurate to refer to a ``canon'' of scripture in the same way that the Western Churches refer to it, the point stands that the books of \enoch and \jub were preserved by, and remain in use for, the Ethiopian Orthodox Tewahedo Church. See 
        \cite{baynes_mason-etal2012}; 
        \cite{asale_js2013}; 
        \cite{cowley_os1974}. On the topic of the relative closedness of the Ethiopian Orthodox Tewahedo Church's canon, See 
        \cite{asale_bt2016}.}

Yet, I am not at all convinced that substituting the term ``scripture'' for ``Bible'' meaningfully affects the way that scholars have continued to discuss the topic at hand. While I agree that ``the Bible'' as we know it from the \ce{early centuries} did not exist during the late \secondtemple period, I likewise find the strict reading of ``Bible'' to mean ``the Hebrew Bible (as we know it)'' unnecessarily rigid. To say that \jub was a part of the \qumran Community's ``Bible'' does not carry a vastly different nuance, it seems to me, than to say that the \qumran Community considered \jub to be ``scripture.'' Insofar as a particular group---given a set of texts---can determine which it considers to be ``scripture'' it has, at least in common parlance, a ``Bible.'' That said, I can appreciate the desire to fine-tune our terminology to reflect the scholarly discourse. 

It could, however, be argued that the term ``scripture'' is no more ancient a term than ``Bible.'' Scholars such as James \vanderkam have done important work in trying to discern which texts may have been considered ``authoritative scripture'' at \qumran.%
    \autocite{vanderkam_dsd1998}
But the fact remains that such endeavors start with the assumption that the ancients utilized a notion similar to what we consider ``textual authority.''%
    \autocite{brooke_rev-qumran2012}
While there is good reason to believe that some texts were more important than others during the \secondtemple period (e.g., the Pentateuch, Isaiah, et al.), the degree to which they considered them ``scripture'' is not at all clear. Thus, replacing the term \rwb with Rewritten Scripture, it seems to me, may very well shift the semantic burden from a well defined modern category of text to an ill-defined ancient category. 

For the sake of simplicity, I will follow \vermes in this study and simply use the term ``\rwb.'' In doing so, I realize that I am deviating from what has become the common scholarly terminology. Yet, I find some comfort in \vermes's own take on the matter, who writes, ``Frankly, replacing `Bible' by `Scripture' strikes me as a mere quibble\ldots{} I suggest therefore that we stick with the `Rewritten \emph{Bible}' and let the music of the argument begin.''%
    \autocite[original emphasis]{vermes_zsengeller2014} 

\section{\rwb: A Genre, Process, or Something Else?}

 One of the central issues with the term \rwb is whether it should be treated as a ``genre,'' ``process,'' or ``activity.'' \vermes, is not particularly helpful in clarifying the issue: 

\begin{quote}
    The question has been raised whether the ``Rewritten Bible'' corresponds to a process or a genre? In my view, it verifies both. The person who combined the biblical text with its interpretation was engaged in a process, but when his activity was completed, it resulted in a literary genre.%
    \autocite[8]{vermes_zsengeller2014}
\end{quote} 
\noindent
Within \vermes's schema of aggadic development, \rwb occupied a liminal space outside the genres of classical Jewish texts such as Targum and Midrash. Because these texts eluded categorization within established text categories, \vermes's treatment of \rwb as a discrete group was not unreasonable. A number of scholars have since upheld the categorical approach and argued for \rwb as a literary \emph{genre}. 

The parade example of this perspective is Philip Alexander's 1988 article ``Retelling the Old Testament,'' which, although dated, remains the most widely cited exemplar of the ``genre'' perspective.%
    \footnote{%
        \Cite{alexander_carson-williamson1988}.
        \vermes himself even put his stamp of approval on it, see
        \cite[4]{vermes_zsengeller2014}.}
Alexander takes up four \rwb texts (\jub, \ga, \lab, and \antiquities) to determine whether there exists a set of concrete criteria by which scholars may admit or exclude texts from the category. Although I ultimately disagree with his conclusion that \rwb should be treated as a literary genre, his list of nine ``principle characteristics'' make a number of useful observations about the nature of \rwb texts generally, and are summarized below: 

\begin{quote}
\begin{enumerate}
    \item \rwb texts are \emph{narratives} which follow the order of the biblical text. 
    \item \rwb texts are ``free standing'' literary works that take on the same form as the text they rewrite. They do not comment explicitly on their \vorlagen, but weave interpretation into their seamless retelling. 
    \item \rwb texts are not meant to replace the biblical work. 
    \item \rwb texts cover a large portion of the biblical narrative and exhibit a ``centripetal'' relationship to the biblical text. 
    \item \rwb texts follow the biblical text's narrative ordering, but may omit certain, non-essential elements. 
    \item \rwb texts offer an interpretive reading of scripture which offer, quoting \vermes, ``a fuller, smoother and doctrinally more advanced form of the sacred narrative,''\autocite[Citing \vermes in][305]{schurer1986} and implicitly comment on the biblical text. 
    \item \rwb texts are limited by their literary form which only allows a single interpretation of the biblical text that they rewrite. 
    \item \rwb texts are limited by their literary form which does not allow them to explain their exegetical rationale. 
    \item \rwb texts incorporate traditions and material not derived from the biblical text.%
        \autocite{alexander_carson-williamson1988}
\end{enumerate}
\end{quote} 

Despite Alexander's emphatic conclusion affirming the genre of \rwb, I find a number of these criteria to be unconvincing.

First, his criterion that the text be a \emph{narrative} strikes me as arbitrary. While \vermes focused on \rwb as a narrative phenomenon, he has since noted that the reason for this was that his focus was on \emph{aggadic} material, that is, non-halakhic interpretation, which by definition is non-legal. Coupled with the first half of his second criteria---that \rwb texts take on the same form as the text they rewrite---these observations seem self-fulfilling and suffering from a sort of selection bias.%
    \footnote{%
        Although all of the texts he surveyed are narratives, this fact illustrates one of the major shortcomings in Alexander's method, specifically, that his conclusions were based on four texts ``normally included in the genre.'' See 
        \cite[99]{alexander_carson-williamson1988}. 
        Therefore the selection of these four texts was the result of a deductive selection, in part, based on their narrative form.}
By stating that the author was ``limited'' to a single interpretation by the genre of narrative, and therefore could not provide his exegetical rationale, illustrates the major, overarching assumption about Alexander's (and \vermes's) approach to these texts---that their primary function was to explain the Bible.

Second, several of his criteria are comments about the intention of the author or purpose of the work. For example, Alexander states that \rwb texts were not ``meant'' to replace their \vorlagen. Although I believe this to be fundamental to the discussion, as formal characteristics of a genre, Alexander does not address how one determines such purposes and intentions.  In particular when discussing texts---as \vermes does---such as the \pTarg, or (now) the so-called Reworked Pentateuch (4QReworkedPent),%
    \footnote{%
        In fairness, 4QReworkedPent was not available to Alexander or \vermes. Yet, one still may wonder why   the \lxx or \sampent are not included. On the status of 4QReworkedPent as \rwb, see 
        \cite{zahn2011}; 
        \cite{zahn_dsd2008}.}
claiming that \rwb texts ``implicitly comment'' on their \vorlagen speaks to the \emph{intention} of the author, which in many cases is not demonstrable. Such claims overstep the issue of genre and have entered into speculation about the text's social function. Thus, while Alexander does offer some concrete formal characteristics for \rwb, a number of his criteria are actually issues of textual \emph{function}.

Alexander insists that ``[any] text admitted to the genre must display \emph{all} the characteristics.''%
    \autocite[119 n. 11]{alexander_carson-williamson1988}
This principle seems needlessly rigid. Although these characteristics were inductively identified,  Alexander offers no formal rationale for selecting his sample. The texts that he selects represent the \emph{core} of what is generally accepted to be \rwb, but texts on the periphery of a genre by definition will not display \emph{every} characteristic of the core texts. Thus, Alexander's criteria should not be treated as prerequisites for inclusion to the category of \rwb (if we are to treat it as such), instead, they should be used to describe a sort of literary \emph{Idealtypus} for \rwb.%
% TODO: See Fraade's article about the \templescroll
    \footnote{%
        I have borrowed and adapted the well-known term \emph{Idealtypus} from Max Weber. See 
        \cite{weber1978}. For a concise summary of Weber's work, see 
        \cite[12--16]{smith-riley2009}.}

Moshe Bernstein, too, has upheld a Vermesian understanding of \rwb as a literary category, arguing that the boundaries must be clearly demarcated and reasonably narrow for the category to be useful to scholars.%
    \autocite{bernstein_textus2005}
Notably, Bernstein never clearly articulates what it means for a category to be ``useful.'' All the same, he writes that he set out to: 

\begin{quote}
    examine the definition and descriptions of ``rewritten Bible'' proffered by \vermes and several subsequent scholars, in order to delineate the variety of ways in which the term is currently employed and to make some suggestions for how we might use it more clearly and definitively in the future.%
    \autocite[171--172]{bernstein_textus2005}
\end{quote}

Bernstein begins by addressing the few small modifications that he makes to \vermes's list, namely that he does not understand the targums to be examples of \rwb. He excludes targums from his discussion ``\emph{ab initio},'' as well as ``biblical'' books (by which he seems to mean ``Chronicles''), and includes legal texts such as the \templescroll.%
    \footnote{% 
        On the characterization of the \templescroll as \rwb and the difficulties that go along with it, see 
        \cite{fraade_goldstein-etal2017}. 
        I wonder, too, if we accept that the category of ``Bible'' was not operative during the \secondtemple period, what Bernstein has in mind by excluding ``biblical'' works.}
Despite this second exclusion, Bernstein acknowledges that ``[o]ne group's rewritten Bible could very well be another's biblical text!''%
    \footnote{%
        \Cite[175]{bernstein_textus2005}.
        This seems particularly odd, since, an Ethiopian Christian may protest that \jub should be excluded as well.}
Thus, Bernstein concedes that ``matters of canon and audience may play a role,'' but does not address the topic further. 

Bernstein critiques scholars, such as Nickelsburg, Harrington, and Brooke, for excessively expanding the use of the term \rwb at its ``upper bound'' (my term) to the point that they have weakened it, and have ``not aided in focusing scholarly attention on the unifying vs.~divergent traits of some of these early interpretive works.''%
    \footnote{%
        \Cite[179]{bernstein_textus2005}. He critiques 
        \cite{nickelsburg_stone1984}, 
        \cite{harrington_kraft-nickelsburg1986}, and 
        \cite{brooke_schiffman-vanderkam2000}.}
Likewise, Bernstein critiques Tov for including reworked texts (e.g., 4QReworkedPent) and therefore expanding the ``lower bound'' of the category. While Bernstein avers that ``[r]earrangement with the goal of interpretation is probably an earlier stage in the development of biblical `commentary' than supplementation with the goal of interpretation,''%
    \footnote{%
        \Cite[183]{bernstein_textus2005}. I make special note of the fact that Bernstein places the term ``commentary'' in quotes to indicate that he is not saying that \rwb is formally ``commentary.'' Yet, the overarching principle remains that \rwb is implicitly ``commenting'' on the biblical text. See 
        \cite{fraade_bakhos2006}; 
        \cite{fraade_zsengeller2014}.}
he nevertheless distinguishes the former from the category \rwb, declaring that the definitions of \rwb by Tov and \vermes are not even ``remotely compatible'' and that scholars must ``choose between them simply for the purposes of clarity.''%
    \autocite[185]{bernstein_textus2005}
Bernstein ultimately argues that \vermes's category is worth keeping around, and admonishes the reader to maintain a narrow definition of the category because, in his own words, ``the more specific the implications of the term, the more valuable it is as a measuring device,''%
    \autocite[195]{bernstein_textus2005}
and conversely that ``the looser the definition, the less precisely it classifies those items under its rubric.''%
    \autocite[195]{bernstein_textus2005} 

At the other end of the spectrum, a number of important scholars have treated \rwb as a ``process'' or ``activity,'' rather than as a genre or category. These scholars also have tended to be more ``expansive'' when it comes to which texts should be discussed as ``rewritten.'' Harrington, as noted above, is the classic example of those who wish to treat \rwb as a process. He states: 

\begin{quote}
    Nevertheless, establishing that these books are not appropriately described as targums or midrashim is not the same as proving that they all represent a distinctive literary genre called ``rewritten Bible.'' In fact, it seems better to view rewriting the Bible as a kind of activity or process than to see it as a distinctive literary genre of Palestinian Judaism.%
    \autocite[242--243]{harrington_kraft-nickelsburg1986}
\end{quote} 
\noindent
Instead, he observes that while texts such as \jub and \emph{Assumption of Moses} both constitute a rewriting of the Bible, both ``are formally revelations of apocalypses.''%
    \footnote{%
        \Cite[243]{harrington_kraft-nickelsburg1986}. On the characterization of \jub as apocalypse, see
        \cite{collins_mason-etal2012};
        \cite{hanneken2012}.}
This is an important criticism of scholars who see \rwb as a distinct genre. For example, unlike the Gospels, which arguably have the same basic ``form,'' the texts typically described as ``rewritten'' come in a variety of ``forms'' such as narratives (\ga), apocalypses (\jub), and legal texts (\templescroll). In other words, a single \emph{genre}---insofar as the word describes a literary \emph{form}---is not sufficient to subsume the varied \emph{forms} which all can be described as ``rewritten.''%
    \footnote{%
        See especially 
        \cite{fraade_goldstein-etal2017}. Anders Petersen has also recently attempted to bridge this genre/process gap by arguing that \rwb makes sense as a genre from an \emph{etic}, scholarly, perspective, whether or not (he thinks not) such a genre existed in antiquity (i.e., as an \emph{emic} category). Such distinctions, are useful, and I am in broad agreement insofar as Petersen accepts a classical definition of genre (see below). 
        \cite{petersen_hilhorst-puech2007}. 
        Contra Petersen, see the recent work of 
        \cite{tino_jsj2018}.}

More recently, Molly Zahn has attempted to move the conversation forward by interacting with modern genre theory---which is conspicuously absent from most discussions of ``genre'' and \rwb.%
    \footnote{%
        \Cite{zahn_jbl2012}. Daniel Machiela noted the absence of genre theory in his 2010 article, as well, see 
        \cite{machiela_jjs2010}. Brooke is a notable exception. See 
        \cite{brooke_dsd2010}.}
Zahn discusses the difficulty that Harrington addresses by noting that works may participate in multiple genres simultaneously. While older conceptions of genre ``pigeonhole'' texts to specific genres, modern genre theorists---she cites Fowler---now prefer to talk about texts ``participating'' in a genre and are ``less like pigeonholes and more like pigeons.'' She writes: 

\begin{quote}
    Just as a flock of pigeons might change shape, lose and add members, be absorbed into larger flocks of break apart into several smaller flocks, genres and their boundaries are not static.%
    \autocite[277]{zahn_jbl2012}
\end{quote} 

Zahn's contribution is nuanced and deserves to be taken seriously. The implication for \rwb is clear: although \jub is a revelatory text while the \templescroll is a legal text, they can both participate in their respective ``formal'' genres simultaneously with the supposed \rwb genre.%
    \footnote{%
        See also \cite{fraade_goldstein-etal2017}.}
Zahn also explores the ``functional'' aspects of genre. She notes that genres are ``not simply systems of classifications developed and used by literary critics, but are fundamental to all human communication.''%
    \footnote{%
        \Cite[280]{zahn_jbl2012} citing 
        \cite[37-53]{fowler2002}.}
Thus, genres manifest as common patterns recognized by both the author and the reader that aid communication. In this way, genre functions as a sort of ``literary body language.''%
    \footnote{%
        \Cite[276]{zahn_jbl2012}. See also 
        \cite[199]{newsom_grossman2010} and 
        \cite[37-53]{fowler2002}.}
By way of a modern example, it is quite common in modern films for such generically mixed works to exist. Mel Brooke's films, for example, are well known for a particular genre of goofball comedy/farce set against the backdrop of some other well-known genre. Here I have in mind films such as \emph{Young Frankenstein} (1974), \emph{Blazing Saddles} (1974), \emph{Spaceballs} (1987), and others, each of which participates in both a specific comedic genre as well as ``Classic Horror,'' ``Western,'' or ``Star Wars-esque'' genres, respectively. Even in these examples, however, there are formal characteristics in \emph{both} genres which the audience can point to. Although the social function of farce is distinct from classical horror movies, the distinction between classical horror films and farces is not only defined by its social function. 

Yet, it is not at all clear to me what we have gained by upholding \rwb as a genre by simply changing what we mean by ``genre,'' however well rooted in theory.%
    \footnote{%
        Machiela critiques Zahn's approach for similar reasons. See 
        \cite{machiela_jjs2010}.}
If, by Zahn's definition of ``genre,'' we no longer are talking about formal characteristics to \rwb, we are ultimately left with a particular kind of relationship (rewriting) matched to a particular kind of biblical \vorlage. Although this more sophisticated approach to genre appears superior, it seems to me that the kinds of questions that remain to be answered regarding the \emph{purpose} of \rwb fall at the outer limits of generic discourse and may require a different set of analytical and theoretical tools.



\section{Defining the Boundaries of \rwb}

Early adopters of the \vermes's taxonomy experimented with applying the term \rwb to a wide range of \secondtemple Jewish literature. The discussion about which texts should fall under the rubric of \rwb has continued into the present and remains a point of scholarly contention. Insofar as ``rewritten'' texts can be measured by how closely they resemble their \vorlagen, defining the boundaries of \rwb focuses on which texts are \emph{too far} from their \vorlagen to meaningfully be considered ``rewritten,'' forming what I will refer to as the ``upper bound,'' and texts which are \emph{too close} to their \vorlagen to be considered distinct literary works, forming the ``lower bound.'' At the upper bound, for example, the \emph{Book of the Watchers} and the \emph{Book of Giants} clearly are rooted in the biblical tradition, yet most scholars do not consider them sufficiently dependent on the text of Genesis to be ``rewritten.'' Although they take Genesis 6:1--4 as a point of departure, their narratives do not explicitly return to the biblical text in a concrete way. Conversely, at the lower bound, although the Samaritan Pentateuch and the \q{4}{ReworkedPent}{} certainly modify their \vorlagen (and in that sense are ``rewritten''), they are more often considered examples of alternate textual \emph{editions} rather than rewritten works. Likewise, the targums and \lxx, as translations, are frequently excluded from discussions of \rwb at the lower bound because they were, presumably, meant to be perceived as the same literary work as their \vorlagen. 

%% TODO: Add footnote about the perception of \lxx and targums as ``Bible'' using ancient sources.

\subsection{The Upper Bound}

\vermes's use of the term \rwb grew out of the concrete examples of texts that exhibited the sorts of exegetical practices relevant to later aggadic traditions. As others adopted the term, however, the question of how to abstract the concept to something meaningful that could be applied to other texts was explored by a number of scholars.%
    \footnote{%
        See especially
        \cite{nickelsburg_stone1984};
        \cite{harrington_kraft-nickelsburg1986}. See also the more recent contributions such as 
        \cite{crawford2008};
        \cite{falk2007}.}
These early applications of the term \rwb, like \vermes's use, did not tend to carry a technical nuance and instead focused on the ways that numerous texts reappropriated biblical stories, figures, and themes in their own works. 

In his 1984 article ``The Bible Rewritten and Expanded,'' George Nickelsburg discusses a number of texts which are ``very closely related to the biblical texts, expanding and paraphrasing them and implicitly commenting on them.''%
    \autocite[89]{nickelsburg_stone1984}
Notably, although the article does deal with \rwb, it also includes a discussion of texts that Nickelsburg does not consider ``rewritten'' (as the title indicates) and which introduce wholly new material into the traditions of the Bible.%
    \autocite[89--90]{nickelsburg_stone1984}

Nickelsburg does, however, provide a list of texts which he loosely describes as examples of biblical rewriting: \firstenoch, \emph{Book of Giants}, \jub, \ga, \antiquities, the Books of Adam and Eve (\emph{Apocalypse of Moses}, \emph{Life of Adam and Eve}), and some Hellenistic Jewish Poets including Philo's \emph{On Jerusalem}, Theodotus's \emph{On the Jews}, and the \emph{Exagoge} by one ``Ezekiel the Poet of Tragedies.'' Compared to \vermes's list, Nickelsburg's represents a maximalist understanding of the \rwb phenomenon. The inclusion, especially, of \firstenoch illustrates his tendency to include works that build off of the biblical text (in this case, Genesis 6:1--4), but do not track with the biblical narrative for long stretches. 

One of the more interesting contributions to the conversation is Nickelsburg's idea that biblical rewriting followed a trajectory from rewriting smaller units of the Bible---involving short stories that deal with particular events from the biblical text---to longer, more systematic treatments that span multiple biblical books. His treatment of \firstenoch (which is, at least in part, the earliest text that he deals with) is illustrative of this approach. Rather than dealing with \firstenoch as a whole, Nickelsburg addresses the various rewritings of the flood narrative throughout \firstenoch and in the \emph{Book of Giants} which, although not formally a part of \firstenoch, has a clear connection to the work.%
    \autocite[90--97]{nickelsburg_stone1984}
Setting aside for the moment that \firstenoch is a composite work, we can appreciate that the flood story from Gen 6--9 is retold and to varying degrees reinterpreted throughout \firstenoch.%
    \footnote{%
        By my count, there   are six retellings of the flood in \firstenoch: 6--11; 54:7; 64--69;   83--84; 86--89; and 106--107.} 

Although Nickelsburg generally accepts that the rewritten texts ``comment'' on the Bible, he notes that the posture toward the biblical text is also not uniform even among the agreed upon \rwb texts. For example, while the author of \jub's concerns are largely halakhic and makes explicit reference to the biblical text, the authority assumed by the author of \jub does not (at least rhetorically) originate in the exposition of the Torah, but in the ``immutable heavenly tablets.''%    
    \autocite[100--101]{nickelsburg_stone1984}
Nickelsburg thus states: 

\begin{quote}
    This process of transmitting and revising the biblical text reflects a remarkable view of Scripture and tradition. The \psgraphic ascription of the book to an angel of the presence and the attribution of laws to the heavenly tablets invest the author's interpretation of Scripture with absolute divine authority.%   
    \autocite[101]{nickelsburg_stone1984}
\end{quote} 

In contrast, \ga seems to have little interest in halakhic matters and instead is content to elaborate on its \vorlagen by giving detailed geographic information and providing the reader with more dramatic characters.%
    \autocite[106]{nickelsburg_stone1984}
Finally, he observes that \lab likewise differs with \jub in its omission of halakhic matters and its ``highly selective reproduction of the text.''%
    \autocite[110]{nickelsburg_stone1984}
This selectivity also differs from \ga, which otherwise is ``characterized by the addition of lengthy non-biblical incidents.''%
    \autocite[110]{nickelsburg_stone1984} 

Ultimately, Nickelsburg differs from \vermes mainly in the way he views the Bible during the late \secondtemple period. Although Nickelsburg observes that the preoccupation with certain texts suggests that they were held in high regard, he does not have the same interest in tying the exegetical practices of \rwb with earlier inner-biblical, or later haggadic traditions. Because Nickelsburg treats \rwb as a process, he is able to highlight the fact that \firstenoch does indeed ``rewrite'' certain pericopae from Genesis despite the fact that the whole book (which, as already noted, is a composite text to begin with) does not maintain a ``centripetal'' relationship with the biblical narrative. 

Daniel Harrington's 1986 contribution entitled ``Palestinian Adaptations of Biblical Narratives and Prophecies I: The Bible Rewritten (Narratives),'' adopts the term \rwb to talk about texts produced around the turn of the era by Palestinian Judaism that ``take as their literary framework the flow of the biblical text itself and apparently have as their major purpose the clarification and actualization of the biblical story.''% 
    \autocite[239]{harrington_kraft-nickelsburg1986}
In this regard, he follows \vermes closely in how he imagines \rwb to function. Yet, compared to \vermes, he operates with a slightly expanded list of rewritten texts. In addition to \jub, \ga, Ps. Philo's \lab, and Josephus's \ant, he also includes the \emph{Assumption of Moses} and the \templescroll. Furthermore, he makes a point to suggest that a number of other texts may be able to be included in the list, including \emph{Paralipomena of Jeremiah}, \emph{Life of Adam and Eve/Apocalypse of Moses}, and \emph{Ascension of Isaiah}. Harrington's major contribution is his explicit rejection of \rwb as a category or literary genre in favor of a process-oriented approach. Because of this fact, Harrington takes a broad view of rewriting and allows this process to be understood similarly to reception history (although this is my term and not his). Harrington's inclusion of the \templescroll marked a significant deviation from \vermes's use of the term by including non-narrative material under the rubric of \rwb. While several of Harrington's other suggested texts are not considered \rwb by many scholars, the inclusion of other non-narrative texts, in particular the \templescroll, has gained wide acceptance.%
    \footnote{%
        At the time that \citetitle{vermes1961} was published, the \templescroll had not yet been published, so it is hardly fair to expect \vermes to include it in his discussion. That said, my understanding of \vermes's conception of \rwb---even taking into account the existence of works such as the \templescroll---would preclude the inclusion of the \templescroll from the category of \rwb. This is a point at which even those who broadly agree with \vermes, such as Moshe Bernstein, take issue with \vermes's definition. See 
        \cite[183--184]{bernstein_textus2005}.}
%
Building on the notion that \rwb could also include non-narrative material, George Brooke, in a more recent treatment of the topic, defines \rwb as ``any representation of an authoritative scriptural text that implicitly incorporates interpretive elements, large or small, in the retelling itself.''%
    \autocite[777]{brooke_schiffman-vanderkam2000}
Adopting a ``loose'' definition of the term, Brooke includes in his discussion biblical texts that rewrite other biblical texts, such as Deuteronomy and Chronicles, in addition to examples of texts which ``rewrite'' portions of each of the major divisions of the Hebrew Bible, most of which were found at \qumran.%
    \footnote{%
        Brooke categorizes the texts as follows: Reworked Pentateuchs, Rewritten Pentateuchal narratives, Rewritten Pentateuchal laws, Rewritten Former Prophets, Rewritten Latter Prophets, and Rewritten Writings.
        \cite[778--780]{brooke_schiffman-vanderkam2000}. See also 
        \cite{brooke_herbert-tov2002}. See also the important work of 
        \cite{falk2007}.}

The purposes of rewriting, according to Brooke, are manifold, but in each case the (re)writer augmented or repurposed an authoritative base text for some new context. He writes: 

\begin{quote}
    The rewriting seems to have a variety of purposes, among which are the following: to improve an unintelligible base text, making it more comprehensible (11Q19); to improve a text by removing inconsistencies---often through internal harmonization (\q{4}{paleoExod}{m}); to justify some particular content by providing explanations for certain features in the base text (\q{1}{apGen}{}); to make an authoritative text serve a particular function, perhaps in a liturgical setting (4Q41); to encourage the practice of particular legal rulings (\jub); and to make an old text have contemporary appeal (\templescroll).%
    \autocite[778]{brooke_schiffman-vanderkam2000}
\end{quote} 

While I am sympathetic to the more maximalist approaches of Nickelsburg, Harrington, and Brooke, none of these treatments offer any concrete criteria for delineating between \rwb and texts that merely allude to biblical stories. Philip Alexander has suggested that certain works which are primarily ``expansive'' (the \emph{Book of Giants}, the Book of Noah) should not be considered \rwb because their relationship to the biblical text is ``centrifugal''---that is, they take the biblical text as a point of departure, while formally \rwb texts show a ``centripetal'' relationship to the biblical text---that is, they expand beyond the biblical text, but remain tightly coupled to the text \emph{as it exists in the Bible.} Alexander writes: 

\begin{quote}
    Rewritten Bible texts are centripetal: they come back to the Bible again and again. The rewritten Bible texts make use of the legendary material, but by placing that material within an extended biblical narrative (in association with passages of more or less literal retelling of the Bible), they clamp the legends firmly to the biblical framework, and reintegrate them into the biblical history.%
    \autocite[117]{alexander_carson-williamson1988}
\end{quote} 

This ``centripetal'' relationship to the biblical text, I believe, should form the upper bound of what is called \rwb. Therefore, for the purposes of this study, works such as \firstenoch, will not be treated because they do not exhibit this close centripetal relationship. On the other hand, I adopt a more expansive understanding of \rwb than that of \vermes, and include works within the Hebrew Bible itself (Deuteronomy and Chronicles), as well as non-narrative works such as the \templescroll.

%% TODO: Add something about Falk?

\subsection{The Lower Bound: Between Bible and \rwb}

Another recent avenue of investigation has been to explore the boundaries between the biblical text, editions, translations, and rewritten biblical texts. \vermes utilized the targums liberally in \emph{Scripture and Tradition}, but his goal was to blur the line between individual post-biblical texts. Most scholars treating \rwb, however, are not inclined to include the targums among \rwb. But the targums---and for that matter the \lxx and Samaritan Pentateuch---do uniquely represent interpretive traditions. Furthermore, the instability of the biblical text during the late \secondtemple period, as exhibited by the varied editions of Jeremiah found at Qumran and other liminal texts, such as 4QReworkedPent, has problematized the question of what may have constituted ``Bible'' (or, more properly, ``scripture'') at the time.%
    \footnote{%
        See \cite{zahn2011} and \cite{zahn_dsd2008}.}

Unsurprisingly, Emanuel Tov has been at the forefront of this investigation. In his 1998 article, ``Rewritten Bible Compositions and Biblical Manuscripts, with Special Attention to the Samaritan Pentateuch,'' Tov's purpose is to specify the ``fine line between biblical manuscripts and rewritten Bible texts.''% 
    \autocite[334]{tov_dsd1998}
By this, Tov means that he is concerned with what I have termed the ``lower bound'' of the definition of \rwb, specifically, the distinction between a text \emph{edition} and a distinct composition, which Tov considers ``rewritten.'' The primary difference between these two categories of texts, according to Tov, is not how dramatically the daughter text diverges from its parent, but the \emph{purpose} of the daughter text.%
    \autocite[334]{tov_dsd1998}
According to Tov, this purpose is mirrored in the putatively authoritative status of the ``biblical'' text \visavis the rewritten text which, he says, is not authoritative (although he seems to suggest that this is up for debate).%
    \autocite[337]{tov_dsd1998}
For example, he notes that the extant texts of Jeremiah, while widely divergent in length and order, still represent ``biblical Jeremiah,'' which carries some authoritative weight. Tov is, however, careful to point out that the nature of this authority is not clear and ``the boundary between the biblical and non-biblical texts was probably not as fixed as we would have liked for the purpose of our scholarly analysis.''%
    \autocite[335]{tov_dsd1998} 

Carrying a similar trajectory, Michael Segal's 2005 article ``Between Bible and Rewritten Bible,'' in the tradition of Alexander, attempts to enumerate a series of criteria by which scholars can define the lower bound and distinguish between ``editions'' of biblical texts and ``rewritten'' texts. 

Segal's understanding of the role of \rwb is rooted in the conviction that a rewritten text is a ``new'' work that derives its own authority by means of its association with a biblical text. The new composition carries with it the purpose and any theological or ideological \emph{Tendenzen} of the new author, building on the authoritative status of the underlying text.%
    \autocite[11]{segal_henze2005} Segal writes: 

\begin{quote}
    Even though these rewritten compositions sometimes contain material contradictory to their biblical sources, their inclusion within the existing framework of the biblical text bestows upon them legitimacy in the eyes of the intended audience \ldots{} the inclusion of this material within the framework of the biblical passages under interpretation transforms the ideas of the later writer into authoritative and accepted beliefs.%
        \autocite[11]{segal_henze2005}
\end{quote} 
\noindent
And further: 
\begin{quote}
    The nature of the relationship between rewritten biblical compositions and their sources constitutes a paradox. On the one hand, the rewritten composition relies upon biblical texts for authority and legitimacy. The author claims that any new information included in the later work already appears in earlier sources. But simultaneously, the insertion of new ideas into the biblical text, ideas that may even contradict the beliefs and concepts of the original biblical authors, undermines the very authority that the rewriter hopes to utilize.%   
    \autocite[11-12]{segal_henze2005}
\end{quote} 
\noindent
While I find Segal's characterization of \rwb texts to be problematic, his main contribution to the discussion are his criteria for distinguishing between ``biblical'' and \rwb texts. He distinguishes between ``external'' and ``internal'' (literary) characteristics. 

\subsubsection{External Characteristics}

Segal identifies two external characteristics of \rwb texts: ``language'' and ``relationship between the source and its revision.'' 
\begin{quote}
\begin{enumerate}
    \item Language: While he offers little rationale for this criterion, Segal categorically dismisses the possibility that any \rwb text could have been written in a language other than its \vorlage. This criterion categorically excludes \ga, Josephus's \ant, and \lab.

    \item The Textual Relationship between the Source and Its Revision: The underlying text must be ``visible'' in the \rwb text. He uses the book of Chronicles as the parade example of this relationship and notes the caveats necessary in dealing with \vorlagen from this period (i.e., it is difficult to say what is ``rewritten'' versus what is just another variant in the \vorlage).
\end{enumerate} 
\end{quote}
\noindent
Segal notes that both of these criteria, in fact, apply to textual editions, as well as to \rwb texts.%
    \autocite[20]{segal_henze2005}
In other words, these are not ``distinguishing'' criteria, so much as the baseline for consideration. One may demur, however, that if a single criterion, such as language, categorically excludes several texts which meet all the other criteria (below; by this definition he excludes \ga, Josephus's \ant, and Ps. Philo's \lab), perhaps the problem is with the criterion.

\subsubsection{Internal (Literary) Criteria for \rwb}

It is the ``Internal Criteria'' which Segal, ultimately, believes provide the \emph{definition} of \rwb texts.%  
    \autocite[20]{segal_henze2005}
Segal provides six internal criteria: 

\begin{quote}
\begin{enumerate}
    \item Scope of the Composition: ``Editions'' of texts cover the same material as their source. In other words, one expects an edition of Genesis to cover the same material as the book of Genesis; pluses and minuses do not stray into other works. On the other hand, rewritten texts ``do not generally correspond to the scope of their sources.''%
        \autocite[20]{segal_henze2005}
    For example, he observes that \jub covers Genesis and part of Exodus, and Chronicles covers parts of Samuel and Kings. Oddly, he also notes that Ps. Philo---which is not written in Hebrew---runs from Genesis into 1 Samuel. He writes: ``In all these examples the change in the scope of the composition created a new literary unit.''%
        \autocite[20--21]{segal_henze2005}

    \item New Narrative Frame: Several of the \rwb texts include a framing narrative. His examples include the \templescroll and \jub, both of which re-frame the ``biblical'' material. In the case of both works, the Torah is assumed and the new work presumed to be a reflection of a second, direct revelation of the law to Moses, albeit by different means (and fragmentary, in the case of the \templescroll). In \jub, the angel of the Presence revealed this ``second Torah'' during Moses's second ascent (Exod 24). On the other hand, the \templescroll seems to begin in Exod 34.%
        \autocite[22]{segal_henze2005}
   
    \item Voice: While biblical narratives are generally written in a ``detached'' third person style, Segal observes that both \jub and the \templescroll ``change the voice of the narrator throughout.''%
        \autocite[22]{segal_henze2005}
    As far as I can tell, what Segal means is that in these \rwb texts, certain events which are narrated in the third person in the biblical text are re-framed as, for instance, direct discourse in the first person by an angel, or even by God.%
        \footnote{%
            This may seem like a minor quibble, but the ``narrator'' has a distinct and technical meaning in narrative criticism which should be maintained. I would note, however, that this sort of reframing is not unique to \rwb, since, e.g., Deuteronomy does something similar (perhaps Segal considered Deuteronomy to be \rwb?).}
 
    \item Expansion versus Abridgment: By-and-large, text editions are \emph{additive}. That is to say, when there is a discrepancy between the amount of content (as opposed to the order), typically the shorter text is considered older. Segal is here concerned with editorial changes, and not with scribal errors, which could go in both directions (through parablepsis et al.). This property, he contends, is rooted in the conviction of the scribes that in order to reproduce a text, one must reproduce the \emph{entire} text.%
        \autocite[24]{segal_henze2005}
    \rwb texts, however, felt free to add \emph{or remove} material because their authors understood themselves to be composing entirely new works.%
        \autocite[24]{segal_henze2005} 

    \item Tendentious Editorial Layer: ``Editions'' do not change fundamental ideology of the work. For example, differing editions of Jeremiah may differ but those differences do not change the fundamental ideology of the work. Likewise, expansion and addition to the work (e.g.~additions to Daniel) are in line with the theological \emph{Tendenz} of the shorter book. On the other hand, \rwb texts freely alter the ideologies of the text, for example, \jub.% 
        \autocite[25]{segal_henze2005}

    \item Explicit References to the Source Composition: ``Editions'' cannot (in a meta-discursive sense) reference their base text while \rwb texts can (e.g., \jub reference to the Torah).
\end{enumerate} 
\end{quote} 

Although I fundamentally disagree with Segal's conclusions, he makes a number of useful observations about several literary features which are common to \rwb texts. If one ignores his linguistic criterion, his literary criteria form a solid formal baseline for discussing \rwb texts. Thus, while Segal would categorically dismiss \ga, \ant, \lab, et al. for not being composed in Hebrew, in fact, his literary criteria fit these works very well.

More recently, Tov has returned to the topic of text editions and their relationship to the phenomenon of \rwb.%
    \autocite{tov_krarrer-kraus2008}
Tov addresses three ``strange'' texts from the \lxx which, for one reason or another, differ significantly from the preserved MT (3 Kingdoms, Esther, and Daniel). Evoking a number of Segal's criteria for inclusion in the category of \rwb (which he acknowledges to be well accepted, if not terribly well defined), Tov suggests that these \lxx texts likewise may exhibit 1) a new narrative frame, 2) expansion and abridgment, and 3) a tendentious editorial layer and therefore may be candidates for \rwb. 

It is important to think about what Tov and Segal are trying to accomplish in these articles. They are trying to connect scribal practices which allowed for exegetical additions and emendations to ``authoritative texts''---dramatic examples of which are provided by \sampent and \lxx---to the practices which produced the \emph{new compositions}, which scholars refer to as \rwb texts.%
    \footnote{%
        Although one wonders why the targums are not included here. Perhaps it is because Tov is arguing for Hebrew \vorlagen of these texts, while the targums represent a translation.}

What Tov's articles in particular demonstrate, however, is that the issue of authorial \emph{intent} and \emph{purpose} may be at the heart of the distinction between text edition and \rwb. Of course, such intents are not things that can be objectively proven, but I can not help but feel that such considerations ought to factor into reconstructions of ancient practices, even if we must settle for speculation. Thus, it may be that certain especially troublesome texts, such as 4QReworkedPent, cannot be meaningfully categorized as ``edition'' or ``rewritten'' based on formal characteristics at all. Instead, it may be that such distinctions, ultimately, must be addressed by how we reconstruct a text's function within its social context with due consideration to the fact that such contexts are not monoliths and differ across time and space. For example, I imagine that the author of \ga did not consider himself to be creating a new edition of Genesis, nor do I imagine that the author's contemporaries understood it to be such. The same goes for \jub and Chronicles (for Samuel--Kings). On the other hand, the editors and translators of the \sampent, targums, and \lxx presumably \emph{were} producing texts whose social functions aligned with (albeit, not perfectly) the biblical text.%
    \footnote{%
        I am not claiming that the social function of the \lxx and (esp.) the targums would have been identical to or indistinguishable from their \vorlagen, however, the fact that both the \lxx and targums were at times used liturgically---in any capacity---speaks to a unique social function by comparison to \rwb texts. That Chronicles is used liturgically is only a function of the fact that it became a part of the Hebrew Bible on its own merits. Canonically speaking, Chronicles does not derive its authority based on its relationship to Samuel--Kings.}
%
Therefore, what matters for our purposes is not, necessarily, what formal characteristics distinguish \rwb texts, but rather what social \emph{function} such texts may have held in antiquity \visavis the biblical text.

\section{Conclusion}

While each of these discussions has worked to expand and refine the study of \rwb, the basic trajectory set by \vermes---the conviction that \rwb reflects an effort by the writer to implicitly comment on the biblical (or some ``scriptural'') text---has remained surprisingly consistent.%
    \footnote{%
        Most general introductions to the topic treat it in this way. See
        \cite{crawford_charlesworth2000};
        \cite{brooke_schiffman-vanderkam2000};
        \cite{zahn_lim-collins2010};
        \cite{zahn2011}.}
%
This tendency to focus first and foremost on the exegetical qualities of \rwb reflects \vermes's original purpose for the term quite well. However, given the current state of the discussion, it is worth reconsidering this central tenet. This is not to say that there is \emph{no} exegetical purpose to \emph{any} \rwb text, rather that, given a broader view of \rwb, it is worth considering that the phenomenon of rewriting, as a literary process, was not primarily concerned with, nor necessarily tied to, the \emph{explication} of a scriptural \vorlage.%
    \footnote{%
        See the work of Koskenniemi and Linqvist who similarly conclude that \rwb should be treated as rewritten \emph{story}. They ask several of the same questions that I hope to address in this dissertation. See 
        \cite{koskenniemi-lindqvist_laato-ruiten2008}.}
%
Campbell, for example, has recently suggested that the practice of rewriting may have extended beyond works with scriptural \vorlagen and may better be understood as a more general literary phenomenon of the late \secondtemple period.%
    \autocite{campbell_zsengeller2014} 
In his article, Campbell observes that the rewriting of non-scriptural texts has by-and-large been ignored by scholars of \rwb. He offers a number compelling examples of \secondtemple texts which rewrite non-scriptural material following the same basic process as \rwb. In particular, Campbell notes that while \ant 1--11 focuses on biblical material, \ant 12--13 offers a rewriting of the \emph{Letter of Aristeas} and portions of 1 Maccabees. These rewritings maintain the ``structure and flow'' of their base texts, just like \rwb, but it is generally agreed that neither 1 Maccabees nor the \emph{Letter of Aristeas} were viewed as scripture by Josephus, who, notably, provides us with one of the earliest lists of sacred writings from the period.% 
    \autocite{mason2002_mcdonald-sanders2002}
According to Campbell, ``Josephus handles these compositions in the same way that he treats scriptural material in \ant 1--11.''%
    \footnote{%
        \Cite[70]{campbell_zsengeller2014}. See also
        \cite[126]{mason2002_mcdonald-sanders2002}.}
Furthermore, 4 Macc 5--17 retells the story of the martyrdom of seven brothers along with their mother found in 2 Macc 3--7. As with \ant 12--13, 4 Maccabees follows the structure and ordering of the account in 2 Maccabees while augmenting the story and using it to advance the author's thesis as part of a philosophical treatise. 

What these examples lack, as compared with \vermes's understanding of rewriting, is a tradition of \emph{interpretation} (\emph{aggadah} for \vermes), which can account for the changes between the \vorlage and the rewritten work. None of these examples is primarily concerned with clarifying their \vorlage. At best, we might speculate that the authors refined the stories to better suit their needs (making corrections, emendations, etc.), but we do not imagine that there was any expectation that these ``interpretations'' carry any kind of normative force for understanding the original account. In other words, there is no reason for us to assume that the author of 4 Maccabees intended his reworking of the story of seven brothers to affect the way that readers would understand 2 Maccabees.%
    \footnote{%
        I hasten to point out that the later account certainly \emph{would have} affected readers' understanding of the story in 2 Maccabees, but here I am arguing that we do not imagine the \emph{intent} of the author of 4 Maccabees to be affecting that change.}
On the other hand, this is precisely what \vermes was arguing for in \citetitle{vermes1961}: that the activity of rewriting was meant not only to be descriptive of \emph{how} the authors understood the biblical \vorlagen, but that the \emph{purpose} of these rewritten texts in some way functioned \emph{prescriptively} within the tradition of biblical interpretation, that is, according to \vermes, \rwb texts were, by definition, \emph{about} the texts that they rewrote. 

Although modern genre theory, as presented by Molly Zahn, has offered some avenue for discussing the classification of these texts based on \emph{function}, it provides no means for analyzing the \emph{function itself}. In other words, Zahn's work has opened the possibility that the function of a text may affect its generic profile and that at least one such function could be common to texts that we call \rwb. However, \emph{what} that function was and \emph{how it operated} within its social context falls outside the field of genre studies. Similar limitations cropped up in discussions of the boundaries between text editions and \rwb as framed by Tov. Thus, the social role of \rwb as a literary phenomenon within \secondtemple Judaism remains a central and relatively understudied area.

As a way to address this topic, this dissertation will approach the phenomenon of \rwb through the lens of social and cultural memory studies. Although concrete data for the function of \rwb is still lacking, memory studies offers a set of theoretical tools and models for thinking about the transmission, adaptation, and generation of cultural memory based on observed anthropological behavior and social theory. This dissertation, therefore, is an intentional pivot away from reading \rwb texts as primarily functioning ``exegetically,'' toward reading them as products of cultural transmission and adaptation. To be sure, biblical interpretation played a part in the production of these texts, but other social, cultural, and literary forces were at work behind these texts as well, which have largely been ignored in favor of exegetical and interpretive discourses. In the next chapter, I will provide an overview of memory theory and how it can inform the discussion surrounding \rwb.

