% !TeX root = ../dissertation.tex

\chapter{Narrative, Genre, and \Psy: The \ga as Cultural Memory}
\label{chap:ga}

%% TODO: Remove most subheads
%% TODO: Include primary texts and translations
%% TODO: Be consistent with GA/Jubilees/Enoch ordering

Since its initial discovery and publication, the Aramaic text known as the \ga (\q{1}{apGen ar}{}) has been associated in various ways with the book of Genesis. As one of the first seven scrolls discovered in the Judean desert beginning in 1947, the \ga is also one of the more well-studied works among the Dead Sea Scrolls. When the scroll was initially analyzed by scholars, it could not be fully unrolled and only a small portion of the outer layer of the scroll could be read. These visible portions, written in Aramaic, referenced the antedeluvian Lamech, the father of Noah, and his wife, Bitenosh, known from the book of \jub. The text appeared to be written in the first-person from the perspective of Lamech leading Trevor to conclude that the scroll was a copy of the so-called ``Book of Lamech'' listed as an apocryphal work by a seventh Century CE Greek canon list.\footnote{This fact led Trevor to refer to the scroll as the ``Ain Feshkha Lamech Scroll'' and Milik to refer to it as the ``Apocalypse of Lamech'' for the publication of the fragment in DJD 1. See \cite[9--10]{trevor_basor1949} and ``Apocalypse de Lamech'' in \cite[86--87]{djd_1}} Once the scroll was completely unrolled, it became obvious that the scroll contained more than just a first-person account from Lamech and instead contained additional first-person accounts from figures found in the Genesis stories including Noah and Abram. Thus, the more descriptive title, \citetitle{avigad-yadin1956}, was given to the scroll by Avigad and Yadin for the publication of its \emph{editio princeps} in 1956.\footnote{Hebrew: \he{מגילה חיצונית לבראשית}. See \cite{avigad-yadin1956}.} While the name \ga remains in wide use, it is notable that the name has been criticized and a number of alternative titles have been suggested; most notably: ``Book of the Patriarchs''\autocite[Hebrew: \he{ספר אבות}. As suggested by Mazar in][379 n. 2]{flusser_ks1956}, ``Memoirs of the Patriarchs''\autocite[358]{gaster1976}, and \aram{כתב אבהן}\footnote{\cite[14 n. 1.]{milik1959}.  Fitzmyer suggests \aram{כתב אבהתא} would be, perhaps, even more suitable. See \cite[16]{fitzmyer2004}.} In this chapter, I will retain the traditional title, \ga.

Although much of the scroll was very badly damaged, illegible, or missing, enough survived for Avigad and Yadin to make the generalized observations that \ga followed the basic order and events of Genesis from the Flood (Gen 6) into the Abram narrative (ending in Gen 15). The events are generally (though, not exclusively) narrated in a series of first person accounts---what I will refer to as ``memoirs''%
%
\footnote{I will use the term ``memoir'' throughout this chapter as a way of referring to the distinct (mostly) first-person narratives found in the \ga. The term is meant to highlight the formal characteristic of being written in the first person voice without any reference to the authenticity of the work and in alignment with the convention of referring to first-person narratives in the Bible as ``memoirs'' (e.g., the ``Nehemiah Memoir'' or the ``Isaiah Memoir'').}%
---by Lamech, Noah, and Abram, respectively and show a clear affinity with the roughly contemporaneous works of \firstenoch and \jub.\autocite[16--37]{avigad-yadin1956} The literary relationship of \ga to both \firstenoch and (especially) \jub remains a matter of debate, with Avigad and Yadin suggesting that \ga more probably preceded \jub, while the recent prevailing opinion seems to prefer the opposite.%
%
\footnote{\cite[38]{avigad-yadin1956}; cf. \cite[20--21]{fitzmyer2004}. Fitzmyer cites Hartman's suggestion, building on Fitzmyer's own work, treating the similarity between \ga's and Jubilee's chronology of Abram's life. Because the chronology seems to have been closely tied to Jubilee's uniquely structured calendar, it follows that \ga drew from \jub. See \cite[497]{hartman_cbq1966}.}

The name given to the \ga in the \emph{editio princeps} set the agenda for scholarly inquiry on the work into the modern era by connecting it to the biblical book of Genesis while simultaneously categorizing it as apocryphal. Much of the attention given to the \ga, therefore, has focused on its literary genre and its relationship to the Bible and resemblance of the Targums and later midrashic works. As already noted, \vermes's treatment of \ga focused on the role that it played in showing the continuity between the interpretation of Jewish scripture during the \secondtemple period and the aggadic traditions of early rabbinic Judaism. In \emph{Scripture and Tradition}, \vermes treats in detail the relationship between Gen 12:8--15:4 and \ga \cols{19}{22}, ultimately declaring \ga to be ``the most ancient midrash of all'' and the ``lost link between the biblical and the Rabbinic midrash.''\autocite[124]{vermes1961} The result of this framing (whether one considers it appropriate or not) has been that much of the scholarly attention paid to \ga has focused on its relationship to Genesis and especially how its author(s) may have been addressing exegetical issues found within the (later) biblical work. Yet, as Fitzmyer observes, the roots of biblical midrash are now generally accepted to be found within the Hebrew Bible itself.\autocite[20]{fitzmyer2004} Together with the fact that a number of Targums have been found at Qumran makes the presence of targumic and midrashic qualities in \ga less remarkable and frees us from any obligation to try and fit it cleanly within either category.

Although few scholars insist on rigidly defining \ga as either targum or midrash, the treatment of \ga as primarily \emph{exegetical} tacitly implies that the purpose of \ga was to explain or interpret Genesis. Put another way, the discussions surrounding \ga are often preoccupied with gleaning information about how \secondtemple Jews read \emph{Genesis}---treating \ga from the perspective of ``biblical interpretation.'' While there is no question that such an approach has been fruitful, treating \ga as \emph{only} or even \emph{primarily} an example of biblical interpretation, I think, cannot offer a comprehensive reading of the work. In this chapter, therefore, I will focus on the ways that the author of \ga engages with a  constellation of discourses surrounding events and characters \emph{known from} the book of Genesis, as well as those from other texts not as biblical \emph{interpretation}, but as processes of \emph{memory}.%
%
\footnote{What remains uncertain about the \ga is what its function may have been for its original audience. I am in agreement with Fitzmyer that it seems unlikely that \ga would have been used liturgically and that the general character of the work is ``for a pious and edifying purpose.''\textcite[20]{fitzmyer2004}. Yet, I can not help but feel somewhat dissatisfied with this answer. How might \ga have edified its readers? Works such as \jub and \firstenoch, perhaps, have more obvious rhetorical aims, but for all its similarities to these texts, \ga maintains a different character which has generally eluded commentators. While I have no illusions that I will be able to offer a satisfactory answer to the question of \ga's specific purpose, approaching \ga as an object of cultural memory, I believe, is a good place to start. The advantage that a memory approach has in addressing this problem is that it offers a way to talk about the manifold ways that \ga both builds from its cultural memory and speaks back into it.}

From the perspective of cultural memory, therefore, \ga operates within a stream of traditions and participates in discourses surrounding early foundational figures in Jewish tradition: Lamech, Noah, and Abram. As such, it is both the recipient and progenitor of cultural memory whose participation in the mnemonic process affected the memory itself. Thus, building on the theoretical framework of chapter two, \ga may be understood to have taken part in three discrete mnemonic processes: 1) the reception of cultural memory, 2) the reshaping of memory by contemporary social frameworks, and 3) the active construction, codification, and reintegration of memory for future transmission. These three distinct processes are observable within the text of \ga. In this chapter, therefore, I have chosen to frame the discussion of \ga around these processes. First, and as a point of departure, I will discuss the ways that the \ga functions as the recipient of cultural memory through its engagement with what I refer to as  ``biblical memory.'' Second, I will discuss the ways that \ga was shaped by the social frameworks which inherited it through a discussion of literary genre and shared formal characteristics with contemporary texts. Finally, I will discuss how \ga participated in the construction of cultural memory through its use of \psgraphical discourse.


\section{\ga and Biblical Memory}

Although it is anachronistic to suggest that the ``Bible''  existed during the late \secondtemple period, the ``biblical'' texts found at Qumran provide evidence to support the idea that many of the texts and traditions that were later formalized as the ``Bible'' were present in a reasonably stable and even privileged state near the turn of the era. In other words, while I wish to push the discussion away from what I see as a preoccupation with the relationship between \rwb texts their putative biblical counterparts, I do not mean to deny the reality that certain ``biblical'' texts certainly held uniquely privileged positions within the religious and social milieu of \secondtemple Judaism. Therefore, I think it is a mistake to jettison any discussion of \rwb texts as they relate to the texts that would later become the Hebrew Bible. On the other hand, restricting our discussion to those later biblical texts would likewise not do justice to the wide variety of texts and traditions in existence during the \secondtemple period which undoubtedly influenced \ga. In an effort to strike a middle ground, therefore, I have opted to refer to \ga as it relates to ``biblical memory,'' a term which deserves some unpacking. By biblical memory, I have in mind the constellation and confluence of stories and traditions---irrespective of any notion of authority or canon---which participate in the cultural memory which became formalized in the Hebrew Bible.%
%
\footnote{I would like to emphasize that I am not suggesting that ``biblical memory'' represents a qualitatively unique form of memory, only that the scope of the traditions under consideration relate to texts that later became the Bible, and, in all likelihood, held at least some sort of special privilege within the memory of many \secondtemple Jews.}
%
This rhetorical move is meant to blur the line between the \rwb text's putative biblical ``\emph{Vorlage}'' and the broader cultural perception---both official and popular---of the events and characters with which the \rwb text deals. In this section, therefore, I will discuss the ways that the \ga participated in biblical memory through a discussion of its relationship to both the biblical text and other \secondtemple texts such as \jub and \firstenoch.

% What was GA rewriting: Sources for the parts of GS
\subsection{What was the \ga Rewriting?}

Although the \ga is generally touted as one of the more clear-cut examples of the \rwb, it is noteworthy that its relationship to the biblical text is not, in fact, entirely uniform.\footnote{\cite[333]{bernstein_berthelot-etal2010}.}


% First Columns (Lamech Memoir: Summary and relation to 1 Enoch)
% \subsubsection{The Lamech Memoir (Columns 0--\rom{5})}

The earliest columns of the \ga (cols. 0--\rom{5}), which are narrated from the perspective of Lamech (the ``Lamech Memoir'' by my terminology), Noah's father, essentially offer a rewriting of \firstenoch 106--107.\footnote{\cite[174]{nickelsburg2005}. The birth of Noah seems to have been a matter of some interest; a number of other texts likewise discuss the exceptional qualities of Noah at his birth. See \q{4}{534}{}[\q{4}{BNoah}{a-d}], \q{1}{Noah} as well as \cite{vanderkam_kapera1992}. Note also \cite{stuckenbruck_berthelot-etal2010}.} In this section, Lamech, recounts the birth of Noah and Lamech's fear that his wife, Bitenosh, had conceived Noah by means of the \aram{עירין} ``Watchers.'' Despite Bitenosh's assurances, Lamech petitions his father, Methuselah, to ask \emph{his} father, \enoch, for further assurance, which he ultimately gives. Although this section is fragmentary, its close resemblance to \firstenoch 106--107 makes the scholarly reconstruction of the missing sections quite plausible. While it may be tempting to suggest that this section of \ga represents a variant edition of \firstenoch 106--107, rather than a rewriting, the fact that the version of the story preserved in \ga is told in the first-person from the point of view of Lamech, while \firstenoch 106--107 is told in the third-person, makes this suggestion highly unlikely. Moreover, because both \firstenoch and \ga were composed in Aramaic, the differences between the two tellings cannot be attributed to translational issues. In other words, although cols. 0--\rom{5} deal, nominally, with events in Genesis 5:28--29, for all intents and purposes, the story recounted in these columns is a retelling of events known from the Enochic tradition and \emph{not} the book of Genesis.%
%
\footnote{It is not clear what the precise relationship between the Enochic traditions and the \ga actually were. Here I have more-or-less assumed the priority of \firstenoch, but I wish to leave ambiguous whether \ga represents a rewriting of the \emph{text} of \firstenoch, or whether they simply draw on a common tradition. Thus, I have chosen to refer to the tradition ``known from'' \firstenoch, rather than \firstenoch itself. See Stuckenbruck's treatment of these traditions in \cite*{stuckenbruck_berthelot-etal2010}; Nickelsburg's concise but thorough treatment of the similarities and differences in of these texts is also quite helpful. See \cite[173--174]{nickelsburg2005} as well as \cite[122--123]{fitzmyer2004}.} 


% Second Section (Noah Memoir)
% \subsubsection{The Noah Memoir (\Cols{5}{17})}
The second major section of \ga begins with a superscription identifying What follows as a \aram{[פרשגן] כתב מלי נוח} or ``[A copy of] the Book of the Words of Noah'' (\col{5}, 29) and continues through \col{17} (and, likely, onto the beginning of \col{18}).%
%
\footnote{\cite[174--175]{nickelsburg2005}; Regarding the superscription, see \cite{steiner_dsd1995}. On the topic of the existence of a so-called ``book of Noah'' see \cite{dimant_vanderkam-etal2006} and \cite{werman_chazon-etal1999}.}

% Summary
Although this section accounts for the bulk of the scroll, significant portions are missing or unreadable. This ``Noah Memoir'' begins with a description of Noah's righteousness\autocite{vanderkam_collins-nickelsburg1980} (affirmed even in-utero) and his early family life (\col{5}, 29--\rom{6}, 9), followed by a vision predicting the flood (\col{6}, 9--\rom{7}, 9) which comes about due to the evil behavior of the Nephilim. \Cols{7}{8} are highly fragmentary, but most likely described the events of the flood, while \cols{9}{12} (which are slightly less fragmentary) describe the ark's putting in on Mt. Ararat, God's instructions to  and blessing of Noah (including the prohibition of consuming blood), and Noah's subsequent interest in viticulture. \Cols{13}{15} recount a dream-vision in which Noah is depicted as a cedar tree with shoots representing his sons, including a fragmentary explanation of the dream. Finally, \cols{16}{17} describe the division of the land by Noah to his sons.

% ``Sources'' and Relationship of NM to Jubilees & 1 Enoch?
As with the Lamech Memoir, the Noah Memoir clearly draws from traditions outside of those preserved in Genesis. This fact was acknowledged even from the scroll's initial publication.\autocite[38]{avigad-yadin1956} Although the flood account in Gen 6:9--9:17 is a longer and more developed story in its own right than is the account of Noah's birth (which the Lamech Memoir takes as its point of departure), characterizing either cols. 0--\rom{5} or \cols{6}{17} of \ga as \emph{primarily} a rewriting of the Genesis does not give due consideration to the additional traditions which influenced its composition. The extended treatment of Lamech's reaction of Noah's birth in cols. 0--\rom{5}, including the mention of Lamech's wife Bitenosh betray the scrolls extra-biblical sources (esp. \firstenoch and \jub, respectively; more on this below). Moreover, the mention of the Watchers (Aramaic: \aram{עירין}) and the Nephilim in \cols{6}{7} evinces a clear thematic affinity to the Book of Watchers in \firstenoch 6--11.\footnote{\cite[174]{nickelsburg2005}.} Moreover, the explicit reference to the ``the [Book] of the Words of Enoch'' in \col{19}, 25 suggests that the \ga was familiar with \firstenoch, or at the very least a tradition of enochic writings.%
%
\footnote{It is worth noting, of course, that this reference occurs in the latter Abram section which some have argued originates in a different source than the first two memoirs. See esp. \cite{bernstein_berthelot-etal2010} and \cite{bernstein_as2010}.}

Especially plain, however, is the Noah Memoir's connection to the book of \jub, which seems to offer a consistent point of contact with this section of the \ga.\autocite[20]{fitzmyer2004} In fact, it was the explicit identification of Lamech's wife Bitenosh which first prompted Trevor's initial identification of the (unopened) scroll with the so-called Book of Lamech.\autocite{trevor_basor1949} Although an exhaustive treatment of the parallels between \jub and \ga is outside the scope of this chapter, it will suffice to note a few of the most significant points of contact between the Lamech and Noah Memoirs and \jub. James \vanderkam has recently offered a detailed, yet concise, summary of these similarities and differences, which, while too long to reproduced in full, can be summarized as follows:%
\footnote{See \cite[374--376]{vanderkam_feldman-etal2017}. For additional treatments of this topic, see also \cite{machiela2009} and \cite[305--342]{kugel2012} previously published as \cite{kugel_roitman-etal2011}} 

\begin{enumerate}
    \item Several personal and geographic\footnote{%
        Mahaq Sea (\col{16}, 9; \jub 8.22), Tina River (\col{16}, 15; \jub 8.12), Mount Lubar (\col{12}, 13; \jub 5.28), Erythrean/Red Sea (\col{17}, 7; \jub 8.21), and Gadeira (\col{16}, 11; \jub 8.26).}%
        %
        names which are never mentioned in the Bible show up in both \ga and \jub (including Bitenosh, which is a part of the Lamech Memoir).
    \item Both \jub and \ga utilize ``\jub'' as a significant chronological unit (\ga to a lesser degree than \jub).
    \item Several shared stories, themes, and phrases such as 1) ``in the days of Jared,'' 2) Enoch remaining accessible after his departure from normal terrestrial life, 3) Noah making atonement for the ``whole earth,'' and 4) stories recounting Noah and his vineyard.
    \item The ``division of the earth,'' while different in several specifics are strikingly similar and offer, perhaps, the most compelling case for a direct, genetic relationship between the two texts.\footnote{See also Machiela's extensive treatment of this section where he argues for the theory that both texts could be drawing from a shared cartographical source in \cite*[105--130]{machiela2009}. See also \cite{alexander_jjs1982}.}
\end{enumerate}

The striking similarities between the early columns of \ga and \jub (and to a lesser degree, \firstenoch) over and against the biblical text, complicates the characterization of \ga as \rwb or strictly exegetical in nature. In other words if \ga drew from \jub (or if they drew from some common source) I think it is fair to scrutinize whether this section of \ga should be considered a rewriting of \emph{Genesis}. What is clear, instead, is that for the author of the \ga, the memory of the flood (and adjacent characters and events) did not center on the biblical account. The question could, therefore, be asked whether cols. 0--\rom{17}---the bulk of the scroll---would meet the generic criteria of \rwb according to the definitions of Bernstein and other more conservative commentators. My guess is that if \cols{19}{22} had also been lost, \ga would more commonly be categorized as ``parabiblical'' like \firstenoch.\footnote{Notably, this is the preferred nomenclature of Fitzmyer, though he also considers \ga to be ``a good example of the so-called rewritten Bible.'' \cite[20]{fitzmyer2004}.}

% Third Section (Abram) Description
\subsubsection{The Abram Memoir (\Cols{19}{22})}

The final surviving columns of the scroll, \cols{19}{22}, represent the longest and most complete sustained narrative preserved in \ga, here referred to as the ``Abram Memoir.'' More so than the previous sections, the Abram Memoir maps very closely onto the events narrated in Genesis.%
%
\footnote{On the particular text tradition that the \ga builds from, see \cite{vanderkam1978}.}%
%
These columns parallel Genesis 12:10--15:14, retelling the stories of Abram and Sarai's sojourn in Egypt (|| Gen 12:10--20), Abram's subsequent conflict with Lot (|| Gen 13:1--18), the Elamite campaign (|| Gen 14:1--24), and the beginning of Abram's vision (|| Gen 15:1--4). \ga's retelling of these stories follows the chronology of Gen 12--15 very closely, but embellishes and augments the narrative throughout. Like the Lamech and Noah Memoirs, this section of the \ga is largely written as a first-person narrative, this time in Abram's voice. The transition between the Noah Memoir and the Abram memoir is missing, so there is no superscription or title for this section, however, the phrase ``I, Abram'' shows up a number of times, making it clear who the narrator is. This fact is complicated, however, by the fact that, although the narrative begins the in the first-person, beginning at \col{21}, 23 the narrator transitions to the third person and remains so through the end of the surviving portion of the scroll.\footnote{It is worth pointing out that the final surviving sheet of parchment was not the final sheet of the scroll originally. Avigad and Yadin note that although only four sheets of the work were present, the seem between the fourth and (what would be) the fifth sheets is visible on the edge of the fourth sheet. \cite*[14]{avigad-yadin1956}.} This inconsistency, perhaps more than any other feature of \ga, has complicated its generic classification.

% First Half (Midrash)
The earlier portions of the Abram Memoir strike a balance between fidelity and innovation with regard to the \emph{biblical} text that the other sections lack. For example, the narrative of Abram and Sarai's descent into Egypt is clearly and recognizably built from the story preserved in the Hebrew Bible. The events and chronology of the story map directly onto Gen 12:10--20, but the \ga offers---in addition to the first-person point of view---a number of expansions that seem plainly to be innovative or, as \vermes would put it and example or prototype of ``midrash.''%
%
\footnote{\cite[124]{vermes1961}. Notably, the characterization of \ga as \rwb is typically based on an analysis of the Abram Memoir. Although the earlier portions of the scroll were known, \vermes's treatment of \ga only dealt with \cols{19}{22}. Together with the fact that these are the best-preserved and most complete columns, this fact has, I think, impacted the characterization of \ga as a whole, perhaps unfairly. On the characterization pre-rabbinic texts as ``midrash,'' see \cite[esp. 298--305]{mandel2017}; \cite{mandel_bakhos2006}.}
% 
Numerous small additions and emendations occur throughout the retelling such as making explicit how long Sarai and Abram lived in Egypt prior to Sarai's notice by Pharoah's princes, how long Sarai was with Pharoah, numerous geographical and personal names, etc. A number of these details, as with earlier sections of \ga, are also found in \jub, which again illustrates the close (if poorly understood) relationship between the two texts.

More noticeable are the larger expansions present in the \ga such as Abram's portentous dream (\col{19}, 14--17), the \emph{waṣf} put on the lips of Pharoah's princes about Sarai (\col{20}, 2--8), Abram's prayer following Sarai's abduction (\col{20}, 12--16), the details of Pharoah's afflictions (\col{20}, 16--21), Harkenosh's discussion with Lot (\col{20}, 21--\rom{20}, 24), and Abram's intervention on Pharoah's behalf (\col{20}, 24--32).%
%
\footnote{Other changes from later in the memoir include a description of Abram walking the length and width of the land as well as a notable abbreviation of Abram and Lot's conflict in Gen 13:5--12.}

The explanation of these expansions, according to \vermes---which has been adopted by most treatments of \ga---is as a means of ``correcting'' or otherwise supplementing the biblical text in order to engage the reader and to \emph{explain} the biblical text.\autocite[126]{vermes1961} \vermes writes:

\begin{quote}
The author of GA does indeed try, by every means at his disposal, to make the biblical story more attractive, more real, more edifying, and above all more intelligible. Geographic data are inserted to complete biblical lacunae or to identify altered  place names, and various descriptive touches are added to give the story substance\dots To this work of expansion and development Genesis Apocryphon adds another, namely, the reconciliation of unexplained or apparently conflicting statements in the biblical text in order to allay doubt and worry.\autocite[125]{vermes1961}
\end{quote}

% Latter Half (Targum)
By contrast, the latter portion of the Abram Memoir, beginning at \col{21}, 23 at times borders on a word-for-word translation of Genesis into Aramaic with comparatively few significant changes. This quality provided occasion for a number of (especially early) scholars to compare \ga with the Targums.\footnote{\cite[193]{black1983}. Though, he notably amended his opinion later \cite*{black_black-fohrer1968}.} Although the change from first-person to third-person is, perhaps, the most significant literary shift that occurs in the \ga, other literary features of the Abram Memoir agree against the Lamech and Noah Memoirs in such a way that gives reason to suppose the Abram Memoir makes up a literary unit.\footnote{Specifically, Moshe Bernstein has noted based on the divine names that are used throughout the work that the primary division is between the Lamech/Noah Memoirs and the Abram Memoir; the earlier sections utilizing a specific set of divine titles and the latter section(s) using a different set. See \cite{bernstein_jbl2009}; See also \cite[97]{falk2007}. Regarding the genre(s) and unity of \ga more generally see Bernstein's later work \cite*{bernstein_berthelot-etal2010} and \cite*{bernstein_as2010}.} It is not clear, however, why there seems to be such a dramatic difference in narrative voice beginning at \col{21}, 23.

\subsection{Exegesis and Memory}

Modern treatments of the \ga have tended to speak about the work as ``Rewritten Bible'' as a third category somewhere between Targum and Midrash, with a preference to the latter.%
%
\footnote{\cite{evans_revq1988}; \cite[19]{fitzmyer2004}. Esther Eshel has proposed the term ``narrative midrash,'' but I am in agreement with Harrington and Bernstein in eschewing later categories such as ``midrash'' for these pre-rabbinic sources. See \cite[182]{eshel_roitman-etal2011}; Cf. \cite[242]{harrington_kraft-nickelsburg1986}; \cite[327 n. 33; 328--329]{bernstein_berthelot-etal2010}.}

Yet, as I have illustrated, although portions of the \ga relate clearly to the text of Genesis (notably, the Abram Memoir), much of the earlier portions of the scroll only nominally relate to Genesis, and instead show an affinity to the traditions associated with \firstenoch and \jub. Thus, characterizing the work as a whole as focused primarily on the explanation of Genesis (as \vermes suggests), seems to me to be ill-founded. Indeed, the disjunction between the various parts of \ga have been observed by numerous scholars, even by those who broadly accept the \ga to be a literary unity, but such discussions still seem to focus on generic classification.\footnote{Notably \cite{bernstein_as2010} and \cite{falk2007}. Cf. \cite{eshel_roitman-etal2011}.}

To illustrate this difficulty, I would like to focus on Moshe Bernstein's treatment of the ``Genre(s)'' of the \ga.\autocite[As argued in][]{bernstein_berthelot-etal2010} Bernstein's basic thesis is to note that the \ga, as a composite work, must be treated as multigeneric, rather than simply as ``rewritten Bible'' or ``parabiblical'' or the like because, as noted above, the \ga does not relate uniformly to the biblical text. The difficulty, for Bernstein, comes when one must decide how to characterize the work as a whole. While works such as \jub and Pseudo-Philo could be viewed as works that have been uniformly ``rewritten'' (that is, that the entirety of the work is a single rewriting), works such as \ga (he also includes the \templescroll) could be viewed as ``a series of mini-rewritings of limited scope.''%
%
\footnote{\cite[336]{bernstein_berthelot-etal2010}. I am reminded here of Nickelsburg's similar sentiment regarding the ways that \firstenoch rewrites the flood story several times, arguing that the phenomenon of rewriting moved from smaller units of rewriting to larger, more systematic rewritings. See \cite[89--90]{nickelsburg_stone1984}.}
%
According to such a characterization, Bernstein writes, ``we have no choice but to refer to Part I [the Lamech and Noah Memoirs] as `parabiblical' and Part II [the Abram Memoir] as `rewritten Bible''' based on the fact that, while the Abram Memoir rewrites portions of Genesis, the Lamech and Noah Memoirs really only take Genesis as a point of departure for their stories (and may, in fact, be rewriting other texts).\autocite[337]{bernstein_berthelot-etal2010} To refer to the entirety of \ga as \rwb or as two different kinds of \rwb is, according to Bernstein, unacceptably imprecise. While I am happy to accept a multigeneric characterization of \ga (and any number of other texts), I think Bernstein has sidestepped a more fundamental question by suggesting that the relationship between the \ga and its sources is best addressed as an issue of genre. The assumption made by Bernstein is that there was a qualitative difference between the sources utilized by \ga\footnote{While I am sympathetic to viewing \ga as secondary to \jub and \firstenoch, here, I am simply stating this as Bernstein's position.} which forms the basis of his characterization of \ga as ``multigeneric.'' This pluriformity is in tension with his larger assertion affirming the unity of the work. 

It seems to me that the situation may be better analyzed in reverse, namely that the genre of \ga is consistent and it its the assumed qualitative distinction between its sources that should be interrogated. After all, formally speaking, \ga is composed of three (broadly) first-person accounts told from the perspective of three significant patriarchs. In other words, rather than characterizing \ga as a work that utilized both ``biblical'' and ``non-biblical'' sources, it is just as reasonable to begin with the assumption that \ga's method is consistent and that the use of ``non-biblical'' sources actually points to the possibility that \jub and \firstenoch were just as legitimate of sources as Genesis. One possible inference from this observation could be that these other works may have been on equal footing as Genesis and enjoyed some special ``scriptural'' (or otherwise authoritative) position for the author of \ga or that such categories were not operative at this time.\footnote{See esp. the work of Eva Mroczek in \cite*[114--155]{mroczek2016}.} To be clear, the terminology of ``\rwb'' is not what is at stake here, but rather the way that we imagine the relationship(s) between the \ga and the traditions that surround it.

Although the scholarly consensus since the initial publication of \ga has been that \firstenoch, \jub, and \ga all participate in overlapping or adjacent traditions,\footnote{\cite[38]{avigad-yadin1956}; \cite[20--22]{fitzmyer2004}; \cite[110--116]{crawford2008}; \cite[8--19]{machiela2009}.} what remains unclear is the nature and directionality (if any) of these relationships. While Avigad and Yadin suspected that \ga was a source for \firstenoch and \jub,\autocite[38]{avigad-yadin1956} it is now widely acknowledged that no definitive evidence has yet been assembled to argue one way or another.%
%
\footnote{At the risk of over-simplifying the issue, Fitzmyer, Kugel, \vanderkam, and Nickelsburg tend to see \ga as secondary to \jub, while Machiela, Werman, and Segal have argued the reverse.
See \cite{vanderkam_feldman-etal2017}, \cite[20--22]{fitzmyer2004}, \cite[174]{nickelsburg2005}, \cite[305--342]{kugel2012}. Cf. \cite{segal_as2010}, \cite[140--142]{machiela2009}, and \cite[171--177]{werman_chazon-etal1999}.}

Thinking about \ga in terms of cultural memory means thinking about its composition not simply in source-critical terms, but rather as the synthesis of traditions which, regardless of whether they were considered religiously ``authoritative,'' were operative within the \emph{cultural discourse} of late \secondtemple Judaism. In other words, viewing \ga as the product of cultural memory means taking seriously the idea that the combination of traditions in \ga should not primarily be understood as the genius of an author/editor, but rather that the author/editor should be viewed as the instrument by which cultural memory was codified as text. Of course, we must allow for singular, creative contributions of the author/editor of \ga, but even those original contributions should not be treated as if they arose out of a vacuum.\footnote{Such conscious contributions are examples of memory construction. If from this perspective, one supposed that the author of \ga was responsible for the synthesis of these traditions, the \ga would represent another iteration of the process of memory construction and reintegration into the cultural memory of \secondtemple Judaism.} The cultural memory that surrounded the book of Genesis---the biblical memory of Genesis---was more broad than the text of Genesis alone and included traditions that we know from \jub and \firstenoch (whether or not they were directly informed by the \emph{texts} of \jub and \firstenoch). The fact that these traditions coexist within the \ga speaks to the notion that these traditions participated in the same set of discourses, which I have called ``biblical memory,'' and that the author of \ga viewed all such sources as useful for his narrative purposes.

\section{Abram in the Diaspora: The Literary Frameworks of \GA}

Having discussed how the \ga functioned as the \emph{recipient} of a cultural memory that was broader than the text of Genesis, we may now turn our attention to the ways that the \ga was adapted to address its audience within the frameworks of their \emph{social} memory. This section will focus on the way that the \ga was shaped by the social frameworks of its compositional milieu, specifically the ways that contemporary cultural discourses and literary forms molded the biblical memory of Genesis (specifically, the Abram narrative) into a meaningful participant in the collective memory of \secondtemple Judaism.

As I have already noted, the narrative of the \ga is not simply a straight-forward retelling of Genesis from the perspectives of Lamech, Noah, and Abram, but participates more broadly in the ``biblical memory'' of Genesis (which includes related tradition like those represented in \firstenoch and \jub). However, what is most compelling about \rwb texts very often is the \emph{ways} that they adapt biblical memory. These adaptations can come at the level of story---by adding, removing, or rearranging events---or at the level of narrative discourse by describing events differently or with different emphases. In the case of \ga, and in particular in the account of Abram's descent into Egypt in \cols{19}{20}, the biblical narrative has been recast as a (first-person) Hellenistic novella in a similar vein to other well-known Second Temple Jewish works such as the narrative portions of Daniel (including the Greek additions), Esther, Tobit, and (arguably) the so-called Joseph novella of Genesis 37 and 39--50.\footnote{See especially Lawrence Wills work on the Jewish novels and novellas in antiquity: \cite*{wills2002} as well as his important earlier works \cite*{wills1995} and \cite{wills1990}.}

For the convenience of the reader, I have provided the text and my translation of the \ga's account of Abram's descent into Egypt (\q{1}{apGen ar}{} cols. \rom{19}, 10--\rom{20}, 32):%
    \footnote{The Aramaic text is that of M. Abegg in \cite[1:534--537]{perry-tov2014}, in consultation with \cite{fitzmyer2004}. The translation is my own.}

% !TeX root = ../dissertation.tex

\begin{aramaictext}
\versenum{XIX 7}
◦◦◦[\hspace{1.5em}ובנית תמן מדבח]א֯ וקרית תמן ב[שם ]א̇ל[הא] ואמרת א֯נתה הוא
\versenum{8}
א֯ל[הי א]ל[ה ע]ל[מיא] ◦◦◦[\hspace*{1.5em}]◦◦◦ עד כען לא֯ ד֯ב֯קתה לטורא קדישא ונגדת
\versenum{9}
ל◦◦◦ והוית אזל לדרומא̇̇ ◦◦◦ ואתית̇ עד די דב̇קת לחב̇רון ול[ה זמנא] 
א̇ת̇ב̇[נ]יאת חברון ויתבת
\versenum{10}
[תרתין שנין ת]מ̇ן \vacat

והוא כפנא בארעא ד̇א כ̇ולא̇ ושמעת די ע[בו]ר֯א 
ה֯[וא] במצרין ונגדת
\versenum{11}
ל̇[מעל] לאר̇̇ע̇ מ̇צרין [\hspace*{1.5em}]◦◦◦[\hspace*{1.5em}עד די דבק]ת לכ̇רמונא נהרא חד מן 
\versenum{12}
רא֯ש֯י֯ נה̇ר֯א֯ ◦◦◦[\hspace*{1.5em}]עד כען אנחנא בגו א֯ר֯ע֯נ֯א֯ [וח]לפת שבעת ראשי נהרא דן 
די
\versenum{13}
◦◦◦[\hspace*{1.5em}]א כען חלפנא א̇ר̇ענא ועלנא לארע בני חם לארע מצרין

\versenum{14}
\vacat
וחלמת אנה אברם חלם בלילה מעלי לאר̇ע̇ מ̇צ̇ר֯י֯ן וחזית בחלמי֯ [וה]א 
ארז חד ותמרא
\versenum{15}
כחדא צמ[חו] מן ש֯ר֯[ש חד] ובני אנוש אתו ובעון למק̇ץ ולמ̇עקר ל[א]ר̇זא 
ולמ̇ש̇ב̇ק̇ ת̇מ̇ר̇\textsuperscript{ת}א בלחודיהה
\versenum{16}
ואכליאת תמרתא ואמרת אל תקוצ̇ו ל[א]רזא ארי תרינא מן שרש [ח]ד֯ 
צ֯מ֯ח֯[נ]א ושביק ארזא בטלל תמרתא
\versenum{17}
ולא [אתקץ] \vacat

ואת̇עירת בליליא מן שנתי ואמרת לשרי אנתתי חלם 
\versenum{18}
חלמת [אנה וא]דחל [מן] ח֯למ̇̇א֯ דן ואמרת לי א̇שתעי לי חלמך ואנדע ושרית 
לאשתעיא לה חלמא דן
\versenum{19}
[וחוית] ל[ה פשר] חלמא [דן ו]אמ[רת] ◦◦◦ די יבעון למקטלני ולכי למשבק
[ב]ר̇ם דא כול טבותא
\versenum{20}
[די תעבדין עמי] בכול א֯ת֯ר֯ די [נהך לה אמרי] עלי די אחי הוא ואחי בטליכי
ותפלט̇ נפשי בדיליכי
\versenum{21}
[\hspace*{1.5em}יבעון] לאע[ד]יותכי מני ולמקטלני ובכת שרי ע̇ל מ̇לי בליליא דן
\versenum{22}
[\hspace*{1.5em}]◦◦◦[\hspace{8em}] ופרעו צ[ען\hspace*{1.5em}] שרי למפ̇נה לצען
\versenum{23}
[עמי והסתמרת י]ת̇ירא בנפשה די לא יחזנה כול [אנש חמש שני]ן ולסוף
חמש שניא אלן
\versenum{24}
[\hspace*{1.5em}אתו] תלתת גברין מן רברבי מצרי[ן\hspace*{0.75em}] די פרע[ו] צע[ן] על מל[י] ועל
אנתתי והווא יהבין
\versenum{25}
[לי מתנן שגיאן ובעו] ל[י] ל[אודעא] טבתא וחכמתא וקושטא וקרית קודמיהון
ל[כתב] מלי חנוך
\versenum{26}
[\hspace{8em}] בכפנא די [\hspace*{1.5em}]◦◦◦ ולא ◦◦◦ין למקם עד די ◦◦◦ מלי
\versenum{27}
[\hspace{2em}]ל[\hspace*{1.5em}]במאכל שגי ובמשתה[\hspace{8em}]חמרא
\hspace*{1.5em} 
\versenum{28}
[\hspace{3em}]

\end{aramaictext}
% !TeX root = ../dissertation.tex
\begin{aramaictext}
    \versenum{XX 1}
    ◦◦◦[\hspace{3em}]◦◦◦
    \versenum{2}
    ◦◦◦[\hspace{9em}] כמ̇ה ◦◦◦ ושפיר לה צלם אנפיהא וכמא
    \versenum{3}
    [נ]ע֯י֯ם̇ ו֯כ֯מ֯א֯ רקיק לה שער ראישה כמא יאין להון לה עיניהא ומא רגג הוא
    לה אנפהא וכול נץ
    \versenum{4}
    אנפיהא ◦◦◦ כמא יאא לה חדיה וכמא שפיר לה כול לבנהא דרעיהא מא
    שפירן וידיהא כמא
    \versenum{5}
    כלילן וחמיד כול מחזה יד̇[י]הא כמא יאין כפיהא ומא אריכן וקטינן כול
    אצבעת ידיהא רגליהא
    \versenum{6}
    כמא שפירן וכמא שלמא להן לה שקיהא וכל בתולן וכלאן די יעלן לגנון
    לא ישפרן מנהא ועל כול
    \versenum{7}
    נשין שופר שפר̇ה ועליא שפרהא לעלא מן כולהן ועם כול שפרא דן חכמא
    שגיא עמהא ודלידיהא
    \versenum{8}
    יאא וכדי שמע מלכא מלי חרקנוש ומלי תרין חברוהי די פם חד תלתהון
    ממללין שגי רחמה ושלח
    \versenum{9}
    לעובע דברהא וחזהא ואתמה על כול שפרהא ונסבהא לה לאנתא ובעא
    למקטלני ואמרת שרי
    \versenum{10}
    למלכא דאחי הוא כדי הוית מתגר על דילהא ושביקת אנה אברם בדילהא
    ולא קטילת ובכית אנה
    \versenum{11}
    אברם בכי תקיף אנה ולוט בר אחי עמי בליליא כדי דבירת מני שרי באונס
    \vacat
\end{aramaictext}

\begin{translation}
    \versenum{XX 1}
    [\hspace{1em}]
    \versenum{2}
    ``[\hspace{1em}] how [ ] and pleasing is the form of her face. And how
    \versenum{3}
    soft and how fine is the hair of her head!
    How beautiful are her eyes and how desirable her nose and the whole blossom 
    \versenum{4}
    of her face! [\hspace{1em}] how beautiful are her breasts and how pleasing is all her fairness! Her arms? How beautiful! Her hands? How 
    \versenum{5}
    perfect and desirable is every sight of her hands! How pleasing  are her palms and how straight and petite are all the digits of her hands! Her feet?
    \versenum{6}
    How pleasing! How perfect are her thighs! No virgin or bride who enters into the bridal chamber is more beautiful than her. Surpassing all
    \versenum{7}
    women's beauty is her beauty, and her beauty is above all of them. And with all this beauty, great wisdom is with her, and that which she possesses is 
    \versenum{8}
    beautiful.'' And when the king heard the words of Hyrcanus and the words of his two companions (the three of whom spoke as one), he greatly desired her. So he sent
    \versenum{9}
    to quickly lead her (to him). And he looked and he was amazed on account of all her beauty. And he took her as a wife and he sought to kill me, but Sarai said
    \versenum{10}
    to the king, ``he is my brother.'' Therefore I profited because of her and I was left alone---I Abram---and I was not killed. But I wept
    \versenum{11}
    ---I Abram---grievously---both me and Lot, my brother's son with me---in the night when Sarai was taken from me by force.
\end{translation}

\begin{aramaictext}
    \versenum{12}
    בליליא דן צלית ובעית ואתחננת ואמרת באתעצבא ודמעי נחתן בריך אנתה
    אל עליון מרי לכול
    \versenum{13}
    עלמים די אנתה מרה ושליט על כולא ובכול מלכי ארעא אנתה שליט
    למעבד בכולהון דין וכען
    \versenum{14}
    קבלתך מרי על פרעו צען מלך מצרין די דברת אנתתי מני בתוקף עבד לי
    דין מנה ואחזי ידך רבתא
    \versenum{15}
    בה ובכול ביתה ואל ישלט בליליא דן לטמיא אנתתי מני וי\textsuperscript{נ}דעוך מרי די
    אנתה מרה לכול מלכי
    \versenum{16}
    ארעא ובכית וחשית בליליא דן שלח לה אל עליון רוח מכדש למכתשה
    ולכול אנש ביתה רוח
    \versenum{17}
    באישא והואת כתשא לה ולכול אנש ביתה ולא יכל למקרב בהא ואף לא
    ידעהא והוא עמה
    \versenum{18}
    תרתין שנין ולסוף תרתין שנין תקפו וגברו עלוהי מכתשיא ונגדיא ועל כול
    אנש ביתה ושלח
    \versenum{19}
    קרא לכול ח֯כ֯י֯מ̇[י] מצרין ולכול אשפיא עם כול אסי מצרין הן יכולון
    לאסי֯ו֯תה מן מכתשה דן ולאנש
    \versenum{20}
    ביתה ול֯א֯ י֯כ֯לו֯ כ֯ול א̇ס̇יא֯ ואשפיא וכול חכימיא למקם לאסיותה ארי הוא
    רוחא כתש לכולהון
    \versenum{21}
    וערקו \vacat
\end{aramaictext}
\begin{translation}
    \versenum{12}
    That night I prayed and begged and sought mercy and said in sadness---my tears running down, (saying) ``Blessed are you, God Most High, my Lord of all
    \versenum{13}
    eternity, because you are the Lord and ruler over everything and over all the kings of the earth you have the power to make judgment on all of them. And now,
    \versenum{14}
    I petition you, oh Lord against Pharoah Zoan, the king of Egypt, because my wife was taken from me by force. Make a restitution for me from him and make visible your mighty hand 
    \versenum{15}
    upon him and upon his whole house. Let him not have the power this night to defile my wife from me. And they will know you, my Lord, that you are the Lord of all the kings of
    \versenum{16}
    the earth.'' And I wept silently. That night, God Most High sent him a spirit of disease to strike him and all the men of his house---an evil
    \versenum{17}
    spirit---and it tormented him and all the men of his house such that he was not able to to come near to her; indeed he did not know her though she was with him 
    \versenum{18}
    for two years. And at the end of two years the plagues and scourges became stronger and prevailed over them and over all the men of his house. So he sent out 
    \versenum{19}
    a call for all the wise men of Egypt and all the magicians with all the physicians of Egypt (to see) if they would be able to heal him and the men of 
    \versenum{20}
    his house from the plague. But all the physicians and magicians and all the wise men were not able to rise up to heal him.  Instead, the spirit afflicted all of them
    \versenum{21}
    and they ran away.
\end{translation}

\begin{aramaictext}
    באדין אתה עלי חרקנוש ובעא מני די אתה ואצלה על
    \versenum{22}
    מלכא֯ ו֯אסמ֯וך ידי עלוהי ויחה ארי ב[ח]לם ח֯ז[ני] ואמר֯ לה לו֯ט לא יכ֯ו֯ל
    אברם דדי לצלי֯א על
    \versenum{23}
    מלכא ושרי אנ֯ת֯ת֯ה֯ ע̇מ̇ה וכען אזל אמר למלכא וישלח אנתתה מנה לבעלהא
    ויצלה עלוהו ויחה
\end{aramaictext}

\begin{translation}
    Then Hyrcanus came to me and and asked that I might come and pray on behalf of
    \versenum{22}
    the king and that I might put my hands upon him that he might live because [he saw me in a dream]. And Lot said to him, ``Abram, my uncle is not able to pray for
    \versenum{23}
    the king while Sarai his wife is with him. Now, depart and tell the king that he should send his wife from him to her husband so that he will pray for him and live.''
\end{translation}

\begin{aramaictext}
    \versenum{24}
    \vacat
    ו֯כ֯די שמע חרקנוש מלי לוט אזל אמר למלכא כול מכתשיא ונגדיא
    \versenum{25}
    אלן די מתכתש ומתנגד מר̇י מלכא בדיל שרי אנתת אברם י֯ת֯יבו נה לשרי
    לא̇ב֯רם בעלה
    \versenum{26}
    ויתוך מ̇נכה מ̇כ̇ת̇ש̇א̇ דן ורוח שחלניא וקרא [מ]ל[כ]א לי ואמר לי מ̇א עבדתה
    לי בדיל [שר]י ותאמר
    \versenum{27}
    לי די אחתי היא והיא הואת אנתתך ונסבתהא לי לאנתה הא אנתתך ד̇ב֯ר̇ה֯
    א̇זל ועדי לך מן
    \versenum{28}
    כול מדינת מצרין וכען צ֯לי ע֯לי ו֯ע֯ל ביתי ותתגער מננה רוחא דא באיש֯ת̇א
    וצ̇לית֯ עלו֯ה֯י֯ מ֯ג֯דפא
    \versenum{29}
    הו וסמכת ידי ע֯ל֯ [ראי]שה ואתפלי מנה מכתשא ואתגערת [מנה רוחא]
    ב֯אישתא וחי ו֯ק֯ם ו֯י֯הב
    \versenum{30}
    ל̇י֯ מלכא ב[יומא] ד̇נא̇ מנתנ[ן] ש̇גיא̇ן וימ֯א לי מלכא במומה די לא֯ ◦◦◦[  
    ]הא ואת̇יב̇ לי
    \versenum{31}
    לש̇ר̇י ויהב לה מלכא̇ [כסף וד]הב ש̇גיא ולבוש שגי די בוץ וארגואן ו[\hspace{1em}]
    \versenum{32}
    קודמיהא ואף להגר וא[ש]למה לי ומני עמי אנוש די ינפקו֯נני֯ ול◦◦◦ מן
    מ̇צ̇רין \vacat
\end{aramaictext}

\begin{translation}
    \versenum{24}
    When Hyrcanus heard  the words of Lot, he departed and said to the king, ``All these scourges and plagues
    \versenum{25}
    that have scourged and plagued my lord the king  are on account of Sarai, the wife of Abram. Let him return Sarai to Abram, her husband,
    \versenum{26}
    so that this plague might cease from (afflicting) you and also the spirit of oozing.'' And the king summoned me and he said to me, ``What have you done to me on account of [Sarai]? You said
    \versenum{27}
    to me `she is my sister,' but she is really your wife! I took her for myself as a wife! Here is your wife; take her, depart, and go from
    \versenum{28}
    all he provinces of Egypt. But now, pray for me and for my house that this evil spirit be exorcised from us.'' And I prayed for him, that blasphemer,
    \versenum{29}
    and I put my hands upon his [head] and the plague was cleansed from him and the evil spirit was driven from him and he lived. And the king stood up and gave
    \versenum{30}
    to me on that [day] many gifts. And the king swore an oath to me that [he had not defiled her?]. And he returned Sarai to me
    \versenum{31}
    and the king gave her much [silver and] gold and lots linen clothing and purple linen and [\hspace{1em}]
    \versenum{32}
    before her and also Hagar and he returned her to me and allotted men for me who brought me out and [\hspace{0.5em}] from Egypt.
\end{translation}

% [[TODO: Add text and my translation]]
% % TODO: Add  Analysis of Cols. 19-20?

The reading of \ga \cols{19}{20} as a Hellenistic Jewish novella has recently been very thoroughly explicated by Blake Jurgens, who has further argued that the utilization of Hellenistic literary motifs and structures in \ga altered the overall presentation of the pericope for the purpose of edifying Jews living in the Hellenistic world in the shadow of empire.\autocite{jurgens_jsj2018} Although much of Jurgens's paper is based on long-established observations about the literary influences on \ga, he makes an important discursive turn toward the audience by claiming that the \ga was meant to be useful to readers:

\begin{quote}
By imbuing its story with literary tropes and techniques similar to those found in Dan 1--6, Esther, and other Jewish texts arising out of the Hellenistic period, the author successfully attends to the narratival ambiguities of Gen 12:10--20 through interpretive expansion upon the latent exegetical links of the text while concurrently modifying the narrative to appeal to contemporary literary expectations.\autocite[27]{jurgens_jsj2018} \end{quote}

Thinking in terms of social memory, however, we can appreciate the way that the stories that the \ga retells are ``remembered into'' the social context of Hellenistic Judaism and are fitted into contemporary social frameworks by the utilization of common literary techniques. In other words, the changes that Jurgens identifies as authorial decisions intended to engage with readers can also be framed as \emph{determined by} the social location of the author and the literary tools available to him. In other words, as societies change over time, new kinds of literary forms overtake older ones and older stories take on new meanings for new contexts. This adaptation into new forms and meanings is socially determined and should be understood as an example of how social frameworks (in this case, literary frameworks), shape memory \emph{in the present}.

\subsection{Abram in the Diaspora}

One of the primary features of Jewish Hellenistic novellas is their setting. Jurgens notes that, typically, these Jewish novellas are set in the diaspora, which invariably place the Jewish (or, in Tobit and Judith's case, Israelite) protagonist under the hegemony of a foreign power. In the case of \ga, although not properly ``diaspora,'' Abram is a sojourner in a foreign land and is under foreign hegemony. Moreover, from a modern perspective, these stories have a tendency to commit rather egregious factual errors about certain historical particulars such as the names of rulers (Judith 1:1; Dan 4; Tobit) and geographic items (Tobit 5:6). Likewise, \ga seems to utilize details which almost certainly were inventions of the author (or an earlier tradant) such as referring to ``Pharaoh Zoan'' (we know of no such figure) and Herqanos, a name popular in the Ptolemaic period, but not attested otherwise as well as referring to the ``Karmon River'' (probably the Kharma canal), as the one of the seven heads of the Nile river, which it is not.\autocites[7]{jurgens_jsj2018}[See also][50--59]{machiela_as2010}[197--199]{fitzmyer2004} These details, according to Jurgens, are meant to create a sense of verisimilitude and authenticity within the narrative. Thus, although the story of Abram's sojourn in Egypt as narrated in the biblical text engages with discourses of the \emph{foundation} of Israel, the narrative of the \ga seems to be turning the story to engage with the contemporary discourses around the idea of \emph{diaspora}. In other words the way that Abram's sojourn in Egypt was remembered in the \secondtemple period, at least in part, took on new meaning for those sojourning in the diaspora and for those living in the land under foreign hegemony.

\subsection{Abram in the Court of a Foreign King: Literary Genre as Social Framework}

If we place the pericope of Abram's journey into Egypt in \ga under the rubric of diaspora literature, the final scene in the pericope bears a striking resemblance to the so called court contest narratives well-known from (especially) the book of Daniel.\footnote{Other court contest narratives include the Joseph Cycle (Gen 41)} Such narratives, as observed by Collins and others, follow particular narrative progressions with common features.%
%
\footnote{\cite[38--52]{collins1993}; \cite{humphreys_jbl1973}; \cite{collins_jbl1975}; \cite{wills1990}. See also \cite{niditch-doran_jbl1977}.}
%
Jurgens has convincingly argued that the \ga's retelling of Abram's sojourn in Egypt fits such a progression by comparing this pericope to to Dan 2, 4, and 5 as well as Gen 41. The resemblance is quite striking. Although based on the earlier work of Collins and Humprheys, Jurgens offers his own outline, which can be summarized as follows\autocite[21]{jurgens_jsj2018}:

\begin{SingleSpace}
\begin{itemize}
    \item The foreign king has a problem that he is unable to solve.
    \item The king's own personnel are charged with solving the problem
    \item The king's personnel are unable to solve the problem
    \item The Jewish protagonist is asked to solve the problem
    \item The Jewish protagonist is able to solve the problem
    \item The Jewish protagonist is rewarded by the king
\end{itemize}
\end{SingleSpace}

The biblical account, however, offers a rather anemic description of the events, but leaves open the specifics of how Pharoah came to know about Abram and how the monarch was relieved from the plagues. 
It is easy to imagine how the author of \ga would conceive of Abram's interaction with Pharoah in Gen 12 as analogous to other well-known court contests from Israel's biblical memory, despite the fact that the biblical version offers almost no detail. Gen 12:17--20 reads: 

\subsubsection{Genesis 12:17--20}
\begin{hebrewtext}
    \versenum{12:17}
    ‏וַיְנַגַּע יְהוָה אֶת־פַּרְעֹה נְגָעִים גְּדֹלִים וְאֶת־בֵּיתוֹ עַל־דְּבַר שָׂרַי אֵשֶׁת אַבְרָם׃
    \versenum{18}
    וַיִּקְרָא פַרְעֹה לְאַבְרָם וַיֹּאמֶר מַה־זֹּאת עָשִׂיתָ לִּי לָמָּה לֹא־הִגַּדְתָּ לִּי כִּי אִשְׁתְּךָ הִוא׃
    \versenum{19}
    לָמָה אָמַרְתָּ אֲחֹתִי הִוא וָאֶקַּח אֹתָהּ לִי לְאִשָּׁה וְעַתָּה הִנֵּה אִשְׁתְּךָ קַח וָלֵךְ׃ 
    \versenum{20}
    וַיְצַו עָלָיו פַּרְעֹה אֲנָשִׁים וַיְשַׁלְּחוּ אֹתוֹ וְאֶת־אִשְׁתּוֹ וְאֶת־כָּל־אֲשֶׁר־לוֹ׃
\end{hebrewtext}

\begin{translation}
    \versenum{Gen 12:17}
    Now, \yahweh afflicted Pharoah and his house with terrible plagues on account of Sarai, the wife of Abram. 
    \versenum{18}
    And Pharoah called for Abram and he said, ``What is this that you have done to me? Why did you not tell me that she was your wife? 
    \versenum{19}
    Why did you say, `she is my sister' such that I took her as a wife? Anyhow, here is your wife. Take her and go. 
    \versenum{20}
    And Pharoah gave his men orders about him and they sent him away along with his wife away and everything he owned. (My~translation)
\end{translation}

From an innerbiblical perspective, the \ga's description of Abram and Pharaoh's interaction might be thought of as a synthesis or exegetical harmonization with the Abimelech doublet in Gen 20, which offers a much more detailed account of the Abimelech's confrontation with Abram/Abraham (Genesis 20:8--18):

\begin{hebrewtext}
    \versenum{Gen 20:8}
    וַיַּשְׁכֵּם אֲבִימֶלֶךְ בַּבֹּקֶר וַיִּקְרָא לְכָל־עֲבָדָיו וַיְדַבֵּר אֶת־כָּל־הַדְּבָרִים הָאֵלֶּה בְּאָזְנֵיהֶם וַיִּירְאוּ הָאֲנָשִׁים מְאֹד׃
    \versenum{9}
    וַיִּקְרָא אֲבִימֶלֶךְ לְאַבְרָהָם וַיֹּאמֶר לוֹ מֶה־עָשִׂיתָ לָּנוּ וּמֶה־חָטָאתִי לָךְ כִּי־הֵבֵאתָ עָלַי וְעַל־מַמְלַכְתִּי חֲטָאָה גְדֹלָה מַעֲשִׂים אֲשֶׁר לֹא־יֵעָשׂוּ עָשִׂיתָ עִמָּדִי׃
    \versenum{10}
    וַיֹּאמֶר אֲבִימֶלֶךְ אֶל־אַבְרָהָם מָה רָאִיתָ כִּי עָשִׂיתָ אֶת־הַדָּבָר הַזֶּה׃
    \versenum{11}
    וַיֹּאמֶר אַבְרָהָם כִּי אָמַרְתִּי רַק אֵין־יִרְאַת אֱלֹהִים בַּמָּקוֹם הַזֶּה וַהֲרָגוּנִי עַל־דְּבַר אִשְׁתִּי׃
    \versenum{12}
    וְגַם־אָמְנָה אֲחֹתִי בַת־אָבִי הִוא אַךְ לֹא בַת־אִמִּי וַתְּהִי־לִי לְאִשָּׁה׃
    \versenum{13}
    וַיְהִי כַּאֲשֶׁר הִתְעוּ אֹתִי אֱלֹהִים מִבֵּית אָבִי וָאֹמַר לָהּ זֶה חַסְדֵּךְ אֲשֶׁר תַּעֲשִׂי עִמָּדִי אֶל כָּל־הַמָּקוֹם אֲשֶׁר נָבוֹא שָׁמָּה אִמְרִי־לִי אָחִי הוּא׃
\end{hebrewtext}

\begin{translation}
    \versenum{Gen 20:8}
    Abimelech rose early in the morning and called all his servants and spoke all these things in their ears. And the men were very frightened. 
    \versenum{9}
    So Abimelech summoned Abraham, and said to him, ``What have you done to us? How have I sinned against you such that you have brought terrible guilt upon me and my kingdom? You have done things to me which should not be done.''
    \versenum{10}
    And Abimelech said to Abraham, “What were you thinking that made you do this thing?” 
    \versenum{11}
    Abraham said, ``I thought `Surely there is no fear of God at all in this place, and they will kill me because of my wife. 
    \versenum{12} 
    Also, honestly, she is my sister; the daughter of my father but not the daughter of my mother and she \emph{became} my wife.
    \versenum{13}
    And when God caused me to wander from my father's house, I said to her, `This is the kindness you must do for me: at every place when we enter there, say  ``He is my brother.'''''
\end{translation}

\begin{hebrewtext}
    \versenum{Gen 20:14}
    וַיִּקַּח אֲבִימֶלֶךְ צֹאן וּבָקָר וַעֲבָדִים וּשְׁפָחֹת וַיִּתֵּן לְאַבְרָהָם וַיָּשֶׁב לוֹ אֵת שָׂרָה אִשְׁתּוֹ׃
    \versenum{15}
    וַיֹּאמֶר אֲבִימֶלֶךְ הִנֵּה אַרְצִי לְפָנֶיךָ בַּטּוֹב בְּעֵינֶיךָ שֵׁב׃
    \versenum{16}
    וּלְשָׂרָה אָמַר הִנֵּה נָתַתִּי אֶלֶף כֶּסֶף לְאָחִיךְ הִנֵּה הוּא־לָךְ כְּסוּת עֵינַיִם לְכֹל אֲשֶׁר אִתָּךְ וְאֵת כֹּל וְנֹכָחַת׃
    \versenum{17}
    וַיִּתְפַּלֵּל אַבְרָהָם אֶל־הָאֱלֹהִים וַיִּרְפָּא אֱלֹהִים אֶת־אֲבִימֶלֶךְ וְאֶת־אִשְׁתּוֹ וְאַמְהֹתָיו וַיֵּלֵדוּ׃
    \versenum{18}
    כִּי־עָצֹר עָצַר יְהוָה בְּעַד כָּל־רֶחֶם לְבֵית אֲבִימֶלֶךְ עַל־דְּבַר שָׂרָה אֵשֶׁת אַבְרָהָם׃  
\end{hebrewtext}

\begin{translation}
    \versenum{Gen 20:14}
    Then Abimelech took sheep and cattle, and male and female slaves, and gave them to Abraham, and restored his wife Sarah to him.
    \versenum{15}
    Abimelech said, ``See, my land is before you. Settle where it seems good to you.'' 
    \versenum{16}
    But to Sarah he said, ``Look, I have given your brother a thousand pieces of silver; it is your exoneration before all who are with you; you are completely vindicated.'' 
    \versenum{17}
    Then Abraham prayed to God; and God healed Abimelech, and also healed his wife and female slaves so that they bore children. 
    \versenum{18}
    For \yahweh had completely closed all the wombs of the house of Abimelech because of the word of Sarah, Abraham's wife. (My~translation)
\end{translation}

While the Gen 12 account is very terse, the Gen 20 account includes a dream-revelation (20:6--7), specifies that the plagues that afflicted the monarch impeded his sexual activities (specifically with Sarah), and describes Abraham praying over Abimelech and his household to heal them. Similar details are given in the \ga's account which likewise includes a dream-revelation, notes that the plague were sexual in nature, and describes Abram praying over Pharoah and his household for healing. 

However, while these similarities may indeed represent some kind of literary conflation between the two accounts,%
%
\footnote{From a memory perspective I would prefer to account for the \ga's adoption of certain details from Gen 20 in more passive terms where the specifics of the Gen 12 story are, where absent, supplied from another well-known, typologically similar, source. It is also, perhaps, worth noting that this doublet has traditionally been attributed source critically to different hands (Gen 12 = J; Gen 20 = E) See \cite[15]{driver1956} .} 
%
at the level of genre and structure, conflation with Gen 20 cannot account for the \ga's reframing as a court-contest. For example. the dream-revelation in Gen 20 is given to Abimelech, rather than to Abram as in \ga. Moreover although Abraham prays for healing for Abimelech and his household in a very similar fashion to the way he is portrayed in \ga praying for Pharoah, in Gen 20, he does so only after Abimelech effectively ``pays him off.'' It is the revelation given to Abimelech in a dream which causes him to ``repent'' in Gen 20, while in \ga, the miraculous healing of Pharoah and his household functions as the sign and catalyst for Pharoah's rich rewarding of Abram. Although this difference may seem subtle, the primary feature of the court-contest is the demonstration of God's power through the protagonist which leads to the foreign king's repentance/conversion and the rationale for his rewarding of the protagonist. In other words, while it may have been that the \ga used details from Gen 20 to supplement the account from Gen 12, \ga's framing of Abram's contest with Pharoah cannot be solely attributed to a harmonization of the Gen 12/20 doublet. Thus, drawing on details from, or perhaps just inspired by, the Abimelech doublet in Gen 20, the author of \ga was able to reframe this portion of the Abram narrative to conform to the common court-contest pattern, which, as Jurgens rightly notes, surely would have been an effective and entertaining adaptation by comparison to the account from Genesis.

\subsection{Other Literary features and Motifs}
A number of other generic and literary motifs which diverge from the Genesis account, but which are at home in the \secondtemple period can be identified in this portion of the \ga as well.

\subsubsection{Abram as Oracle}
Although the Abimelech story in Gen 20 includes a dream-revelation, it is noteworthy that in \ga, Abram himself is given the dream as a means of warning him about how the Egyptians would attempt to kill him on account of Sarai's beauty. Where the biblical text credits Abram's intuition for anticipating the Egyptians' desire for Sarai (though, we are left to wonder whether he would have been killed had the ruse not been realized), the \ga describes Abram receiving a portentous dream vision characteristic of other \secondtemple literature.%
%
\footnote{\cite{gevirtz_maarav1992}; \cite[184]{fitzmyer2004}; \cite{dacy_tzoref2013}}
%

Although dream-visions are not unique to the \secondtemple period, their ubiquity within Jewish literature from the \secondtemple period is indisputable. In his treatment of the Dream-Visions among the Aramaic \dss, Andrew Perrin describes both Abram and Noah as being ``recast as a dreamer[s]'' within the \ga.\footnote{\cite[52--57]{perrin2015}. See also \cite{eshel_klostergaard-etal2009} and \cite{machiela_falk-etal2010}.} While Noah is not described as a dreamer within the biblical text, within the \ga, he seems to have been the recipient of as many as five dream-visions.\footnote{\cite[53]{perrin2015}. Elsewhere in the Enochic literature dreams weight heavily in the events surrounding the flood, if not always given to Noah (\firstenoch, Book of Giants, etc.).} Restricting the discussion to Abram, however, Perrin suggests that the insertion of a dream-vision into the story on the eve of Abram and Sarai's descent into Egypt functioned as part of a larger project to ``extend Abram's prophetic credentials in light of Gen 20:7.''\autocite[55]{perrin2015} 

Fitzmyer notes that the component parts of this dream---``cedar'' (Aramaic: \aram{ארז}) and ``date-palm'' (Aramaic: \aram{תמרא})---are derived from Ps 92, which declares ``the righteous will flourish like the date palm (Hebrew: \he{כַּתָּמָר}); like a cedar (Hebrew: \he{כְּאֶרֶז}) in Lebanon he will grow'' (Ps 92:13). The identification of Abram and Sarai with the cedar and date-palm, respectively, is plain enough by the parallel to what happens next in the narrative. The interpretation is supported by the grammatical gender of the terms \aram{ארז} (masc.) and \aram{תמרא} (fem.) which correspond to the gender of the characters. Although grammatical gender does not correspond to natural gender, the identification of Sarai with the date-palm is supported further by the use of ``Tamar'' as a feminine personal name within the Bible (Gen 38:6; 2 Sam 13:1; 14:27).

The dream itself provides an allegorical vision that credits the date-palm (Sarai) with saving the cedar (Abram) from the people seeking to destroy it. Although there is a question whether the beginning of \col{19}, 14--15 should read \aram{והא אזר חד ותמרא כחדא חמחו מן שרש חד} ``a cedar and a date-palm \emph{growing from a single root}'' (so DJD), or \aram{ והא אזר חד ותמרא חדא יאיא שגי} ``a cedar and a date-palm [which was] \emph{very beautiful}'' (so, Fitzmyer), all editions understand \col{19}, 16 to read \aram{ארי תרינא מן שרש חד זמחנא} ``for the two of us grow from a single root.'' Thus, the purpose of the dream  is to show Abram the way that he should avoid being ``cut down and uprooted'' by the Egyptians, namely, by claiming that he and Sarai ``sprung from the same root,'' viz. are siblings. This interpretation is also offered by Abram himself in \col{19}, 19--21.

Significantly, the later \GenRabbah connects the this section of Genesis with Ps 92 and utilizes the cedar/date-palm imagery as well, albeit in a different manner. Specifically, during its treatment of the description of the plagues which God inflicted on Pharoah, \GenRabbah begins with a citation of Ps 92, ``The righteous will flourish like the date-palm (Hebrew: \he{תמר}); like a cedar (Hebrew: \he{ארז}) in Lebanon he will grow'' and this comparison to date-palms and cedars makes several digressions. First, building on the idea of righteousness, \GenRabbah observes that both cedars and date-palms are ``straight'' trees, largely without crooks and crotches.\footnote{Although the Psalm uses the typical term for ``righteousness'' (Hebrew: \he{צַדִיק}), another common biblical term for a person who acts in an upright manner is ``straight'' (Hebrew: \he{יָשָׁר}). The author of \GenRabbah seems to be playing off of this association. Furthermore, according to \GenRabbah, tall trees cast long shadows; the length of these shadows represent the fact that the reward for such righteous people will only come later.} The second digression focuses on the ability of date-palms to produce fruit (including through grafting) and the usefulness of, especially date-palms for all manner of practical concerns. \GenRabbah then extends the comparison to the whole of Israel:

\begin{quote}
As no part of the palm has any waste, the dates being eaten, the branches used for Hallel, the twigs for covering [booths], and bast for ropes, the leaves for besoms, and the planed boards for ceiling rooms, so are there none worthless in Israel, some being versed in Scripture, some in Mishnah, some in Talmud, others Haggadah. (\genrab{41.1})
\end{quote}

The final comparison makes the claim that, like dealing with Israel, climbing these tall trees is perilous. The proof, for \GenRabbah, brings us back to the verse at hand. That Pharaoh was plagued by Yahweh when he took Sarai for himself demonstrates the danger in engaging with Israel as an adversary. What is significant here is that the authors of both \ga and \GenRabbah connect Ps 92 with this section of Genesis, but importantly, they do so with different sets of interpretive ``evidence.'' In other words, the two texts agree on the \emph{fact} of the connection between Ps 92 and the story in Gen 12, but they arrive at their interpretations through different means.
%
\footnote{%
\begin{SingleSpace}Perhaps based on the Psalm's later reference to bearing children in one's old age:
\begin{quote}
    Planted in the house of Yahweh; they will flourish in the courts of our God\\
    They will still bring forth fruit in old age; they will be full of sap and green\\
    (Ps 92:14--15)
\end{quote}
\end{SingleSpace}}
%
Yet, the fact that Ps 92 and Gen 12 are explicitly connected in both \ga and \GenRabbah seems more than a coincidence. Thus, while \ga's use of the cedar/date-palm imagery may rely on some previous tradition, the dream revelation itself is best understood as an example of the author of \ga utilizing the literary tropes of his own time and place. 

% \subsubsection{Abram as Sage}
% [TODO: A Brief discussion of Abram being depicted as a sage (sought for knowledge; reads from the ``book of the words of Enoch,'' etc)]

% \subsubsection{A \emph{waṣf} about Sarai}
% [TODO: Talk about Waṣf]

\subsection{Conclusions}

The recasting of Abram's sojourn as ``diaspora,'' his conflict with Pharaoh as a court-contest along with the portrayal of Abram and Noah as dreamers can be understood in terms of social memory as the author of \ga pressing the stories of Genesis into existing literary genres. Insofar as ``genres'' can be understood as commonly understood literary conventions---a social ``contract'' of expectations between the author and her audience---they are socially defined and, for our purposes, function as what \halbwachs would call ``social frameworks.'' As Abram's sojourn in Egypt could take on new meanings within the context and social framework of diaspora Judaism during the \secondtemple period, so too the common trope of the court-contest---well-known from the book of Daniel---provided a new framework into which the story of Gen 12 could be read. Thus, Jurgens's basic premise---that these stories are ``updated'' for a new audience---takes for granted what the memory approach makes explicit: \secondtemple Jews had their own ways of thinking about the way that God interacted with the ancients, and how pious Jews acted in particular circumstances. These social frameworks provided new structures for understanding the stories that they inherited from the biblical tradition. Thus, rather than only thinking about how the author was trying to ``fix'' the biblical account, from the memory perspective we can imagine the author of \ga not only interpreting the biblical tradition, but making efforts to contextualize it within his own literary frame of reference.

\section{\ga as \psa}

% Intro about third level of discourse
While the \ga can be seen engaging with its received cultural memory through its sources and engaging with its contemporary social memory at the level of literary form and genre, the \ga also participates in the construction of cultural memory going forward. Although \emph{all} literary and cultural products can participate in constructing cultural memory, in this section, I will argue that \ga's \psgraphic form participates in this constructive act differently than other forms of literature, in particular the biblical text.% TODO: What does this mean? Tease all of this out.
\footnote{I continue to reiterate that although the term ``biblical'' is anachronistic for the late \secondtemple period, it is a usefully concise term for my purposes.}

\subsection{The Hebrew Bible as a Baseline}

The vast majority of the Hebrew Bible is narrated in the third-person omniscient and is formally anonymous. There are, of course, exceptions to this generalization, most notably within the prophetic corpus (such as Isa 6--8), the so-called Nehemiah Memoir (Neh 11--13), and perhaps works such as Deuteronomy and Song of Songs. But for the lion's share of the biblical text, the the author (and narrator) operates invisibly.

The rhetorical force of this particular authorial voice, as observed by Erhard Blum, is significant for the function of the Hebrew Bible's participation in the collective memory of the communities that claim it as their own. Although the implied author does occasionally engage directly with the reader by offering explanatory observations (for example where the author inserts phrases like ``this is why\ldots{}'' or ``\ldots{}until this day''), for all intents and purposes, the author presents as both \emph{reliable} and \emph{authoritative} without a hint of subjectivity. As Blum puts it, ``In this sense the narrative does not distinguish the depiction from the depicted.''\autocite[33]{blum_barton-etal2007} Put another way, the text does not acknowledge that it \emph{has} an author, it simply \emph{is}. The rhetorical effect of this invisible, omniscient author is to collapse the knowledge gap between the reader and the events narrated by removing the author from view. This move, according to Blum, allows the text to convey ``an unmediated truth claim which is not based on the author's distinguishable critical judgments and convictions.''\autocite[33]{blum_barton-etal2007} The effectiveness of this implied author, according to Blum, is tied to the pragmatics of the text, that is, tied to the context of the biblical narratives as scripture (though, Blum does not refer to ``scripture'' \emph{per se}). The implied audience of the biblical narratives by-and-large can be understood as group-insiders for whom the biblical text worked to reinforce group identity.

Of course, the ``unmediated truth claims'' of the biblical text \emph{were}, in fact, mediated and reinforced by those who (orally or otherwise) transmitted the tradition from one generation to another.\autocite[33]{blum_barton-etal2007} Individuals within the community---teachers and religious leaders and even parents---become the voice of the biblical text as it is passed on. In other words, one might say that the narrator of the biblical text is the community itself---its collective memory. Blum writes:

\begin{quote}
If we assume that the traditional literature was primarily transmitted through oral means, than the narrator who is speaking supplies the material with a personal presence; he is not present as an author who judges and evaluates his sources from a critical distance, but as a `transmitter' who participates in the tradition itself and is able to lend it credence through his own personality, his standing, and/or his office.\autocite[33]{blum_barton-etal2007}
\end{quote}

In other words, the authoritative claims of ``biblical'' texts are actually made by their communities and not by the text itself. Thus, the way biblical texts participate in the collective memory is determined by their \emph{use}---how their \emph{readers} frame their function and how the text relates to the collective memory. 

\subsection{On \Psy and the \Psa}

Because significant portions of the \ga are written in the first person as though written by Lamech, Noah, and Abram, \ga may be formally included in the literary category of \psy. Before moving on, however, it is worth taking a moment to clearly define what is meant by ``\psy,'' ``\psa,'' and related terms.\autocites[The topic of \psy has received a large amount of very sophisticated attention in recent years. See especially][]{mroczek2016}{tigchelaar_tigchelaar2014}{reed_towsend-moulie2011}{reed_jts2009}{reed_ditomasso-turcescu2008}{najman_hilhorst-puech2007}{najman2003} In the simplest terms, \psa are texts which are fictively purported to be written by figures (typically) from the ancient past. 

The ancient use of the term \psa denoted spurious texts which Church leaders believed to be intentionally misleading about their authorship.\footnote{See esp. Eusebius's \emph{Hist. eccl.}~6.12.2 where the Bishop of Antioch, Serapion, refers to the \emph{Gospel of Peter} among the a number of works ``falsely attributed'': \gk{γάρ, ἀδελφοί, καὶ Πέτρον καὶ τοὺς ἄλλους ἀποστόλους ἀποδεχόμεθα ὡς Χριστόν, τὰ δὲ ὀνόματι αὐτῶν   ψευδεπίγραφα ὡς ἔμπειροι παραιτούμεθα, γινώσκοντες ὅτι τὰ τοιαῦτα οὐ   παρελάβομεν}. ``For we, brothers, accept both Peter and the other apostles as Christ, but we skillfully reject those falsely ascribed writings, knowing that they were not handed down to us.''} Thus, the term has tended to carry a somewhat negative connotation, even when such a connotation is not warranted. Implicit in the negative use of the term is the assumption that ``false'' attribution was malicious, or at the very least intentionally misleading. Yet, the number of (esp.~Jewish) \psgraphical texts discovered within the past century provide good reason to question the assumption that pseudonymous authors' intentions were to deceive their readers.\autocites[53--58]{mroczek2016}[See also][]{reed_jts2009} On the contrary, the sheer number of \psgraphical works now known to us suggests that the historical reality and social function of \psgraphical works was not simply a matter of being ``falsely attributed.''

At the other end of the spectrum, because so many early Jewish texts seem to fall into the category of \psa, in some scholarly discourse, the term ``\psa'' has become generalized to encompass any text written in around the turn of the era which did not make it into the canon of rabbinic Judaism or early Christianity. Bernstein observes, for example, that although the first volume of James Charlesworth's two-volume \emph{Old Testament Pseudepigrapha} contains a number of formally \psgraphic works, the second volume includes many which do not meet the formal definition of \psa.%
%
\footnote{\cite[2]{bernstein_chazon-etal1999}. See also \fullcite{charlesworth_OTP}}
%
This expansive practice is not particularly helpful for clarifying the term and so I will attempt to restrict my usage to a more clearly defined set of criteria.

Moshe Bernstein, in his discussion of the phenomenon of \psy distinguishes between ``authoritative'' \psy and ``decorative'' \psy.\footnote{He also identifies a third form, ``convenient'' \psy which is located somewhere between the two. \cite[3--7]{bernstein_chazon-etal1999}.} By ``authoritative'' \psy, Bernstein refers to texts that \emph{portray themselves} as being written by a particular figure. Portions of \firstenoch (in particular the latter three books, Astronomical Writings [72--82], Dream Visions [83--90], and the Epistle of Enoch [91--107]), which present themselves as if they were written by Enoch himself, are prime examples of ``authoritative'' \psy. The \ga, too would fall into this category. Psalm 23, on the other hand, although attributed to David, was presumably not \emph{actually} written by David. Moreover, whoever did write Ps 23, (again, presumably) did not intend to write it \emph{as if} it had been written by David. Rather, the Psalm was simply \emph{attributed} to David, along with many others, in part due to the tradition of David being a musician.\footnote{See, for example 2~Chr~23:18 and Ezra~3:2,10. \cite{mays_interpretation1986}; \cite{sarna_stein-loewe1979}} Thus, Ps 23 could be classified as ``decorative'' \psy. Thus, the difference between ``authoritative'' and ``decorative'' \psy can, in some sense, be boiled down to the notoriously difficult issue of authorial intent---whether a text was \emph{intended} to be read as \psa or whether the work was anonymous, and later attributed to an explicit author.

Less clear-cut examples, however, require a more nuanced treatment. For example, Deuteronomy is not generally referred to as among the \psa, yet, from a literary perspective, it is framed as \he{הדברִם אשׁר דבר משׁה אל־כל־ישׂראל} ``the words which Moses spoke to all Israel'' (Deut 1:1a). Although the whole narrative is not written in the first person, long sections of the book are treated as verbatim recountings of Moses's speech. Was Moses the author of Deuteronomy? Traditionally, most critical scholars have dated Deuteronomy to the late monarchic period and thus have eschewed the traditional attribution. But whether Deuteronomy was \emph{written} as \psa or just attributed to Moses after the fact is difficult to say with certainly and the matter is further complicated by the editorial processes that the book likely underwent through the centuries.\autocite[143--172]{toorn2007} What we \emph{can} say is that there are concrete literary cues within Deuteronomy which make the attribution to Moses easier. Framing Deuteronomy as ``the words which Moses spoke,'' while not formally ``\psa'' participates in the construction of memory in a similar fashion as \psa proper.

\subsection{Pseudepigrapha, \ga, and Memory Construction}

If we take seriously Blum's characterization of the way that the anonymous, third-person omniscient biblical text may have engaged with the collective memory of Israel based on formal, narratological features within the text, it stands to reason that the \ga as first-person \psy would engage that collective memory in a different way, despite the fact that the stories within the \ga are found in the book of Genesis. In other words, the literary form of the \ga affects how it relates \emph{back} to the biblical memory, and how it can be used in the further \emph{construction of} that memory.

The \psgraphic quality of \ga shapes the way that the text engages with the remembered past by describing the biblical story through the mouths of important figures.%
%
\footnote{Here ``story'' refers to the abstract sequence of actions which the narrative describes. The \emph{way} a story is recounted, on the other hand, is referred to by narratologists as \emph{narrative discourse.} Thus, the \ga's change from third-person omniscient to a \psgraphical first-person narrative can be understood as a change in \emph{narrative discourse} which, broadly, retains the same \emph{story} as that of the biblical text. See \cite[13--27, esp. 18--19]{abbott2008}.}
%
This explicitness changes the way that the reader understands how the text fits into the collective memory by shifting the locus of authenticity onto the text's putative author and away from the mediating figures within the community. In other words, as an example of \psy, the \ga can be thought of as a set of fictional \emph{primary sources} that bypass the received tradition. As these sources are used and enter into the discourse of the broader biblical memory, they are able to function not simply as ``alternate'' versions of events but as qualitatively distinct contributions to the tradition as it is passed on to the next generation.\footnote{On analogy to Hindy Najman's notion of ``Mosaic Discourse,'' here I am saying that the \ga is participating in a broader ``biblical'' discourse insofar as it participates in discourses surrounding Lamech, Noah, and Abram. See \cite[1--40]{najman2003}.}

Of course, referring to \psy as ``fictional primary sources'' may overstate these texts' importance or otherwise misunderstand how ``authentic'' these texts were thought to be by various and sundry religious groups in antiquity. On the one hand, it could be that readers understood that such novel fictional adaptations took certain artistic license with their biblical \emph{Vorlagen}. By way of analogy, modern adaptations of biblical narratives into film are expected to deviate to a certain degree from their source material, despite the fact that the Hebrew Bible remains a sacred, authoritative text for many modern Jews and Christians. Such adaptations are not, typically, understood to be superseding the Bible because viewers understand intuitively that there is a qualitative difference between their scriptures and a movie. On the other hand, there certainly are examples of \psgraphical texts which ultimately \emph{did} become authoritative for certain religious groups.%
%
\footnote{For example, the Ethiopian Orthodox Church includes \firstenoch among its scriptures. Tobit, too may, under certain rubrics, be considered \psa, which is included within the Roman Catholic and Eastern Orthodox deuterocanon. Insofar as deutero- and trito- Isaiah were penned as is written by Isaiah, they too could be considered \psa. And, of course, a number of the so called ``disputed'' Pauline letters within the Christian New Testament likely were not penned by Paul and are properly \psgraphical.}
%
My point here is not to suggest that there were multiple ways to understand \psgraphical writing in antiquity so much as to point out that discussions of ``false'' or ``authentic'' attribution are generally from later periods and do not tell us anything meaningful about \emph{why} such a text was written or \emph{how} it would have been understood by its original readers.

The \ga, of course, was never considered ``scripture'' so far as we know, but that does not mean that it did not participate in the broader biblical memory, even if only in the popular imagination. But even at the level of the popular imagination---even as an entertaining fiction---the \ga participated in how its society conceived of the Genesis narratives. Regardless of whether the memoirs in \ga were thought to be ``authentic,'' they represent both an interpretive understanding of biblical memory and an original contribution to that memory.\section{Conclusions}

As I have demonstrated, \ga may be understood to have taken part in three discrete mnemonic processes: 1) the reception of cultural memory, 2) the reshaping of memory by contemporary social frameworks, and 3) the active construction, codification, and reintegration of memory for future transmission. 

First, \ga functions as the recipient of cultural memory through its engagement with what I refer to as  ``biblical memory.'' I argued that \ga drew from more than just the biblical text and instead drew from a whole constellation of traditions and stories surrounding the early figures of Lamech, Noah, and Abram. Although the nature of the relationship(s) between \firstenoch, \jub, and \ga is not well understood, what is clear is that the cultural memory that surrounded the book of Genesis---the biblical memory of Genesis---was more broad at the time that the \ga was composed than simply the text of Genesis. The cultural memory from which \ga drew included additional traditions adjacent to the text of Genesis that we know from \jub and \firstenoch whether or not \ga itself drew from the \emph{texts} of \jub and \firstenoch and vice versa.

% Second, I will discuss the ways that \ga was shaped by the social frameworks which inherited it through a discussion of literary genre and shared formal characteristics with contemporary texts.

But the presentation of these traditions was not a straight-forward synthesis of their content. The author of the \ga utilized generic and thematic elements common to the social location in which it was written. Although the account of Abram's encounter with Pharoah in Gen 12 is a rather anemic narrative, the \ga does not simply fill-out missing details but recasts the final confrontation as a court-contest in the tradition of Daniel and Joseph. Even the Abraham/Abimelech doublet in Gen 20, although a more detailed narrative, cannot account for this transformation. Instead, I have proposed that the utilization of the court-contest (as well as the depiction of Abram as a dreamer and his sojourn as diaspora) was a way for the author of the \ga to not only make his narrative entertaining, but to fit it into the extant social frameworks (read: genres) of the late \secondtemple period.

Finally, I discussed how \ga participated in the construction of cultural memory through its use of \psgraphical discourse. By participating in the genre of \psy, the author of the \ga engaged in the further construction of biblical memory by presenting the text of the \ga as a first person narrative. Although we cannot know specifically how the \ga was received by its audience in antiquity, the fact that it presents as a ``primary source'' for the stories of Genesis (or, more precisely, the stories which participate in the biblical memory of Genesis) was meant as a queue to the reader for how to understand the \ga's claim to authority, whether that claim was minimal (as with a modern film-adaptation where the audience expects certain artistic license) or genuinely intended co-opt the authority of its pseudonymous author (as with Paul's disputed letters).

As a product of memory, the \ga fits this three-fold schema well. Treating \ga simply or even primarily as a way of explaining the book of Genesis does not do justice to the complex and varied processes and traditions that informed the production of \ga nor adequately account for the plurality of purposes for which the \ga could have been intended. Instead, memory studies offers a way to talk about how \ga was able receive, recontextualize, and codify the received traditions about Genesis (remembered) for himself and his contemporaries.

%% Is "Remembered Genesis" a good term?

% TODO: You need to elaborate a little more on this last point.
%% TODO: Note that this approach better deals with GA as a *whole* rather than just cols. 19ff.