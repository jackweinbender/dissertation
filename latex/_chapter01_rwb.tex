\chapter{The Rewritten Bible}\label{the-rwb}

In his seminal work
\emph{Scripture and Tradition in Judaism: Haggadic Studies}, Geza Vermes
introduced the term ``rewritten Bible'' to the discussion of
Second Temple Jewish literature as part of a larger project to trace the
development of certain haggadic traditions from the late Second Temple
period into the rabbinic period. Vermes used the term rewritten Bible to
describe a number of texts which follow closely certain portions of the
biblical narrative but also augment, elide, and emend the text in ways
which produced new literary works in their own right. In this exegetical
process, ``the midrashist inserts haggadic development into the biblical
narrative'' in order to ``anticipate questions, and to solve problems in
advance.''\autocites[95]{vermes1961}[see also][]{vermes_zsengeller2014}
These interpretive traditions could be traced historically and
demonstrate a degree of interpretive continuity between the
Second Temple and rabbinic periods. In other words, according to Vermes,
the authors of rewritten Bible texts \emph{implicitly} made use of
interpretive traditions that later works such as the Talmud and Mishnah
expressed \emph{explicitly}.

Since the publication of \emph{Scripture and Tradition}, Vermes's
concept of rewritten Bible has taken on a life of its own and developed
into its own discreet area of study as scholars from various related
disciplines have reused, reinterpreted, and redefined it. The scope and
nuance of of the term rewritten Bible has shifted a great deal in the
intervening years, yet, the trajectory set by Vermes nearly sixty years
ago has remained reasonably consistent by focusing on the relationships
that exist between these unique works and the scriptural texts that
inspired them.

In this chapter I will trace the emergence and evolution of the concept
of rewritten Bible from Vermes's use in \emph{Scripture and Tradition}
to the present focusing on three key questions and ideas which have
shaped the scholarly discourse around rewritten Bible studies: 1) what
works should fall under the rubric of rewritten Bible? 2) the
terminology surrounding rewritten Bible, and 3) whether rewritten Bible
constitutes a literary genre, a process, or some combination of the two.
\textbf{Then I'm gong to say something else about what I think}

\section{Scripture and Tradition}\label{scripture-and-tradition}

The primary purpose of \emph{Scripture and Tradition} was not to offer a
clear definition of the term ``rewritten Bible,'' but to lay the
groundwork for the historical, diachronic, study of aggadic traditions,
of which rewritten Bible makes only a small
part.\autocite[3]{vermes_zsengeller2014} As Vermes recounts, prior to
the mid-twentieth century, the prevailing approach to the study of
aggadic exegesis was to treat the aggadah as originating during the
Tannaitic period. The aggadah were viewed as ``the result of the
adoption, and anonymous repetition, of popular interpretations by
favourite preachers,'' \autocite[3]{vermes1961} the earliest of which
were from the second century CE and were represented by Targums Onkelos
and Jonathan. Furthermore, studies of ancient Jewish literature at this
time focused on texts which modern Judaism considered authentic. As a
result, a good number of earlier texts---for example, the apocrypha,
pseudepigrapha, and sectarian texts---were often categorically excluded
from discussions of the origins of aggadic
exegesis.\autocite[2]{vermes1961}

A series of publications and discoveries beginning in the 1930's,
however, began to undermine the notion that these early exegetical
traditions began in the second century CE. Vermes credits this
broadening of aggadic studies to a series of major studies and
discoveries such as Rappaport's \emph{Agada und Exegese bei Flavius
Josephus},\autocite{rappaport1930} Paul Kahle's Schweich Lectures at the
British Academy on the Cairo Geniza (given in 1941, published
1947),\autocite{kahle1947} Kisch's new text edition of Ps. Philo's
Liber antiquitatum biblicarum (1949),\autocite{kisch1949} the discovery
of the Dead Sea Scrolls (1948) and Codex Neofiti (1956), as well as (and
perhaps especially) Renée Bloch's work on midrash
\autocites{bloch1954}{bloch1955_repr}[3--7]{vermes1961}. The overarching
theme among these works was the evidence for continuity between biblical
interpretive traditions prior to the second century, and later aggadah.
For example, Vermes notes that Rappaport's work on Jewish Antiquities
identified substantial overlaps between Josephus's text and Rabbinic
aggadah and suggested, therefore, that Josephus had drawn from an
already living tradition of interpretation. The implication of his
suggestion is that the aggadah of the second century were not novel
exegetical works, but were themselves products of earlier exegetical
traditions.

Building on these recent advancements, the explicitly stated purpose of
\emph{Scripture and Tradition} was to push the field beyond synchronic
analysis of haggadah toward diachronic, historical analyses to trace the
development of these exegetical traditions \autocites[1]{vermes1961}[See
also][]{bloch1955_repr}. The book is eight chapters long and is divided
into four parts.

The first three chapters of \emph{Scripture and Tadition} make up the
first part of Vermes's study and is entitled ``The Symbolism of Words.''
In this section, Vermes focuses on the way that individual words and
ideas take on localized symbolic meanings which are then applied
globally to the interpretation of the Bible. In his first chapter, he
notes the divergent treatment of Gen 44:18--19 among ancient
commentators and proceeds through a synoptic study of this passage in
the Fragmentary Targum, Targum Neofiti, and the Tosefta of Targum
Yerushalmi. He concludes that the Fragmentary Targum represents the most
primitive work, whose interpretive strategy is essentially
inner-biblical (by harmonizing Gen 44 and 49 with Exod 7--9), followed
by the Tosefta, which seems to depend on the Fragmentary Targum but
takes a different interpretive stance, and finally Targum Neofiti, which
combines the two. In his second chapter, Vermes examines the way that
the word ``Lebanon'' came to be used symbolically in the Hebrew Bible
and other jewish literature and how those symbolic meanings developed,
particularly the association of Lebanon with Jerusalem and the Temple.
He identifies the Song of Songs as the intermediary text which helped to
establish this tradition within post-exilic Judaism, noting that the
book occupies the unique position as the only biblical text which
clearly uses the name Lebanon symbolically for the Temple. Importantly,
Vermes shows that the symbolic use of Lebanon to represent Jerusalem and
the Temple is rooted in \emph{biblical} exegesis. This is a key idea for
Vermes because it establishes a continuity between the production of the
biblical text and its later interpretation. In chapter three, Vermes
examines other words which take on symbolic meaning in later Jewish
texts (``lion,'' ``Damascus,'' ``\emph{Meḥoqeq},'' and ``Man''), and
shows the relative similarity of the process between the
Dead Sea Scrolls texts and the targumic and midrashic materials.

It is in the second part of \emph{Scripture and Tradition} that the
topic of Rewritten Bible is addressed most directly. This
section---titled ``The Rewritten Bible''---covers two chapters (chapters
four and five), both focusing on the figure Abraham and the aggadic
traditions surrounding his life.

In chapter four, Vermes embarks on what he calls a ``retrogressive
historical study'' by which he means beginning with the traditions in
their later, more developed forms, and working back toward their
origins. In this case, Vermes begins with the 11th century CE text
\emph{Sefer ha-Yashar} (and the varied extra-biblical traditions
contained therein) then works backward to identify sections of the text
which exhibit earlier traditions, for example, in Targums, Josephus,
Jubilees, (Ps.) Philo, and others. The purpose of this chapter is to
demonstrate that even late texts can contain valuable information about
earlier methods of exegesis. As Vermes puts it, ``{[}Sefer ha-Yashar{]}
manifests a direct continuity with the corresponding tradition of the
time of the second Temple, but reflects also the influence of the
haggadah of the Tannaim and Amoraim.''\autocite[95]{vermes1961}

In chapter five, Vermes proceeds with a ``progressive historical
study,'' by which he means a study beginning with the oldest materials
and working forward. Still focusing on the figure Abraham, Vermes treats
in detail the relationship between Gen 12:8--15:4 and cols. 19--22 of
the Genesis Apocryphon. Vermes treats Genesis Apocryphon as ``the most
ancient midrash of all''\autocite[124]{vermes1961} and views it as the
``lost link between the biblical and the Rabbinic midrash''
\autocite[124]{vermes1961}. As I understand it, the Genesis Apocryphon,
for Vermes, occupies a unique position just one step removed from
inner-biblical exegesis. The task of Genesis Apocryphon's author was
``to make the biblical story more attractive, more real, more edifying,
and above all more intelligible'' and he accomplishes this through a
variety of means \autocite[125]{vermes1961}. The work of
Genesis Apocryphon's author, therefore, was to ``{[}reconcile{]}
unexplained or apparently conflicting statements in the biblical text in
order to allay doubt and worry.''\autocite[125]{vermes1961} According to
Vermes, the interpretation of Genesis is ``organically bound'' to the
text of Genesis and the additions that \emph{were} made sprung from the
interpretation of the Bible itself and not whole-sale from the mind of
the author \autocite[126]{vermes1961}. Where texts like Jubliees sought
to systematically advance a theological vision, according to Vermes, the
author of Genesis Apocryphon intended to simply ``explain the biblical
text,'' calling it illustrative of ``the unbiased rewriting of the
Bible.''\autocite[126]{vermes1961}

The third part of \emph{Scripture and Tradition} is titled ``Bible and
Tradition'' and is composed of a single chapter engaging in a lengthy
analysis of the traditions surrounding the seer Balaam from Number
22--24. Vermes Observes that while the majority of post-biblical texts
treat Balaam as a villain, in \emph{LAB} he is treated as a sort of
tragic hero \autocite[173]{vermes1961}. The more traditional portrayal
of Balaam as a wicked prophet began within the nexus of biblical
tradition itself. The various documentary strata of the Balaam story
cast the prophet in differing lights, and it is the final stratum, the P
layer, got the final say---within the biblical text---about him. Vermes
points out, however, that ignoring the Preistly additions yields a story
somewhat similar to that of \emph{LAB}. Thus, Vermes concludes that the
exegetical traditions found in the later Targums and rabbinic works are
simply the continuation of the exegetical strategies employed within the
Bible itself, which he calls ``biblical midrash or haggadah''
\autocite[176]{vermes1961}.

The last two chapters make up the final section of Vermes's study,
titled ``Theology and Exegesis,'' and push the discussion to include
early Christianity. Chapter seven is entitled ``Circumcision and Exodus
4:24--26'' but offers a subtitle of ``Prelude to the Theology of
Baptism,'' which gives some hint at the ultimate, if tacit, goal of the
chapter. Discussing the topic of circumcision in {[}Ex
4:24--26{]}{[}bib@exod 4:24-26{]} and its treatment among the early
exegetes, Vermes's primary observation is simply that the theology of
circumcision and the exegetical traditions which surrounded it, were
affected by historical forces and theological ideologies. For instance,
he claims that Jubilees omitted the rather odd statement that God was
going to kill Moses---who was saved by the circumcision of his son by
Zipporah---because ``It was impossible for its author to accept that God
tried to kill Moses as it was for him to believe that Moses neglected to
circumcise his son on the eighth day after his
birth.''\autocite[185]{vermes1961} Similarly, he notes that after the
Bar Kokhba rebellion, the practice of circumcision was outlawed and so,
``it is not surprising, therefore, to find the spiritual authorities of
Palestinian Judaism emphasizing the greatness and necessity of this
essential rite, and explaining away \ldots{} every possible biblical
excuse for delaying the circumcision of their
children.''\autocite[189]{vermes1961} He ends the chapter by suggesting
that the early Christian association of baptism with circumcision
(citing Rom 4:3--4 and Col 2:11--12) was enabled by the traditional
Jewish association of circumcision withe blood sacrifice (``the Blood of
the Covenant'')\autocite[190]{vermes1961}. That Paul (to whom he
attributes both Romans and Colossians) associated baptism with
circumcision therefore, was ``not due, therefore, to his own insight,
but springs directly from the contemporary Jewish doctrine of
circumcision which he adopted and adapted.''\autocite[191]{vermes1961}

Vermes makes a similar move in chapter eight, entitled ``Redemption and
Genesis XXII: The Binding of Isaac and the Sacrifice of Jesus.'' In it,
he compares a number of ancient works' treatment of the Aqedah and
demonstrates how the (near-) sacrifice of Isaac became a prototype for
the entire sacrificial system in later judaism. The sacrifice of animals
in the Temple functioned as a ``reminder'' to God of the faithfulness of
Abraham. Furthermore, he shows the ways the tradition grew to focus on
the willingness of Isaac to be sacrificed and his function as a
proto-matryr. Thus, he ends the chapter by addressing the New
Testament's portrayal of Jesus as a willing sacrifice to God and its
putative relationship to the Aqedah. Vermes makes the case that the
redemptive theology of the NT---typically attributed to Paul---was not
original to him. He writes:

\begin{quote}
For although {[}Paul{]} is undoubtedly the greatest theologian of the
Redemption, he worked with inherited materials and among these was, by
his own confession, the tradition that ``Christ dies for us according to
the Scriptures.''\autocite[221]{vermes1961}
\end{quote}

He then proceeds to push the origin of this theology back further into
the first century CE, and, in rather dramatic fashion, suggests that the
introduction of the Aqedah motif into Christian theology---by means of
the Suffering Servant---may have been by Jesus himself
\autocite[223]{vermes1961}.

Vermes concludes the chapter by discussing the Aqedah and the Eucharist.
Just as the whole sacrificial system pointed back toward the binding of
Isaac in targumic exegesis, the eucharistic rite likewise was
intended---according to Vermes---to point back to Jesus's redemptive
sacrifice. Thus he concludes:

\begin{quote}
Although it would be inexact to hold that the Eucharistic doctrine of
the New Testament, together with the whole Christian doctrine of
Redemption, is nothing but a Christian version of the Jewish Akedah
theology, it is nevertheless true that in the formation of this doctrine
the targumic representation of the Binding of Isaac has played an
essential role.
\end{quote}

\begin{quote}
Indeed, without the help of Jewish exegesis it is impossible to perceive
any Christian teaching in its true
perspective.\autocite[227]{vermes1961}
\end{quote}

The arc of Vermes's study, therefore, is meant to establish a continuity
between the earliest traditions of biblical interpretation with the
later traditions of both Rabbinic Judaism and Early Christianity and to
trace the evolution of those traditions historically.

\subsection{Vermes's Use of Rewritten Bible}\label{vermess-use-of-rwb}

The fact that Vermes spent so little time explaining precisely what he
meant by the term RwB bears witness to the fact that Vermes thought the
term was self-explanatory. Vermes makes this sentiment clear in his
short retrospective on the origins of the term, expressing shock over
the debate that his term prompted and the scholarly confusion
surrounding it. He writes:

\begin{quote}
The notion {[}of RwB{]}, which over fifty years ago I thought was quite
clear, seemed to the majority of the more recent practitioners nebulous
and confused, and lacked methodological
precision.\autocite[3]{vermes_zsengeller2014}
\end{quote}

Only a few scholars, according to Vermes, managed to remain true to his
original vision.\footnote{He specifically references
  \textcite{alexander_carson-williamson1988} and
  \textcite{bernstein_textus2005}.} Instead, many subsequent studies,
according to Vermes, ``moved the goalposts'' to better ``suit the
interest of their inquiry.''\autocite[4]{vermes_zsengeller2014} Yet, one
cannot help but push back against Vermes here as scholars' desire to
narrow the scope of the term is, I think, a reasonable impulse. After
all, Vermes's use of RwB covers texts written in several languages,
across centuries, in no particular geographical region, and, while all
the texts are ``narratives,'' the formal similarities between
Genesis Apocryphon, Jewish Antiquities, Jubilees, and the
Palestinian Targums stop there. Vermes specifically laments the
narrowing of the term RwB to primarily focusing on the Dead Sea Scrolls
texts. Of course, when \emph{Scripture and Tradition} was first
published in 1961 (Vermes notes that the manuscript, in fact, was
submitted for publication in 1959), only a small portion of the scrolls
were published or accessible to more than a few specific scholars. But
the field's subsequent preoccupation with the Qumran material, he
suggests, is misguided.\footnote{I am sympathetic to what Vermes
  perceived as ``moving the goalposts''---I think the context and
  purpose of how he used the term RwB is often ignored---but it is worth
  pointing out that the reason the term RwB is so often applied to the
  Qumran texts likely has less to do with a conscious, scholarly effort,
  and more to do with the fragmentation of the various fields that deal
  with the texts in question. A scholar with a background primarily
  focused on the New Testament or Hebrew Bible may not be as familiar
  with the texts and traditions of rabbinic Judaism that Vermes
  discusses in \emph{Scripture and Tradition}.}

This perception is---it seems to me--a bit over-blown. On the one hand,
Genesis Apocryphon and the Temple Scroll receive a lot of scholarly
attention, but Jubilees and Jewish Antiquities do as well. Even so,
whatever narrowing of the discussion of RwB toward the Qumran scrolls
has occurred is likely symptomatic of the ``methodological
{[}im{]}precision'' attributed to \emph{Scripture and Tradition} and
Vermes's use of rewritten Bible. For example, Vermes's inclusion of the
medieval Sefer ha-Yashar muddies the waters for those who wish to
discuss RwB as a process of scriptural interpretation which can be
situated historically. On the other hand, his inclusion of the
Palestinian Targums makes sense diachronically, but formally, the
Targums are translations and not ``new compositions'' in the same sense
that Jubilees or Genesis Apocryphon are. Within
\emph{Scripture and Tradition}, of course, Vermes treats these texts
with due care and nuance---in the case of Sefer ha-Yashar, he endeavors
to show that traditions preserved in the text can be traced back to the
Second Temple period---but the fact that Vermes sought to situate
haggadic developments diachronically while implementing a category that
spanned such broad socio-religious (Qumran, Early Christian, Rabbinic,
Medieval), chronological (1st -- 12th centuries CE), and literary
(translations, narrative, revelatory/apocalyptic, history?) horizons has
given some scholars a reasonable challenge when attempting to use the
term in their own work. Thus, simply because Vermes set the
``goalposts'' (to suit his \emph{own} thesis, I might add), does not
mean that others cannot or should not move them when appropriate, though
hopefully along with a well-reasoned explanation for the change.

\section{Rewritten Bible, Rewritten Scripture, and Parabiblical
Texts}\label{rwb-rewritten-scripture-and-parabiblical-texts}

Since Vermes coined the term RwB a number of scholars have suggested
that the term be modified to more accurately reflect the (now, well
established) fact that there was no ``Bible'' in the late Second Temple
period and that many of the works that would eventually make up the
Hebrew Bible did not have stable textual witnesses which could be
meaningfully ``rewritten.'' Because of these difficulties, scholars
have, in recent years, suggested alternate designations for the
phenomenon under investigation, the most widely used of which is
``rewritten \emph{scripture}.'' While original term RwB was a product of
its time which took for granted the existence of a canonical ``Bible''
that more-or-less resembled the Bible used by the rabbis in the early
centuries CE, the term rewritten scripture was intended to correct what
scholars perceived as an anachronistic reference to a canon of scripture
during the late Second Temple period.\footnote{\textcite[58--59]{campbell_zsengeller2014}.
  See also \textcite{ulrich_mcdonald-sanders2002} and
  \textcite{ulrich_zsengeller2014}}

\section{Defining the Boundaries of
Rewritten Bible}\label{defining-the-boundaries-of-rwb}

Early adopters of the Vermes's taxonomy experimented with applying the
term RwB to a wide range of Second Temple Jewish literature and the
discussion about which texts should fall under the rubric of RwB has
continued into the present. Insofar as ``rewritten'' texts can be
measured by how closely they resemble their \emph{Vorlagen}, defining
the boundaries of RwB focuses on which texts are \emph{too far} from
their \emph{Vorlagen} to meaningfully be considered ``rewritten,''
forming the ``upper bound'' and texts which are \emph{too close} to
their \emph{Vorlagen} to be considered distinct literary works, forming
the ``lower bound.'' At the upper bound, for example, the \emph{Book of
the Watchers} and the \emph{Book of Giants} clearly are rooted in the
biblical text, yet most scholars do not consider them sufficiently
dependent on the text of Genesis to be considered ``rewritten.'' They
take Genesis 6:1--4 as a point of departure, but do not return to the
biblical text in a meaningful way. Conversely, at the lower bound, the
Samaritan Pentateuch and the 4QReworkedPent, although they certainly
modify their \emph{Vorlagen} (and in that sense are ``rewritten''), are
more often considered examples of alternate textual ``editions'' rather
than rewritten works. Likewise, the Targums and LXX, as translations,
are frequently excluded from discussions of RwB at the lower bound
because they were meant to be perceived as the same literary work as
their \emph{Vorlagen}.

\subsection{The Upper Bound}\label{the-upper-bound}

Vermes's use of the term RwB grew out of the concrete examples of texts
that exhibited the sorts of exegetical practices relevant to later
aggadic traditions. As others adopted the term, however, the question of
how to abstract the concept to something meaningful that could be
applied to other texts was explored by a number of scholars. These early
applications of the term RwB, like Vermes's use, did not tend to carry a
technical nuance and instead focused on the ways that numerous texts
reappropriated biblical stories, figures and themes in their own works.

In his 1984 article ``The Bible Rewritten and Expanded,'' George
Nickelsburg discusses a number of texts which are ``very closely related
to the biblical texts, expanding and paraphrasing them and implicitly
commenting on them.''\autocite[89]{nickelsburg_stone1984} We should note
that, although the article does deal with RwB, it includes a discussion
of texts which even Nickelsburg does not consider ``rewritten'' (as the
title indicates) discussing texts which introduce wholly new material
into the traditions of the Bible
\autocite[89--90]{nickelsburg_stone1984}.

Nickelsburg does, however, provide a list of texts which he loosely
describes as examples of biblical rewriting: \emph{1 Enoch}, \emph{Book
of Giants}, Jubilees, Genesis Apocryphon, Jewish Antiquities, the Books
of Adam and Eve (\emph{Apocalypse of Moses}, \emph{Life of Adam and
Eve}), and some Hellenistic Jewish Poets including Philo's \emph{On
Jerusalem}, Theodotus's \emph{On the Jews}, and the \emph{Exagoge} by
one ``Ezekiel the Poet of Tragedies.'' Compared to Vermes's list,
Nickelsburg's represents a maximalist understanding of the RwB
phenomenon. The inclusion, especially, of \emph{1 Enoch} illustrates his
tendency to include texts that build off of the biblical text (in this
case, Genesis 6:1--4), but do track with the biblical narrative for long
stretches.

One of the more interesting contributions that Nickelsburg makes to the
conversation is his idea that biblical rewriting followed a trajectory
from rewriting smaller units of the Bible---involving short stories that
deal with particular events from the biblical text---to longer, more
systematic, treatments which span multiple biblical books. His treatment
of 1 Enoch (which is, at least in part, the earliest text that he deals
with) is illustrative of this approach. Rather than dealing with 1 Enoch
as a whole, Nickelsburg addresses the various rewritings of the flood
narrative throughout 1 Enoch as well as in the Book of Giants (which is
not formally a part of 1 Enoch, but has a clear connection to the work).
Setting aside for the moment that 1 Enoch is a composite work, we can
appreciate that the flood story from Gen 6--9 is retold and to varying
degrees reinterpreted throughout 1 Enoch.\footnote{By my count, there
  are six retellings of the flood in 1 Enoch: 6--11; 54:7; 64--69;
  83--84; 86--89; and 106--107.}

Although Nickelsburg generally accepts that the rewritten texts
``comment'' on the Bible, he notes that the posture toward the biblical
text is also not uniform even among the agreed upon RwB texts. He notes,
for example, that while the author of Jubilees's concerns are largely
halakhic and the book makes explicit reference to the biblical text, the
authority assumed by the author of Jubilees does not (at least
rhetorically) originate in the exposition of the Torah, but in the
``immutable heavenly
tablets.''\autocite[100--101]{nickelsburg_stone1984} Nickelsburg thus
states:

\begin{quote}
This process of transmitting and revising the biblical text reflects a
remarkable view of Scripture and tradition. The pseudepigraphic
ascription of the book to an angel of the presence and the attribution
of laws to the heavenly tablets invest the author's interpretation of
Scripture with absolute divine
authority.``\autocite[101]{nickelsburg_stone1984}
\end{quote}

In contrast, Genesis Apocryphon seems to have very little interest in
halakhic matters and instead seems to just elaborate on the story by
giving detailed geographic information and providing the reader with
more dramatic characters \autocite[106]{nickelsburg_stone1984}. Finally,
he observes that Liber antiquitatum biblicarum likewise differs with
Jubilees in its omission of halakhic matters and its ``highly selective
reproduction of the text.''\autocite[110]{nickelsburg_stone1984} This
selectivity also differs from the Genesis Apocryphon, which otherwise is
``characterized by the addition of lengthy non-biblical
incidence.''\autocite[110]{nickelsburg_stone1984}

Ultimately, Nickelsburg differs from Vermes mainly in the way he views
the Bible during the late Second Temple period. Although Nickelsburg
observes that the preoccupation with certain texts suggests that they
were held in high regard, he does not have the same interest in tying
the exegetical practices of, for example, Jubilees, with earlier
inner-biblical or later haggadic traditions. Because Nickelsburg treats
RwB as a process, he is able to highlight the fact that, e.g., 1 Enoch
does indeed ``rewrite'' certain pericopae from Genesis despite the fact
that the whole book (which, we should note, is a composite text to begin
with) does not maintain a ``centripetal'' relationship with the biblical
narrative.

Daniel Harrington's 1986 contribution entitled ``Palestinian Adaptations
of Biblical Narratives and Prophecies I: The Bible Rewritten
(Narratives),'' adopts the term RwB to talk about texts produced around
the turn of the era by Palestinian Judaism that ``take as their literary
framework the flow of the biblical text itself and apparently have as
their major purpose the clarification and actualization of the biblical
story.''\autocite[239]{harrington_kraft-nickelsburg1986} In this regard,
he follows Vermes closely in how he imagines RwB to function. Yet,
compared to Vermes, he operates with a slightly expanded list of
rewritten texts. In addition to Jubilees, Genesis Apocryphon, Ps.
Philo's Liber antiquitatum biblicarum and Josephus's Jewish Antiquities,
he also includes the \emph{Assumption of Moses} and 11QTemple
(Temple Scroll). Furthermore, he makes a point to suggest that a number
of other texts may be able to be included in the list, including
\emph{Paralipomena of Jeremiah}, \emph{Life of Adam and Eve/Apocalypse
of Moses}, and \emph{Ascension of Isaiah}. Harrington's major
contribution is his explicit rejection of RwB as a category or literary
genre (more on this, below) in favor of a process-oriented approach.
Because of this fact, Harrington takes a broad view of rewriting and
allows, to some degree, that this process be understood similar to a
reception history (although, this is my term, and not his).

Harrington's inclusion of the Temple Scroll marked a significant
deviation from Vermes's use of the term by including non-narrative
material under the rubric of RwB. While several of Harrington's other
suggested text are not considered RwB by the many scholars, the
inclusion of other non-narrative texts, in particular the Temple Scroll,
has gained wide acceptance.\autocite{bernstein_textus2005}

Building on the notion that RwB could also include non-narrative
material, George Brooke, in a more recent treatment of the topic,
defines RwB as ``any representation of an authoritative scriptural text
that implicitly incorporates interpretive elements, large or small, in
the retelling itself.''\autocite[777]{brooke_schiffman-vanderkam2000}
Adopting a ``loose'' definition of the term Brooke includes in his
discussion biblical texts that rewrite other biblical texts such as
Deuteronomy and Chronicles in addition to examples of texts which
``rewrite'' portions of each of the major division of the Hebrew Bible,
most of which were found at Qumran.\autocite[Brooke categorizes the
texts as follows: Reworked Pentateuchs, Rewritten Pentateuchal
narratives, Rewritten Pentateuchal laws, Rewritten Former Prophets,
Rewritten Latter Prophets, and Rewritten
Writings.][778--780]{brooke_schiffman-vanderkam2000}

The purposes of rewriting, according to Brooke, are manifold, but in
each case the (re)writer augmented or repurposed an authoritative base
text for some new context. He writes:

\begin{quote}
The rewriting seems to have a variety of purposes, among which are the
following: to improve an unintelligible bas text, making it more
comprehensible (11Q19); to improve a text by removing
inconsistencies---often through internal harmonization
(4QpaleoExod\textsuperscript{m}); to justify some particular content by
providing explanations for certain features in the base text (1QapGen);
to make an authoritative text serve a particular function, perhaps in a
liturgical setting (4Q41); to encourage the practice of particular legal
rulings (Jubilees); and to make an old text have contemporary appeal
(Temple Scroll).\autocite[778]{brooke_schiffman-vanderkam2000}
\end{quote}

While I am sympathetic to the more maximalist approaches of Nickelsburg,
Harrington, and Brooke, none of these treatments offer any concrete
criteria for delineating between RwB and texts that merely allude to
biblical stories. Philip Alexander has suggested that certain works
which are primarily ``expansive'' (the Book of Giants, the Book of Noah)
should not be considered RwB because their relationship to the biblical
text is ``centrifugal''---that is, they take the biblical text as a
point of while formally RwB texts show a ``centripetal'' relationship to
the biblical text---that is, they expand beyond the biblical text, but
remain tightly coupled to the text \emph{as it exists in the Bible.}
Alexander writes:

\begin{quote}
Rewritten Bible texts are centripetal: they come back to the Bible again
and again. The rewritten Bible texts make use of the legendary material,
but by placing that material within an extended biblical narrative (in
association with passages of more or less literal retelling of the
Bible), they clamp the legends firmly to the biblical framework, and
reintegrate them into the biblical history.
\autocite[117]{alexander_carson-williamson1988}
\end{quote}

This ``centripetal'' relationship to the biblical text, I believe,
should form the upper bound of what is called RwB. Therefore, for the
purposes of this study, works such as 1 Enoch, will not be treated
because they do not exhibit this close centripetal relationship. On the
other hand, I adopt a more expansive understanding of RwB than that of
Vermes and include works within the Hebrew Bible itself (Deuteronomy and
Chronicles), as well as non-narrative works such as the Temple Scroll
which, I believe, do exhibit a centripetal relationship to the biblical
text.

\subsection{The Lower Bound: Between Bible and
Rewritten Bible}\label{the-lower-bound-between-bible-and-rwb}

Another recent avenue of investigation has been to explore the
boundaries between the biblical text, editions, translations, and
rewritten biblical texts. Vermes, of course, utilized the targums
liberally in \emph{Scripture and Tradition}, but his goal was to blur
the line between post-biblical texts via the haggadah. Most scholars
treating RwB, however, are not inclined to include the targums among
RwB. But the targums---and for that matter the LXX and Samaritan
Pentateuch---do uniquely represent interpretive traditions. Furthermore,
the instability of the biblical text during the late Second Temple
period, as exhibited by the varied editions of Jeremiah found at Qumran
and other liminal texts, such as 4QReworkedP, has problematized the
question of what may have constituted ``Bible'' (or, more properly,
``scripture'') at the time.

Unsurprisingly, Emanuel Tov has been at the forefront of this
investigation. In his 1998 article, ``Rewritten Bible Compositions and
Biblical Manuscripts, with Special Attention to the Samaritan
Pentateuch,'' Tov's purpose is to specify the ``fine line between
biblical manuscripts and rewritten Bible
texts.''\autocite[334]{tov_dsd1998} By this, Tov means that he is
concerned with what I have termed the ``lower bound'' of the definition
of RwB, specifically, the distinction between a text \emph{edition} and
a distinct composition, which Tov considers ``rewritten.'' The primary
difference between these two categories of texts, according to Tov, is
not how dramatically the daughter text diverges from its parent, but the
\emph{purpose} of the daughter text \autocite[334]{tov_dsd1998}.
According to Tov, this purpose is mirrored in the putatively
authoritative status of the ``biblical'' text vis-a-vis the rewritten
text which, he says, is not authoritative (although, he seems to suggest
that this is up for debate\autocite[337]{tov_dsd1998}). For example, he
notes that the extant texts of Jeremiah, while widely divergent in
length and order, still represent ``biblical Jeremiah'' which carries
some authoritative weight. Tov is, however, careful to point out that
the nature of this authority is not clear and ``the boundary between the
biblical and non-biblical texts was probably not as fixed as we would
have liked for the purpose of our scholarly
analysis.''\autocite[335]{tov_dsd1998}

Tov makes explicit that he understands the SP as participating in this
same sort of process as the rewritten texts from Qumran, making special
mention of the 11QT\textsuperscript{a} (Temple Scroll),
Genesis Apocryphon, and Jubilees. Tov, therefore, is attempting to draw
a parallel between the sorts of exegetical additions included in these
three LXX texts and those included in the SP and the ``classical'' RwB
texts from Qumran.

The more significant contribution to this area, however, is Michael
Segal's 2005 article ``Between Bible and Rewritten Bible,'' which, in
the tradition of Alexander, attempts to enumerate a series of criteria
by which scholars can distinguish between editions of biblical texts and
so-called rewritten texts.

Segal's understanding of the role of RwB is rooted in the conviction
that a rewritten text is a ``new'' work that derives its own authority
by means of its association with a biblical text. The new composition
carries with it the purpose and any theological or ideological
\emph{Tendenzen} of the new author, but keeps piggy-backs off of the
status of the underlying text.\autocite[11]{segal_henze2005} Segal
writes:

\begin{quote}
Even though these rewritten compositions sometimes contain material
contradictory to their biblical sources, their inclusion within the
existing framework of the biblical text bestows upon them legitimacy in
the eyes of the intended audience \ldots{} the inclusion of this
material within the framework of the biblical passages under
interpretation transforms the ideas of the later writer into
authoritative and accepted beliefs.\autocite[11]{segal_henze2005}
\end{quote}

And further:

\begin{quote}
The nature of the relationship between rewritten biblical compositions
and their sources constitutes a paradox. On the one hand, the rewritten
composition relies upon biblical texts for authority and legitimacy. The
author claims that any new information included in the later work
already appears in earlier sources. But simultaneously, the insertion of
new ideas into the biblical text, ideas that may even contradict the
beliefs and concepts of the original biblical authors, undermines the
very authority that the rewriter hopes to
utilize``\autocite[11-12]{segal_henze2005}
\end{quote}

While I find Segal's characterization of RwB texts problematic, his main
contribution to the discussion are his criteria for distinguishing
between ``biblical'' and RwB texts. He distinguishes between
``external'' and ``internal'' characteristics.

\subsubsection{External Characteristics}\label{external-characteristics}

Segal's external characteristics are by far his weakest. He notes two
external characteristics of RwB texts: ``language'' and ``relationship
between the source and its revision.''

\begin{enumerate}
\def\labelenumi{\arabic{enumi}.}
\tightlist
\item
  Language: While he offers little rationale for this criterion, Segal
  categorically dismisses the possibility that any RwB text could have
  been written in a language other than its \emph{Vorlage}. Notably,
  this criterion excludes, Genesis Apocryphon, Josephus's
  Jewish Antiquities, and Ps. Philo's \emph{LAB}.\footnote{Segal
    needlessly undercuts himself here. One might wonder that if a single
    criterion categorically excludes several texts which meet all the
    other criteria, perhaps the problem is with the criterion. In his
    discussions of other criteria, he begins by giving the principle by
    which the ``edition'' would assert itself as equal to its
    \emph{Vorlage}, then contrast that with the RwB (see, esp. Expansion
    v. Abridgment). His reasoning is sound for a text edition (although
    I think the issue of textual authority and \emph{translation} is,
    perhaps, too hastily ignored in this case), and could easily be
    contrasted with, for example the Genesis Apocryphon, which meets
    nearly all of his internal criteria.}
\end{enumerate}

\begin{enumerate}
\def\labelenumi{\arabic{enumi}.}
\setcounter{enumi}{1}
\tightlist
\item
  The Textual Relationship between the Source and Its Revision: The
  underlying text must be ``visible'' in the RwB text. He uses the book
  of Chronicles as the parade example of this relationship and notes the
  caveats necessary in dealing with \emph{Vorlagen} from this period
  (i.e., it is difficult to say what is `rewritten' versus what is just
  another variant in the \emph{Vorlage}).
\end{enumerate}

Segal notes that both of these criteria, in fact, apply to textual
editions, as well as to RwB texts \autocite[20]{segal_henze2005}. In
other words, these are not ``distinguishing'' criteria, so much as the
baseline for consideration.

\subsubsection{Literary Criteria for
Rewritten Bible}\label{literary-criteria-for-rwb}

It is the ``Literary Criteria'' which Segal, ultimately, believes
provide the \emph{definition} of RwB
texts.\autocite[20]{segal_henze2005} Segal provides six internal
criteria:

\begin{enumerate}
\def\labelenumi{\arabic{enumi}.}
\item
  Scope of the Composition: ``Editions'' of texts cover the same
  material as their source. In other words, one expects an edition of
  Genesis to cover the same material as the book of Genesis; pluses and
  minuses do not stray into other works. On the other hand, rewritten
  texts ``do not generally correspond to the scope of their sources''
  \autocite[20]{segal_henze2005}. For example, he observes that Jubilees
  Covers Genesis and part of Exodus, and Chronicles covers parts of
  Samuel and Kings. Oddly, he also notes that Ps. Philo---which is not
  written in Hebrew---runs from Genesis into 1 Samuel. He writes: ``In
  all these examples the change in the scope of the composition created
  a new literary unit.'' \autocite[20--21]{segal_henze2005}
\item
  New Narrative Frame: Several of the RwB texts include a framing
  narrative. His examples include the Temple Scroll and Jubilees, both
  of which re-frame the ``biblical'' material. In the case of both
  works, the Torah is assumed and the new work presumed to be a
  reflection of a second, direct revelation of the law to Moses, albeit
  by different means (and fragmentary, in the case of the
  Temple Scroll). In Jubilees, the angel of the Presence revealed this
  ``second Torah'' during Moses's second ascent (Exod 24). On the other
  hand, the Temple scroll seems to begin in Exod 34
  \autocite[22]{segal_henze2005}.
\item
  Voice: While biblical narratives are generally written in a
  ``detached'' third person style, Segal observes that both Jubilees and
  the Temple Scroll ``change the voice of the narrator throughout''
  \autocite[22]{segal_henze2005}. As far as I can tell, what Segal means
  is that in these RwB texts, certain evens which are narrated in the
  third person in the biblical text are re-framed as, for instance,
  direct discourse in the first person by an angel, or even by
  God.\footnote{This may seem like a minor quibble, but the ``narrator''
    has a distinct and technical meaning in narrative criticism which
    should be maintained. I would note, however, that this sort of
    reframing is not unique to RwB, since, e.g., Deuteronomy does
    something similar (perhaps Segal considerd Deuteronomy to be RwB?).
    This is a weak criteria, in my mind.}
\end{enumerate}

\begin{enumerate}
\def\labelenumi{\arabic{enumi}.}
\setcounter{enumi}{3}
\item
  Expansion versus Abridgment: By-and-large, text editions are
  \emph{additive}. That is to say, when there is a discrepancy between
  the amount of content (as opposed to the order), typically the shorter
  text is considered older. Segal is here concerned with editorial
  changes, and not with scribal errors, which, of course could go in
  both directions (through parablepsis et al.). This property, he
  contends, is rooted in the conviction of the scribes that in order to
  reproduce a text, one must reproduce the \emph{entire}
  text.\autocite[24]{segal_henze2005} Rewritten bible texts, however,
  felt free to add \emph{or remove} material because they understood
  themselves to be composing an entirely new
  work.\autocite[24]{segal_henze2005}
\item
  Tendentious Editorial Layer: ``Editions'' do not change fundamental
  ideology of the work. For example, differing editions of Jeremiah may
  differ but those differences do not change the fundamental ideology of
  the work. Likewise, expansion and addition to the work (e.g.~additions
  to Daniel) are in line with the theological \emph{Tendenz} of the
  shorter book. On the other hand, RwB texts freely alter the ideologies
  of the text, for example, Jubilees.\autocite[25]{segal_henze2005}
\item
  Explicit References to the Source Composition: ``Editions'' cannot (in
  a meta-discusive sense) reference its base text. RwB texts can.
\end{enumerate}

In a more recent article, Tov returns to the topic of text editions and
their relationship to the phenomenon of
RwB\autocite{tov_krarrer-kraus2008}. Tov addresses three ``strange''
texts from the LXX which, for one reason or another, differ
significantly from the preserved MT (3 Kingdoms, Esther, and Daniel).
Evoking a number of Segal's criteria\autocite{segal_henze2005} for
inclusion in the category (which he acknowledges to be well accepted, if
not terribly well defined), Tov suggests that these LXX texts likewise
may exhibit 1) a new narrative frame, 2) expansion and abridgment, and
3) a tendentious editorial layer and therefore may be candidates for
RwB.

\subsubsection{Scribalism and Rewritten Bible}\label{scribalism-and-rwb}

It is important to think about what Tov and Segal are trying to
accomplish in these articles: They are trying to connect scribal
practices which allowed for exegetical additions and emendations to
``authoritative texts''---dramatic examples of which are provided by SP
and LXX (though one wonders why the Targums aren't included here;
perhaps because Tov is arguing for Hebrew \emph{Vorlagen} of these
texts, while the Targums represent a translation)---to the practices
which produced the \emph{new compositions} which scholars refer to as
RwB texts. This is very similar to Vermes, albeit from a more
``textual'' perspective.

What Tov's articles in particular demonstrate, however, is that the
issue of authorial \emph{intent} and \emph{purpose} may be at the heart
of the distinction between text edition and RwB. Of course, this is not
something that can be objectively proven, but it \emph{must} factor into
the conversation, even if we must settle for speculation. The result is
that, e.g., 4QReworked Pent. should be understood as RwB insofar as we
imagine the author attempting something \emph{other} than creating a
text edition of the Pentateuch. Presumably the author of GA did not
imagine himself creating a new edition of Genesis; the same with
Jubilees and Chronicles. The issue of whether the resultant text was
used authoritatively after the fact is beside the point; what matters
was whether the text was either intended to be (or from the reader's
perspective, whether the text was treated as) a copy of the text's
\emph{Vorlage}. And in the case of SP and LXX (and the Targums, I'd
say), this seems to have been the case. Thus, it seems like these should
not be treated as RwB.

\section{Rewritten Bible: A Genre, Process, or Something
Else?}\label{rwb-a-genre-process-or-something-else}

One of the central issues with the term Rewritten Bible is whether it
should be treated as a ``genre'' or as a ``process'' or ``activity.''
Vermes, is not particularly helpful in clarifying the issue:

\begin{quote}
The question has been raised whether the ``Rewritten Bible'' corresponds
to a process or a genre? In my view, it verifies both. The person who
combined the biblical test with its interpretation was engaged in a
process, but when his activity was completed, it resulted in a literary
genre.\autocite[8]{vermes_zsengeller2014}
\end{quote}

Within Vermes's schema of aggadic development, RwB occupied a liminal
space outside the genres of classical Jewish texts. Because these texts
eluded categorization within these established text groups (such as
Targums, or midrash), Vermes's treatment of RwB as a discrete group was
not unreasonable. A number of scholars have since upheld the categorical
approach and argued for RwB as a literary genre.

\subsection{Rewritten Bible as a Genre}\label{rwb-as-a-genre}

The parade example of this perspective is Philip Alexander's 1988
article ``Retelling the Old Testament,'' which, although dated, remains
the most widely cited exemplar of the ``genre'' perspective.\footnote{\textcite{alexander_carson-williamson1988}.
  Vermes himself even put his stamp of approval on it, see
  \textcite[4]{vermes_zsengeller2014}.} Alexander takes up four
rewritten Bible texts (Jubilees, Genesis Apocryphon,
Liber antiquitatum biblicarum, and Jewish Antiquities) to determine
whether there exists a set of concrete criteria by which scholars can
admit or exclude text from the category. Although I ultimately disagree
with his conclusion that RwB should be treated as a literary genre, his
list of nine ``principle characteristics'' make a number of useful
observations about the nature of RwB texts generally and are summarized
as follows:

\begin{enumerate}
\def\labelenumi{\arabic{enumi}.}
\tightlist
\item
  RwB texts are \emph{narratives} which follow the order of the biblical
  text.
\item
  RwB texts are ``free standing'' literary works that take on the same
  form as the text they rewrite. They do not comment explicitly on their
  \emph{Vorlagen}, but weave interpretation into their seamless
  retelling.
\item
  RwB texts are not meant to replace the biblical work.
\item
  RwB texts cover a large portion of the biblical narrative and exhibit
  a ``centripetal'' relationship to the biblical text.
\item
  RwB texts follow the biblical text's narrative ordering, but may omit
  certain, non-essential elements.
\item
  RwB texts offer an interpretive reading of scripture which, quoting
  Vermes offer, ``a fuller, smoother and doctrinally more advanced form
  of the sacred narrative''\autocite[Citing Vermes in][305]{schurer1986}
  and implicitly comment on the biblical text.
\item
  RwB texts are limited by their literary form which only allows a
  single interpretation of the biblical text that they rewrite.
\item
  RwB texts are limited by their literary form which does not allow them
  to explain their exegetical rationale.
\item
  RwB texts incorporate traditions and material not derived from the
  biblical text.
\end{enumerate}

Despite Alexander's emphatic conclusion affirming the genre of
Rewritten Bible, I find a number of these criteria to be unconvincing.

First, his criterion that the text be a \emph{narrative} strikes me as
arbitrary. While Vermes focused on RwB as a narrative phenomenon, he has
since noted that the reason for this was that his focus was on
\emph{aggadic} material, that is, non-halakhic interpretation, which by
definition is non-legal. Coupled with the first half of his second
observation---that RwB texts take on the same form as the text they
rewrite---these observations seem self-fulfilling and suffering from a
sort of selection bias.\footnote{Although, all of the texts he surveyed
  are narratives, this fact illustrates one of the major shortcomings in
  Alexander's method, specifically, that his conclusions were based on
  four texts ``normally included in the
  genre.''\autocite[99]{alexander_carson-williamson1988} Therefore the
  selection of these four texts was the result of a deductive selection,
  in part, based on their narrative form.}

Second, several of his criteria are comments about the intention of the
author or purpose of the work, which are both unverifiable. For example,
the question of whether a RwB text was ``meant'' to replace its
\emph{Vorlage}, is not clear, particularly when discussing texts---as
Vermes does---such as the Palestinian Targums, or (now) the so-called
Reworked Pentateuch (4QReworkedPent)\footnote{In fairness,
  4QRewrokedPent was not available to Alexander or Vermes. Yet, one
  still may wonder why the LXX or Samaritan Pentateuch are not included.}
Similarly, claiming that RwB texts ``implicitly comment'' on their
\emph{Vorlagen} speaks to the \emph{intention} of the author, which one
should be wary of doing. The fact that Alexander states that the author
was ``limited'' by the genre of narrative to a single interpretation and
could not provide his exegetical rationale illustrates the major,
overarching assumption about Alexander's (and Vermes's) approach to
these texts---that the essential function of the texts and the purposes
of their authors are the same as the later exegetes.

Alexander insists that ``Any text admitted to the genre must display
\emph{all} the characteristics,''\autocite[119 n.
11]{alexander_carson-williamson1988} but he offers no formal rationale
for selecting his sample. The texts that he selects, indeed, represent
the \emph{core} of what is generally accepted to be RwB, but texts on
the periphery of a genre, almost by definition, will not display
\emph{every} characteristic of the core texts. From my perspective,
however, Alexander's criteria should not be treated as prerequisites for
inclusion to the category of RwB, if we are to treat it as such.
Instead, they should be used to describe a sort of literary
\emph{Idealtypus} for RwB.

Moshe Bernstein, too, has upheld a Vermesian understanding of RwB as a
literary category and has argued that for the category to be useful to
scholars, the boundaries must be clearly demarcated and reasonably
narrow.\autocite{bernstein_textus2005} Notably, Bernstein never clearly
articulates what it means for a category to be ``useful.'' All the same,
he sets out to:

\begin{quote}
``examine the definition and descriptions of''rewritten Bible" proffered
by Vermes and several subsequent scholars, in order to delineate the
variety of ways in which the term is currently employed and to make some
suggestions for how we might use it more clearly and definitively in the
future." \autocite[171--172]{bernstein_textus2005}
\end{quote}

Bernstein begins by addressing the few small modifications that he makes
to Vermes's list, namely that Bernstein does not understand the Targums
to be examples of RwB. He excludes Targums from his discussion
``\emph{ab initio},'' as well as ``biblical'' books, (by which he seems
to mean ``Chronicles''), and includes legal texts such as the
Temple Scroll. Despite this second exclusion, Bernstein acknowledges
that ``One group's rewritten Bible could very well be another's biblical
text!'' \autocite[175. This seems particularly odd, since, and Ethiopian
Christian may protest that Jubilees should be excluded as
well.]{bernstein_textus2005}. Thus, Bernstein concedes that ``matters of
canon and audience may play a role,'' but doesn't address the topic
further.

Bernstein critiques scholars such as
Nicklesburg,\autocite{nickelsburg_stone1984}
Harrington,\autocite{harrington_kraft-nickelsburg1986} and
Brooke\autocite{brooke_schiffman-vanderkam2000} for excessively
expanding the use of the term RwB at its ``upper bound'' (my term) to
the point that they have weakened the term and have ``not aided in
focusing scholarly attention on the unifying vs.~divergent traits of
some of these early interpretive
works.''\autocite[179]{bernstein_textus2005} Likewise, Bernstein
critiques Tov for including reworked texts (e.g., 4QRP) and therefor
expanding the ``lower bound'' of the category. While Bernstein avers
that ``Rearrangement with the goal of interpretation is probably an
earlier stage in the development of biblical `commentary' than
supplementation with the goal of interpretation,'' he nevertheless
distinguishes the former from the category RwB and declares that ``the
definitions of `rewritten Bible' furnished by Tov and Vermes are
{[}not{]} even remotely compatible, and we need to choose between them
simply for the purposes of
clarity.''\autocite[185]{bernstein_textus2005} Bernstein, ultimately,
argues that Vermes's category is worth keeping around, and admonishes
the reader to maintain a narrow definition of the category, because, in
his own words, ``the more specific the implications of the term, the
more valuable it is as a measuring
device,''\autocite[195]{bernstein_textus2005} and conversely that ``the
looser the definition, the less precisely it classifies those items
under its rubric.'' \autocite[195]{bernstein_textus2005}

\subsection{Rewritten Bible as a Process}\label{rwb-as-a-process}

At the other end of the spectrum, a number of important scholars have
treated RwB as a ``process'' or ``activity'' rather than as a genre or
category. These scholars also have tended to be more ``expansive'' when
it comes to which texts should be discussed as ``rewritten.''

HARRINGTON He states:

\begin{quote}
Nevertheless, establishing that these books are not appropriately
described as targums or midrashim is not the same as proving that they
all represent a distinctive literary genre called ``rewritten Bible.''
In fact, it seems better to view rewriting the Bible as a kind of
activity or process than to see it as a distinctive literary genre of
Palestinian
Judaism``\autocite[242--243]{harrington_kraft-nickelsburg1986}
\end{quote}

Instead, he observes that while texts such as Jubilees and
\emph{Assumption of Moses} both constitute a rewriting of the Bible,
both ``are formally revelations of
apocalypses.''\autocite[243]{harrington_kraft-nickelsburg1986} This is
an important criticism of scholars who see RwB as a distinct genre.
Unlike, for example, the Gospels, which arguably have the same basic
``form,'' the texts typically described as ``rewritten'' come in a
variety of ``forms'' such as narratives (Genesis Apocryphon),
apocalypses (Jubilees), and, legal (Temple Scroll). In other words, a
single \emph{genre}---insofar as the word describes a literary
\emph{form}---is not sufficient to subsume the varied \emph{forms} which
all can be described as ``rewritten.'' From my perspective, this
observation is at the heart of the discussion.

More recently, Molly Zahn has attempted to move the conversation forward
by interacting with modern genre theory---which is conspicuously absent
from most discussions of ``genre'' and
RwB.\autocites{zahn_jbl2012}[Daniel Machiela noted the absence of genre
theory in his 2010 article, as well, see][]{machiela_jjs2010}[Notable
exceptions include][]{brooke_dsd2010} Zahn addresses the difficulty that
Harrington addresses by noting that works may participate in multiple
genres simultaneously. While older conceptions of genre ``pigeonhole''
texts to specific genres, modern genre theorists---she cites
Fowler---now prefer to talk about texts ``participating'' in a genre.
Citing Fowler, Zahn notes that ``genres are less like pigeonholes and
more like pigeons'' and further augments the metaphor to suggest that
genres are ``more like flocks of pigeons.'' She writes:

\begin{quote}
Just as a flock of pigeons might change shape, lose and add members, be
absorbed into larger flocks of break apart into several smaller flocks,
genres and their boundaries are not static.\autocite[277]{zahn_jbl2012}
\end{quote}

This is a solid post-structuralist understanding of genre, and one that
should be taken seriously. The implications for RwB is clear: although
Jubilees is a revelatory text while the Temple Scroll is a legal text,
they can both participate in their respective ``formal'' genres
simultaneously with the supposed RwB genre.

Zahn also explores the ``functional'' aspects of genre. She notes that
genres are ``not simply systems of classifications developed and used by
literary critics, but are fundamentally to all human
communication.''\autocite[280]{zahn_jbl2012} Thus, genres manifest as
common patterns recognized by both the author and the reader which aid
communication;\autocites[276]{zahn_jbl2012}[See
also][199]{newsome_grossman2010} in this way, genre functions as a sort
of ``literary body language.''

The question of whether RwB should be considered a literary genre is
dependent on two very important and interrelated issues.
