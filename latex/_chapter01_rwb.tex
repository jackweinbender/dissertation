\chapter{The \textasciitilde{}RwB}\label{the-rwb}

\section{Emergence of the Concept of
\textasciitilde{}rwb}\label{emergence-of-the-concept-of-rwb}

In his seminal work \textasciitilde{}stj-full, Geza Vermes introduced
the term ``\textasciitilde{}rwB'' to the discussion of Second Temple
Jewish literature to describe an exegetical process by which ``the
midrashist inserts haggadic development into the biblical narrative'' in
order to ``anticipate questions, and to solve problems in
advance.''\autocites[95]{vermes1961}[see also][]{vermes_zsengeller2014}
The authors of these texts, therefore, make use of \emph{implicitly}
those interpretive traditions which later exegetical texts make
\emph{explicit}.

\subsection{Scripture and Tradition in Judaism
(1961)}\label{scripture-and-tradition-in-judaism-1961}

The primary purpose of \emph{Scripture and Tradition}, however, was not
to offer a clear definition of the term ``\textasciitilde{}rwB,'' but to
lay the groundwork for the historical, diachronic, study of aggadic
traditions, of which \textasciitilde{}rwB makes only a small
part.\autocite[3]{vermes_zsengeller2014} As Vermes recounts, prior to
the mid-twentieth century, the prevailing approach to the study of
aggadic exegesis was to treat the aggadah as originating during the
Tannaitic period. The aggadah were viewed as ``the result of the
adoption, and anonymous repetition, of popular interpretations by
favourite preachers,'' \autocite[3]{vermes1961} the earliest of which
were from the second century CE and were represented by Targums Onkelos
and Jonathan. Furthermore, studies of ancient Jewish literature at this
time focused on texts which modern Judaism considered authentic. As a
result, a good number of earlier texts---for example, the apocrypha,
pseudepigrapha, and sectarian texts---were often categorically excluded
from discussions of the origins of aggadic
exegesis.\autocite[2]{vermes1961}

A series of publications and discoveries beginning in the 1930's,
however, began to undermine the notion that these early exegetical
traditions began in the second century CE. Vermes credits this
broadening of aggadic studies to a series of major studies and
discoveries such as Rappaport's \emph{Agada und Exegese bei Flavius
Josephus},\autocite{rappaport1930} Paul Kahle's Schweich Lectures at the
British Academy on the Cairo Geniza (given in 1941, published
1947),\autocite{kahle1947} Kisch's new text edition of Ps. Philo's
\textasciitilde{}lab (1949),\autocite{kisch1949} the discovery of the
Dead Sea Scrolls (1948) and Codex Neofiti (1956), as well as (and
perhaps especially) Renée Bloch's work on midrash
\autocites{bloch1954}{bloch1955_repr}[3--7]{vermes1961}. The overarching
theme among these works was the evidence for continuity between biblical
interpretive traditions prior to the second century, and later aggadah.
For example, Vermes notes that Rappaport's work on \textasciitilde{}ant
identified substantial overlaps between Josephus's text and Rabbinic
aggadah and suggested, therefore, that Josephus had drawn from an
already living tradition of interpretation. The implication of his
suggestion is that the aggadah of the second century were not novel
exegetical works, but were themselves products of earlier exegetical
traditions.

Building on these recent advancements, the explicitly stated purpose of
\emph{Scripture and Tradition} was to push the field beyond synchronic
analysis of haggadah toward diachronic, historical analyses to trace the
development of these exegetical traditions \autocites[1]{vermes1961}[See
also][]{bloch1955_repr}. The book is eight chapters long and is divided
into four parts.

The first three chapters of \emph{Scripture and Tadition} make up the
first part of Vermes's study and is entitled " The Symbolism of Words."
In this section, Vermes focuses on the way that individual words and
ideas take on localized symbolic meanings which are then applied
globally to the interpretation of the Bible. In his first chapter, he
notes the divergent treatment of Gen 44:18--19 among ancient
commentators and proceeds through a synoptic study of this passage in
the Fragmentary Targum, Targum Neofiti, and the Tosefta of Targum
Yerushalmi. He concludes that the Fragmentary Targum represents the most
primitive work, whose interpretive strategy is essentially
inner-biblical (by harmonizing Gen 44 and 49 with Exod 7--9), followed
by the Tosefta, which seems to depend on the Fragmentary Targum but
takes a different interpretive stance, and finally Targum Neofiti, which
combines the two. In this second chapter, Vermes examines the way that
the word ``Lebanon'' came to be used symbolically in the Hebrew Bible
and other jewish literature and how those symbolic meanings developed,
particularly the association of Lebanon with Jerusalem and the Temple.
He identifies the Song of Songs as the intermediary text which helped to
establish this tradition within post-exilic Judaism, noting that the
book occupies the unique position as the only biblical text which
clearly uses the name Lebanon symbolically for the Temple. Importantly,
Vermes shows that the symbolic use of Lebanon to represent Jerusalem and
the Temple is rooted in \emph{biblical} exegesis. This is a key idea for
Vermes because it establishes a continuity between the production of the
biblical text and its later interpretation. In chapter three, Vermes
examines other words which take on symbolic meaning in later Jewish
texts (``lion,'' ``Damascus,'' ``\emph{Meḥoqeq},'' and ``Man''), and
shows the relative similarity of the process between the DSS texts and
the targumic and midrashic materials.

It is in the second part of \emph{Scripture and Tradition} that the
topic of \textasciitilde{}RwB is addressed most directly. This
section---titled ``The \textasciitilde{}RwB''--- covers two chapters
(chapters four and five), both focusing on the figure Abraham and the
aggadic traditions surrounding his life.

In chapter four, Vermes embarks on what he calls a ``retrogressive
historical study'' by which he means beginning with the traditions in
their later, more developed forms, and working back toward their
origins. In this case, Vermes begins with the 11th century CE text
\emph{Sefer ha-Yashar} (and the varied extra-biblical traditions
contained therein) then works backward to identify sections of the text
which exhibit earlier traditions, for example, in Targums, Josephus,
Jubilees, (Ps.) Philo, and others. The purpose of this chapter is to
demonstrate that even late texts can contain valuable information about
earlier methods of exegesis. As Vermes puts it, ``{[}Sefer ha-Yashar{]}
manifests a direct continuity with the corresponding tradition of the
time of the second Temple, but reflects also the influence of the
haggadah of the Tannaim and Amoraim.''\autocite[95]{vermes1961}

In chapter five, Vermes proceeds with a ``progressive historical
study,'' by which he means a study beginning with the oldest materials
and working forward. Still focusing on the figure Abraham, Vermes treats
in detail the relationship between Gen 12:8--15:4 and cols. 19--22 of
the \textasciitilde{}ga. Vermes treats \textasciitilde{}ga as ``the most
ancient midrash of all''\autocite[124]{vermes1961} and views it as the
``lost link between the biblical and the Rabbinic midrash''
\autocite[124]{vermes1961}. As I understand it, the \textasciitilde{}ga,
for Vermes, occupies a unique position just one step removed from
inner-biblical exegesis. The task of \textasciitilde{}ga's author was
``to make the biblical story more attractive, more real, more edifying,
and above all more intelligible'' and he accomplishes this through a
variety of means \autocite[125]{vermes1961}. The work of
\textasciitilde{}ga's author, therefore, was to ``{[}reconcile{]}
unexplained or apparently conflicting statements in the biblical text in
order to allay doubt and worry.''\autocite[125]{vermes1961} According to
Vermes, the interpretation of Genesis is ``organically bound'' to the
text of Genesis and the additions that \emph{were} made sprung from the
interpretation of the Bible itself and not whole-sale from the mind of
the author \autocite[126]{vermes1961}. Where texts like Jubliees sought
to systematically advance a theological vision, according to Vermes, the
author of \textasciitilde{}ga intended to simply ``explain the biblical
text,'' calling it illustrative of ``the unbiased rewriting of the
Bible.''\autocite[126]{vermes1961}

The final two chapters make up the final section, titled ``Theology and
Exegesis'' and push the discussion to include early Christianity.
Chapter seven is entitled ``Circumcision and Exodus 4:24--26'' but
offers a subtitle of ``Prelude to the Theology of Baptism,'' which gives
some hint at the ultimate, if tacit, goal of the chapter. Discussing the
topic of circumcision in Ex 4:24--26 and its treatment among the early
exegetes, Vermes's primary observation is simply that the theology of
circumcision and the exegetical traditions which surrounded it, were
affected by historical forces and theological ideologies. For instance,
he claims that Jubilees omitted the rather odd statement that God was
going to kill Moses---who was saved by the circumcision of his son by
Zipporah---because ``It was impossible for its author to accept that God
tried to kill Moses as it was for him to believe that Moses neglected to
circumcise his son on the eighth day after his
birth.''\autocite[185]{vermes1961} Similarly, he notes that after the
Bar Kokhba rebellion, the practice of circumcision was outlawed and so,
``it is not surprising, therefore, to find the spiritual authorities of
Palestinian Judaism emphasizing the greatness and necessity of this
essential rite, and explaining away \ldots{} every possible biblical
excuse for delaying the circumcision of their
children.''\autocite[189]{vermes1961} He ends the chapter by suggesting
that the early Christian association of baptism with circumcision
(citing Rom 4:3--4 and Col 2:11--12) was enabled by the traditional
Jewish association of circumcision withe blood sacrifice (``the Blood of
the Covenant'')\autocite[190]{vermes1961}. That Paul (to whom he
attributes both Romans and Colossians) associated baptism with
circumcision therefore, was ``not due, therefore, to his own insight,
but springs directly from the contemporary Jewish doctrine of
circumcision which he adopted and adapted.''\autocite[191]{vermes1961}

Vermes makes a similar move in chapter eight, entitled ``Redemption and
Genesis XXII: The Binding of Isaac and the Sacrifice of Jesus.'' In it,
he compares a number of ancient works' treatment of the Aqedah and
demonstrates how the (near-) sacrifice of Isaac became a prototype for
the entire sacrificial system in later judaism. The sacrifice of animals
in the Temple functioned as a ``reminder'' to God of the faithfulness of
Abraham. Furthermore, he shows the ways the tradition grew to focus on
the willingness of Isaac to be sacrificed and his function as a
proto-matryr. Thus, he ends the chapter by addressing the New
Testament's portrayal of Jesus as a willing sacrifice to God and its
putative relationship to the Aqedah. Vermes makes the case that the
redemptive theology of the NT---typically attributed to Paul---was not
original to him. He writes:

\begin{quote}
For although {[}Paul{]} is undoubtedly the greatest theologian of the
Redemption, he worked with inherited materials and among these was, by
his own confession, the tradition that ``Christ dies for us according to
the Scriptures.''\autocite[221]{vermes1961}
\end{quote}

He then proceeds to push the origin of this theology back further into
the first century CE, and, in rather dramatic fashion, suggests that the
introduction of the Aqedah motif into Christian theology---by means of
the Suffering Servant---may have been by Jesus himself
\autocite[223]{vermes1961}.

He concludes the chapter by discussing the Aqedah and the Eucharist.
Just as the whole sacrificial system pointed back toward the binding of
Isaac in targumic exegesis, the eucharistic rite likewise was
intended---according to Vermes---to point back to Jesus's redemptive
sacrifice. Thus he concludes:

\begin{quote}
Although it would be inexact to hold that the Eucharistic doctrine of
the New Testament, together with the whole Christian doctrine of
Redemption, is nothing but a Christian version of the Jewish Akedah
theology, it is nevertheless true that in the formation of this doctrine
the targumic representation of the Binding of Isaac has played an
essential role.
\end{quote}

\begin{quote}
Indeed, without the help of Jewish exegesis it is impossible to perceive
any Christian teaching in its true
perspective.\autocite[227]{vermes1961}
\end{quote}

The arc of Vermes's study, therefore, is meant to establish a continuity
between the earliest traditions of biblical interpretation with the
later traditions of both Rabbinic Judaism and Early Christianity and to
trace the evolution of those traditions historically.

\subsection{Vermes's Use of
\textasciitilde{}RwB}\label{vermess-use-of-rwb}

Although the term \textasciitilde{}rwb need not be restricted to the
original intent of Vermes, it is worthwhile, I think, to

\section{Genre and Process}\label{genre-and-process}

One of the central issues with the term \textasciitilde{}RwB is whether
it should be treated as a ``genre'' or as a ``process'' or ``activity.''
Because these texts eluded categorization within traditional established
text groups such as Targums, or midrash, early treatment of
\textasciitilde{}rwb as a discrete group was not unreasonable. Although
scholars such as Daniel Harrington, argued against the treatment of
\textasciitilde{}rwb as a generic category quite
early,\autocite{harrington_kraft-nickelsburg1986} a number of scholars
have since argued the reverse.

The parade example of this perspective is Philip Alexander's 1988
article ``Retelling the Old Testament,'' which, although dated, remains
the most widely cited exemplar of the ``genre''
perspective.\autocite{alexander_carson-williamson1988} Alexander takes
up four \textasciitilde{}rwB texts (\textasciitilde{}jub,
\textasciitilde{}ga, \textasciitilde{}lab, and \textasciitilde{}ant) to
determine whether there exists a set of concrete criteria by which
scholars can admit or exclude text from the category. Although I
ultimately disagree with his conclusion that \textasciitilde{}rwb should
be treated as a literary genre, his list of nine ``principle
characteristics'' make a number of useful observations about the nature
of \textasciitilde{}rwb texts generally and are summarized as follows:

\begin{enumerate}
\def\labelenumi{\arabic{enumi}.}
\tightlist
\item
  \textasciitilde{}rwb texts are \emph{narratives} which follow the
  order of the biblical text.
\item
  \textasciitilde{}rwb texts are ``free standing'' literary works that
  take on the same form as the text they rewrite. They do not comment
  explicitly on their \emph{Vorlagen}, but weave interpretation into
  their seamless retelling.
\item
  \textasciitilde{}rwb texts are not meant to replace the biblical work.
\item
  \textasciitilde{}rwb texts cover a large portion of the biblical
  narrative and exhibit a ``centripetal'' relationship to the biblical
  text.
\item
  \textasciitilde{}rwb texts follow the biblical text's narrative
  ordering, but may omit certain, non-essential elements.
\item
  \textasciitilde{}rwb texts offer an interpretive reading of scripture
  which, quoting Vermes offer, ``a fuller, smoother and doctrinally more
  advanced form of the sacred narrative''\autocite[Citing Vermes
  in][305]{schurer1986} and implicitly comment on the biblical text.
\item
  \textasciitilde{}rwb texts are limited by their literary form which
  only allows a single interpretation of the biblical text that they
  rewrite.
\item
  \textasciitilde{}rwb texts are limited by their literary form which
  does not allow them to explain their exegetical rationale.
\item
  \textasciitilde{}rwb texts incorporate traditions and material not
  derived from the biblical text.
\end{enumerate}

Despite Alexander's emphatic conclusion affirming the genre of
\textasciitilde{}RwB, a few of these criteria are unconvincing. While
Vermes, too, affirmed that one characteristic of \textasciitilde{}rwb
was only a narrative phenomenon,{[}vermes GET THIS{]} texts such as the
Temple Scroll q{[}11QT\textsuperscript{a}{]} have caused some, such as
Bernstein, to call this conclusion into
question.\autocite{bernstein_textus2005} This caveat illustrates one of
the major shortcomings in Alexander's method, specifically, that his
conclusions were based on four texts ``normally included in the
genre.''\autocite[99]{alexander_carson-williamson1988} Alexander insists
that ``Any text admitted to the genre must display \emph{all} the
characteristics,''\autocite[119 n. 11]{alexander_carson-williamson1988}
but he offers no formal rationale for selecting his sample and so his
conclusions suffer from a sort of selection bias. The texts that he
selects, indeed, represent the \emph{core} of what is generally accepted
to be \textasciitilde{}rwb, but texts on the periphery of a genre,
almost by definition, will not display \emph{every} characteristic of
the core texts. Thus the criteria that Alexander proposes should not be
treated as prerequisites for inclusion to the category of RwB, but as
attributes of a sort of literary \emph{Idealtypus}.

Alexander's third criteria also suffers

\section{Bible and Scripture}\label{bible-and-scripture}

\section{Text Editions v.
\textasciitilde{}rwb}\label{text-editions-v.-rwb}

\section*{Texts (Lists, etc. Parabiblical v
\textasciitilde{}rwb)}\label{texts-lists-etc.-parabiblical-v-rwb}
\addcontentsline{toc}{section}{Texts (Lists, etc. Parabiblical v
\textasciitilde{}rwb)}
