% !TEX root = dissertation.tex

%%%%%%%%%%%%%%%%%%%%%%%%%%%%%%%%%%%%%%%
%%% This file relies on the package %%%
%%% utexas-dissertation-frontmatter %%%
%%%%%%%%%%%%%%%%%%%%%%%%%%%%%%%%%%%%%%%

% Document meta
\author{Jack Lawrence Weinbender III}
\date{August 2019}
\title{Remembering and Rewriting}
% utexas-dissertation-frontmatter specific meta
\subtitle{Reframing Rewritten Bible through Memory Studies}
\doctype{Dissertation}
\institution{The University of Texas at Austin}
\degree{Doctor of Philosophy}
\degreeabbr{Ph.D.}


%% Copyright Page Info
\copyrightyear{2019}
%%Copyright Page
\makecopyright


%% Signature Page Details
\numberofmembers{4}
\supervisor{Jonathan Kaplan}
\memberone{Jo Ann Hackett}
\membertwo{John Huehnergard}
\memberthree{Steven Fraade}
% Signature Page
\makesignatures

%% Title Page
\maketitle

\clearpage
\vspace*{3in}
\begin{center}
  {\large\itshape For my teachers.}
\end{center}

\clearpage
\vspace*{2in}
\begin{quote}
  \emph{The existing order is complete before the new work arrives; for order to persist after the supervention of novelty, the whole existing order must be, if ever so slightly, altered; and so the relations, proportions, values of each work of art toward the whole are readjusted; and this is conformity between the old and the new. Whoever has approved this idea of order, of the form of European, of English literature, will not find it preposterous that the past should be altered by the present as much as the present is directed by the past. And the poet who is aware of this will be aware of great difficulties and responsibilities.}

  \begin{center}
    -- T. S. Elliot
  \end{center}
\end{quote}

\clearpage
\chapter*{Acknowledgments}

%%%%%%%%%%%%%%%%%%%%%%%%%%%%

I would first and foremost like to thank my dissertation supervisor, Jonathan Kaplan, for all of his encouragement and support over the past several years. I genuinely do not know if I would have finished this project without his constant guidance and well-timed prodding. Throughout the two years that I spent on this dissertation, he met with me on a weekly basis. He never made me feel like a burden. When we had finished discussing my work, he was always quick to ask how I was doing personally and was happy to sit and chat about life. ``After all,'' he'd say, ``this is your time.''

I would also like to thank the other members of my committee. Professors Hackett and Huehnergard (affectionately, ``John and Jo Ann'') have been models for methodological rigor, attention to detail, and genuine human kindness. Their feedback on this project consistently pushed me to think more clearly and carefully as a philologist and biblical scholar. Prof. Fraade was kind enough to meet with me at the Association for Jewish Studies conference this past December and offered many insightful comments. He also pointed me toward a number of important primary and secondary texts for which I am exceedingly grateful.

A number of other colleagues at UT deserve mention as well. Although Prof. Pat-El was not on the committee, she was immensely supportive as an instructor during my coursework and has been a strong advocate for all of us graduate students as the department chair. The coursework I completed with Professors Friesen and Schofer in the department of Religious Studies has impacted this dissertation immensely and I am grateful to both of them for their insights and guidance.

Several friends and colleagues provided me with constructive feedback on various sections of this project. Chris Keith of St Mary's University, Twickenham and Daniel Pioske of Georgia Southern University both provided expert critiques of my chapter on memory studies and offered exceedingly helpful suggestions to make it better. Where I ignored their advice, I did so at my own peril. Drew Keane, also of Georgia Southern University, gave me helpful feedback on several chapters, as did Christopher Frisina and Nathaniel Greene.

Several former teachers of mine---though not formally involved with the dissertation---should also be mentioned; it was largely the foundation that they laid that enabled me to succeed as a doctoral student and to conduct this research. Christopher Rollston of the George Washington University, Jason Bembry of Emmanuel Christian Seminary, Rafael Rodríguez of Johnson University, and John Rumple all played formative roles in cultivating my interest in studying the languages, cultures and religions of the ancient Near East, the Hebrew Bible and Second Temple Judaism.

My classmates at UT have been a constant source of support. In particular Sigrid Kjær and Øyvind Bjøru have been the finest folks with whom to endure this long journey. I will always think fondly on the evenings Oda, Sigrid, Øyvind and I spent at the Crown \& Anchor drinking pitchers and complaining about the things that graduate students complain about. Y'all are the best.

To my former colleagues---many of whom also have or are currently finishing PhD's---who have continued to be close friends: Adam Bean, Christopher Frisina, and Nathaniel Greene. Thank you all for your empathy and camaraderie and for putting up with me when I become obnoxious in the group chat.

I would also like to thank my parents, who have never been anything but supportive of me throughout this process, They were my first and most formative teachers.

Finally, I would like to thank my wife, Tiffany Weinbender. She is my very best friend and my most enthusiastic cheerleader. She has endured more than anyone else and sacrificed more for this degree than I have. Thank you for your love and your affection. Without you none of this would matter.

This dissertation is dedicated to my teachers. It took me twenty-six years, but I'm finally done. Thank you all.

%%%%%%%%%%%%%%%%%%%%%%%%%%%%

\begin{abstract}
  This project introduces the theoretical framework of memory studies to the topic of \rwb to illustrate the usefulness of social and cultural memory theory for describing the relationships between \rwb texts, their putative \vorlagen, and the societies that produced them. Thus, this dissertation is a first step toward a broader theory of \rwb that frames \rwb texts by their participation in social discourses about their remembered past. Although I do not contest the idea that portions of \rwb texts reflect ``biblical interpretation,'' I argue that characterizing these texts as \emph{primarily} or \emph{essentially} focused on explaining the biblical text limits how scholars can talk about how these texts may have \emph{functioned} in the ancient world. Instead, I suggest that memory studies provides a robust model and useful taxonomy for describing \rwb texts as cultural products that participated in the construction of cultural memory by receiving and adapting traditions of Israel's remembered past, represented---in part---by the biblical text. Memory theory offers models to describe the reception of traditions into contemporary discourses, the effects of contemporary discourses on the remembered past, and the construction of new memories that inform future generations. We can benefit from using this language to broaden our discussion about what roles \rwb texts might have played in antiquity. This approach takes seriously the fact that \rwb texts utilized the Bible as a major source of tradition without limiting the discussions of \rwb to how each text related to the Bible or its scriptural predecessors.
\end{abstract}


\tableofcontents*
\begin{SingleSpace}
  \printbiblist[heading=biblistintoc]{abbreviations}
\end{SingleSpace}