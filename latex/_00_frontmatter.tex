% !TEX root = dissertation.tex

%%%%%%%%%%%%%%%%%%%%%%%%%%%%%%%%%%%%%%%
%%% This file relies on the package %%%
%%% utexas-dissertation-frontmatter %%%
%%%%%%%%%%%%%%%%%%%%%%%%%%%%%%%%%%%%%%%

% Document meta
\author{Jack Lawrence Weinbender III}
\date{August 2019}
\title{Remembering and Rewriting}
% utexas-dissertation-frontmatter specific meta
\subtitle{Reframing Rewritten Bible through Memory Studies}
\doctype{Dissertation}
\institution{The University of Texas at Austin}
\degree{Doctor of Philosophy}
\degreeabbr{Ph.D.}


%% Copyright Page Info
\copyrightyear{2019}
%%Copyright Page
\makecopyright


%% Signature Page Details
\numberofmembers{4}
\supervisor{Jonathan Kaplan}
\memberone{Jo Ann Hackett}
\membertwo{John Huehnergard}
\memberthree{Steven Fraade}
% Signature Page
\makesignatures

%% Title Page
\maketitle

\clearpage
\vspace*{3in}
\begin{center}
  {\large\itshape For my teachers.}
\end{center}

\clearpage
\vspace*{2in}
\begin{quote}
  \emph{The existing order is complete before the new work arrives; for order to persist after the supervention of novelty, the whole existing order must be, if ever so slightly, altered; and so the relations, proportions, values of each work of art toward the whole are readjusted; and this is conformity between the old and the new. Whoever has approved this idea of order, of the form of European, of English literature, will not find it preposterous that the past should be altered by the present as much as the present is directed by the past. And the poet who is aware of this will be aware of great difficulties and responsibilities.}

  \begin{center}
    -- T. S. Elliot
  \end{center}
\end{quote}

\clearpage
\chapter*{Acknowledgments}

%%%%%%%%%%%%%%%%%%%%%%%%%%%%

% Kaplan
% Huehnergard
% Hackett
% Fraade

% Na'ama, Friesen, Schofer, Chad Seales

% Rollston, Bembry
% Rumple, Rodriguez

% Chris Keith, Dan Pioske, Drew Keane

% Adam, Ned, CJ

% Øyvind, Sigrid & classmates

% Family, Friends

% Tiffany

% All my Teachers


%%%%%%%%%%%%%%%%%%%%%%%%%%%%

\begin{abstract}
  This project introduces the theoretical framework of memory studies to the topic of \rwb to illustrate the usefulness of social and cultural memory theory for describing the relationships between \rwb texts, their putative \vorlagen, and the societies that produced them. Thus, this dissertation is a first step toward a broader theory of \rwb that frames \rwb texts by their participation in social discourses about their remembered past. Although I do not contest the idea that portions of \rwb texts reflect ``biblical interpretation,'' I argue that characterizing these texts as \emph{primarily} or \emph{essentially} focused on explaining the biblical text limits how scholars can talk about how these texts may have \emph{functioned} in the ancient world. Instead, I suggest that memory studies provides a robust model and useful taxonomy for describing \rwb texts as cultural products that participated in the construction of cultural memory by receiving and adapting traditions of Israel's remembered past, represented---in part---by the biblical text. Memory theory offers models to describe the reception of traditions into contemporary discourses, the effects of contemporary discourses on the remembered past, and the construction of new memories that inform future generations. We can benefit from using this language to broaden our discussion about what roles \rwb texts might have played in antiquity. This approach takes seriously the fact that \rwb texts utilized the Bible as a major source of tradition without limiting the discussions of \rwb to how each text related to the Bible or its scriptural predecessors.
\end{abstract}


\tableofcontents*
\begin{SingleSpace}
  \printbiblist[heading=biblistintoc]{abbreviations}
\end{SingleSpace}